\documentclass[a4paper,14pt]{article}
\usepackage[T2A]{fontenc}
\usepackage[utf8x]{inputenc}
\usepackage[english, russian]{babel}
\usepackage{cmap}
\usepackage{indentfirst}
\setlength{\parindent}{1.25cm}
\setlength{\parskip}{6pt}
\usepackage{float}
\usepackage{multirow}
\usepackage{booktabs} 
\usepackage{amsmath}
\usepackage{tabularx}
\usepackage{caption}
\usepackage{mathtools}
\usepackage{diagbox}
\usepackage{graphicx}
\usepackage{hyphenat}
\usepackage{setspace}
\usepackage{fancyhdr}
\usepackage{longtable}
\usepackage{graphicx}
\usepackage{alltt}
\usepackage{breqn}
\usepackage{esint}
\usepackage{enumitem}
\usepackage{amsfonts}
\usepackage{tocloft} %для глав
\usepackage{hyperref}
\usepackage[linewidth=1pt]{mdframed}
\usepackage{lipsum}
\captionsetup{justification=centering}
\usepackage{xcolor,fancyhdr}

\usepackage{tikz}
\usetikzlibrary{calc}
\usetikzlibrary{decorations.pathmorphing}

\pagestyle{fancy}
\fancyhf{}
\fancyhead[L]{ПМ-32 Коновалов Я. А. N=1}

\renewcommand\headrulewidth{0pt}
\fontfamily{ptm}\selectfont
\cfoot{}
\renewcommand{\baselinestretch}{1.5}
\usepackage[left=3cm,right=1cm,top=2cm,bottom=2cm]{geometry}

\fancyfoot[R]{\thepage}
\fancyfootoffset{0mm}

\fancypagestyle{plain}{ %
  \fancyhf{} % remove everything
  \renewcommand{\headrulewidth}{0pt} % remove lines as well
  \renewcommand{\footrulewidth}{0pt}
  \rfoot{\thepage}}


\begin{document}
\thispagestyle {empty}

\newgeometry{top=20mm,bottom=20mm,left=30mm,right=10mm}

\begin{tikzpicture} [overlay,remember picture]
    \draw [line width=0.5mm ] 
   ($ (current page.north west) + (3cm, -3cm) $)
    rectangle
    ($ (current page.south east) + (-1.5cm,3.0cm) $);
\end{tikzpicture}

\begin{Large} 
\begin{center}

\vspace{30px}

САРОВСКИЙ ГОСУДАРСТВЕННЫЙ \\ ФИЗИКО-ТЕХНИЧЕСКИЙ ИНСТИТУТ

\vspace{40px}

КАФЕДРА ПРИКЛАДНОЙ МАТЕМАТИКИ

\vspace{40px}

\bf{ПОЯСНИТЕЛЬНАЯ ЗАПИСКА\\ К КУРСОВОЙ РАБОТЕ}

\end{center}
\end{Large}

\vspace{10px}

\begin{center}
\begin{large}
на тему: применение численных методов \par 
для решения различных математических задач
\end{large}
\end{center}

\vspace{40px}

СТУДЕНТ \underline{ {\hspace{4cm} Коновалов Ярослав {\hspace{4.65cm} }}}

\vspace{5px}

ГРУППА \underline{ {\hspace{5cm} ПМ-32 {\hspace{6.05cm} }}}

\vspace{5px}

ДАТА \underline{\hspace{13.1cm}}

\vspace{40px}

РУКОВОДИТЕЛЬ РАБОТЫ: \underline{ {\hspace{1.45cm} Кидямкина Дина Николаевна

{\hspace{2.4cm} }}}

\vspace{5px}

ЗАВ. КАФЕДРОЙ: \underline{ {\hspace{2.35cm} Шагалиев Рашит Мирзагалиевич 

{\hspace{2.7cm} }}}

\vspace{50px}

\begin{center}
\textbf{г. Саров, 2025 г.}
\end{center}
\restoregeometry


\newpage
\thispagestyle{empty}

\pagenumbering{arabic}
\setcounter{page}{2}% начать нумерацию с

\newgeometry{top=20mm,bottom=20mm,left=30mm,right=12.5mm}
\begin{tikzpicture} [overlay,remember picture]
    \draw [line width=0.5mm ] 
   ($ (current page.north west) + (3cm, -3cm) $)
    rectangle
    ($ (current page.south east) + (-1.5cm,3.0cm) $);
\end{tikzpicture}
\begin{Large} 
\begin{center}

\vspace{30px}

САРОВСКИЙ ГОСУДАРСТВЕННЫЙ \\ ФИЗИКО-ТЕХНИЧЕСКИЙ ИНСТИТУТ

\vspace{40px}

КАФЕДРА ПРИКЛАДНОЙ МАТЕМАТИКИ

\vspace{40px}

\bf{ЗАДАНИЕ\\НА КУРСОВОЕ ПРОЕКТИРОВАНИЕ}

\end{center}
\end{Large}

\vspace{40px}

по курсу \underline{ {\hspace{4cm} Численные методы {\hspace{5.2cm} }}}

\vspace{5px}

СТУДЕНТ \underline{ {\hspace{4cm} Коновалов Ярослав {\hspace{4.65cm} }}}

\vspace{5px}

ГРУППА \underline{ {\hspace{5cm} ПМ-32 {\hspace{6.05cm} }}}

\vspace{40px}

РУКОВОДИТЕЛЬ РАБОТЫ: \underline{ {\hspace{1.42cm} Кидямкина Дина Николаевна {\hspace{2.4cm} }}}

\vspace{5px}

КОНСУЛЬТАНТ ПРОЕКТА: \underline{ {\hspace{1.42cm} Милешин Иван Геннадьевич {\hspace{2.7cm} }}}

\vspace{100px}

\begin{center}
\textbf{г. Саров, 2025 г.}
\end{center}
\restoregeometry


\newpage

\textbf{Наименование темы:} \underline{ {\hspace{0.7cm} Применение численных методов для решения {\hspace{2.27cm}} }}

\underline{ {\hspace{6cm} различных математических задач {\hspace{2.9cm}}}} 

\textbf{Место выполнения:} \underline{Отделение 08 ИТМФ -РФЯЦ-ВНИИЭФ {\hspace{4.5cm}}}

\textbf{Исходные данные:} \underline{тексты 21 задания по темам: {\hspace{6.44cm}}}

\begin{itemize}[leftmargin=1.7cm]
\setlength\itemsep{0em}
\item \underline{интерполирование и приближение функций; {\hspace{6.95cm}}}

\item \underline{приближенное дифференцирование и интегрирование; {\hspace{5.3cm}}}

\item \underline{решение алгебраических и нелинейных уравнений; {\hspace{5.9cm}}}

\item \underline{алгебра матриц; {\hspace{11.5cm}}}

\item \underline{решение систем линейных уравнений; {\hspace{8cm}}}

\item \underline{интегрирование дифференциальных уравнений и уравнений в частных {\hspace{2.48cm}} }

\underline{производных, в том числе с применением разностных схем. {\hspace{4.49cm}}}
\end{itemize}

\textbf{Содержание работы:} \underline{промежуточные выкладки, полученные формулы {\hspace{2.47cm}} }

\underline{и результаты решения поставленных задач с использованием численных методов {\hspace{1.29cm}} } 

\textbf{Экспериментальная часть:} \underline{сравнение точных значений и значений, {\hspace{3.09cm}}}

\underline{полученных в результате расчетов с использованием численных методов. {\hspace{2.7cm}}} 


\begin{center}
\textbf{Допуск к защите}
\end{center}

\qquad \qquad \qquad \qquad \qquad \ \   К защите представляется:

\begin{itemize}[leftmargin=5cm]
\setlength\itemsep{0em}
\item Пояснительная записка \hspace{3cm} \underline{\hspace{2.1cm}} страниц

\item Чертежно-графический материал \hspace{1.42cm} \underline{\hspace{2cm}} листов

\item Иллюстрационный материал \hspace{2.2cm} \underline{\hspace{2cm}} листов

\item Экпериментальные макеты \hspace{2.5cm} \underline{\hspace{2cm}} шт.

\item Приложение \hspace{4.9cm} \underline{\hspace{2cm}} страниц
\end{itemize}
Студент $\underset{\text{индекс группы, фамилия, имя, отчество
}}{\underline{ {\hspace{1cm} \text{Коновалов Ярослав Александрович ПМ-32} {\hspace{2.3cm} }}}}$ допущен к защите
%Name: $\underset{\text{(full name)}}{\underline{\hspace{5cm}}}$
\vspace{10px}

%Руководитель \underline{{\hspace{12.2cm}}}
Руководитель $\underset{\text{дата, подпись}}{\underline{ {\hspace{12.2cm} }}}$
\vspace{10px}

Работа защищена с оценкой \underline{{\hspace{9.9cm}}}

\vspace{10px}

{\hspace{8cm}} Члены комиссии: \underline{{\hspace{3.5cm}}}

{\hspace{10.85cm}} \underline{{\hspace{3.55cm}}}

{\hspace{10.85cm}} \underline{{\hspace{3.55cm}}}

\newpage

\begin{center}
\renewcommand{\cftsecleader}{\cftdotfill{\cftdotsep}}
\tableofcontents
\newpage
\end{center}



\refstepcounter{section} %гиперссылка
\addcontentsline{toc}{section}{Аннотация}
\begin{center}
\section*{\large Аннотация}
\end{center}
\begin{large}
\qquad Данная курсовая работа посвящена применению численных методов для решения задач, относящихся к таким темам, как интерполирование функций, численные дифференцирование и интегрирование, решение алгебраических и нелинейных уравнений, алгебра матриц, решение систем линейных уравнений, численное интегрирование обыкновенных дифференциальных уравнений и уравнений в частных производных, а также применение разностных схем для их решения.
\end{large}
\newpage

\refstepcounter{section} %гиперссылка
\addcontentsline{toc}{section}{Введение}
\begin{center}
\section*{\large Введение}
\end{center}
\begin{large}
\qquad Численные методы играют ключевую роль в современной науке и инженерии, позволяя решать сложные математические задачи, которые не поддаются аналитическому решению или требуют значительных упрощений. С развитием вычислительной техники эти методы приобрели особую значимость, так как дают возможность моделировать физические, экономические, биологические и другие процессы с высокой точностью.

\end{large}
\newpage

\begin{center}
\refstepcounter{section} %гиперссылка
\addcontentsline{toc}{section}{Лабораторная работа №1}
\section*{\large Лабораторная работа №1\\
Интерполирование и приближение функций.\\ Интерполяционные формулы Лагранжа и Ньютона.}
\end{center}
\renewcommand{\labelenumi}{\textbf{\arabic{enumi}.}}
\renewcommand{\labelenumii}{\textbf{\arabic{enumi}.\arabic{enumii}}}
\renewcommand{\labelenumiii}{\textbf{\arabic{enumi}.\arabic{enumii}.\arabic{enumiii}}}
\renewcommand{\labelenumiv}{\textbf{\arabic{enumi}.\arabic{enumii}.\arabic{enumiii}.\arabic{enumiv}}}

\begin{enumerate}
\large\item {\large \textbf{Постановка задачи}}
\begin{enumerate}
Для функции $f(x) = e^{Nx}$, где $N \ -$ порядковый номер:
\item {Построить (вычислить значения коэффициентов $a_i$, интерполяционного многочлена $a_0 + a_1x^1 +a_2x^2+ ... + a_nx^n$) интерполяционный многочлен Лагранжа с коэффициентами Лагранжа и Ньютона седьмой степени с равноотстоящими узлами. Узлы для интерполяции взять на отрезке [-1,1] с шагом $h = \frac{b-a}{7}$, включая концы отрезка.}
\item {Построить графики полученных полиномов и сравнить их с исходной функцией.}
\item {Вычислить относительную погрешность полученных значений по сравнению с точными значениями функции в точках отрезка [-1, 1], взятых с шагом $h = \frac{b-a}{15}$ (под точными значениями будем понимать здесь и далее значения функции при непосредственной подстановке значения аргумента).}
\item {Вычислить значения полученных полиномов по исходной формуле и используя схему Горнера в 100000 точках, сравнить времена расчета.}
\end{enumerate}

\large\item {\large \textbf{Теоретический материал}}

\begin{enumerate}
\item {\textbf{Интерполяция}}\par
\qquad Интерполяцией [3,10,13] называют такую разновидность аппроксимации, при которой кривая построенной функции восстанавливается по известным узлам и значениям функции в них и проходит точно через имеющиеся точки.\par

\item {\large \textbf{Интерполяционная формула Лагранжа}}\par
\qquadПусть в узлах $x_1, x_2,..., x_{n+1}$ заданы значения функции $y(x)$, равные\par
$y_1, y_2,..., y_{n+1}$, и требуется построить многочлен степени $n$ соответствующий формуле (1.1):\par
\begin{equation} \tag{1.1}
    P(x)= \displaystyle\sum_{k=0}^{n} a_k x^k ,
\end{equation}

где $P(x_i) = y_i \ (i = 1,2, \dots ,n+1)$. \par
\newpage
\qquad Пусть все $x_i$ различны, тогда решение этой задачи можно записать в виде уравнения (1.2):
\begin{equation} \tag{1.2}
    \begin{vmatrix}
    \ P       & \ 1   &\ x       &\ x^{2}       & \dots  &\ x^n      \\
    \ y_1     & \ 1   &\ x_1     &\ x^{2}_1     & \dots  &\ x^n_1   \\
    \ y_2     & \ 1   &\ x_2     &\ x^{2}_2     & \dots  &\ x^n_2    \\
    \ \dots   & \dots & \dots    & \dots        & \dots  &\dots      \\
    \ y_{n+1} & \ 1   &\ x_{n+1} &\ x^{2}_{n+1} & \dots  & x^n_{n+1}
    \end{vmatrix}
    = 0.
\end{equation}
\qquad Если раскрыть этот определитель и выразить из полученного уравнения $P$ через $x$, то получится многочлен степени $n$, который при $x=x_i$, принимает значение $y_i$.\par

\qquad Идея метода Лагранжа [3,10,13] состоит в том, чтобы сначала построить
многочлены $L_i(x) \ (i=1, 2, \dots, n+1)$, для которых будут выполняться условия (1.3): 
\begin{equation} \tag{1.3}
\begin{cases}    
    L_i(x_j)=1, & i=j \\
    L_i(x_j)=0, & i \neq j
\end{cases}
\end{equation}

\qquad Тогда искомый многочлен можно записать в виде (1.4):
\begin{equation} \tag{1.4}
    P(x) = \displaystyle\sum_{i=1}^{n+1}L_i(x) \cdot y_i.
\end{equation}
\qquad Формула (1.4) называется интерполяционной формулой Лагранжа.
Многочлены $L_i(x)$ получаются из формулы (1.5):
\begin{equation} \tag{1.5}
    L_i(x) = \frac{(x-x_1)\dots (x-x_{i-1})(x-x_{i+1})\dots (x-x_{n+1})}{(x_i-x_1)\dots (x_i-x_{i-1})(x_1-x_{i+1})\dots (x_i-x_{n+1})}.
\end{equation}

\item {\large \textbf{Интерполяционная формула Ньютона}}

\qquad Если узлы интерполяции являются равноотстоящими по величине, т.е. выполняется (1.6):
\begin{equation}\tag{1.6}
    x_{i+1}-x_i=h=const,
\end{equation}
где $h$ - шаг интерполяции, то интерполяционный многочлен
можно записать в формах, предложенных Ньютоном [3,10,13].
\begin{enumerate}
    \item {\large \textbf{Первая интерполяционная формула Ньютона}}\par
\qquad Интерполяционный полином ищется в виде (1.7):
\begin{equation}\tag{1.7}
    P_n(x)=a_0+a_1(x-x_0)+a_2(x-x_0)(x-x_1)+ \dotsc  +a_n(x-x_0)\dots (x-x_{n-1})
\end{equation}
\qquad Построение многочлена сводится к определению коэффициентов $a_i$. При поиске коэффициентов пользуются конечными разностями. Конечные разности первого порядка записываются в виде (1.8):
\begin{equation}\tag{1.8}
\begin{gathered}    
    \Delta y_0 = y_1-y_0;\\
    \Delta y_1 = y_2-y_1;\\
     \dots               \\
    \Delta y_{n-1} = y_n-y_{n-1}.
\end{gathered}
\end{equation}

\qquad Конечные разности второго порядка записываются в виде (1.9):
\begin{equation}\tag{1.9}
\begin{gathered}    
\Delta^2 y_0 = \Delta y_1-\Delta y_0;\\
\Delta^2 y_1 = \Delta y_2-\Delta y_1;\\
 \dots                               \\
\Delta^2 y_{n-1} = \Delta y_{n}-\Delta y_{n-1}.
\end{gathered}
\end{equation}

\qquad Конечные разности высших порядков записываются в виде (1.10):
\begin{equation}\tag{1.10}
\begin{gathered}
\Delta^k y_0 = \Delta^{k-1} y_1-\Delta^{k-1} y_0;\\
\Delta^k y_1 = \Delta^{k-1} y_2-\Delta^{k-1} y_1;\\
 \dots                                              \\
\Delta^k y_{n-1} = \Delta^{k-1} y_{n}-\Delta^{k-1} y_{n-1}.
\end{gathered}    
\end{equation}

\qquad Первая интерполяционная формула Ньютона имеет вид (1.11):

\begin{equation}\tag{1.11}
\begin{gathered}    
    y(x) = P_n(x) = y_0+q\Delta y_0 + \frac {q(q-1)}{2!} \Delta^2 y_0 +\dotsc +\\
    +\frac {q(q-1)\dots (q-n+1)}{n!}\Delta^n y_0,
\end{gathered}
\end{equation}
где $q = \frac{x-x_0}{h}.$ \par
\qquad Остаточный член имеет вид (1.12):
\begin{equation}\tag{1.12}
     R_n(x) = h^{n+1}\frac {q(q-1)\dots (q-n)}{(n+1)!}f^{n+1}(\xi),
\end{equation}
    где $\xi$ — некоторая внутренняя точка наименьшего помежутка, содержащего все узлы $x_i \ (i = 0,1,...,n)$ и точку $x$.
\newpage
\item {\large \textbf{Вторая интерполяционная формула Ньютона}}\par
\qquad Вторая интерполяционная формула Ньютона имеет вид (1.13):
\begin{equation}\tag{1.13}
\begin{gathered}
    y(x) = P_n(x) = y_n+q\Delta y_{n-1} + \frac {q(q+1)}{2!} \Delta^2 y_{n-2} +\dotsc +\\
    +\frac {q(q+1)...(q+n-1)}{n!}\Delta^n y_0,
\end{gathered}
\end{equation}

где $q = \frac{x-x_n}{h}.$\par
\qquad Остаточный член имеет вид (1.14): 
\begin{equation}\tag{1.14}
     R_n(x) = h^{n+1}\frac {q(q+1)\dots (q+n)}{(n+1)!}f^{n+1}(\xi),
\end{equation}

где $\xi$ $-$ некоторая внутренняя точка наименьшего промежутка, содержащего все узлы $x_i \ (i = 0,1,...,n)$ и точку $x$ .
\end{enumerate}

\item {\large \textbf{Вычисление значений многочлена с помощью схемы Горнера}}\par
\qquad Пусть дан многочлен n-й степени (1.15):

\begin{equation}\tag{1.15}
    P(x)=a_nx^n + a_{n-1}x^{n-1}+ \dotsc +a_0
\end{equation}

с действительными коэффициентами $a_k (k = 0, 1, ..., n)$ и требуется найти значение этого многочлена в точке $\xi$, т.е. формула (1.15) приобретает вид (1.16):
\begin{equation}\tag{1.16}
    P(\xi)=a_n\xi^n + a_{n-1}\xi^{n-1}+...+a_0.
\end{equation}

\qquad Вычисление значения $P(\xi)$ удобнее всего производить в виде (1.17):
\begin{equation}\tag{1.17}    
    P(\xi)=(...(((a_n\xi + a_{n-1})\xi + a_{n-2})\xi + a_{n-3})\xi +\dotsc +a_0).
\end{equation}


\qquad Если ввести числа (1.18)
\begin{equation}\tag{1.18}
\begin{cases}
  \begin{array}{ccc}
    &&b_0=a_0 \\
    c_1 = b_0\xi, && b_1=a_1 + c_1\\
    c_2 = b_1\xi, && b_2=a_2 + c_2\\
    \cdots \cdots \cdots && \cdots \cdots \cdots \cdots \\
    c_n = b_n\xi  && b_n=a_n + c_n\\
  \end{array}
\end{cases}
,
\end{equation}
то $b_n = P(\xi)$.\\
\qquad Вычисление значения многочлена $P(x)$ при $x=\xi$ сводится к повторению последовательности элементарных операций (1.19):
\begin{equation}\tag{1.19}
   c_k=b_{k-1}\xi, \quad b_k = a_k+c_k \quad (k = 1,2,\dots ,n).
\end{equation}
\end{enumerate}

\large\item {\large \textbf{Результаты численных расчётов}}\par
\qquad При выборе узлов для интерполяции на отрезке [-1;1] и при шаге $h = 0.285714$ получаем 8 узлов.\par
\qquad Узлы и значения функции $f(x)= e^{Nx}$ в узлах приведены в Таблице 1.\par
Таблица 1 — Узлы и значения функции в них
\begin{table}[H]
\centering
\begin{tabular}{|c|c|c|}
 \hline
 n & x & $f(x)$ \\ \hline
 1 & -1.000000 &       	0.367879  \\ \hline
 2 & -0.714286 &        	0.489541  \\  \hline
 3 & -0.428571 &       	0.651439  \\  \hline
 4 & -0.142857 &       	0.866877  \\  \hline
 5 &  0.142857 & 		1.153565  \\  \hline
 6 &  0.428571 &  	1.535063   \\  \hline
 7 &  0.714286 & 		2.042727   \\  \hline
 8 &  1.000000 & 		2.718282   \\  \hline
\end{tabular}
\end{table} \par

\qquad Значения коэффициентов полиномов Лагранжа и Ньютона приведены в Таблице 2.\par
Таблица 2 — Коэффициенты полиномов Лагранжа и Ньютона
\begin{table}[H]
\centering
\begin{tabular}{|c|c|c|c|}
 \hline
 $a_i$ & Лагранж & 1-я формула Ньютона & 2-я формула Ньютона  \\  \hline
 0 &   1.000000                    &      0.999999                    &      0.999999                    \\ \hline
 1 &   1.000000                    &      0.999999                    &      1.000000                    \\ \hline
 2 &   0.500003                    &      0.500003                    &      0.500003                    \\ \hline
 3 &   0.166667                    &      0.166667                    &      0.166667                    \\ \hline
 4 &   $0.416467\cdot 10^{-1}$     &      $0.416447\cdot 10^{-1}$     &      $0.416447\cdot 10^{-1}$     \\ \hline
 5 &   $0.833046\cdot 10^{-2}$     &      $0.833046\cdot 10^{-2}$     &      $0.833046\cdot 10^{-2}$     \\ \hline
 6 &   $0.143284\cdot 10^{-2}$     &      $0.143284\cdot 10^{-2}$     &      $0.143284\cdot 10^{-2}$     \\ \hline
 7 &   $0.203692\cdot 10^{-3}$     &      $0.203692\cdot 10^{-3}$     &      $0.203692\cdot 10^{-3}$     \\ \hline
\end{tabular}
\end{table}

\newpage
Таблица 3 — Cравнение времени вычисления полинома  с использованием схемы Горнера и по исходной формуле
\begin{table}[H]
\centering
\begin{tabular}{|c|c|c|}
 \hline
 Количество точек & Исходная формула, с & Схема Горнера, с \\ \hline
 100000 & 0.015625 & 0.0 \\ \hline
 1000000& 0.031250 & 0.015625 \\ \hline
\end{tabular}
\end{table} \par

\qquad При выборе узлов для интерполяции на отрезке [-1,1] и при шаге 
$h = 0.133333$ получаем 16 узлов.\par
\qquad Из таблицы 3 видно, что время вычисления значения полинома с помощью схемы Горнера при обоих разбиениях оказывается меньше, чем вычисления значения полинома путем прямой подстановки.\par

Таблица 4 — Значения исходной функции, полученного полинома в узлах, абсолютной и относительной погрешностей
\begin{table}[H]
\centering
\begin{tabular}{|c|c|c|c|c|}
 \hline
 $x$ & $f(x)$ & $P(x)$ & $|f(x)-P(x)|$ & $\frac{|f(x)-P(x)|}{P(x)}$ \\  \hline
-1.000000        &   0.367879    &   0.367879    &   0.0                            &   0.0                            \\ \hline
-0.866666        &   0.420350    &   0.420350    &   $5.729642\cdot 10^{-7}$        &   $1.363063\cdot 10^{-6}$        \\ \hline
-0.733333        &   0.480305    &   0.480305    &   $5.455983\cdot 10^{-8}$        &   $1.135940\cdot 10^{-7}$        \\ \hline
-0.599999        &   0.548811    &   0.548811    &   $1.363895\cdot 10^{-7}$        &   $2.485179\cdot 10^{-7}$        \\ \hline
-0.466666        &   0.627089    &   0.627089    &   $3.699988\cdot 10^{-8}$        &   $5.900260\cdot 10^{-8}$        \\ \hline
-0.333333        &   0.716531    &   0.716531    &   $5.691264\cdot 10^{-8}$        &   $7.942799\cdot 10^{-8}$        \\ \hline
-0.199999        &   0.818730    &   0.818730    &   $3.142150\cdot 10^{-8}$        &   $3.837831\cdot 10^{-8}$        \\ \hline
-0.066666        &   0.935507    &   0.935506    &   $3.614980\cdot 10^{-8}$        &   $3.864194\cdot 10^{-8}$        \\ \hline
0.066666         &   1.068939    &   1.068939    &   $3.668750\cdot 10^{-8}$        &   $3.432141\cdot 10^{-8}$        \\ \hline
0.200000         &   1.221402    &   1.221402    &   $3.284455\cdot 10^{-8}$        &   $2.689084\cdot 10^{-8}$        \\ \hline
0.333333         &   1.395612    &   1.395612    &   $6.127387\cdot 10^{-8}$        &   $4.390464\cdot 10^{-8}$        \\ \hline
0.466666         &   1.594669    &   1.594669    &   $4.103006\cdot 10^{-8}$        &   $2.572950\cdot 10^{-8}$        \\ \hline
0.600000         &   1.822119    &   1.822118    &   $1.557867\cdot 10^{-7}$        &   $8.549760\cdot 10^{-8}$        \\ \hline
0.733333         &   2.082009    &   2.082009    &   $6.419183\cdot 10^{-8}$        &   $3.083168\cdot 10^{-8}$        \\ \hline
0.866666         &   2.378968    &   2.378968    &   $6.944029\cdot 10^{-7}$        &   $2.918925\cdot 10^{-7}$        \\ \hline
1.000000         &   2.718282    &   2.718282    &   $1.988187\cdot 10^{-12}$       &   $7.314132\cdot 10^{-13}$       \\ \hline
\end{tabular}
\end{table}


\begin{figure}[H]
    \centering
    \includegraphics[width=0.9\linewidth]{../pics/lagrange.pdf}\par
    Рисунок 1 — Полином Лагранжа и исходная функция\\
    $1$ — исходная функция, $2$ — полином Лагранжа
\end{figure}


\begin{figure}[H]
    \centering
    \includegraphics[width=0.9\linewidth]{../pics/newton_forward.png}\par
    Рисунок 2 — 1-й полином Ньютона и исходная функция\\
    $1$ — исходная функция, $2$ — 1-й полином Ньютона
\end{figure}

\begin{figure}[H]
    \centering
    \includegraphics[width=0.9\linewidth]{../pics/newton_backward.png}\par
    Рисунок 3 — 2-й полином Ньютона и исходная функция\\
    $1$ — исходная функция, $2$ — 2-й полином Ньютона
\end{figure}\par

\begin{figure}[H]
    \centering
    \includegraphics[width=0.9\linewidth]{../pics/relative_error.png}\par
    Рисунок 4 — График относительной погрешности между функцией и полиномом\\
    $1$ — относительная погрешность
\end{figure}\par

\qquad Из рисунков 1,2 и 3 следует, что полученные полиномы при заданном значении $N$ достаточно близки к исходной функции почти на всём отрезке $[-1,1]$.\par
 

\large\item {\large \textbf{Вывод}}\par
\qquad В ходе выполнения лабораторной работы были построены интерполяционные многочлены Лагранжа и Ньютона седьмой степени с равноотстоящими узлами. Были построены графики полученных полиномов. Проведено сравнение времени вычисления значений через полученные полиномы по исходной формуле и времени вычисления с помощью схемы Горнера. Была вычислена относительная погрешность полученных значений при увеличении количества узлов.
\end{enumerate}
\newpage


\begin{center}
\refstepcounter{section} %гиперссылка
\addcontentsline{toc}{section}{Лабораторная работа №2}
\section*{\large Лабораторная работа №2\\
Среднеквадратическое приближение функций\\ (случай дискретных точек). }
\end{center}

\renewcommand{\labelenumi}{\textbf{\arabic{enumi}.}}
\renewcommand{\labelenumii}{\textbf{\arabic{enumi}.\arabic{enumii}}}
\renewcommand{\labelenumiii}{\textbf{\arabic{enumi}.\arabic{enumii}.\arabic{enumiii}}}
\renewcommand{\labelenumiv}{\textbf{\arabic{enumi}.\arabic{enumii}.\arabic{enumiii}.\arabic{enumiv}}}

\begin{enumerate}
\large\item {\large \textbf{Постановка задачи}}
\begin{enumerate}
    \item В декартовой системе координат на плоскости построить экспериментальные точки $M_1(x_1,y_1),M_2(x_2,y_2), ...,M_n(x_n,y_n)$. Соединить построенные точки кривой.
    \item По виду полученной кривой подобрать 3 типа функциональных зависимостей из представленных: $ y=ax+b, y=a+b\ln x, y=a+\frac{b}{x}, y=ax^2+b, y=ax^2+bx, y=a+be^x$.
    \item Для каждых из подобранных функциональных зависимостей  составить сумму квадратов отклонений экспериментальных значений $y_i$ от расчетных $f(x_i, a, b)$, то есть величину (2.1)
    \begin{equation}\tag{2.1}
    S(a,b) = \displaystyle\sum_{i=1}^{n} [f(x_i, a, b)-y_i]^2.
    \end{equation}
    Найти частные производные  $\frac{\partial S}{\partial a}$, $\frac{\partial S}{\partial b}$, составить систему уравнений (2.2)
    
\begin{equation}\tag{2.2}
 \begin{cases}
   \frac{\partial S}{\partial a} = 0 
   \\
   \frac{\partial S}{\partial b} = 0
 \end{cases}
\end{equation}

Из системы найти параметры $a$ и $b$  .
\item Оценить результаты аппроксимации: для каждой из полученных эмпирических формул вычислить сумму $ S= \displaystyle\sum_{p=1}^{n} [y_p - y_x]^2$ . Сравнивая эти суммы, выбрать эмпирическую формулу, которая более точно описывает результаты эксперимента.
\item Построить график полученной аппроксимирующей функции и сравнить его с расположением экспериментальных точек.
\end{enumerate}

\large\item {\large \textbf{Теоретический материал}}

\begin{enumerate}
\item {\textbf{Аппроксимация}}\par
\qquad Аппроксимация [13] — метод восстановления функции по набору узлов, при котором для нахождения
неизвестных промежуточных значений искомая функция может не проходить через данные узлы.

\qquad Задача аппроксимации заключается в определении параметров таким образом, чтобы общая аналитическая зависимость наилучшим образом соответствовала данным эксперимента.\par

\item {\textbf{Метод наименьших квадратов}}\par

\qquad Если некоторая функция $y=f(x)$ приближает экспериментальные данные
$A_1(x_1,y_1); A_2(x_2,y_2); ...;A_n(x_n,y_n)$,
то отклонения между приближёнными и экспериментальными значениями имеют вид (2.3):
\begin{equation}\tag{2.3}
\begin{gathered}
    e_{1}=y_{1}-f\left(x_{1}\right) \\
    e_{2}=y_{2}-f\left(x_{2}\right) \\
    ....\\
    e_{n}=y_{n}-f\left(x_{n}\right).
\end{gathered}
\end{equation}

\qquad Для определения точности приближения используется метод наименьших квадратов.\par
\qquad Идея метода наименьших квадратов [13] заключается в нахождении таких значений $x_i$, при которых сумма квадратов отклонений (2.4) стремилась бы к наименьшему значению.
\begin{equation}\tag{2.4}
 \displaystyle \sum_{i=1}^{n} (y_i - f(x_i))^2 \to min.
\end{equation}
\qquad Так как каждое значение $x_i$ сопровождается соответствующим коэффициентом $a_i \  (i = 0, 1, 2, …, n)$, то задача сводится к поиску коэффициентов, которые минимизируют функцию (2.5), зависящую от этих коэффициентов

\begin{equation}\tag{2.5}
f(a_0,a_1, ..., a_n) = \displaystyle\sum_{i=1}^{n}(y_i - f(x_i))^2 . 
\end{equation}

\qquad Так как функция (2.5) выпуклая относительно всех параметров, то она имеет
единственный локальный экстремум, совпадающий глобальным.
Путём дифференцирования функции (2.5) по каждой переменной $a_i$ получается система (2.6).
Коэффициенты $a_i$ вычисляются путём решения системы (2.6):
\begin{equation}\tag{2.6}
 \begin{cases}
   \frac{\partial f}{\partial a_0} = 0 
   \\
   \frac{\partial f}{\partial a_1} = 0
   \\
   ...
   \\
   \frac{\partial f}{\partial a_n} = 0
 \end{cases}.
\end{equation}
\end{enumerate}


\newpage
\large\item {\large \textbf{Результаты численных расчётов}}\par
\qquad Заданные точки представлены в таблице 1.

Таблица 1 — Заданные точки для аппроксимации.

\begin{table}[H]
\centering
\begin{tabular}{|c|c|c|c|c|c|c|}
 \hline
 x & 0 & 5 & 10 & 15 & 20 & 25  \\ \hline
 y & 21 & 39 & 51 & 63 & 70 & 90  \\ \hline
\end{tabular}
\end{table} \par

\qquad На рисунке 1 представлено построение заданных точек таблицы 1.\par
\begin{center}
    \includegraphics[width=0.9\linewidth]{../pics/points.png}\par
    Рисунок 1 — Построенные заданные точки. \par
    $1$ — ломаная по заданным точкам.
\end{center}

\qquad Из таблицы 2 можно сделать вывод, что функция $y=ax^2+b$ наиболее точно описывает результат эксперимента.

Таблица 2 — Выбранные аппроксимирующие функции, их вычисленные параметры $a, b$ и сумма квадратов отклонений от заданных точек.
\begin{table}[H]
\centering
\begin{tabular}{|c|c|c|c|}
 \hline
  & $y=ax+b$ & $y=a + b\ln x$ & $y=ax^2 + b$  \\ \hline
 a & 2.57142854 & 14.8066015 & 0.0942084938 \\ \hline
 b & 23.5238094 & 19.1014004&34.0772209 \\ \hline
 $\displaystyle\sum_{i=1}^{n}(f(x_i,a,b) -y_i)^2 $  & 46.4761925 & 346.448212 & 305.420837 \\ \hline
\end{tabular}
\end{table} \par

\begin{figure}[H]
    \centering
    \includegraphics[width=0.9\linewidth]{../pics/approx.png}\par
    Рисунок 2 — Заданные точки и аппроксимирующая функция.\par
        $1$ — ломаная по заданным точкам, $2$ — аппроксимирующая функция   $y = ax + b$.
\end{figure}

\large\item {\large \textbf{Вывод}}\par
\qquad В ходе выполнения лабораторной работы для заданных точек была получена и построена аппроксимирующая функция. Для этого в процессе выполнения лабораторной работы на графике были построены заданные точки, основываясь на расположении которых были подобраны три функциональные зависимости, для каждой из которых были вычислены параметры a, b и составлены суммы квадратов отклонений. После этого была выбрана формула с наименьшей суммой квадратов отклонений, которая наиболее точно описывала результаты эксперимента. В конечном итоге был построен график, который наглядным образом показывает сравнение расположения полученной аппроксимирующей функции и заданных точек.
\end{enumerate}

\newpage

\begin{center}
\refstepcounter{section} %гиперссылка
\addcontentsline{toc}{section}{Лабораторная работа №3}
\section*{\large Лабораторная работа №3\\
Среднеквадратическое приближение функций в интегральной метрике \\ (полиномы Лежандра и Чебышёва, тригонометрические полиномы). }
\end{center}

\renewcommand{\labelenumi}{\textbf{\arabic{enumi}.}}
\renewcommand{\labelenumii}{\textbf{\arabic{enumi}.\arabic{enumii}}}
\renewcommand{\labelenumiii}{\textbf{\arabic{enumi}.\arabic{enumii}.\arabic{enumiii}}}
\renewcommand{\labelenumiv}{\textbf{\arabic{enumi}.\arabic{enumii}.\arabic{enumiii}.\arabic{enumiv}}}

\begin{enumerate}
\large\item {\large \textbf{Постановка задачи}}
\begin{enumerate}
    \item 	Считая функцию 
        \begin{equation*}
        f(x) =
         \begin{cases}
         
                0; &\ -2 \leq x < -a
              \\
                x+a; &\ -a \leq x < 0
              \\
                -x+a; &\ 0 \leq x < a
               \\
                    0; &\ a \leq x < 2
              
        \end{cases}
           ,\qquad \text{где}\quad  a= 0.2 + 0.002N
        \end{equation*}
        периодической с периодом 4:
        \begin{itemize}
            \item получить ее приближение при помощи тригонометрических полиномов 8 и 9 степени (разложение в ряд Фурье);
            \item для каждого приближения вычислить значения в точках \\ $ x = -2 + 0.1n, (n=1,\dotsc, 39) $ и сравнить с точным значением функции $f(x)$ и между собой в этих точках;
            \item построить графики полученных полиномов и сравнить их с исходной функцией.
        \end{itemize}
        \item Для функции $ f(x) = |x^2 -2ax + 1.5|$, $a = 1.5 - 0.02N$, на отрезке $x \in [0.5;2.5]$:
        \begin{itemize}
            \item найти приближения функции при помощи полиномов Лежандра\\$P_i\{i=0,\dotsc,N; N=5,10\} $
            \item найти приближения функции при помощи полиномов Чебышёва \\$T_i\{i=0,\dotsc ,N; N=5,10\} $
            \item для каждого приближения вычислить значения в точках \\ $x= 0.5, 1.0, 1,5, 2.0, 2.5$ и сравнить с точным значением функции  $f(x)$ и между собой в этих точках;
            \item построить графики полученных полиномов и сравнить их с исходной функцией.
        \end{itemize}
\end{enumerate}
 Возникающие интегралы считать численно, сетку выбрать самостоятельно.
\newpage

\large\item {\large \textbf{Теоретический материал}}

\begin{enumerate}\item {\textbf{Приближение функций тригонометрическим полиномом}}\par
\qquad Обобщённым тригонометрическим рядом Фурье называется числовой \linebreakряд вида:
    \begin{equation}\tag{3.1}
        \frac {a_0}{2}  + \displaystyle\sum_{n=1}^{\infty} (a_n\cos{(\frac{n\pi x}{l})} +b_n\sin{(\frac{n\pi x}{l})}).
    \end{equation}
\qquad Для приближения исходной функции $f(x)$ используем разложение этой функции в тригонометрический ряд Фурье [13]:
 \begin{equation}\tag{3.2}
        f(x)=\frac {a_0}{2}  + \displaystyle\sum_{n=1}^{\infty} (a_n\cos{(\frac{n\pi x}{l})} +b_n\sin{(\frac{n\pi x}{l})}).
    \end{equation}
\qquad Умножим (3.2) на $\cos{\frac{m\pi x}{l}}$ и проинтегрируем от $-l$ до $l$:
\begin{equation}\tag{3.3}
\begin{gathered}
\int_{-l}^{l} f(x) \cos{\left(\frac{m\pi x}{l}\right)} \, dx= \frac{a_0}{2} \int_{-l}^{l} \cos{\left(\frac{m\pi x}{l}\right)} \, dx +\\
 + \sum_{n=1}^{\infty} \left( a_n \int_{-l}^{l} \cos{\left(\frac{n\pi x}{l}\right)} \cos{\left(\frac{m\pi x}{l}\right)} \, dx \right. + b_n \int_{-l}^{l} \sin{\left(\frac{n\pi x}{l}\right)} \cos{\left(\frac{m\pi x}{l}\right)} \, dx)
\end{gathered}
\end{equation}

\qquad Воспользуемся ортогональностью тригонометрических функций на \linebreak интервале \([-l, l]\):
\begin{equation}\tag{3.4}
\begin{cases}
    \int_{-l}^{l} \cos\left(\frac{m\pi x}{l}\right) \, dx = 0 \\
    \int_{-l}^{l} \sin\left(\frac{n\pi x}{l}\right) \cos\left(\frac{m\pi x}{l}\right) \, dx = 0 \\
    \int_{-l}^{l} \cos\left(\frac{n\pi x}{l}\right) \cos\left(\frac{m\pi x}{l}\right) \, dx = 
    \begin{cases} 
        0, & n \neq m \\
        l, & n = m \neq 0 
    \end{cases}
    
\end{cases}
\end{equation}


\qquad Получим выражение для $a_{n}$:
        \begin{equation}\tag{3.5}
             a_n=\frac{1}{l}\int\limits_{-l}^{l} f(x)\cos{(\frac{n\pi x}{l})}dx, 
        \end{equation}


\qquad Умножим (3.2) на $\sin{\frac{m\pi x}{l}}$ и проинтегрируем от $-l$ до $l$:
\begin{equation}\tag{3.6}
\begin{gathered}
\int_{-l}^{l} f(x) \sin{\left(\frac{m\pi x}{l}\right)} \, dx= \frac{a_0}{2} \int_{-l}^{l} \sin{\left(\frac{m\pi x}{l}\right)} \, dx +\\
 + \sum_{n=1}^{\infty} \left( a_n \int_{-l}^{l} \cos{\left(\frac{n\pi x}{l}\right)} \sin{\left(\frac{m\pi x}{l}\right)} \, dx \right. + b_n \int_{-l}^{l} \sin{\left(\frac{n\pi x}{l}\right)} \sin{\left(\frac{m\pi x}{l}\right)} \, dx)
\end{gathered}
\end{equation}
\newpage
\qquad Воспользуемся ортогональностью тригонометрических функций на \linebreak
интервале \([-l, l]\):
\begin{equation}\tag{3.7}
\begin{cases}
    \int_{-l}^{l} \sin\left(\frac{m\pi x}{l}\right) \, dx = 0 \\
    \int_{-l}^{l} \sin\left(\frac{m\pi x}{l}\right) \cos\left(\frac{n\pi x}{l}\right) \, dx = 0 \\
    \int_{-l}^{l} \sin\left(\frac{n\pi x}{l}\right) \sin\left(\frac{m\pi x}{l}\right) \, dx = 
    \begin{cases} 
        0, & n \neq m \\
        l, & n = m \neq 0 
    \end{cases}
    
\end{cases}
\end{equation}


\qquad Получаем выражение для $a_{n}$:
        \begin{equation}\tag{3.8}
             b_n=\frac{1}{l}\int\limits_{-l}^{l} f(x)\sin{(\frac{n\pi x}{l})}dx, 
        \end{equation}        

\qquad Проинтегрируем (3.2)  от $-l$ до $l$:
\begin{equation}\tag{3.9}
\begin{gathered}
\int_{-l}^{l} f(x)  \, dx= \frac{a_0}{2} \int_{-l}^{l}  \, dx +\\
 + \sum_{n=1}^{\infty} \left( a_n \int_{-l}^{l} \cos{\left(\frac{n\pi x}{l}\right)}  \, dx \right. + b_n \int_{-l}^{l} \sin{\left(\frac{n\pi x}{l}\right)}  \, dx)
\end{gathered}
\end{equation}
\qquad Воспользовавшись формулами (3.7), (3.8), получаем :
\begin{equation}\tag{3.10}
\begin{gathered}
 a_0 = \frac{1}{l}\int_{-l}^{l} f(x)  \, dx 
\end{gathered}
\end{equation}


\item  {\textbf{Приближение функций при помощи полиномов Лежандра}}\par

\qquad Многочлены Лежандра $L_n(x)$  [13] определены на отрезке $[-1;1]$
соотношением:
\begin{equation}\tag{3.11}
       L_n(x) =  \frac{1}{2^nn!} \frac{d^n}{dx^n}(x^2-1)^n,
 \end{equation}

     \qquad Рекурентное соотношение для нахождения многочленов Лежандра имеет вид:
  \begin{equation}\tag{3.12}
       L_{n+1}(x) = \frac{2n+1}{n+1}x L_{n}(x) - \frac{n}{n+1} L_{n-1}(x)
 \end{equation}.

\qquad Полиномы Лежандра образуют ортогональную систему  с весом $\rho (x)=1$:
 \begin{equation}\tag{3.13}
		\int\limits_{-1}^1L_n(x)L_m(x)dx=
		\begin{cases}
			\frac{2}{2m+1}, \hspace{5mm} m = n \\
			0, \hspace{5mm} m\neq n
		\end{cases}
		\label{pol:syst_Leg}
	\end{equation}

\newpage

\qquad На отрезке $[-1;1]$ функция $f(x)$ может быть приближена с помощью полиномов Лежандра полиномом вида:
	\begin{equation}\tag{3.14}
		P_n(x)=\sum\limits_{i=0}^na_iL_i.
 \end{equation}

\qquad Умножим (3.14) На $L_{i}$ и проинтtгрируем от $-1$ до $1$:
\begin{equation}\tag{3.15}
		\int\limits_{-1}^1f(x)L_{i} = a_1\int\limits_{-1}^1L_{1}L_{i} + a_{2}\int\limits_{-1}^1L_{2}L_{i} +...+a_{i}\int\limits_{-1}^1L_{i}L_{i}+... + a_n \int\limits_{-1}^1L_nL_i
\end{equation}

 \qquad Воспользовавшись ортогональностью полиномов Лежандра, получаем выражение для $a_i$:
	\begin{equation}\tag{3.16}
		a_i=\frac{\int\limits_{-1}^1f(x)L_i(x)dx}{\int\limits_{-1}^1L_i^2(x)dx}
		\label{pol:a_i_Cheb}.
	\end{equation}



 
 \item  {\textbf{Приближение функций при помощи полиномов Чебышева}}\par

\qquad Многочлены Чебышева $T_n(x)$ [13] определены на отрезке $[-1;1]$
соотношением:
 \begin{equation}\tag{3.17}
       T_n(x) = \cos{(n \arccos{x})}.
 \end{equation}

 \qquad Рекурентное соотношение для нахождения многочленов Чебышева имеет вид:
  \begin{equation}\tag{3.18}
       T_{m+n}(x)+  T_{m-n}(x)= 2T_m(x)T_n(x).
 \end{equation}.

\qquad Полиномы Чебышева образуют ортогональную систему  с весом $\rho (x)=\frac{1}{\sqrt{1-x^2}}$:
	\begin{equation}\tag{3.19}
		\int\limits_{-1}^1 \frac{T_n(x)T_m(x)}{\sqrt{1-x^2}}dx=
		\begin{cases}
			\pi, \hspace{5mm} m=n=0 \\
			\frac{\pi}{2}, \hspace{5mm} m = n=1,2,... \\
			0, \hspace{5mm} m\neq n
		\end{cases}
		\label{pol:syst_Cheb}
	\end{equation}

\qquad На отрезке $[-1;1]$ функция $f(x)$ может быть приближена с помощью полиномов Чебышева полиномом вида:
	\begin{equation}\tag{3.20}
		P_n(x)=\sum\limits_{i=0}^na_iT_i.
		\label{pol:approx_Cheb}
	\end{equation}

\qquad Умножим (3.20) На $\frac{T_{i}}{\sqrt{1-x^2}}$ и проинтегрируем от $-1$ до $1$:
\begin{equation}\tag{3.21}
						\int\limits_{-1}^1 \frac{f(x)T_{i}}{\sqrt{1-x^2}}= a_1\int\limits_{-1}^1\frac{T_{1}T_{i}}{\sqrt{1-x^2}} +...+a_{i}\int\limits_{-1}^1\frac{T_{i}T_{i}}{\sqrt{1-x^2}}+... + a_n \int\limits_{-1}^1\frac{T_{n}T_{i       }}{\sqrt{1-x^2}}
\end{equation}



\qquad Воспользовавшись ортогональностью полиномов Чебышева, получаем выражение для $a_i$:
	\begin{equation}\tag{3.22}
		a_i=\frac{\int\limits_{-1}^1\frac{1}{\sqrt{1-x^2}}f(x)T_i(x)dx}{\int\limits_{-1}^1\frac{1}{\sqrt{1-x^2}}T_i^2(x)dx}
		\label{pol:a_i_Cheb}.
	\end{equation}
 \end{enumerate}


\large\item {\large \textbf{Результаты численных расчётов}}

Таблица 1 — Заданные точки, приближения функции при помощи тригонометрических полиномов 8-й ($P_8(x)$) и 9-й степеней ($P_9(x)$) в точках и значения функции $f(x)$. 

\begin{longtable}{|c|c|c|c|c|}
\hline
 $n$ & $x$ & Значения $P_8(x)$ & Значения $P_9(x)$& $f(x)$ \\
\hline
\endfirsthead
\hline
\endhead
\hline
\endfoot
\hline
\endlastfoot
\hline
1 &    -1.90 &  0.007330  &  0.005284 &   0.0 \\ \hline
2 &    -1.80 &  0.013740  &  0.003600 &   0.0 \\ \hline
3 &    -1.70 &  0.018193  &  0.007631 &   0.0 \\ \hline
4 &    -1.60 &  0.017348  &  0.014359 &   0.0 \\ \hline
5 &    -1.50 &  0.011278  &  0.018122 &   0.0 \\ \hline
6 &    -1.40 &  0.003755  &  0.015254 &   0.0 \\ \hline
7 &    -1.30 &  0.000033  &  0.007484 &   0.0 \\ \hline
8 &    -1.20 &  0.003356  &  0.001120 &   0.0 \\ \hline
9 &    -1.10 &  0.012394  &  0.002163 &   0.0 \\ \hline
10 &   -1.00 & 0.021465  &  0.010982 &   0.0  \\ \hline
11 &   -0.90 & 0.023758  &  0.020956 &   0.0  \\ \hline
12 &   -0.80 & 0.015908  &  0.022907 &   0.0  \\ \hline
13 &   -0.70 & 0.001177  &  0.012680 &   0.0  \\ \hline
14 &   -0.60 & 0.010959  & -0.003657 &   0.0  \\ \hline
15 &   -0.50 & 0.009123  & -0.011548 &   0.0  \\ \hline
16 &   -0.40 & 0.013758  &  0.003441 &   0.0  \\ \hline
17 &   -0.30 & 0.055655  &  0.045254 &   0.0  \\ \hline
18 &   -0.20 & 0.104873  &  0.102259 &   0.002  \\ \hline
19 &   -0.10 & 0.144454  &  0.151606 &   0.102  \\ \hline
20 &    0.00 & 0.159596  &  0.171101 &   0.202 \\ \hline
21 &    0.10 & 0.144454  &  0.151606 &   0.102  \\ \hline
22 &    0.20 & 0.104873  &  0.102259 &   0.002  \\ \hline
23 &    0.30 & 0.055655  &  0.045253 &   0.0  \\ \hline
24 &    0.40 & 0.013758  &  0.003441 &   0.0  \\ \hline
25 &    0.50 & 0.009123  & -0.011548 &   0.0  \\ \hline
26 &    0.60 & 0.010959  & -0.003657 &   0.0  \\ \hline
27 &    0.70 & 0.001177  &  0.012680 &   0.0  \\ \hline
28 &    0.80 & 0.015908  &  0.022907 &   0.0  \\ \hline
29 &    0.90 & 0.023758  &  0.020956 &   0.0  \\ \hline
30 &    1.00 & 0.021465  &  0.010982 &   0.0  \\ \hline
31 &    1.10 & 0.012394  &  0.002163 &   0.0  \\ \hline
32 &    1.20 & 0.003356  &  0.001120 &   0.0  \\ \hline
33 &    1.30 & 0.000033  &  0.007484 &   0.0  \\ \hline
34 &    1.40 & 0.003755  &  0.015254 &   0.0  \\ \hline
35 &    1.50 & 0.011278  &  0.018122 &   0.0  \\ \hline
36 &    1.60 & 0.017348  &  0.014359 &   0.0  \\ \hline
37 &    1.70 & 0.018193  &  0.007631 &   0.0  \\ \hline
38 &    1.80 & 0.013740  &  0.003600 &   0.0  \\ \hline
39 &    1.90 & 0.007330  &  0.005284 &   0.0  \\ \hline
\end{longtable}
\newpage
Таблица 2 — Абсолютные погрешности между функцией и приближениями и абсолютные погрешности между самими приближениями.
\begin{longtable}{|c|c|c|c|c|}
\hline
 $n$ & $x$ & $|f(x)-P_8(x)|$ & $|f(x)-P_9(x)|$ & $|P_8(x)-P_9(x)|$ \\
\hline
\endfirsthead
\hline
\endhead
\hline
\endfoot
\hline
\endlastfoot
\hline
     1   & -1.90         &  0.004500   &  0.002579     &  0.001920  \\ \hline
     2   & -1.80         &  0.010282   &  0.000760     &  0.009521  \\ \hline
     3   & -1.70         &  0.014423   &  0.004507     &  0.009916  \\ \hline
     4   & -1.60         &  0.013857   &  0.011051     &  0.002806  \\ \hline
     5   & -1.50         &  0.008521   &  0.014948     &  0.006426  \\ \hline
     6   & -1.40         &  0.001689   &  0.012485     &  0.010796  \\ \hline
     7   & -1.30         &  0.001933   &  0.005061     &  0.006995  \\ \hline
     8   & -1.20         &  0.000698   &  0.001400     &  0.002099  \\ \hline
     9   & -1.10         &  0.008639   &  0.000965     &  0.009605  \\ \hline
     10  & -1.00         &  0.017006   &  0.007163     &  0.009842  \\ \hline
     11  & -0.90         &  0.019711   &  0.017080     &  0.002630  \\ \hline
     12  & -0.80         &  0.013451   &  0.020023     &  0.006571  \\ \hline
     13  & -0.70         &  0.000588   &  0.011389     &  0.010801  \\ \hline
     14  & -0.60         &  0.011048   &  0.004192     &  0.006856  \\ \hline
     15  & -0.50         &  0.011729   &  0.014006     &  0.002277  \\ \hline
     16  & -0.40         &  0.004877   &  0.004809     &  0.009687  \\ \hline
     17  & -0.30         &  0.037529   &  0.027762     &  0.009766  \\ \hline
     18  & -0.20         &  0.074802   &  0.072347     &  0.002454  \\ \hline
     19  & -0.10         &  0.006718   &  0.013434     &  0.006715  \\ \hline
     20  &  0.00         &  0.081018   &  0.070215     &  0.010802  \\ \hline
     21  &  0.10         &  0.006719   &  0.013434     &  0.006715  \\ \hline
     22  &  0.20         &  0.074802   &  0.072347     &  0.002454  \\ \hline
     23  &  0.30         &  0.037528   &  0.027762     &  0.009766  \\ \hline
     24  &  0.40         &  0.004877   &  0.004809     &  0.009687  \\ \hline
     25  &  0.50         &  0.011729   &  0.014006     &  0.002277  \\ \hline
     26  &  0.60         &  0.011048   &  0.004192     &  0.006856  \\ \hline
     27  &  0.70         &  0.000588   &  0.011389     &  0.010801  \\ \hline
     28  &  0.80         &  0.013451   &  0.020023     &  0.006571  \\ \hline
     29  &  0.90         &  0.019711   &  0.017080     &  0.002630  \\ \hline
     30  &  1.00         &  0.017006   &  0.007163     &  0.009842  \\ \hline
     31  &  1.10         &  0.008639   &  0.000965     &  0.009605  \\ \hline
     32  &  1.20         &  0.000698   &  0.001400     &  0.002099  \\ \hline
     33  &  1.30         &  0.001933   &  0.005061     &  0.006995  \\ \hline
     34  &  1.40         &  0.001689   &  0.012486     &  0.010796  \\ \hline
     35  &  1.50         &  0.008521   &  0.014948     &  0.006426  \\ \hline
     36  &  1.60         &  0.013857   &  0.011051     &  0.002806  \\ \hline
     37  &  1.70         &  0.014423   &  0.004507     &  0.009916  \\ \hline
     38  &  1.80         &  0.010282   &  0.000760     &  0.009521  \\ \hline
     39  &  1.90         &  0.004500   &  0.002579     &  0.001920  \\ \hline
\end{longtable}

Таблица 3 — Относительные погрешности между функцией и приближениями.
\begin{longtable}{|c|c|c|c|}
\hline
 $n$ & $x$ & $\frac{|f(x)-P_8(x)|}{f(x)}$ & $\frac{|f(x)-P_9(x)|}{f(x)}$ \\
\hline
\endfirsthead
\hline
\endhead
\hline
\endfoot
\hline
\endlastfoot
\hline
     1  & -1.900000 &                 —  &                —  \\ \hline
     2  & -1.800000 &                 —  &                —  \\ \hline
     3  & -1.700000 &                 —  &                —  \\ \hline
     4  & -1.600000 &                 —  &                —  \\ \hline
     5  & -1.500000 &                 —  &                —  \\ \hline
     6  & -1.400000 &                 —  &                —  \\ \hline
     7  & -1.300000 &                 —  &                —  \\ \hline
     8  & -1.200000 &                 —  &                —  \\ \hline
     9  & -1.100000 &                 —  &                —  \\ \hline
     10 & -1.000000 &                —  &                —  \\ \hline
     11 & -0.900000 &                —  &                —  \\ \hline
     12 & -0.800000 &                —  &                —  \\ \hline
     13 & -0.700000 &                —  &                —  \\ \hline
     14 & -0.600000 &                —  &                —  \\ \hline
     15 & -0.500000 &                —  &                —  \\ \hline
     16 & -0.400000 &		 —  &                —  \\ \hline
     17 & -0.300000 & 		 —  &                —  \\ \hline
     18 & -0.200000 &  51.43608303  	       & 50.12907523     	\\ \hline
     19 & -0.100000 &   0.41621855	          &    0.48633506  \\ \hline
     20 &  0.000000 &   0.20991956  	       &  0.15296203  \\ \hline
     21 &  0.100000 &   0.41621924  	       &  0.48633574  \\ \hline
     22 &  0.200000 &  51.43770911  	        &50.130657252  \\ \hline
    23  &  0.300000 & 		 —  &                —  \\ \hline
    24  &  0.400000 &  		 —  &                —  \\ \hline
    25  &  0.500000 &                 — &                —  \\ \hline
    26  &  0.600000 &                 — &                —  \\ \hline
    27  &  0.700000 &                 — &                —  \\  \hline
    28  &  0.800000 &                 — &                —  \\ \hline
    29  &  0.900000 &                 — &                —  \\ \hline
    30  &  1.000000 &                 — &                —  \\ \hline
    31  &  1.100000 &                 — &                —  \\ \hline
    32  &  1.200000 &                 — &                —  \\ \hline
    33  &  1.300000 &                 — &                —  \\ \hline
    34  &  1.400000 &                 — &                —  \\ \hline
    35  &  1.500000 &                 — &                —  \\ \hline
    36  &  1.600000 &                 — &                —  \\ \hline
    37  &  1.700000 &                 — &                —  \\ \hline
    38  &  1.800000 &                 — &                —  \\ \hline
    39  &  1.900000 &                 — &                —  \\ \hline
\end{longtable}

\begin{figure}[H]
    \centering
    \includegraphics[width=0.9\linewidth]{../pics/fourier.png}\par
   Рисунок 1 — График приближений $P_8(x)$ и $P_9(x)$ и исходной функции $ f(x)$.\par 
   $1$ — график исходной функции, $2$ — тригонометрический полином 8-й степени,\par
   $3$ — тригонометрический полином 9-й степени.
\end{figure}\par

Таблица 4 — Заданные точки, приближения при помощи полиномов Лежандра 5-й ($L_5(x)$) и 10-й степеней ($L_{10}(x)$), значения функции $f(x)$.

\begin{table}[H]
\centering
\begin{tabular}{|c|c|c|c|c|}
 \hline
 n & x & Значения ($L_5(x)$)  &Значения ($L_{10}(x)$)   & $f(x)$\\ \hline
1 &    0.5 &    0.136901 &    0.296201 &    0.27 \\ \hline
2 &    1.0 &    0.415708 &    0.468904 &    0.46 \\ \hline
3 &    1.5 &    0.726424 &    0.697391 &    0.69 \\ \hline
4 &    2.0 &    0.370632 &    0.431434 &    0.42 \\ \hline
5 &    2.5 &    0.271490 &    0.351588 &    0.35 \\ \hline
\end{tabular}
\end{table} \par

Таблица 5 — Абсолютные погрешности между функцией и приближениями и между самими приближениями полиномами Лежандра.

\begin{table}[H]
\centering
\begin{tabular}{|c|c|c|c|c|}
 \hline
 $n$ & $x$ & $|f(x)-L_5(x)|$ & $|f(x)-L_{10}(x)|$ & $|L_5(x)-L_{10}(x)|$ \\ \hline
1 &    0.5 &    0.133098 &    0.026201 &    0.159300 \\ \hline
2 &    1.0 &    0.044291 &    0.008904 &    0.053196 \\ \hline
3 &    1.5 &    0.036424 &    0.007391 &    0.029033 \\ \hline
4 &    2.0 &    0.049367 &    0.011434 &    0.060802 \\ \hline
5 &    2.5 &    0.078509 &    0.001588 &    0.080098 \\ \hline
\end{tabular}
\end{table} \par

Таблица 6 — Относительные погрешности между функцией и приближениями полиномами Лежандра.

\begin{table}[H]
\centering
\begin{tabular}{|c|c|c|c|}
 \hline
 $n$ & $x$ & $\frac{|f(x)-L_5(x)|}{f(x)}$ & $\frac{|f(x)-L_{10}(x)|}{f(x)}$ \\ \hline
1 & 0.5 & 0.49295614 & 0.09704411  \\ \hline
2 & 1.0 & 0.09628498 & 0.01935850  \\ \hline
3 & 1.5 & 0.05278928 & 0.01071242  \\ \hline
4 & 2.0 & 0.11754129 & 0.02722551  \\ \hline
5 & 2.5 & 0.22431348 & 0.00454005  \\ \hline
\end{tabular}
\end{table} \par


\begin{figure}[H]
    \centering
    \includegraphics[width=0.9\linewidth]{../pics/legendre.png}\par
   Рисунок 2 — График сравнения полиномов $L_{5}(x)$ и $L_{10}(x)$ с исходной функцией $ f(x)$. \par
   $1$ — график исходной функции, $2$ — полином Лежандра 5-й степени, \par 
   $3$ — полином Лежандра 10-й степени.
\end{figure}\par

\begin{figure}[H]
    \centering
    \includegraphics[width=0.9\linewidth]{../pics/leg5_error.png}\par
   Рисунок 3 — График относительной погрешности для полинома Лежандра 5-й степени. \par
   $1$ — относительная погрешность полинома Лежандра 5-й степени.
\end{figure}\par

\begin{figure}[H]
    \centering
    \includegraphics[width=0.9\linewidth]{../pics/leg10_error.png}\par
   Рисунок 4 — График относительной погрешности для полинома Лежандра 10-й степени. \par
   $1$ — относительная погрешность полинома Лежандра 10-й степени.
\end{figure}\par


\qquad Из рисунков 2, 3, 4 можно сделать вывод, что полиномы Лежандра 5-й и 10-й степеней достаточно точно приближают заданную функцию.

Таблица 7 — Заданные точки, приближения при помощи полиномов Чебышёва 5-й $T_5(x)$ и 10-й степеней $T_{10}(x)$, значения функции $f(x)$.

\begin{table}[H]
\centering
\begin{tabular}{|c|c|c|c|c|}
 \hline
 n & x & Значения $T_5(x)$  & Значения $T_{10}(x)$  & $f(x)$\\ \hline
1 &    0.5 &    0.218559 &    0.267335 &    0.27 \\ \hline
2 &    1.0 &    0.401700 &    0.474379 &    0.46 \\ \hline
3 &    1.5 &    0.747058 &    0.703875 &    0.69 \\ \hline
4 &    2.0 &    0.355750 &    0.435436 &    0.42 \\ \hline
5 &    2.5 &    0.321163 &    0.333411 &    0.35 \\ \hline
\end{tabular}
\end{table} \par

Таблица 8 — Абсолютные погрешности между функцией и приближениями и между самими приближениями полиномами Чебышёва.

\begin{table}[H]
\centering
\begin{tabular}{|c|c|c|c|c|}
 \hline
 $n$ & $x$ & $|f(x)-T_5(x)|$ & $|f(x)-T_{10}(x)|$ & $|T_5(x)-T_{10}(x)|$ \\ \hline
 1 &  0.5 &  0.051440  &   0.002664 &  0.051992 \\ \hline
 2 &  1.0 &  0.058299  &   0.014379 &  0.072467 \\ \hline
 3 &  1.5 &  0.057058  &   0.013875 &  0.044254 \\ \hline
 4 &  2.0 &  0.064249  &   0.015436 &  0.079286 \\ \hline
 5 &  2.5 &  0.028836  &   0.016588 &  0.016254 \\ \hline
\end{tabular}
\end{table} \par

Таблица 9 — Относительные погрешности между функцией и приближениями полиномами Чебышёва.

\begin{table}[H]
\centering
\begin{tabular}{|c|c|c|c|}
 \hline
 $n$ & $x$ & $\frac{|f(x)-T_5(x)|}{f(x)}$ & $\frac{|f(x)-T_{10}(x)|}{f(x)}$ \\ \hline
1 &    0.5 &    0.190519 &    0.009868 \\ \hline
2 &    1.0 &    0.126737 &    0.031260 \\ \hline
3 &    1.5 &    0.082693 &    0.020108 \\ \hline
4 &    2.0 &    0.152976 &    0.036754 \\ \hline
5 &    2.5 &    0.082390 &    0.047397 \\ \hline
\end{tabular}
\end{table} \par


\begin{figure}[H]
    \centering
    \includegraphics[width=0.9\linewidth]{../pics/chebyshev.png}\par
   Рисунок 5 — График сравнений полученных полиномов $T_{5}(x)$ и $T_{10}(x)$ с исходной функцией $ f(x)$. \par
   $1$ — график исходной функции, $2$ — полином Чебышева 5-й степени, \par 
   $3$ — полином Чебышева 10-й степени.
\end{figure}\par

\begin{figure}[H]
    \centering
    \includegraphics[width=0.9\linewidth]{../pics/cheb5_error.png}\par
   Рисунок 6 — График относительной погрешности для полинома Чебышева 5-й степени. \par
  $1$ — относительная погрешность для полинома Чебышева 5-й степени.
\end{figure}\par

\begin{figure}[H]
    \centering
    \includegraphics[width=0.9\linewidth]{../pics/cheb10_error.png}\par
   Рисунок 7 — График относительной погрешности для полинома Чебышева 10-й степени.\par
  $1$ — относительная погрешность для полинома Чебышева 5-й степени.
\end{figure}\par
\qquad Из рисунков 5, 6, 7 можно сделать вывод, что полиномы Чебышева 5-й и 10-й степеней достаточно точно приближают заданную функцию.
\large\item {\large \textbf{Вывод}}

\qquad В результате выполнения лабораторной работы были построены приближения заданной функции при помощи тригонометрических полиномов Фурье 8-й и 9-й степеней.
Для каждого из приближений были вычислены значения в экспериментальных точках, проведено сравнение между собой и с точными значениями функции $f(x)$ в этих точках, построены графики полученных полиномов и проведено сравнение с исходной функцией.
Были найдены приближения для экспериментальной функции при помощи полиномов Лежандра и Чебышёва 5-й и 10-й степеней.
Для каждого приближения вычислены значения в экспериментальных точках, проведено сравнение между собой и с точными значениями функции $f(x)$ в этих точках, построены графики полученных полиномов и сравнены с графиком исходной функции.
\end{enumerate}

\newpage
\begin{center}
\refstepcounter{section} %гиперссылка
\addcontentsline{toc}{section}{Лабораторная работа №4}
\section*{\large Лабораторная работа №4\\
Численное дифференцирование.\\ Интерполяционные многочлены Ньютона, Лагранжа.}
\end{center}


\renewcommand{\labelenumi}{\textbf{\arabic{enumi}.}}
\renewcommand{\labelenumii}{\textbf{\arabic{enumi}.\arabic{enumii}}}
\renewcommand{\labelenumiii}{\textbf{\arabic{enumi}.\arabic{enumii}.\arabic{enumiii}}}
\renewcommand{\labelenumiv}{\textbf{\arabic{enumi}.\arabic{enumii}.\arabic{enumiii}.\arabic{enumiv}}}

\begin{enumerate}
\large\item {\large \textbf{Постановка задачи}}

Для функции $f(x)=\sqrt{x+a}$, $a=0.01N$:
\begin{enumerate}
\item Cоставить таблицу значений функции для аргументов \\
	$x_i=0.01i$, $(i=0,1,...,19)$;
\item По значениям функции в этой таблице найти значения первой и второй производных указанной функции в точках $x=0.07$ и $x=0.14$ с точностью $10^{-4}$, используя:
\begin{itemize}
\item интерполяционный многочлен Лагранжа;
\item интерполяционный многочлен Ньютона;
\end{itemize}
\end{enumerate}


\large\item {\large \textbf{Теоретический материал}}

\begin{enumerate}
\item {\textbf{Получение формул численного дифференцирования при помощи интерполяционного многочлена Лагранжа}}\par

\qquad Формула дифференцирования первого порядка с использованием многочлена Лагранжа [3,4,10,13] имеет вид (4.1):
\begin{equation}\tag{4.1}
    y'(x)= \frac{1}{h}\displaystyle\sum_{i=0}^{n} \left ( \frac{(-1)^{n-i}y_i}{i!(n-i)!}\left ( \displaystyle\sum_{k=0}^{n} \left ( \frac{q^{[n+1]}}{(q-i)(q-k)}\right)-\frac{q^{[n+1]}}{(q-i)^2}\right) \right),
\end{equation}
где $q^{[n+1]}=q(q-1)(q-2)...(q-n)$.\par
\qquad Формула дифференцирования второго порядка с использованием многочлена Лагранжа [3,4,10,13] имеет вид (4.2):
\begin{equation}\tag{4.2}
  \begin{gathered}
    y''(x)=\frac{1}{h^2}\displaystyle\sum_{i=0}^{n} \Bigg ( \frac{(-1)^{n-i}y_i}{i!(n-i)!} \bigg ( \displaystyle\sum_{k=0}^{n} \Big ( \displaystyle\sum_{p=0}^{n} \Big ( \frac{q^{[n+1]}}{(q-p)(q-i)(q-k)} \Big )- \\
    -\frac{2q^{[n+1]}}{(q-i)^2(q-k)}-\frac{q^{[n+1]}}{(q-i)(q-k)^2}\Big ) \bigg ) + \frac{2q^{[n+1]}}{(q-i)^3} \Bigg ) .
    \end{gathered}
\end{equation}

\item {\textbf{Получение формул численного дифференцирования при помощи интерполяционного многочлена Ньютона}}\par

\qquad Формула дифференцирования первого порядка с использованием многочлена Ньютона [3,4,10,13] имеет вид (4.3):

\begin{equation}\tag{4.3}
    y'(x)=\frac{1}{h}\left (y_0 - \frac{\Delta^2y_0}{2!}(2q-1)+  \frac{\Delta^3y_0}{3!}(3q^2-6q+2)+... \right ),
\end{equation}
где 
$\Delta^k y_{i} = \Delta^{k-1} y_{i+1}-\Delta^{k-1} y_{i}$ $-$ конечные разности $k$-го порядка,\\
$q=\frac{x-x_0}{h}$,
$h=x_{i+1}-x_i$ $-$ шаг, \ (i = 0, 1, ..., n - 1). \par
\qquad Формула дифференцирования второго порядка с использованием многочлена Ньютона [3,4,10,13] имеет вид (4.4):

\begin{equation}\tag{4.4}
     y''(x)=\frac{\Delta^2 y_0}{h^2} + \frac{\Delta^3 y_0}{h^2}(q-1)+ \dotsc\ .
\end{equation}

\end{enumerate}
\large\item {\large \textbf{Результаты численных расчётов}}\par
\qquad В таблице 1 приведены значения функции для заданных точек.\par

Таблица 1 — Значения исходной функции для заданных точек.

\begin{longtable}{|c|c|c|c|}
\hline
 $i$ & $x$ & $f(x)$ \\
\hline
\endfirsthead
\hline
 $i$ & $x$ & $f(x)$ \\
\hline
\endhead
\hline
\endfoot
\hline
\endlastfoot
 0  & 0.00 &  0.100000    \\ \hline
 1  & 0.01 &  0.141421    \\ \hline
 2  & 0.02 &  0.173205    \\ \hline
 3  & 0.03 &  0.200000    \\ \hline
 4  & 0.04 &  0.223606    \\ \hline
 5  & 0.05 &  0.244948    \\ \hline
 6  & 0.06 &  0.264575    \\ \hline
 7  & 0.07 &  0.282842    \\ \hline
 8  & 0.08 &  0.300000    \\ \hline
 9  & 0.09 &  0.316227    \\ \hline
 10 & 0.10 &  0.331662    \\ \hline
 11 & 0.11 &  0.346410    \\ \hline
 12 & 0.12 &  0.360555    \\ \hline
 13 & 0.13 &  0.374165    \\ \hline
 14 & 0.14 &  0.387298    \\ \hline
 15 & 0.15 &  0.399999    \\ \hline
 19 & 0.16 &  0.412310    \\ \hline
 17 & 0.17 &  0.424264    \\ \hline
 18 & 0.18 &  0.435889    \\ \hline
 19 & 0.19 &  0.447213    \\ \hline
\end{longtable}

\begin{figure}[H]
    \centering
    \includegraphics[width=0.9\linewidth]{../pics/points_laba4.png}\par
    Рисунок 1 — График исходной функции.\par
    $1$ — исходная функция.
\end{figure}

Таблица 2 — Точные значения первой и второй производных исходной функции в точках $x=0.07$ и $x=0.14$.
\begin{table}[H]
\centering
\begin{tabular}{|c|c|c|}         \hline
  $x$     &  $y'(x)$ & $y''(x)$ \\ \hline
 0.07 & 1.767766 & -11.048543 \\ \hline
 0.14 & 1.290994 & -4.303314 \\ \hline
\end{tabular}
\end{table} \par

\qquad Из таблиц 3—5 можно сделать вывод, что для первой заданной точки $x=0.07$ наибольшую точность имеют формулы численного дифференцирования, полученные при помощи многочленов Ньютона. Для второй заданной точки $x=0.14$ наибольшую точность имеют формулы численного дифференцирования, полученные при помощи формулы Ньютона.\par


Таблица 3 — Значения первой и второй производных ($L^{(n)}(x)$) исходной функции в экспериментальных точках, полученные при помощи формулы численного дифференцирования многочлена Лагранжа, абсолютная и относительная погрешности с точными значениями.
\begin{table}[H]   
\centering
\begin{tabular}{|c|c|c|c|c|c|c|}
 \hline
 $x$ & $L'(x)$ & $L''(x)$ & $|y'(x)-L'(x)|$ & $|y''(x)-L''(x)|$ & $\frac{|y'(x)-L'(x)|}{y'(x)}$ & $\frac{|y''(x)-L''(x)|}{y''(x)}$\\\hline
0.07 &  1.767766 & -11.048730 &   0.0000002494 &   0.0001870023 &   0.0000001411 &   0.0000169255 \\ \hline
0.14 &  1.290993 &  -4.303316 &   0.0000007223 &   0.0000014663 &   0.0000005595 &   0.0000003407 \\ \hline
\end{tabular}
\end{table} \par

\begin{figure}[H]
    \centering
    \includegraphics[width=0.9\linewidth]{../pics/lab4_l1.png}\par
    Рисунок 2 — График первой производной и приближенных значений вычисленных с помощью полинома Лагранжа.\par
    $1$ — первая производная исходной функции, $2$ — значения первой производной вычисленные с помощью полинома Лагранжа.
\end{figure}

\begin{figure}[H]
    \centering
    \includegraphics[width=0.9\linewidth]{../pics/lab4_l2.png}\par
    Рисунок 3 — График второй производной и приближенных значений вычисленных с помощью полинома Лагранжа.\par
    $1$ — вторая производная исходной функции, $2$ — значения второй производной вычисленные с помощью полинома Лагранжа.
\end{figure}

Таблица 4 — Значения первой и второй производных ($P^{(n)}(x)$) исходной функции в экспериментальных точках, полученные при помощи формулы численного дифференцирования многочлена Ньютона, абсолютная и относительная погрешности с точными значениями.
\begin{table}[H]
\centering
\begin{tabular}{|c|c|c|c|c|c|c|}
 \hline
 $x$ & $P'(x)$ & $P''(x)$ & $|y'(x)-P'(x)|$ & $|y''(x)-P''(x)|$ & $\frac{|y'(x)-P'(x)|}{y'(x)}$ & $\frac{|y''(x)-P''(x)|}{y''(x)}$   \\ \hline
 0.070000 &  1.767766 & -11.048621 &  0.0000002819 &  0.0000776926 &  0.0000001594 &  0.0000070319 \\ \hline
 0.140000 &  1.290993 &  -4.303724 &  0.0000007159 &  0.0004098069 &  0.0000005545 &  0.0000952305 \\ \hline
\end{tabular}
\end{table} \par

\begin{figure}[H]
    \centering
    \includegraphics[width=0.9\linewidth]{../pics/lab4_n1.png}\par
    Рисунок 4 — График первой производной и приближенных значений вычисленных с помощью полинома Ньютона.\par
    $1$ — первая производная исходной функции, $2$ — значения первой производной вычисленные с помощью полинома Ньютона.
\end{figure}

\begin{figure}[H]
    \centering
    \includegraphics[width=0.9\linewidth]{../pics/lab4_n2.png}\par
    Рисунок 5 — График второй производной и приближенных значений вычисленных с помощью полинома Ньютона.\par
    $1$ — вторая производная исходной функции, $2$ — значения второй производной вычисленные с помощью полинома Ньютона.
\end{figure}

\large\item {\large \textbf{Вывод}}\par
\qquad В ходе выполнения лабораторной работы была составлена таблица для значений исходной функции для заданных точек.
Были вычислены точные значения первой и второй производных исходной в экспериментальных точках функции.
По значениям функции в таблице были найдены значения первой и второй производных исходной функции в экспериментальных точках при помощи дифференцирования многочленов Лагранжа, Ньютона.
Были посчитаны абсолютные погрешности производных многочленов.
\end{enumerate}
\newpage

\begin{center}
\refstepcounter{section} %гиперссылка
\addcontentsline{toc}{section}{Лабораторная работа №5}
\section*{\large Лабораторная работа  №5\\
Численное дифференцирование.\\ Метод неопределенных коэффициентов.\\Уточнение по методу Рунге-Ромберга.}
\end{center}

\renewcommand{\labelenumi}{\textbf{\arabic{enumi}.}}
\renewcommand{\labelenumii}{\textbf{\arabic{enumi}.\arabic{enumii}}}
\renewcommand{\labelenumiii}{\textbf{\arabic{enumi}.\arabic{enumii}.\arabic{enumiii}}}
\renewcommand{\labelenumiv}{\textbf{\arabic{enumi}.\arabic{enumii}.\arabic{enumiii}.\arabic{enumiv}}}

\begin{enumerate}
\large\item {\large \textbf{Постановка задачи}}

\begin{enumerate}

\item Получить методом неопределенных коэффициентов формулу для вычисления первой производной функции $f(x)=e^{\left(2\sqrt{x}+\frac{1}{2\sqrt{x}}\right)}$. Значения $f(x_i)$ вычислить в точках $x_i=0.25+(0.05+0.001N)i$, где $i=0,1,2,3,4$. Вычислить по полученной формуле значение первой производной в точке $x_1$ и $x_3$: $x_1=0.3+0.001N$, $x_3=0.4+0.003N$. Сравнить полученное значение первой производной с точным значением в этих же точках.
\item Найти значения первой производной функции $y(x)=e^{sin(100x+N)}$ в точке $x=\frac{\pi}{200}$ по формуле $y'_2=\frac{1}{2h}(y_2-y_0)$, где $h=\frac{\pi}{800}$, $y_0=y(x-2h)$, $y_2=y(x)$, а остаточный член $R=-\frac{h^2}{6}y^{(3)}$.
\begin{itemize}
\item 	Уточнить полученные значения первой производной по методу Рунге-Ромберга для $r=3$ (т.е. новый шаг $\hbar=rh$).
\item 	Сравнить полученные значения первой производной, как уточненные, так и неуточненные, с точными  значениями. 
\item 	В случае, когда уточнить не удаётся, подобрать шаг таким образом, чтобы результат уточнения был ближе к точному значению. Выбор шага обосновать.
\end{itemize}
\end{enumerate}

\large\item {\large \textbf{Теоретический материал}}

\begin{enumerate}
\item {\textbf{Метод неопределённых коэффициентов}}\par

\qquad Метод неопределенных коэффициентов [10] применяется для численного дифференцирования таблично заданной функции с произвольным расположением узлов. Записываем искомую формулу для производной $k$-го порядка в некоторой точке $x = x_i$ в виде:
\begin{equation}\tag{5.1}
    y_i^k = C_0 y_0 + C_1 y_1 + \ldots + C_n y_n \
\end{equation} \par

Формула имеет место для многочленов: 
\begin{equation}\tag{5.2}
 y = 1, \, y = x - x_i, \, \ldots, \, y = (x - x_i)^n \ 
\end{equation}

Подставляя последовательно многочлены (5.2) в формулу (5.1), получим систему линейных алгебраических уравнений для определения коэффициентв $C_0$, $C_1$, ..., $C_n$, которая имеет вид: 
\begin{equation}\tag{5.3}
\begin{cases} 
C_0 + C_1 + C_2 + C_3 + C_4 = 0 \\ 
C_1(x_1 - x_0) + C_2(x_2 - x_0) + C_3(x_3 - x_0) + C_4(x_4 - x_0) = 0 \\ 
C_1(x_1 - x_0)^2 + C_2(x_2 - x_0)^2 + C_3(x_3 - x_0)^2 + C_4(x_4 - x_0)^2 = 2(x - x_0) \\ 
C_1(x_1 - x_0)^3 + C_2(x_2 - x_0)^3 + C_3(x_3 - x_0)^3 + C_4(x_4 - x_0)^3 = 3(x - x_0)^2 \\ 
C_1(x_1 - x_0)^4 + C_2(x_2 - x_0)^4 + C_3(x_3 - x_0)^4 + C_4(x_4 - x_0)^4 = 4(x - x_0)^3
\end{cases}
\end{equation} \par

Так как узлы равностроящие, введем обозначение $x_{i+1} - x_i = h$:
\begin{equation}\tag{5.4}
 x_2 - x_0 = 2h, \, x_3 - x_0 = 3h, \, x_4 - x_0 = 4h \ 
\end{equation}

Перепишем систему (5.3) с учетом (5.4) в виде: 

\begin{equation}\tag{5.6}
\begin{cases} 
C_0 + C_1 + C_2 + C_3 + C_4 = 0 \\ 
C_1h + C_2h + C_3h + C_4h = 0 \\ 
C_1h^2 + C_2h^2 + C_3h^2 + C_4h^2 = 2(x - x_0) \\ 
C_1h^3 + C_2h^3 + C_3h^3 + C_4h^3 = 3(x - x_0)^2 \\ 
C_1h^4 + C_2h^4 + C_3h^4 + C_4h^4 = 4(x - x_0)^3
\end{cases}
\end{equation} \par

Из системы (5.6) выражаем коэффициенты $C_1, C_2, C_3, C_4$ и получаем искомую формулу для нахождения первой производной.


\item{\textbf{Уточнение методом Рунге-Ромберга}}\par

\qquad При нахождении производных точность вычислений быстро понижается, поэтому используются методы уточнения решения. Наиболее простым и эффективным способом при фиксированном числе узлов является метод \textit{Рунге-Ромберга} [10,13].\par
\qquad Пусть некоторая производная $y^{(k)}(x)$ определена формулой (5.4):
\begin{equation}\tag{5.4}
    y^{(k)}(x) = \varphi(x,h)+\psi(x)h^p + O(h^{p+1}),
\end{equation}
где $p$ характеризует порядок точности формулы дифференцирования.\par
\qquad Вычислим ту же производную $y^{(k)}(x)$ по фоpмуле (5.4), но изменив в сетке шаг на $h_1=rh$.
Тогда получим формулу (5.5): 
\begin{equation}\tag{5.5}
    y^{(k)}(x) = \varphi(x,rh)+\psi(x)\cdot(rh)^p + O((rh)^{p+1}).
\end{equation}
\qquad Из формул (5.4) и (5.5) получается оценка погрешности $R$ формулы (5.4), определяемая формулой (5.6):
\begin{equation}\tag{5.6}
   R\approx \psi(x)h^p = \frac{\varphi(x,h)-\varphi(x,rh)}{r^p - 1} + O(h^{p+1}).
\end{equation}
\qquad Формула (5.6) называется первой формулой Рунге.\par
\qquad Значение производной $y^{(k)}(x)$ уточняется по формуле (5.7):
 \begin{equation}\tag{5.7}
     y^{(k)}(x) = \varphi(x,h) + \frac{\varphi(x,h)-\varphi(x,rh)}{r^p - 1} + O(h^{p+1}).
 \end{equation}
\qquad Формула (5.7) называется второй формулой Рунге.
\end{enumerate}


\large\item {\large \textbf{Результаты численных расчётов}}\par
\qquad В таблице 1 приведены экспериментальные точки $x_i$ при i=0,1,2,3,4 и значения $f(x_i)=e^{\left(2\sqrt{x_i}+\frac{1}{2\sqrt{x_i}}\right)}$.

Таблица 1 — Экспериментальные точки и значения f(x) в этих точках.
\begin{table}[H]
\centering
\begin{tabular}{|c|c|c|}
 \hline
 $i$ & $x_i$ & $f(x_i)$ \\ \hline
 0& 0.25000000 & 7.38905610  \\ \hline
 1& 0.30100000 & 7.45304451  \\ \hline
 2& 0.35200000 & 7.60907478  \\ \hline
 3& 0.40300000 & 7.82455582  \\ \hline
 4& 0.45400000 & 8.08213169  \\ \hline
\end{tabular}
\end{table} \par

\begin{figure}[H]
    \centering
    \includegraphics[width=0.9\linewidth]{../pics/points.png}\par
    Рисунок 1 — График исходной функции.\\
    $1$ — исходная функция.
\end{figure}

Таблица 2 — Экспериментальные точки $x_1$ и $x_3$, в которых необходимо посчитать 1-ую и 2-ую производные.
\begin{table}[H]
\centering
\begin{tabular}{|c|c|}
 \hline
 $x_1$ & $x_3$ \\ \hline
  0.30100000&   0.40300000 \\ \hline
\end{tabular}
\end{table} \par
Таблица 3 — Коэффициенты $C_i$\ $(i=0,1,2,3,4)$ формул неопределённых коэффициентов для первой производной заданной функции в точках $x_1$ и $x_3$.

\begin{table}[H]
\centering
\begin{tabular}{|c|c|c|}
 \hline
 $i$ & $C_i для x_1$ & $C_i для x_3$ \\ \hline
 0&   -4.90196088 &  -1.63398696 \\ \hline
 1&  -16.33986959 &   9.80392175 \\ \hline
 2&   29.41176526 & -29.41176526 \\ \hline
 3&   -9.80392175 &  16.33986959 \\ \hline
 4&    1.63398696 &   4.90196088 \\ \hline
\end{tabular}
\end{table} \par
\qquad Точные значения производной функции $f'_T(x)$ в точках $x_1$ и $x_3$ были посчитаны аналитически. \par
Таблица 4 — Точные значения $f'_T(x)$ и вычисленные с помощью формулы неопределённых коэффициентов приближённые значения $f'_P(x)$ 1-й производной $f'(x)$ в точках $x_1$ и $x_3$, их абсолютная и относительная погрешности.

\begin{table}[H]
\centering
\begin{tabular}{|c|c|c|c|c|}
 \hline
           & $f'_T(x)$   & $f'_P(x)$  & $|f'_T(x)-f'_P(x)|$ & $\frac{|f'_T(x)-f'_P(x)|}{f'_T(x)}  $ \\ \hline
 $x_1$& 2.288446 &  2.301728 & 0.013281 & 0.005770\\ \hline
 $x_3$& 4.669637 &  4.679436 & 0.009798 & 0.002094\\ \hline
\end{tabular}
\end{table} \par

\begin{figure}[H]
    \centering
    \includegraphics[width=0.9\linewidth]{../pics/lab5_mnk.png}\par
    Рисунок 2 — График первой производной исходной функции и приближенных значений вычисленных по методу неопределенных коэффициентов.\\
    $1$ — первая производная исходной функции, $2$ — приближенные значения вычисленные по методу неопределенных коэффициентов.
\end{figure}

\qquad В таблице 5 представлены точное значение и приближенное значение посчитанное по заданной формуле первой производной функции $y(x)=e^{sin(100x+11)}$, а также уточнение посчитанного значения по формуле Рунге-Ромберга для r=3, абсолютная погрешность посчитанных значений с точным значением. Точное значение производной функции $f'_T(x)$ было посчитано аналитически.\par

\quad Из таблицы 5 можно сделать вывод, что точность значения, посчитанного по заданной формуле, и точность уточнения являются недостаточными.\par

Таблица 5 — Точное значение ($f'_T(x)$) и значение, посчитанное по заданной формуле первой производной ($f'_P(x)$), уточнение ($f'_Y(x)$), абсолютная и относительная погрешности.

\begin{table}[H]
\centering
\resizebox{\textwidth}{!}{%
\begin{tabular}{|c|c|c|c|c|c|c|c|}
 \hline
 $h$ & $f'_T(x)$ & $f'_P(x)$& $|f'_T(x)-f'_P(x)|$ & $\frac{|f'_T(x)-f'_P(x)|}{f'_T(x)}$ & $f'_Y(x)$ &  $|f'_T(x)-f'_Y(x)|$ & $\frac{|f'_T(x)-f'_Y(x)|}{f'_T(x)}$ \\ \hline
$\frac{\pi}{800}$ & -144.440655 & -119.698771 & 24.741884 & 0.171294 & -137.203312 & 7.237343 & 0.050106 \\ [0.5ex]\hline
\end{tabular}}
\end{table} \par

\begin{figure}[H]
    \centering
    \includegraphics[width=0.9\linewidth]{../pics/lab5_RR1.png}\par
    Рисунок 3 — График первой производной исходной функции и приближенных значений.\\
    $1$ — первая производная исходной функции, $2$ — приближенное значение вычисленное по заданной формуле, $3$  приближенное значение уточненное по формуле Рунге-Ромберга.
\end{figure}

\qquad Повторяем счет значения по заданной формуле и его уточнения, пока не получим уточненное значение с точностью $\varepsilon = 10^{-4}$ . Для этого уменьшаем шаг в 10 раз при подсчете каждого нового приближенного значения.

Таблица 6 — Посчитанные приближённые ($f'_P(x)$) и уточнённые значения ($f'_Y(x)$), абсолютная и относительная погрешности между приближенными и точными.
\begin{table}[H]
\centering
\resizebox{\textwidth}{!}{%
\begin{tabular}{|c|c|c|c|c|c|c|}
 \hline
 $h$ & $f'_P(x)$ & $|f'_T(x)-f'_P(x)|$ & $\frac{|f'_T(x)-f'_P(x)|}{f'_T(x)}$ &  $f'_Y(x)$ &  $|f'_T(x)-f'_Y(x)|$ & $\frac{|f'_T(x)-f'_Y(x)|}{f'_T(x)}$ \\ \hline
 $\frac{\pi}{8000}$      & -145.276551 & 0.835897 &$5.787128\cdot 10^{-3}$ &  -145.534913 & 1.094258 & $7.575831\cdot 10^{-3}$ \\ [0.5ex]\hline
 $\frac{\pi}{80000}$     & -144.550894 & $1.102393\cdot 10^{-1}$   &  $7.632152\cdot 10^{-4}$ &  -144.553380 & $1.127247\cdot 10^{-1}$   &  $7.804227\cdot 10^{-4}$ \\ [0.5ex]\hline
 $\frac{\pi}{800000}$    & -144.451935 & $1.128067\cdot 10^{-2}$   &  $7.809903\cdot 10^{-5}$ &  -144.451960 & $1.130542\cdot 10^{-2}$   &  $7.827041\cdot 10^{-5}$  \\ [0.5ex]\hline
 $\frac{\pi}{8000000}$   & -144.441785 &$1.130625\cdot 10^{-3}$   &  $7.827610\cdot 10^{-6}$ &  -144.441786 & $1.130872\cdot 10^{-3}$   &  $7.829323\cdot 10^{-6}$  \\ [0.5ex]\hline
 $\frac{\pi}{80000000}$  & -144.440768 & $1.130869\cdot 10^{-4}$   &  $7.829303\cdot 10^{-7}$ &  -144.440768 & $1.130894\cdot 10^{-4}$   &  $7.829475\cdot 10^{-7}$  \\ [0.5ex]\hline
\end{tabular}}
\end{table} \par

\begin{figure}[H]
    \centering
    \includegraphics[width=0.9\linewidth]{../pics/lab5_RRmnogo.png}\par
    Рисунок 4 — График первой производной исходной функции и приближенных значений уточненных по формуле Рунге-Ромберга.\\
    $1$ — первая производная исходной функции, $2$ — приближенные значения вычисленные по заданной формуле, $3$ — приближенные значения уточненные по формуле Рунге-Ромберга.
\end{figure}

\large\item {\large \textbf{Вывод}}\par
\qquad В ходе выполнения лабораторной работы были посчитаны коэффициенты для вычисления первой производной заданной функции в экспериментальных точках методом неопределённых коэффициентов, сравнены посчитанные значения с точными.
Во второй части работы было посчитано значение первой производной заданной функции по заданной формуле, уточнено посчитанное значение по методу Рунге-Ромберга, проведено сравнение посчитанных значений первой производной с точным значением, посчитано уточнённое значение по заданной формуле с точностью $\varepsilon = 10^{-4}$ путём уменьшения шага.
\end{enumerate}

\newpage
\begin{center}
\refstepcounter{section} %гиперссылка
\addcontentsline{toc}{section}{Лабораторная работа №6}
\section*{\large Лабораторная работа №6\\
Численное интегрирование.\\Формулы Ньютона-Котеса: трапеций, левых, правых, центральных прямоугольников, Симпсона и Ньютона (правило трёх восьмых).}
\end{center}

\renewcommand{\labelenumi}{\textbf{\arabic{enumi}.}}
\renewcommand{\labelenumii}{\textbf{\arabic{enumi}.\arabic{enumii}}}
\renewcommand{\labelenumiii}{\textbf{\arabic{enumi}.\arabic{enumii}.\arabic{enumiii}}}
\renewcommand{\labelenumiv}{\textbf{\arabic{enumi}.\arabic{enumii}.\arabic{enumiii}.\arabic{enumiv}}}

\begin{enumerate}
\large\item {\large \textbf{Постановка задачи}}

\begin{enumerate} 
\item Вычислить интегралы:
		\begin {itemize}
				\item 	$	\int\limits_0^1 \frac{(x^p+x^{-p})}{1+x^2}dx$, где $p=0.969+0.001N$;
				\item $\int\limits_0^{\frac{\pi}{2}} \cos^2x\cos (px)dx$, где $p=10.2+0.01N$
		\end{itemize}
по формулам трапеции, левых, правых, центральных прямоугольников, Симпсона и Ньютона. Число разбиений отрезка интегрирования во всех случаях положить равным 20. Сравнить полученные приближенные значения между собой и с точными значениями соответствующих интегралов. 
\item Вычислить интегралы:
		\begin {itemize}
				\item 	$	\int\limits_0^1 x^p\left(\ln\frac{1}{x}\right)^2dx$, где $p=1+0.02N$;
				\item $\int\limits_0^1 \frac{dx}{[(\alpha-1)x^2-2x+1]^{\frac{3}{2}}}$, где $\alpha=3+0.01N$
		\end{itemize}
по формуле трапеций и Симпсона с точностью $\varepsilon=10^{-4}$ , определяя шаг интегрирования по оценке остаточного члена. Указать конкретные значения полученных шагов интегрирования.
\end{enumerate}

\large\item {\large \textbf{Теоретический материал}}

\begin{enumerate}
\item {\textbf{Формулы Ньютона-Котеса}}\par
    \qquad Пусть для заданной функции $y=f(x)$ на отрезке $[a,b]$ требуется вычислить интеграл (6.1):
    \begin{equation}\tag{6.1}
     \int\limits_a^b ydx.
    \end{equation}
    \qquad Выберем шаг (6.2):
     \begin{equation}\tag{6.2}
     h = x_i - x_{i-1} = \frac{b-a}{n},
    \end{equation}
где $n$ — количество отрезков разбиения. \par
\qquad Затем разобьём отрезок $[a,b]$ на $(n+1)$-равноотстоящих точек (6.3): 
    \begin{equation}\tag{6.3}
     x_0 = a, \ x_i = x_0 + ih \ \ (i = 0, 1, 2,\dotsc , n-1), \  x_n = b,
    \end{equation} 
    а значения функции в этих точках будут заданы с помощью (6.4): 
    \begin{equation}\tag{6.4}
     y_i = f(x_i) \ \ (i = 0, 1, 2,\dotsc , n).
    \end{equation} 
     \qquad Заменяя функцию $y$ соответствующим интерполяционным полиномом Лагранжа $L_n(x)$, получим приближенную квадратурную формулу (6.5):
            \begin{equation}\tag{6.5}
                 \int\limits_{x_0}^{x_n} ydx = \sum_{i=0}^{n} A_iy_i,
             \end{equation}
    где $A_i$ - некоторые постоянные коэффициенты.\par
     \qquad Такие квадратурные формулы с равноотстоящими узлами называются \textit{формулами Ньютона-Котеса} [3,4,10,13]. 
    \begin{enumerate}
        \item {\textbf{Формула трапеций}}\par
    \qquad \textit{Формула трапеций} имеет вид (6.6):
             \begin{equation}\tag{6.6}
                 \int\limits_{a}^{b}f\left(x\right)dx\approx h\left(\frac{y_{0}+y_{n}}{2}+y_{1}+y_{2}+...+y_{n-1}\right) ,
             \end{equation}
     где $y_i = f(x_i) \ \ (i = 0, 1, 2,\dotsc , n)$.\par
     \qquad Остаточный член формулы трапеций имеет вид (6.7):
     \begin{equation}\tag{6.7}
                 R = - \frac{nh^3}{12}f''(\xi) = - \frac{(b-a)h^2}{12}f''(\xi), \ \
                 a<\xi<b.
    \end{equation}
    \par
        \item {\textbf{Формулы прямоугольников}}\par
    \qquad Рассмотрим отрезки разбиения $[x_{i-1};x_i]$, где $i=1,2,\dotsc ,n$ и $a=x_0, \ b=x_n$. Середину каждого элементарного отрезка разбиения можно представить в виде (6.8):
     \begin{equation}\tag{6.8}
                 \xi _i = \frac{x_{i-1} + x_i}{2}.
    \end{equation}
    \qquad Тогда приближённое значение интеграла $\int\limits_{a}^{b} f(x)dx$ можно посчитать с помощью \textit{формулы центральных прямоугольников} (6.9):
     \begin{equation}\tag{6.9}
     \int\limits_{a}^{b} f(x)dx \approx h\sum_{i=1}^n f(\xi _i).
     \end{equation}
     \qquad Если выбирать точку на левой границе данных элементарных отрезков, то получим \textit{формулу левых прямоугольников} (6.10):
     \begin{equation}\tag{6.10}
     \int\limits_{a}^{b} f(x)dx \approx h \cdot \sum_{i=0}^{n-1} f(x_i).
     \end{equation}
      \qquad Если выбирать точку на правой границе данных элементарных отрезков, то получим\textit{ формулу правых прямоугольников} (6.11):
     \begin{equation}\tag{6.11}
     \int\limits_{a}^{b} f(x)dx \approx h \cdot \sum_{i=1}^{n} f(x_i).
     \end{equation}
      
       \item {\textbf{Формула Симпсона}}\par
    \qquad \textit{Формула Симпсона} является точной формулой для многочленов до третьей степени включительно. Число узлов в этой формуле обязательно выбирается нечётным, т.е. количество отрезков разбиения $n$ должно быть чётное $(n = 2m )$.\par
    \qquad Формула Симпсона имеет вид (6.12):
    \begin{equation}\tag{6.12}
         \int\limits_{a}^{b} f(x)dx \approx \frac{h}{3}[ y_0 + y_{2m} + 2(y_2 +y_4 +\dotsc + y_{2m-2}) + 4(y_1+y_3+\dotsc +y_{2m-1})],
    \end{equation}
    для которой шаг $h$ выбирается согласно (6.13):
    \begin{equation}
    \tag{6.13}
         h= \frac{b-a}{n} =  \frac{b-a}{2m}.
    \end{equation}
     \qquad Остаточный член формулы Симпсона имеет вид (6.14):
     \begin{equation}\tag{6.14}
                 R = - \frac{mh^5}{90}f^{(4)}(\xi) = - \frac{(b-a)h^4}{180}f^{(4)}(\xi), \ \
                 a<\xi<b.
    \end{equation}

     \item {\textbf{Формула Ньютона (правило трёх восьмых)}}\par
    \qquad В \textit{формуле Ньютона} число узлов разбиения обязательно выбирается равным $(3m+1)$, т.е. количество отрезков разбиения $n$ должно быть кратным трём $(n=3m)$.\par
    \qquad Формула Ньютона имеет вид (6.15):
    \begin{equation}\tag{6.15}
        \begin{gathered}
         \int\limits_{a}^{b} f(x) dx \approx \frac{3h}{8}[ y_0 + y_{3m} + 2(y_3 +y_6 +\dotsc + y_{3m-3})+\\
         + 3(y_1+y_2+y_4+y_5+\dotsc +y_{3m-2}+y_{3m-1})],
        \end{gathered}
    \end{equation}
    для которой шаг $h$ выбирается согласно (6.16):
    \begin{equation}\tag{6.16}
         h= \frac{b-a}{n} =  \frac{b-a}{3m}.
    \end{equation}
     \qquad Остаточный член формулы Ньютона имеет вид (6.17):
     \begin{equation}\tag{6.17}
                 R = - \frac{3mh^5}{80}f^{(4)}(\xi) = - \frac{(b-a)h^4}{80}f^{(4)}(\xi), \ \
                 a<\xi<b.
     \end{equation} \par
   \end{enumerate}
\end{enumerate}
\large\item {\large \textbf{Результаты}} \par
\qquad 3.1. Рассмотрим интеграл $\int\limits_0^1 \frac{(x^p+x^{-p})}{1+x^2}dx,\  p=0.970$.\par
\qquad Данный интеграл имеет особенность в точке $x = 0$.

$\int\limits_0^1 \frac{(x^p+x^{-p})}{1+x^2}dx = 
 \int\limits_0^1 \frac{x^p}{1+x^2}dx + 
 \int\limits_0^1 \frac{x^{-p}}{1+x^2}dx =
 \int\limits_0^1 \frac{x^p}{1+x^2}dx + 
 \frac{1}{-p+1}\int\limits_0^1 \frac{1}{1+x^2}dx^{-p+1} = \\ = 
 \int\limits_0^1 \frac{x^p}{1+x^2}dx +
 \left.(\frac{1}{-p+1} \cdot x^{-p+1}\cdot \frac{1}{1+x^2})\right|_0^1 + \frac{1}{-p+1}\int\limits_0^1 \frac{2x^{-p+2}}{(1+x^2)^2}dx = \\ =
  \int\limits_0^1 \frac{x^p}{1+x^2}dx +
 \frac{1^{-p+1}}{2(-p+1)}+ \frac{1}{-p+1}\int\limits_0^1 \frac{2x^{-p+2}}{(1+x^2)^2}dx =
 \frac{1}{2(-p+1)}+\int\limits_0^1 \left( \frac{x^p}{1+x^2}+ \frac{2x^{-p+2}}{(-p+1)(1+x^2)^2} \right)dx. $\par

Таблица 1 $-$ Значения заданного интеграла.
\begin{table}[H]
\centering
\begin{tabular}{|c|c|c|c|c|c|c|}
 \hline
 Метод    & Трапеций  & Ц. прям.  & Л. прям.   & П.прям.   & Симпсона  & Ньютона    \\ \hline
 Значение & 33.346011 & 33.330077 & 33.353565  & 32.900911 & 33.759244 & 33.057988  \\ \hline
\end{tabular}
\end{table} \par

\qquad В Таблице 2 абсолютные погрешности между соответствующими значениями представлены на пересечении нужного столбца и строки.\par
Таблица 2 — Абсолютные погрешности между приближенными значениями и между приближенными и точными значениями соответствующего интеграла.
\begin{table}[H]
\resizebox{\textwidth}{!}{%
\centering
\begin{tabular}{|c|c|c|c|c|c|c|c|}
 \hline
 Метод   & Т. знач.                & Трапеций                & Ц. прям.                & Л. прям.                & П.прям.                 &Симпсона                 & Ньютона \\ \hline
Т. знач. 	&                    &   0.000338    &   0.015595    &   0.007892    &   0.444761    &   0.413571    &   0.287684   \\ \hline
Симпсона 	& 0.000338    &                      &   0.015934    &   0.007553    &   0.445100    &   0.413232    &   0.288022   \\ \hline
Трапеций 	& 0.015595    &   0.015934    &                      &   0.023487    &   0.429166    &   0.429166    &   0.272088   \\ \hline
Ц. прям. 	& 0.007892    &   0.007553    &   0.023487    &                      &   0.452653    &   0.405678    &   0.295576   \\ \hline
Л. прям. 	& 0.444761    &   0.445100    &   0.429166    &   0.452653    &                      &   0.858332    &   0.157077   \\ \hline
П. прям. 	& 0.413571    &   0.413232    &   0.429166    &   0.405678    &   0.858332    &                      &   0.701255   \\ \hline
Ньютона  	& 0.287684    &   0.288022    &   0.272088    &   0.295576    &   0.157077    &   0.701255    &                \\ \hline
\end{tabular}}
\end{table} \par
\qquad 3.2. Рассмотрим интеграл $\int\limits_0^{\frac{\pi}{2}} \cos^2x\cos (px)dx, \  p=10.21 $.\par
Таблица 3 — Значения заданного интеграла.
\begin{table}[H]
\centering
\begin{tabular}{|c|c|c|c|c|c|c|}
 \hline
 Метод    & Трапеций & Ц. прям. & Л. прям.  & П.прям.   & Симпсона & Ньютона   \\ \hline
 Значение &   0.00063843    &    0.00063186    &   0.00063394    &   0.03990177  &   -0.03863805   &   0.00063873  \\ \hline
\end{tabular}
\end{table} \par

\qquad В Таблице 4 абсолютные погрешности между соответствующими значениями представлены на пересечении нужного столбца и строки.\par
Таблица 4 — Абсолютные погрешности между посчитанными приближенными значениями и между приближенными и точными значениями соответствующего интеграла.
\begin{table}[H]
\resizebox{\textwidth}{!}{%
\centering
\begin{tabular}{|c|c|c|c|c|c|c|c|}
 \hline
Т. знач. 	&               &   0.00000547    &   0.00000111    &   0.00000097    &   0.03926880    &  0.03927102    &   0.00000576     \\ \hline
Симпсона 	& 0.00000547    &                 &   0.00000657    &   0.00000450    &   0.03926334    &  0.03927648    &   0.00000029     \\ \hline
Трапеций 	& 0.00000111    &   0.00000657    &                 &   0.00000208    &   0.03926991    &  0.03926991    &   0.00000687     \\ \hline
Ц. прям. 	& 0.00000097    &   0.00000450    &   0.00000208    &                 &   0.03926783    &  0.03927199    &   0.00000479     \\ \hline
Л. прям. 	& 0.03926880    &   0.03926334    &   0.03926991    &   0.03926783    &                 &  0.07853982    &   0.03926304     \\ \hline
П. прям. 	& 0.03927102    &   0.03927648    &   0.03926991    &   0.03927199    &   0.07853982    &                &   0.03927678     \\ \hline
Ньютона  	& 0.00000576    &   0.00000029    &   0.00000687    &   0.00000479    &   0.03926304    &  0.03927678    &                  \\ \hline
\end{tabular}}
\end{table} \par

\qquad 3.3. Рассмотрим интеграл $\int\limits_0^1 x^p\left(\ln\frac{1}{x}\right)^2dx, \  p=1.22 $ .\par

\qquad По оценке остаточного члена, при округлении в меньшую сторону для получения целого n: шаг для метода Трапеций равен 0.005 , а для метода Симпсона 0.011. Оценка остаточного члена была произведена с помощью программы "WolframAlpha".\par
Таблица 5 — Точное значение интеграла, численно посчитанное значение, абсолютная
и относительная погрешности.
\begin{table}[H]
\centering
\begin{tabularx}{0.78\textwidth} { 
 | >{\centering\arraybackslash}X 
 | >{\centering\arraybackslash}X
 | >{\centering\arraybackslash}X 
 | >{\centering\arraybackslash}X | }
 \hline
 & Метод Симпсона & Метод Трапеций & Точное значение  \\ \hline
 Значение                      & 0.24255022     & 0.24254816      & 0.24264754 \\ \hline
 Абсолютная погрешность        & 0.00009732     & 0.00009938      & 0 \\ \hline
 Относительная погрешность     & 0.00040110     & 0.00040958      & 0 \\ \hline
\end{tabularx}
\end{table} \par

\qquad 3.4. Рассмотрим интеграл $\int\limits_0^1 \frac{dx}{[(\alpha-1)x^2-2x+1]^{\frac{3}{2}}} \  a=3.01 $ .\par
\qquad  По оценке остаточного члена, при округлении в меньшую сторону для получения целого n: шаг для метода Трапеций равен 0.01, а для метода Симпсона 0.02. Оценка остаточного члена была произведена с помощью программы "WolframAlpha".\par
Таблица 6 — Точное значение интеграла, численно посчитанное значение, абсолютная
и относительная погрешности.
\begin{table}[H]
\centering
\begin{tabular}{|c|c|c|c|}
 \hline
  & Метод Симпсона & Метод Трапеций & Точное значение \\ \hline
 Значение                      & 1.98505595     & 1.98503775      & 1.98513615 \\ \hline
 Абсолютная погрешность        & 0.00008020     & 0.00009840      & 0 \\ \hline
 Относительная погрешность     & 0.00004040     & 0.00004957      & 0 \\ \hline
\end{tabular}
\end{table} \par

\newpage
\large\item {\large \textbf{Вывод}}\par
\qquad В ходе выполнения лабораторной работы были посчитаны приближённые значения заданных интегралов по формулам трапеций, левых, правых, центральных прямоугольников, Ньютона и Симпсона, сравнены полученные значения между собой и с точными значениями. Затем были вычислены приближённые значения заданных интегралов по формулам трапеций и Симпсона с заданной точностью, для чего был определён шаг интегрирования по оценке остаточного члена.
\end{enumerate}

\newpage
\begin{center}
\refstepcounter{section} %гиперссылка
\addcontentsline{toc}{section}{Лабораторная работа №7}
\section*{\large Лабораторная работа №7\\
Численное интегрирование.\\ Квадратурные формулы Гаусса-Кристоффеля.}
\end{center}


\renewcommand{\labelenumi}{\textbf{\arabic{enumi}.}}
\renewcommand{\labelenumii}{\textbf{\arabic{enumi}.\arabic{enumii}}}
\renewcommand{\labelenumiii}{\textbf{\arabic{enumi}.\arabic{enumii}.\arabic{enumiii}}}
\renewcommand{\labelenumiv}{\textbf{\arabic{enumi}.\arabic{enumii}.\arabic{enumiii}.\arabic{enumiv}}}

\begin{enumerate}
\large\item {\large \textbf{Постановка задачи}}

Вычислить интегралы, используя квадратурные формулы типа Гаусса:
\begin{enumerate}
\item $\int\limits_1^2 \frac{dx}{x\sqrt{4x^2 + bx}}$, где $b = 1 + 0.01N$, в качестве узлов квадратурной формулы взять корни многочлена Лежандра седьмой степени.

\item $\int\limits_1^2 \frac{dx}{\sqrt{x-0.01N}\sqrt{x+1}}$, в качестве узлов квадратурной формулы взять корни многочлена Чебышёва седьмой степени.

\item $\int\limits_0^\infty \frac{e^{-(1-\alpha)x}}{\sqrt{x}} dx$, где $\alpha = 1 - 0.02N$, в качестве узлов квадратурной формулы взять корни многочлена Лагерра седьмой степени.

\item $\int\limits_{-\infty}^\infty e^{{-\pi}x^2} \cos \frac{2\sqrt{\pi}\alpha}{\alpha+0.01}x dx$, где $\alpha = -0.5 + 0.001N$, в качестве узлов квадратурной формулы взять корни многочлена Эрмита седьмой степени.
\end{enumerate}
Вычислить точные значения этих интегралов и сравнить их с полученными приближенными значениями.


\large\item {\large \textbf{Теоретический материал}} \par
\qquad Общий вид квадратурных формул Гаусса-Кристоффеля [1,3,4,10,13] представлен формулой (7.1):
\begin{equation}\tag{7.1}
    \int \limits_a^b f(x)\rho(x) dx = \sum_{k=1}^{n} C_kf(x_k) + R_n(f),
\end{equation}
где a, b - границы отрезка интегрирования, f(x) - подынтегральная функция, $\rho(x)$ - весовая функция, $C_k$ - некоторые постоянные коэффициенты, $x_k$ - выбранные узлы интерполяции, $R_n(f)$ - остаточный член; $k=\overline{1,n}$.

\begin{enumerate}
\item {\textbf{Полином Лежандра}}\par

\qquad Полином Лежандра определяется формулой Родрига (7.2):
\begin{equation}\tag{7.2}
    P_n(x)= \frac{1}{{2^n}{n!}} \frac{d^n}{dx^n} (x^2 - 1)^n.
\end{equation}

\qquad Этот полином определен на отрезке $[-1,1]$. Поэтому, если пределы интегрирования отличаются от отрезка $[-1,1]$, то необходимо произвести замену (7.3):
\begin{equation}\tag{7.3}
    x=\frac{b+a}{2}+\frac{b-a}{2}t.
\end{equation}

\qquad Тогда квадратурная формула Гаусса-Кристоффеля с полиномом Лежандра [3] примет вид (7.4):
\begin{equation}\tag{7.4}
    \int \limits_a^b f(x)dx = \int \limits_{-1}^{1} g(t)\cdot\rho(t)dt \approx \sum_{k=1}^{n} C_k^{(n)}\cdot g(t_k),
\end{equation}
где $\rho(t)$ - весовая функция (для полинома Лежандра $= 1$), $C_k^{(n)}$ - постоянные коэффициенты при использовании полинома степени n, $t_k$ - нули полинома Лежандра, $k=\overline{1,n}$.\par
\qquad Коэффициенты $C_k^{(n)}$ могут вычисляться как по формуле (7.5), так и по формуле (7.6):

\begin{equation}\tag{7.5}
    C_k^{(n)} = \frac{2}{(1-{x_k}^2)\cdot[P_n(x_k)]^2},
\end{equation}
\begin{equation}\tag{7.6}
    C_k^{(n)} = \frac{2\cdot(1-{x_k}^2)}{n^2\cdot P_{n-1}(x_k)}.
\end{equation}

\item {\textbf{Полином Чебышёва}}\par
\qquad Полином Чебышёва может быть определен либо формулой (7.7), либо формулой (7.8):
\begin{equation}\tag{7.7}
    T_n(x)=\cos(n\cdot \arccos(x)),
\end{equation}
\begin{equation}\tag{7.8}
    T_n(x)=\frac{(x+\sqrt{x^2 - 1})^n + (x-\sqrt{x^2 - 1})^n}{2}.
\end{equation}
\qquad Этот полином также определен на отрезке $[-1,1]$. Поэтому, если пределы интегрирования отличаются от отрезка $[-1,1]$, то необходимо произвести замену (7.9):
 
\begin{equation}\tag{7.9}
    x=\frac{b+a}{2}+\frac{b-a}{2}t.
\end{equation}

\qquad Тогда квадратурная формула Гаусса-Кристоффеля с полиномом Чебышёва [3] примет вид (7.10):
\begin{equation}\tag{7.10}
    \int\limits_a^b f(x)dx = \int\limits_{-1}^{1} g(t)\cdot\rho(t)dt \approx \sum_{k=1}^{n} C_k^{(n)}\cdot g(t_k),
\end{equation}
где $\rho(t)$ - весовая функция (для полинома Чебышёва $= \frac{1}{\sqrt{1-t^2}}$), $C_k^{(n)} =$ постоянные коэффициенты при использовании полинома степени n, $t_k = \cos\frac{2k - 1}{2n} \pi$ - нули полинома Чебышёва $T_n(t)$, $k=\overline{1,n}$. \par

\qquad Коэффициенты $C_k^{(n)}$ от параметра k не зависят и все вычисляются по формуле (7.11):
\begin{equation}\tag{7.11}
    C_k^{(n)} = \frac{\pi}{n},
\end{equation}
где n - степень используемого полинома Чебышёва.

\item {\textbf{Полином Лагерра}}\par 
\qquad Полином Лагерра определяется формулой (7.12):
\begin{equation}\tag{7.12}
    L_n(x) = \frac{x^{-\alpha} e^x}{n!} \frac{d^n}{dx^n} (x^{n+\alpha} e^{-x}).
\end{equation}
\qquad Этот полином определен на интервале $(0,\infty)$. \par
\qquad Квадратурная формула Гаусса-Кристоффеля с полиномом Лагерра [3] имеет вид (7.13):
\begin{equation}\tag{7.13}
    \int\limits_0^{\infty} f(x)dx = \int\limits_0^{\infty} g(x)\cdot\rho(x)dx \approx \sum_{k=1}^{n} C_k^{(n)}\cdot g(x_k),
\end{equation}
где $\rho(x)$ - весовая функция (для полинома Лагерра $= x^{\alpha} e^{-x}$), $C_k^{(n)} =$ постоянные коэффициенты при использовании полинома степени n, \par 
$x_k$ - нули полинома Лагерра, $k=\overline{1,n}$. \par
\qquad Коэффициенты $C_k^{(n)}$ могут вычисляться как по формуле (7.14), так и по формуле (7.15):

\begin{equation}\tag{7.14}
    C_k^{(n)} = \frac{n!\Gamma(\alpha + n + 1)}{x_k [L_n^{'}(x_k)]^2},
\end{equation}
\par
\begin{equation}\tag{7.15}
    C_k^{(n)} = \frac{x_k}{(n+1)^2 [L_{n+1}^{\alpha}(x_k)]^2}.
\end{equation}

\item {\textbf{Полином Эрмита}}\par
\qquad Полином Эрмита определяется формулой (7.16):
\begin{equation}\tag{7.16}
    H_n(x) = (-1)^n e^{x^2} \frac{d^n}{dx^n} (e^{-x^2}).
\end{equation}
\qquad Этот полином определен на интервале $({-\infty},\infty)$. \par

\qquad Квадратурная формула Гаусса-Кристоффеля с полиномом Эрмита [3] имеет вид (7.17):
\begin{equation}\tag{7.17}
    \int\limits_{-\infty}^{\infty} f(x)dx = \int\limits_{-\infty}^{\infty} g(x)\cdot\rho(x)dx \approx \sum_{k=1}^{n} C_k^{(n)}\cdot g(x_k),
\end{equation}
где $\rho(x)$ - весовая функция (для полинома Эрмита $= e^{-x^2}$), $C_k^{(n)} =$ постоянные коэффициенты при использовании полинома Эрмита, \par
$x_k$ - нули полинома Эрмита, $k=\overline{1,n}$. \par
\qquad Коэффициенты $C_k^{(n)}$ вычисляются по формуле (7.18):

\begin{equation}\tag{7.18}
    C_k^{(n)} = \frac{2^{n-1} n! \sqrt{\pi}}{n^2 H_{n-1}^2 (x_k)}.
\end{equation}
\end{enumerate}

\large\item {\large\textbf{Результаты численных расчётов}}\par
\qquad 3.1. Рассмотрим интеграл 1.1,  при $b=1.01$. \par
\qquad В первую очередь приведем данный интеграл к виду квадратурной формулы Гаусса-Кристоффеля, представленной полиномом Лежандра:
\begin{equation}
    \begin{gathered}\notag
    \int\limits_1^2 \frac{dx}{x\sqrt{4x^2 + bx}} = \left\{ x=\frac{3}{2} + \frac{1}{2} t \right\} = \\
    = 2 \int\limits_{-1}^1 \frac{dt}{(3+t)\sqrt{4(3+t)^2 + 2b \cdot (3+t)}}.
\end{gathered}
\end{equation}
\qquad Таким образом выделяем функцию $g(x)$:
\begin{equation}\notag
    g(x) = \frac{2}{\sqrt{(3+x)^3} \cdot \sqrt{12 + 4x + 2b}}.
\end{equation}

Таблица 1 — Точное значение интеграла, численно посчитанное значение, абсолютная и относительная погрешности.

\begin{table}[H]
\centering
\begin{tabular}{|c|c|c|c|}
 \hline
 Точное значение & Посчитанное значение & Абсолютная погрешность  & Относительная погрешность \\ \hline
    0.229314 	& 0.229316 	& $1.401765\cdot 10^{-6}$ 	& $6.112821\cdot 10^{-6}$ \\ \hline
\end{tabular}
\end{table} \par

\qquad 3.2. Рассмотрим интеграл 1.2. \par
\qquad В первую очередь приведем данный интеграл к виду квадратурной формулы Гаусса-Кристоффеля, представленной полиномом Чебышёва:
\begin{equation}
    \begin{gathered}\notag
        \int\limits_1^2 \frac{dx}{\sqrt{x-0.01N}\sqrt{x+1}} = \left\{ x=\frac{3}{2} + \frac{1}{2} t \right\} = \int\limits_{-1}^1 \frac{dt}{\sqrt{3+t-0.02N} \cdot \sqrt{3+t+2}} = \\ = \int\limits_{-1}^1 \frac{1}{\sqrt{1-t^2}} \cdot \frac{\sqrt{1-t^2}}{\sqrt{3+t-0.02N} \cdot \sqrt{3+t+2}} dt.
    \end{gathered}
\end{equation}
\qquad Таким образом выделяем функцию $g(x)$:
\begin{equation}\notag
    g(x) = \frac{\sqrt{1-x^2}}{\sqrt{3+x-0.02N} \cdot \sqrt{5+x}}.
\end{equation}

Таблица 2 — Точное значение интеграла, численно посчитанное значение, абсолютная и относительная погрешности.

\begin{table}[H]
\centering
\begin{tabular}{|c|c|c|c|}
 \hline
 Точное значение & Посчитанное значение & Абсолютная погрешность  & Относительная погрешность \\ \hline
    0.536332  &  0.531590 & $4.741777\cdot 10^{-003}$  &  $8.919993\cdot 10^{-003}$ \\ \hline
\end{tabular}
\end{table} \par

\qquad 3.3. Рассмотрим интеграл 1.3,  при $\alpha = 0.98$. \par
\qquad В первую очередь приведем данный интеграл к виду квадратурной формулы Гаусса-Кристоффеля, представленной полиномом Лагерра:
\begin{equation}
    \begin{gathered} \notag
        \int\limits_0^\infty \frac{e^{-(1-\alpha)x}}{\sqrt{x}} dx =\int\limits_0^\infty \frac{e^{-x} \cdot e^{\alpha x}}{\sqrt{x}} dx = \int\limits_0^\infty e^{-x} \cdot \frac{e^{\alpha x}}{\sqrt{x}} dx. 
    \end{gathered}
\end{equation}
\qquad Таким образом выделяем функцию $g(x)$:
\begin{equation}\notag
    g(x) = \frac{e^{\alpha x}}{\sqrt{x}}.
\end{equation}

Таблица 3 — Точное значение интеграла, численно посчитанное значение, абсолютная и относительная погрешности.

\begin{table}[H]
\centering
\begin{tabular}{|c|c|c|c|}
 \hline
 Точное значение & Посчитанное значение & Абсолютная погрешность & Относительная погрешность \\ \hline
    8.141483 	& 12.533141 	& 4.391658 	& 0.350403 \\ \hline
 \end{tabular}
\end{table} \par

\qquad 3.4. Рассмотрим интеграл 1.4, при $\alpha = -0.499$. \par
\qquad В первую очередь приведем данный интеграл к виду квадратурной формулы Гаусса-Кристоффеля, представленной полиномом Эрмита:
\begin{equation}
    \begin{gathered}\notag
        \int\limits_{-\infty}^\infty e^{{-\pi}x^2} \cos \frac{2\sqrt{\pi}\alpha}{\alpha+0.01}x dx 
        = \left\{ x=\frac{t}{\sqrt{\pi}} \right\} 
        = \frac{1}{\sqrt{\pi}} \int\limits_{-\infty}^\infty e^{-t^2} \cos \frac{2\alpha t}{\alpha+0.01} dt = \\
        = \frac{1}{\sqrt{\pi}} \int\limits_{-\infty}^\infty e^{-t^2}  \cdot \cos \frac{2\alpha t}{\alpha+0.01} dt.  
    \end{gathered}
\end{equation}
\qquad Таким образом выделяем функцию $g(x)$:
\begin{equation}\notag
    g(x) = \frac{1}{\sqrt{\pi}} \cdot \cos \frac{2\alpha x}{\alpha+0.01}.
\end{equation}


Таблица 4 — Точное значение интеграла, численно посчитанное значение, абсолютная и относительная погрешности.

\begin{table}[H]
\centering
\begin{tabular}{|c|c|c|c|}
 \hline
 Точное значение & Посчитанное значение & Абсолютная погрешность  & Относительная погрешность \\ \hline
    0.352717 	& 0.352989 	& $2.726063\cdot 10^{-4}$ 	& $7.722797\cdot 10^{-4}$ \\ \hline
\end{tabular}
\end{table} \par

\large\item {\large \textbf{Вывод}}\par
\qquad В ходе выполнения лабораторной работы были посчитаны приближённые значения заданных интегралов с использованием квадратурных формул Гаусса-Кристоффеля с полиномами Лежандра, Чебышёва, Лагерра, Эрмита и приближенные значения были сравнены с точными.
\end{enumerate}

\newpage
\begin{center}
\refstepcounter{section} %гиперссылка
\addcontentsline{toc}{section}{Лабораторная работа №8}
\section*{\large Лабораторная работа №8\\
Численное интегрирование.\\Приближенное вычисление несобственных интегралов.}
\end{center}

\renewcommand{\labelenumi}{\textbf{\arabic{enumi}.}}
\renewcommand{\labelenumii}{\textbf{\arabic{enumi}.\arabic{enumii}}}
\renewcommand{\labelenumiii}{\textbf{\arabic{enumi}.\arabic{enumii}.\arabic{enumiii}}}
\renewcommand{\labelenumiv}{\textbf{\arabic{enumi}.\arabic{enumii}.\arabic{enumiii}.\arabic{enumiv}}}

\begin{enumerate}
\large\item {\large \textbf{Постановка задачи}}
\begin{enumerate}
\item Вычислить с точностью до трех значащих цифр несобственные интегралы:
				\begin{equation*}
                    \text{1.1}
					\int\limits_{0}^{1} \frac{x^A \sin{[(A+1) \ln{x}]}}{\ln{x}}dx , A=-0.5+0.02N,
					\end{equation*}
				\begin{equation*}
                     \text{1.2}
					\int\limits_{0}^{\infty} \frac{\sqrt[3]{x}dx}{1+2x\cos{\frac{\pi}{6}+0.02N}+x^2},
					\end{equation*}
				\begin{equation*}
                    \text{1.3}
					\int\limits_{0}^{\infty} \ln{\left(\frac{1}{x}\right)}e^{-(3+0.02N)x}\sin{[(4-0.02N)x]}dx.
				\end{equation*}
	\item Для всех вышеприведенных интегралов использовать формулу Симпсона и указать реальные пределы интегрирования с обоснованием их выбора для достижения указанной точности. В случае преобразования интегралов (например, избавление от особенностей) указать, как это проделано. Вычислить точные значения этих интегралов и сравнить их с полученными приближенными значениями.
\end{enumerate}

\large\item {\large \textbf{Теоретический материал}}
    \begin{enumerate}
    \item {\textbf{Интегралы от разрывных функций}}\par
    \qquad Пусть требуется вычислить несобственный интеграл (8.1):
    \begin{equation}\tag{8.1}
     \int\limits_a^b f(x)dx,
    \end{equation} 
    где подынтегральная функция $f(x)$ обращается в бесконечность в некоторой точке $c$ отрезка $[a,b]$. Тогда по определению интеграл (8.1) представляют в виде (8.2):
        \begin{equation}\tag{8.2}
            \int\limits_a^b f(x)dx = \lim_{ \substack{ \delta_1 \to 0 \\ \delta_2 \to 0 } } \left\{ \int\limits_a^{c-\delta_1} f(x)dx + \int\limits_{c+\delta_2}^b f(x)dx \right\}.
        \end{equation}\par
    \qquad Чтобы вычислить сходящийся несобственный интеграл [3,13] с заданной точностью $\varepsilon$, положительные числа $\delta_1$ и $\delta_2$ выбирают столь малыми, чтобы выполнялось неравенство (8.3):
         \begin{equation}\tag{8.3}
             \left| \ \int\limits_{c-\delta_1}^{c+\delta_2} f(x)dx \ \right | < \frac{\varepsilon}{2}.
         \end{equation}
    \qquad Затем по каким-либо квадратурным формулам приближённо вычисляются определённые интегралы (8.4) и (8.5):
         \begin{equation}\tag{8.4}
             \int\limits_a^{c-\delta_1} f(x)dx,    
         \end{equation}
         \begin{equation}\tag{8.5}
             \int\limits_{c+\delta_2}^b f(x)dx.     
         \end{equation}
   \qquad Если $S_1$ и $S_2$ - приближённые значения интегралов (8.4) и (8.5) соотвественно с точность до $\frac{\varepsilon}{4}$ каждое, то значение интеграла (8.1) равно (8.6):
         \begin{equation}\tag{8.6}
             \int\limits_a^b f(x)dx \approx S_1+S_2
         \end{equation} 
    и вычислено с точностью до $\varepsilon$.
    
    \item {\textbf{Метод усечения}}\par
    \qquad Для того, чтобы приближенно вычислить сходящийся несобственный интеграл [3,13] (8.7):
         \begin{equation}\tag{8.7}
             \int\limits_a^{\infty} f(x)dx
         \end{equation} 
    с точностью до $\varepsilon$, то его представляют в виде (8.8):
         \begin{equation}\tag{8.8}
             \int\limits_a^{\infty} f(x)dx =
             \int\limits_a^b f(x)dx + 
             \int\limits_b^{\infty} f(x)dx,   
         \end{equation}
    где число $b$ выбирают настолько большим, что бы имело место неравенство (8.9):
         \begin{equation}\tag{8.9}
            \left | \ \ \int\limits_b^{\infty} f(x)dx \ \ \right | < \frac{\varepsilon}{2}.
         \end{equation}
    \qquad Затем определённый интеграл (8.10):
         \begin{equation}\tag{8.10}
           \int\limits_a^b f(x)dx
         \end{equation}
    из (8,8) вычисляют по одной из квадратурных формул с точностью $\frac{\varepsilon}{2}$ и приближенно полагают, что интеграл (8.7) равен (8.11):
         \begin{equation}\tag{8.11}
           \int\limits_a^{\infty} f(x)dx \approx \int\limits_a^b f(x)dx.
         \end{equation}
\end{enumerate}

\large\item {\large \textbf{Результаты численных расчётов}} \par
3.1. Рассмотрим интеграл 1.1:
        \begin{equation*}
			\int\limits_{0}^{1} \frac{x^A \sin{[(A+1) \ln{x}]}}{\ln{x}}dx , \text{где} \ \ A=-0.5+0.02N.
		\end{equation*}
  \qquad Интеграл имеет особенности в точках $x=1$ и  $x=0$. Выполним преобразования:
        \begin{equation*}
             \int\limits_{0}^{1} \frac{x^A \sin{[(A+1) \ln{x}]}}{\ln{x}}dx = \int\limits_{0}^{a} + \int\limits_{a}^{b}+
             \int\limits_{b}^{1} < \frac{\varepsilon}{3}+\frac{\varepsilon}{3}+\frac{\varepsilon}{3}<\varepsilon.
        \end{equation*}
    \qquad Найдём $a$:
        \begin{equation*}
        \begin{gathered}
             \left | \int\limits_{0}^{a} \frac{x^A \sin{[(A+1) \ln{x}]}}{\ln{x}}dx \right | \leq \int\limits_{0}^{a} \left | \frac{x^A}{\ln{x}} \right | dx \leq \int\limits_{0}^{a} x^A dx \leq \frac{a^{A+1}}{A+1} = \frac{\varepsilon}{3}, \\
             \\
             a = \left ( \frac{\varepsilon}{3} \cdot (A+1) \right )^{\frac{1}{A+1}}.
    \end{gathered}
    \end{equation*}
    \qquad Найдём $b$:
         \begin{equation*}
            \begin{gathered}
                \left | \int\limits_{b}^{1} \frac{x^A \sin{[(A+1) \ln{x}]}}{\ln{x}}dx \right | \leq \int\limits_{b}^{1} \frac{x^A (A+1) \ln{x}}{\ln{x}}dx  \leq \\
                 \leq (A+1)\int\limits_{b}^{1} x^A dx \leq 1 - b^{A+1} = \frac{\varepsilon}{3}, \\
                 b = \left ( 1 - \frac{\varepsilon}{3} \right )^{\frac{1}{A+1}}. 
           \end{gathered}
        \end{equation*}
        \par
\qquad а =$6.982388\cdot 10^{-10}$, b=0.999935, h=0.0000769. \par

3.2. Рассмотрим интеграл 1.2
        \begin{equation*}
			\int\limits_{0}^{\infty} \frac{\sqrt[3]{x}dx}{1+2x\cos{(\frac{\pi}{6})}+0.02N+x^2}.
		\end{equation*}
  \qquad Интеграл имеет особенность в точке $x=\infty$. Выполним преобразования:
        \begin{equation*}
            \begin{gathered}
                \int\limits_{0}^{\infty} \frac{\sqrt[3]{x}dx}{1+2x\cos{(\frac{\pi}{6})}+0.02N+x^2} = \int\limits_{0}^{b} +\int\limits_{b}^{\infty} 
                \ \Rightarrow \\
                \ \Rightarrow
                \left | \int\limits_{b}^{\infty}\frac{\sqrt[3]{x}dx}{1+2x\cos{(\frac{\pi}{6})}+0.02N+x^2} \right |< \int\limits_{b}^{\infty} \frac{x^{\frac{1}{3}}}{x^2}dx =\int\limits_{b}^{\infty} x^{-\frac{5}{3}} dx =\\
                = \frac{3}{2 \cdot b^{\frac{2}{3}}} = \frac{\varepsilon}{2}, \\
                b = \left (\frac{3}{\varepsilon} \right )^{\frac{3}{2}}.
             \end{gathered}
        \end{equation*}
        \par
    \qquad а =0.0, b=5196152.422706, h= 0.017. \par
3.3. Рассмотрим интеграл 1.3
        \begin{equation*}
            \int\limits_{0}^{\infty} \ln{\left(\frac{1}{x}\right)}e^{-(3+0.02N)x}\sin{[(4-0.02N)x]}dx.
		\end{equation*}
  \qquad Интеграл имеет особенности в точках $x=0$ и  $x=\infty$. Выполним преобразования:
         \begin{equation*}
            \int\limits_{0}^{\infty} \ln{\left(\frac{1}{x}\right)}e^{-(3+0.02N)x}\sin{[(4-0.02N)x]}dx =
            \int\limits_{0}^{a} + \int\limits_{a}^{b}+
             \int\limits_{b}^{\infty} <\varepsilon.
		 \end{equation*}
   \qquad Найдём $a$:
        \begin{equation*}
            \begin{gathered}
                 \left | \int\limits_{0}^{a} \ln{\left(\frac{1}{x}\right)}e^{-(3+0.02N)x}\sin{[(4-0.02N)x]}dx \right | < \\
                 < \left |\int\limits_{0}^{a} \ln{\left(\frac{1}{x}\right)}(4-0.02N)x dx \right |  < 
                 \left |(4-0.02N)\int\limits_{0}^{a}\ln{\left(\frac{1}{x}\right)}x dx \right | < \\
                 < \left |(4-0.02N)\int\limits_{0}^{a} \frac {\ln{\left(\frac{1}{x}\right)}}{\frac{1}{x}} dx \right|
                 < \left |(4-0.02N)\int\limits_{0}^{a} \frac {-\frac{1}{x}}{-\frac{1}{x^2}} dx \right| <\\
                 < (4-0.02N)\cdot \frac{a^2}{2} =\frac{\varepsilon}{3}, \\
                 \\
                 a=\sqrt{\frac{2 \cdot \varepsilon}{3 \cdot (4-0.02N)}}.
             \end{gathered}
        \end{equation*}
        \par
        \qquad Найдём $b$:
         \begin{equation}
            \begin{gathered}
             \left | \int\limits_{b}^{\infty} \ln{\left(\frac{1}{x}\right)}e^{-(3+0.02N)x}\sin{[(4-0.02N)x]}dx \right | < \\
             <  \left | \int\limits_{b}^{\infty} \frac{\ln{x}}{e^{(3+0.02N)x}} dx \right | < \frac{1}{3+0.02N} \int\limits_{b}^{\infty} \frac{\frac{1}{x}}{e^{(3+0.02N)x}} dx < \\
             < \frac{e^{-(3+0.02N)b}}{(3+0.02N)^2} = \frac{\varepsilon}{3}, \\
             \\
             b = \frac{\ln{\frac{3}{\varepsilon \cdot (3+0.02N)^2}}}{3+0.02N}. \\
            \end{gathered}
         \end{equation}
         \par
         \qquad а =0.004092, b=2.681602, h=0.00505. \par
         Таблица 1 — Точные значения заданных интегралов, численно посчитанные значения, абсолютная
и относительная погрешности.
\begin{table}[H]
\centering
\begin{tabular}{|c|c|c|c|}
 \hline
                                           & Интеграл 1.1 & Интеграл 1.2 & Интеграл 1.3 \\ \hline
 Точное значение           & 0.785398                   & 1.254957                & 0.237021                 \\  \hline
 Приближённое значение     & 0.785963                   & 1.254304                  & 0.237016                  \\  \hline
 Абсолютная погрешнось     & $5.655719\cdot 10^{-4}$    & $6.529567\cdot 10^{-4}$    & $5.232722\cdot 10^{-6}$     \\ \hline 
 Относительная погрешнось  & $7.201085\cdot 10^{-4}$   & $5.203018\cdot 10^{-4}$   & $2.207697\cdot 10^{-5}$     \\ \hline 
\end{tabular}
\end{table} \par


\large\item {\large \textbf{Вывод}}\par
\qquad В ходе выполнения лабораторной работы заданные несобственные интегралы были преобразованы, были обоснованы выборы их пределов интегрирования и затем они были вычислены с использованием формулы Симпсона. После были вычислены точные значения заданных интегралов и было проведено сравнение точных значений с посчитанными приближёнными значениями.
\end{enumerate}

\newpage
\begin{center}
\refstepcounter{section} %гиперссылка
\addcontentsline{toc}{section}{Лабораторная работа №9}
\section*{\large Лабораторная работа №9\\
Кратные интегралы.\\ Метод повторного интегрирования. Метод Монте-Карло.}
\end{center}


\renewcommand{\labelenumi}{\textbf{\arabic{enumi}.}}
\renewcommand{\labelenumii}{\textbf{\arabic{enumi}.\arabic{enumii}}}
\renewcommand{\labelenumiii}{\textbf{\arabic{enumi}.\arabic{enumii}.\arabic{enumiii}}}
\renewcommand{\labelenumiv}{\textbf{\arabic{enumi}.\arabic{enumii}.\arabic{enumiii}.\arabic{enumiv}}}

\begin{enumerate}
\large\item {\large \textbf{Постановка задачи}}
\par
\begin{enumerate}
\item Методом повторного интегрирования, применяя формулы центральных прямоугольников, трапеций и Симпсона, вычислить интегралы с точностью $\varepsilon=10^{-5}$:
\begin{enumerate}
\item $\int\limits_{0}^{2}dy\int\limits_{0}^{1}(x^2+2y)dx$,
\item $\int\limits_{3}^{4}dx\int\limits_{1}^{2}\frac{dy}{(x+y)^2}$,
\item $\int\limits_{0}^{1}dx\int\limits_{1}^{2}\frac{x^2dy}{1+y^2}$.
\end{enumerate}
\par
\item Вычислить интегралы методом Монте-Карло:
\begin{enumerate}
\item $\iint\limits_G \sqrt{x^2-y^2}dxdy$, \par
где $G$ - треугольник с вершинами $A(0,0), B(1,0), C(1,1)$.
\item $\iint\limits_G e^{\frac{x}{y}}dxdy$, \par
где $G$ - криволинейный треугольник, ограниченный параболой $y^2=x$ и прямыми $x=0, y=1$.
\item $\iiint\limits_G {\frac{dxdydz}{(x+y+z+1)^3}}$, \par
где $G$ - область интегрирования, ограниченная координатными осями и плоскостью  $x+y+z=1$.
\end{enumerate}
\end{enumerate}

\large\item {\large \textbf{Теоретический материал}} \par
\begin{enumerate}
    \item Метод повторного интегрирования квадратурных формул \par
    \qquad Если требуется вычислить повторный интеграл [3,9,13] вида (9.1): 
    \begin{equation}\tag{9.1}
        I = \iint\limits_G f(x,y)dxdy,
    \end{equation}
    где область G предсавляет собой прямоугольник $a\leq x \leq b, \ c\leq y \leq d, $ то вычисление такого интеграла сводится к двукратному интегрированию вида (9.2):
    \begin{equation}\tag{9.2}
        I = \int\limits_a^b dx \int\limits_c^d f(x,y)dy.
    \end{equation}
    \par
    \qquad При введении обозначения (9.3):
    \begin{equation}\tag{9.3}
        F(x) = \int\limits_c^d f(x,y)dy
    \end{equation}
    интеграл (9.2) приобретает вид (9.4):
    \begin{equation}\tag{9.4}
        I = \int\limits_a^b F(x)dx,
    \end{equation}
    который можно вычислить при помощи квадратурных формул. \\ \par
    
    \quad Введем $ n_x - $ количество разбиений отрезка $[a,b]$, $ n_y - $ количество разбиений отрезка $[c,d]$, $h_x = \frac{b - a}{n_x}, \  h_y = \frac{d - c}{n_y}$. \par
    \qquad Тогда узлы разбиений отрезков $[a,b]$ и $[c,d]$ будут заданы соответственно через (9.5) и (9.6):
    \begin{equation}\tag{9.5}
        x_i = a + i\cdot h_x \ (i=\overline{0, n_x}),
    \end{equation}
    \begin{equation}\tag{9.6}
        y_j = c + j\cdot h_y \ (j=\overline{0, n_y}).
    \end{equation}
    
    \begin{itemize}
          \item Формуле центральных прямоугольников соответствует формула (9.7):
        \begin{equation}\tag{9.7}
            \int\limits_a^b F(x)dx = h_x \cdot (F_0 + F_1 + \dotsc + F_{n_x - 1} ),
        \end{equation}
        где $F(x_i)$ вычисляется по формуле (9.8):
        \begin{equation}\tag{9.8}
            \begin{gathered}
            F_i = F(x_i) = \int\limits_c^d f(x_i,y)dy = \\  h_y \cdot \left (f(x_i, y_0) + f(x_i, y_1) + \dotsc + f(x_i, y_{n_y - 1}) \right).
            \end{gathered}
        \end{equation}
        
         \item Формуле трапеций соответствует формула (9.9):
        \begin{equation}\tag{9.9}
            \int\limits_a^b F(x)dx = h_x \cdot ( \frac{F_0 + F_{n_x}}{2} + F_1 + F_2 + \dotsc + F_{n_x - 2} + F_{n_x - 1} ),
        \end{equation}
        где $F(x_i)$ вычисляется по формуле (9.10):
        \begin{equation}\tag{9.10}
            \begin{gathered}
            F_i = F(x_i) = \int\limits_c^d f(x_i,y)dy = \\  h_y \cdot \big (\frac{f(x_i, y_0) + f(x_i, y_{n_y})}{2} + \\ 
            + f(x_i, y_1) + f(x_i, y_2) + \dotsc + f(x_i, y_{n_y - 2}) + f(x_i, y_{n_y - 1})\big).
            \end{gathered}
        \end{equation}
        
        \item Формуле Симпсона соответствует формула (9.11):
        \begin{equation}\tag{9.11}
            \begin{gathered}
            \int\limits_a^b F(x)dx = \frac{h_x}{3} \cdot \big(F_0 + F_{n_x} + \\ 
            + 4\cdot (F_1 + F_3 + \dotsc + F_{n_x - 1}) + 2\cdot (F_2 + F_4 + \dotsc + F_{n_x - 2}) \big),
            \end{gathered}
        \end{equation}
        где  $F(x_i)$ вычисляется по формуле (9.12):
        \begin{equation}\tag{9.12}
            \begin{gathered}
            F_i = F(x_i) = \int\limits_c^d f(x_i,y)dy = \\  
            \frac{h_y}{3} \cdot \big (f(x_i, y_0) + f(x_i, y_{n_y}) + \\
            + 4 \cdot (f(x_i, y_1) + f(x_i, y_3) + \dotsc + f(x_i, y_{n_y - 1})) + \\
            + 2 \cdot ( f(x_i, y_2) + f(x_i, y_4) + \dotsc + f(x_i, y_{n_y - 2}))\big).
            \end{gathered}
        \end{equation}
    \end{itemize}

    \item Метод Монте-Карло \par
    \qquad Пусть требуется вычислить [3,9,13] $m-$кратный интеграл вида (9.13):
    \begin{equation}\tag{9.13}
        I = \idotsint\limits_G f(x_1, x_2, \dotsc, x_m) dx_1 dx_2 \dotsc dx_m.
    \end{equation}
    по области G, лежащей в $m-$мерном единичном клубе. \par
    \qquad Тогда выбирается $N$ равномерно распределённых на отрезке $[0,1]$ последовательностей случайных чисел (9.14):
    \begin{equation}\tag{9.14}
        \begin{matrix}
            x_1^{(1)}, & x_2^{(1)}, & x_3^{(1)}, & \dotsc, & x_N^{(1)} \\
            x_1^{(2)}, & x_2^{(2)}, & x_3^{(2)}, & \dotsc, & x_N^{(2)} \\
            \dots      & \dots      & \dots      & \dotsc  & \dots     \\
            x_1^{(m)}, & x_2^{(m)}, & x_3^{(m)}, & \dotsc, & x_N^{(m)} \\
        \end{matrix},
    \end{equation}
    соответствущие которым точки $M_i(x_i^{(1)}, x_i^{(2)}, \dotsc, x_i^{(m)})$ $(i = \overline{1,N})$ можно рассматривать как случайные, равномерно распределённые в $m-$мерном единичном кубе. \par
    \qquad Пусть из общего числа $N$ случайных точек $n-$точек попали в область G. Тогда при достаточно большом $N$ имеет место приближённая формула (9.15):
    \begin{equation}\tag{9.15}
        I \approx \frac{V_G}{n} \sum_{i=1}^n f(M_i),
    \end{equation}
    где $V_G \ - \ m-$мерный объём области интегрирования. \par
    \qquad Если вычисление объёма $V_G$ затруднительно, то можно приближенно принять $V_G \approx \frac{n}{N}$, и тогда формула приближённого вычисления интеграла (9.15) примет вид (9.16):
    \begin{equation}\tag{9.16}
        I \approx \frac{1}{N} \sum_{i=1}^n f(M_i).
    \end{equation}
\end{enumerate}

\large\item {\large\textbf{Результаты численных расчётов}}\par
\begin{enumerate}
\item Метод повторного интегрирования \par
Таблица 1 — Точное значение интеграла 1.1.1, численно посчитанные с использованием квадратурных формул значения, абсолютная и относительная погрешности.

\begin{table}[H]
\centering
\begin{tabular}{|c|c|c|c|}
 \hline
 & Ц. прям & М. Трапеций & М. Симпсона \\ \hline
Tочные значения       &    4.666666                &   4.666666                &   4.666666                 \\ \hline
Приближенные значения  &   4.666667                &   4.666667                &   4.666650                 \\ \hline
Абсолютная погрешность   &	 $4.922790\cdot 10^{-7}$   &   $8.256124\cdot 10^{-7}$   &   $1.650775\cdot 10^{-5}$    \\ \hline
Относительная погрешность &	$1.054883\cdot 10^{-7}$   &   $1.769169\cdot 10^{-7}$   &   $3.537375\cdot 10^{-6}$    \\ \hline
\end{tabular}
\end{table} \par

Таблица 2 — Точное значение интеграла 1.1.2, численно посчитанные с использованием квадратурных формул значения, абсолютная и относительная погрешности.

\begin{table}[H]
\centering
\begin{tabular}{|c|c|c|c|}
 \hline
 & Ц. прям & М. Трапеций & М. Симпсона \\ \hline
Tочные значения       &    0.040822                   &   0.040822                 &   0.040822                 \\ \hline
Приближенные значения   &  0.040822                   &   0.040822                 &   0.040888                 \\ \hline
Абсолютная погрешность  &  $3.216499\cdot 10^{-9}$      &   $6.470482\cdot 10^{-10}$   &   $6.668029\cdot 10^{-5}$    \\ \hline
Относительная погрешность  &	$6.892498\cdot 10^{-10}$     &   $1.386532\cdot 10^{-10}$   &   $1.428863\cdot 10^{-5}$    \\ \hline
\end{tabular}
\end{table} \par

Таблица 3 — Точное значение интеграла 1.1.3, численно посчитанные с использованием квадратурных формул значения, абсолютная и относительная погрешности.

\begin{table}[H]
\centering
\begin{tabular}{|c|c|c|c|}
 \hline
 & Ц. прям & М. Трапеций & М. Симпсона \\ \hline
Tочные значения        &   0.107250                 &  0.107250                 &  0.107250                  \\ \hline 
Приближенные значения  &   0.107250                 &  0.107250                 &  0.107639                  \\ \hline 
Абсолютная погрешность  &  $1.095230\cdot 10^{-8}$    &  $1.082283\cdot 10^{-7}$    &  $3.888819\cdot 10^{-4}$     \\ \hline 
Относительная погрешность  & $2.346921\cdot 10^{-9}$    &  $2.319178\cdot 10^{-8}$    &  $8.333184\cdot 10^{-5}$     \\ \hline
\end{tabular}
\end{table} \par

\item Метод Монте-Карло \par
Таблица 4 — Точное значение интеграла 1.2.1, численно посчитанные с использованием Метода Монте-Карло значения при различных $N$, абсолютная и относительная погрешности.

\begin{table}[H]
\centering
\begin{tabular}{|c|c|c|c|}
 \hline
 & N=10000  & N=100000 & N=1000000 \\ \hline
Точные значения           & 0.261799               &  0.261799               &  0.261799               \\ \hline 
Приближенные значения     & 0.262207               &  0.260666               &  0.261816               \\ \hline 
Абсолютная погрешность    & $ 4.082930\cdot 10^{-4}$  &  $1.132946\cdot 10^{-3}$  &  $1.711040\cdot 10^{-5}$  \\ \hline 
Относительная погрешность & $1.559566\cdot 10^{-3}$  &  $4.327541\cdot 10^{-3}$  &  $ 6.535704\cdot 10^{-5}$  \\ \hline
\end{tabular}
\end{table} \par

Таблица 5 — Точное значение интеграла 1.2.2, численно посчитанные с использованием Метода Монте-Карло значения при различных $N$, абсолютная и относительная погрешности.

\begin{table}[H]
\centering
\begin{tabular}{|c|c|c|c|}
 \hline
 & N=10000  & N=100000 & N=1000000 \\ \hline
Точные значения           & 0.500000               &  0.500000               &  0.500000               \\ \hline 
Приближенные значения     & 0.492357               &  0.497518               &  0.499551               \\ \hline 
Абсолютная погрешность    & $7.642808\cdot 10^{-3}$  &  $2.482012\cdot 10^{-3}$  &  $4.493571\cdot 10^{-4}$  \\ \hline 
Относительная погрешность & $2.166585\cdot 10^{-2}$  &  $4.964024\cdot 10^{-3}$  &  $8.987142\cdot 10^{-4}$  \\ \hline 
\end{tabular}
\end{table} \par

Таблица 6 — Точное значение интеграла 1.2.3, численно посчитанные с использованием Метода Монте-Карло значения при различных $N$, абсолютная и относительная погрешности.

\begin{table}[H]
\centering
\begin{tabular}{|c|c|c|c|}
 \hline
 & N=10000  & N=100000 & N=1000000 \\ \hline
Точные значения           & 0.034618               &  0.034074               &  0.034074               \\ \hline 
Приближенные значения     & 0.032762               & 0.034352               &  0.034049               \\ \hline 
Абсолютная погрешность    & $5.440619\cdot 10^{-4}$  &  $2.781141\cdot 10^{-4}$  &  $2.456422\cdot 10^{-5}$  \\ \hline 
Относительная погрешность & $1.596707\cdot 10^{-2}$  &  $8.162064\cdot 10^{-3}$  &  $7.209081\cdot 10^{-4}$  \\ \hline
\end{tabular}
\end{table} \par
\end{enumerate}
\large\item {\large \textbf{Вывод}}\par
\qquad В ходе выполнения лабораторной работы методом повторного интегрирования были посчитаны заданные интегралы с требуемой точностью с применением формул центральных прямоугольников, трапеций и Симпсона и были сравнены с точными значениями. После заданные интегралы были вычислены методом Монте-Карло при различных количествах распределенных точек и проведено сравнение посчитанных результатов с точными значениями.
\end{enumerate}

\newpage
\begin{center}
\refstepcounter{section} %гиперссылка
\addcontentsline{toc}{section}{Лабораторная работа №10}
\section*{\large Лабораторная работа №10\\
Вычисление корней уравнений.\\Методы дихотомии, секущих, простой итерации и Ньютона.}
\end{center}

\renewcommand{\labelenumi}{\textbf{\arabic{enumi}.}}
\renewcommand{\labelenumii}{\textbf{\arabic{enumi}.\arabic{enumii}}}
\renewcommand{\labelenumiii}{\textbf{\arabic{enumi}.\arabic{enumii}.\arabic{enumiii}}}
\renewcommand{\labelenumiv}{\textbf{\arabic{enumi}.\arabic{enumii}.\arabic{enumiii}.\arabic{enumiv}}}

\begin{enumerate}
\large\item {\large \textbf{Постановка задачи}}
\begin{enumerate}
\item Найти корни уравнений:
\begin{enumerate}
    \item  $e^{2x}+3x-N=0$ с точностью $\varepsilon=10^{-3}$,
    \item  $x^4-(1+a)x^3-(4-a)x^2+4(1+a)x-4a-\sqrt{x-1}=0$ с точностью $\varepsilon=10^{-4}$, $a=1.01-0.001N$.
\end{enumerate}
\item Локализовать корень графически.
\item Решить методом дихотомии, изобразить графически ход решения.
\item Решить методом секущих, изобразить графически ход решения.
\item Решить методом простых итераций, подбирая функцию $\varphi(x)$ исходя из достаточного условия сходимости, изобразить графически ход решения.
\item Решить методом Ньютона, находя начальное приближение, используя теорему о достаточных условиях сходимости метода Ньютона, изобразить графически ход решения.
\end{enumerate}
\large\item {\large \textbf{Теоретический материал}}
        \begin{enumerate}
            \item {\textbf{Локализация корня графически}}\par
            \qquad Действительные корни уравнения (10.1):
             \begin{equation}\tag{10.1}
                 f(x)=0
             \end{equation}
        приближённо можно определить как абсциссы точек пересечения графика  функции $y=f(x)$ с осью $Ox$. Во многих случаях уравнение (10.1) представляют в более простом виде (10.2):
            \begin{equation}\tag{10.2}
                 f(x)= \varphi(x) - \psi(x) = 0,
             \end{equation}
        где функции $\varphi(x)$ и $\psi(x)$ - более простые функции, чем функция $f(x)$.\par
        \qquad Построив графики функций $y=\varphi(x)$ и $y=\psi(x)$, искомые корни получим как абсциссы точек пересечения этих графиков.
\newpage

         \item {\textbf{Метод Дихотомии}}\par
            \qquad Пусть дано уравнение (10.1), где функция $f(x)$ непрерывна на отрезке $[a, b]$ и f$(a)\cdot f(b) < 0$.\par
            \qquad Для нахождения корня уравнения (10.1) [4,9,13,20], принадлежащего отрезку $[a, b]$, делим этот отрезок пополам. Если $f\left ( \frac{a+b}{2} \right ) = 0$, то  $\xi =\frac{a+b}{2}$ является корнем уравнения. Если $f\left( \frac{a+b}{2}\right ) \neq 0 $, то выбираем ту из половни $[a,\frac{a+b}{2} ]$ или $[\frac{a+b}{2}, b ]$, на концах которой функция $f(x)$ имеет противоположные знаки. Новый полученный отрезок $[a_1, b_1]$ снова делим пополам и повторяем вышеизложенное исследование функции.\par
            \qquadИтерационный процесс будет продолжаться до тех пор, пока после n-ой итерации длина отрезка не станет меньше некоторой заданной величины $\varepsilon$, т.е. пока не будет выполняться условие (10.3):
            \begin{equation}\tag{10.3}
                |b-a| \leq \varepsilon.            \end{equation}  
            \qquad В этом случае искомое значение корня принимается равным полученному приближению, т.е. решение уравнения будет найдено с точностью $\varepsilon$.

            
           \item {\textbf{Метод секущих}}\par
            \qquad 
            Чтобы найти корень уравнения (10.1) на отрезке $[a, b]$ методом секущих [4,9,13,20], необходимо воспользоваться общей формулой итерационного процесса (10.4) :
            \begin{equation}\tag{10.4}
                x_k= x_{k-1} - \frac{f(x_{k-1})}{f(x_{k-1})-f(x_{k-2})}\cdot (x_{k-1} -x_{k-2}),
            \end{equation}
            где $k=2,3, \dotsc $\par
            \qquad Метод секущих является двухшаговым, т.е. для определения нового приближения $x_k$ необходимы две предыдущие итерации $x_{k-1}$ и $x_{k-2}$.\par
            \qquad При вычисления первого приближения для формулы (10.4) принимается, что $x_{1}= b$, а $x_{0} = a$.\par
            \qquad При реализации метода секущих используется критерий остановки (10.5):
            \begin{equation}\tag{10.5}
                |x_k - x_{k-1}| < \varepsilon, 
            \end{equation}
            т.е. изменение $x_k$ после $k-$ой итерации стало меньше значения $\varepsilon$.
            
            \item {\textbf{Метод простой итерации}}\par
            \qquad  Сущность метода [4,9,13,20] заключается в замене исходного уравнения (10.1) уравнением вида (10.6):
            \begin{equation}\tag{10.6}
                x=\varphi(x).
            \end{equation}
            \qquad Сам метод простой итерации реализуется следующим образом. Выбираем из каких-либо соображений приближенное (может быть, очень грубое) значение корня $x_0$ и подставляем его в правую часть (10.6), после чего получим некоторое число (10.7):
            \begin{equation}\tag{10.7}
                x_1 = \varphi(x_0).
            \end{equation}
            \qquad Подставляя в правую часть (10.6) $x_1$, получим $x_2$ и т.д. Таким образом, возникает некоторая последовательность чисел $x_0, x_1, x_2,\dotsc ,x_n,\dotsc ,$ для которой выполняется соотношение (10.8): 
            \begin{equation}\tag{10.8}
                x_i = \varphi(x_{i-1}).
            \end{equation}
            \qquad Итерационный процесс уточнения корней будет продолжаться до тех пор, пока не будет выполняться критерий остановки (10.9):
            \begin{equation}\tag{10.9}
                |x_k - x_{k-1}| < \varepsilon.
            \end{equation}
            \qquad Для применения данного метода необходимо выполнение достаточного условия сходимости итерационного процесса:
            пусть функция $\varphi(x)$  определена и дифференцируема на отрезке $[a,b]$ и все её значения $\varphi(x) \in [a,b]$. Тогда, если выполнено условие $|\varphi'(x)|\leq q < 1$ при $a<x<b$, то:\par
           \qquad 1) Итерационный процесс (10.8) cходится независимо от начального значения $x_0 \in [a,b]$;\par
           \qquad 2) $x= \displaystyle\lim_{n\to\infty} x_n $ является единственным корнем уравнения (10.6) на $[a,b]$.\par
            \item {\textbf{Метод Ньютона}}\par
            \qquad Пусть корень уравнения (10.1) определён на отрезке $[a,b]$, причём первая и вторя производные $f'(x)$ $f''(x)$ непрерывны и знакопостоянны при $x\in [a,b]$.\par
            \qquad В этом случае для построения последовательности приближений к корню может быть использован метод Ньютона [4,9,13,20]: каждое следующее приближение $x_n$ вычисляется через предыдущее приближение $x_{n-1}$ по формуле (10.10):
            \begin{equation}\tag{10.10}
                x_n = x_{n-1}-\frac{f(x_{n-1})}{f'(x_{n-1})}, \ \ n=1,2...
            \end{equation}
            \qquad Таким образом, задав начальное приближение $x_0$, можно получить первое приближение по формуле (10.11):
            \begin{equation}\tag{10.11}
                x_1 = x_{0}-\frac{f(x_{0})}{f'(x_{0})},
            \end{equation}
            затем второе приближение по формуле (10.12):
            \begin{equation}\tag{10.12}
                x_2 = x_{1}-\frac{f(x_{1})}{f'(x_{1})} 
            \end{equation}
            и так далее, до тех пор, пока не будет выполнен критерий остановки (10.13):
             \begin{equation}\tag{10.13}
                |x_n - x_{n-1}| < \varepsilon.
            \end{equation}
    \end{enumerate}
\large\item {\large \textbf{Результаты численных расчётов}} \par
\begin{enumerate}
\item Рассмотрим уравнение 1.1.1: $e^{2x}+3x-N=0$,  $ \varepsilon = 10^{-3}$.\par
\begin{enumerate}
\item Локализация корня графически \par
        \begin {figure}[H]
        \center{\includegraphics[width=0.8\linewidth]{../pics/localized_1.pdf} \\
	Рисунок 1 — График пересечения функции 1.1.1 и оси абсцисс.\par
        $1$ — заданная функция.}
        \end{figure}

 \qquad Из Рисунка 1 можно сделать вывод, что корень следует искать приблизительно на отрезке $[-0.3;1.0]$. Функция принимает разные значения на концах данного отрезка.\par
 \newpage

\item Метод Дихотомии \par
Таблица 1 — Приближенное значение, точное значение, абсолютная и относительная погрешности.
\begin{table}[H]
\centering
\begin{tabular}{|c|c|c|c|}
 \hline
Посчитанное значение & Точное значение & Абсолютная погрешность  & Относительная погрешность \\ \hline
    1.0333862305&    1.0334157340&    0.0000295035&    0.0000285495\\ \hline
\end{tabular}
\end{table} \par


        \begin {figure}[H]
        \center{\includegraphics[width=0.8\linewidth]{../pics/dichotomy_1.pdf} \\ 
	Рисунок 2 — Метод дихотомии.\par
	$1$ — заданная функция, $2$ — метод дихотомии.}
        \end{figure} 

\item Метод секущих \par
Таблица 2 — Приближенное значение, точное значение, абсолютная и относительная погрешности.
\begin{table}[H]
\centering
\begin{tabular}{|c|c|c|c|}
 \hline
Посчитанное значение &Точное значение & Абсолютная погрешность & Относительная погрешность \\ \hline 
    1.0334155299&    1.0334157340&    0.0000002041&    0.0000001975\\ \hline
\end{tabular}
\end{table} \par

 \begin {figure}[H]
        \center{\includegraphics[width=0.8\linewidth]{../pics/secant_1.pdf} \\ 
	Рисунок 3 —  Метод секущих.\par
	$1$ — заданная функция, $2$ — метод секущих.}
        \end{figure} \par

\item Метод простой итерации \par
Для исходной функции была подобрана эквивалентная функция, для которой выполняется достаточное условие сходимости. \par
Таблица 3 — Приближенное значение, точное значение, абсолютная и относительная погрешности.
\begin{table}[H]
\centering
\begin{tabular}{|c|c|c|c|}
 \hline
Посчитанное значение &Точное значение & Абсолютная погрешность & Относительная погрешность \\ \hline
    1.0334239380&    1.0334157340&    0.0000082040&    0.0000079387\\ \hline
\end{tabular}
\end{table} \par
 \begin {figure}[H]
        \center{\includegraphics[width=0.8\linewidth]{../pics/simpiter_1.pdf} \\ 
	Рисунок 4 —  Метод простых итераций.\par
	$1$ — функция $\phi(x)$ , $2$ — функция y = x,\par
	$3$ — метод простых итераций.}
        \end{figure} \par

\item Метод Ньютона \par

Таблица 4 — Приближенное значение, точное значение, абсолютная и относительная погрешности.
\begin{table}[H]
\centering
\begin{tabular}{|c|c|c|c|}
 \hline
Посчитанное значение & Точное значение & Абсолютная погрешность & Относительная погрешность \\ \hline
    1.0334157338&    1.0334157340&    0.0000000002&    0.0000000002\\ \hline
\end{tabular}
\end{table} \par

 \begin {figure}[H]
        \center{\includegraphics[width=0.8\linewidth]{../pics/newton_1.pdf} \\ 
	Рисунок 5 — Метод Ньютона.\par
	$1$ — заданная функция, $2$ — метод Ньютона.}
        \end{figure} \par

\end{enumerate}
\item Рассмотрим уравнение под 1.1.2:  $x^4-(1+a)x^3-(4-a)x^2+4(1+a)x-4a-\sqrt{x-1}=$ \\ $= 0$, a =0.999 \ $ \varepsilon = 10^{-4}$.\par

\begin{enumerate}

\item \textit{Локализация корня графически} \par
        \begin {figure}[H]
        \center{\includegraphics[width=0.8\linewidth]{../pics/localized_2.pdf} \\
	Рисунок 6 — График пересечения функции 1.1.2 и оси абсцисс. \par
        $1$ — заданная функция.}
        \end{figure}

 \qquad Из Рисунка 6 можно сделать вывод, что корень следует искать приблизительно на отрезке $[2;3]$. Функция принимает разные значения на концах данного отрезка.\par

\item Метод Дихотомии \par

 Таблица 5 — Приближенное значение, точное значение, абсолютная и относительная погрешности.
\begin{table}[H]
\centering
\begin{tabular}{|c|c|c|c|}
 \hline
Посчитанное значение & Точное значение & Абсолютная погрешность & Относительная погрешность \\ \hline
    2.1850662231&    2.1850629750&    0.0000032481&    0.0000014865\\ \hline
\end{tabular}
\end{table} \par

        \begin {figure}[H]
        \center{\includegraphics[width=0.8\linewidth]{../pics/dichotomy_2.pdf} \\ 
	Рисунок 7 — Метод дихотомии.\par
	$1$ — заданная функция, $2$ — метод дихотомии.}
        \end{figure} \par


\item Метод секущих \par
Таблица 6 — Приближенное значение, точное значение, абсолютная и относительная погрешности.
\begin{table}[H]
\centering
\begin{tabular}{|c|c|c|c|}
 \hline
Посчитанное значение & Точное значение & Абсолютная погрешность & Относительная погрешность \\ \hline
    2.1850629762&    2.1850629750&    0.0000000012&    0.0000000006\\ \hline
\end{tabular}
\end{table} \par

 \begin {figure}[H]
        \center{\includegraphics[width=0.8\linewidth]{../pics/secant_2.pdf} \\ 
Рисунок 8 — Метод секущих.\par
	$1$ — заданная функция, $2$ — метод секущих.}
        \end{figure} \par

\item Метод простой итерации \par
Для исходной функции была подобрана эквивалентная функция, для которой выполняется достаточное условие сходимости. \par
Таблица 7 — Приближенное значение, точное значение, абсолютная и относительная погрешности.
\begin{table}[H]
\centering
\begin{tabular}{|c|c|c|c|}
 \hline
Посчитанное значение & Точное значение & Абсолютная погрешность & Относительная погрешность \\ \hline
    2.1850229578&    2.1850629750&    0.0000400172&    0.0000183140\\ \hline
\end{tabular}
\end{table} \par

 \begin {figure}[H]
        \center{\includegraphics[width=0.8\linewidth]{../pics/simpiter_2.pdf} \\
	Рисунок 9 — Метод простых итераций.\par
	$1$ — функция $\phi(x)$ , $2$ — функция y = x,\par
	$3$ — метод простых итераций.}
        \end{figure} \par

\item Метод Ньютона \par

Таблица 8 — Приближенное значение, точное значение, абсолютная и относительная погрешности.
\begin{table}[H]
\centering
\begin{tabular}{|c|c|c|c|}
 \hline
Посчитанное значение &Точное значение & Абсолютная погрешность & Относительная погрешность \\ \hline
    2.1850629764&    2.1850629750&    0.0000000014&    0.0000000006\\ \hline
\end{tabular}
\end{table} \par

 \begin {figure}[H]
        \center{\includegraphics[width=0.8\linewidth]{../pics/newton_2.pdf} \\ 
Рисунок 10 — Метод Ньютона.\par
	$1$ — заданная функция, $2$ — метод Ньютона.}
        \end{figure} \par

\end{enumerate}
\end{enumerate}
\large\item {\large \textbf{Вывод}}\par
\qquad В ходе выполнения лабораторной работы сначала были приближенно посчитаны корни заданных уравнений методами дихотомии, секущих, простых итераций и Ньютона. Затем было проведено сравнение вычисленных корней с точными значениями. Для каждого метода поиска корней графически был изображен ход решения.
\end{enumerate}

\newpage
\begin{center}
\refstepcounter{section} %гиперссылка
\addcontentsline{toc}{section}{Лабораторная работа №11}
\section*{\large Лабораторная работа №11\\
Алгебра матриц.\\Вычисление определителей, обратных матриц, норм матриц.}
\end{center}


\renewcommand{\labelenumi}{\textbf{\arabic{enumi}.}}
\renewcommand{\labelenumii}{\textbf{\arabic{enumi}.\arabic{enumii}}}
\renewcommand{\labelenumiii}{\textbf{\arabic{enumi}.\arabic{enumii}.\arabic{enumiii}}}
\renewcommand{\labelenumiv}{\textbf{\arabic{enumi}.\arabic{enumii}.\arabic{enumiii}.\arabic{enumiv}}}

\begin{enumerate}
\large\item {\large \textbf{Постановка задачи}}
\begin{enumerate}
	\item Вычислить методом Гаусса определитель матрицы $C:$ 
            \begin{center}
			$
				\begin{pmatrix}
				  \alpha^2-1& 0.1\alpha+0.1&0.2\alpha+0.2&0.3\alpha+0.3&0.4\alpha+0.4\\
				  0.1\alpha-0.1&\alpha^2-3.99&0.5\alpha+1.02&0.6\alpha+1.23&0.7\alpha+1.44\\
				0.2\alpha-0.2&0.5\alpha-0.98&\alpha^2+0.29&0.8\alpha+0.36&0.9\alpha+0.43\\
				0.3\alpha-0.3&0.6\alpha-1.17&0.8\alpha+0.36&\alpha^2-14.91&\alpha-2.74\\
				0.4\alpha-0.4&0.7\alpha-1.36&0.9\alpha+0.43&\alpha+5.26&\alpha^2-6.54
				\end{pmatrix}
			$
            \end{center}
Положим параметр $\alpha=0.1-0.001N$.

\item Вычислить методом Гаусса обратную матрицу для матрицы $A:$

\begin{center}
			$
				\begin{pmatrix}
				  5& -\frac{4}{5\alpha}&\frac{1}{5\alpha^2}&-\frac{1}{15\alpha^3}&\frac{1}{30\alpha^4}\\
				  -20\alpha&4&-\frac{1}{\alpha}&\frac{1}{3\alpha^2}&-\frac{1}{6\alpha^3}\\
				45\alpha^2&-9\alpha&3&-\frac{1}{\alpha}&\frac{1}{2\alpha^2}\\
				-60\alpha^3&12\alpha^2&-4\alpha&2&-\frac{1}{\alpha}\\
				30\alpha^4&-6\alpha^3&2\alpha^2&-\alpha&1
				\end{pmatrix}
			$
\end{center}
Положим параметр $\alpha=1-0.01N$.
\item Вычислить 3 нормы матриц $A$ и $A^{-1}$, указанных в пункте 1.2 настоящего задания:
					\begin{equation*}
							||A||_m=\max_i \sum\limits_{j}|a_{ij}|, \text{$m$ - норма}
					\end{equation*}

						\begin{equation*}
							||A||_l=\max_j \sum\limits_{i}|a_{ij}|, \text{$l$ - норма}
					\end{equation*}

					\begin{equation*}
							||A||_k= \sqrt{\sum\limits_{i,j}|a_{ij}|^2}, \text{$k$ - норма}
					\end{equation*}
\end{enumerate}
\newpage

\item {\large \textbf{Теоретический материал}} \par
\begin{enumerate}
    \item {\large \textbf{Вычисление определителя методом Гаусса}} \par
    \qquad Определитель треугольной матрицы A равен произведению элементов главной диагонали, поэтому необходимо привести матрицу к треугольному виду при помощи метода Гаусса [1,4,9,13,20]. \par
   \qquad Пусть дана матрица А (11.1):
    \begin{equation}\tag{11.1}
        \text{А =}  
             \begin{pmatrix}
                a_{11}   & a_{12}   & a_{13}   & \cdots & a_{1n}   \\
                a_{21}   & a_{22}   & a_{23}   & \cdots & a_{2n}   \\
                a_{31}   & a_{32}   & a_{33}   & \cdots & a_{3n}   \\
                \cdots   & \cdots   & \cdots   & \cdots & \cdots   \\
                a_{n1}   & a_{n2}   & a_{n3}   & \cdots & a_{nn}   \\
            \end{pmatrix}.
    \end{equation}
    \qquad Тогда, для приведения её к треугольному виду, необходимо умножить её слева на матрицу $M_1$, равную (11.2):
    \begin{equation}\tag{11.2}
        M_1 =   \begin{pmatrix}
                    1         & 0        & 0        & \cdots & 0        \\
                    -m_{21}   & 1        & 0        & \cdots & 0        \\
                    -m_{31}   & 0        & 1        & \cdots & 0        \\
                    \cdots    & \cdots   & \cdots   & \cdots & \cdots   \\
                    -m_{n1}   & 0        & 0        & \cdots & 1        \\
                \end{pmatrix},
    \end{equation}
    где $m_{i1} = \frac{a_{i1}}{a_{11}}, \ i=\overline{2,n}$. \par
    \qquad В результате будет получена матрица, у которой в первом столбце все элементы кроме элемента первой строки будут равны нулю. \par
    \qquad Если полученная матрица ещё не имеет верхний треугольный вид, то нужно умножить её слева на матрицу $M_2$, равную (11.3):
    \begin{equation}\tag{11.3}
        M_2 =   \begin{pmatrix}
                    1         & 0        & 0        & \cdots & 0        \\
                    0         & 1        & 0        & \cdots & 0        \\
                    0         & -m_{32}  & 1        & \cdots & 0        \\
                    \cdots    & \cdots   & \cdots   & \cdots & \cdots   \\
                    0         & -m_{n2}  & 0        & \cdots & 1        \\
                \end{pmatrix},
    \end{equation}
    где $m_{i2} = \frac{a_{i2}}{a_{22}}, \ i=\overline{3,n}$. \par
    \qquad В случае, если полученная матрица не имеет верхний треугольный вид, то процесс умножения слева на матрицы $M_i \ (i=\overline{3, n-1})$ продолжается, пока мы не получим матрицу с верхним треугольным видом. \par
    \qquad Определитель этой верхней треугольной матрицы будет равен произведению ее диагональных элементов.

 \item {\large \textbf{Вычисление обратной матрицы методом Гаусса}} \par
 \qquad Для вычисления обратной матрицы воспользуемся формулой (11.4):
    \begin{equation}\tag{11.4}
        M_{n-1} \dots M_2 M_1 A A^{-1} =  M_{n-1} \dots M_2 M_1 E,
    \end{equation}
    где E - единичная матрица. \par
    \qquad Будем рассматривать $i-$й столбец $\left (a_{1i} \ a_{2i} \ a_{3i} \ \cdots \ a_{ni} \right )^T $
    обратной марицы $A^{-1}$ как столбец переменных $ \left (x_1^{(i)} \ x_2^{(i)} \ x_3^{(i)} \ \cdots \ x_n^{(i)} \right )^T$, а $i-$й столбец $\left (e_{1i} \ e_{2i} \ e_{3i} \ \cdots \ e_{ni} \right )^T$ полученной матрицы $M_{n-1} \dots M_2 M_1 E$ будем рассматривать как столбец свободных членов $\left (b_{1} \ b_{2} \ b_{3} \ \cdots \ b_{n} \right )^T $. \par

    \qquad Тогда нахождение обратной матрицы сводится к решению $n-$ числа СЛАУ вида (11.5):
    \begin{equation}\tag{11.5}
        A' \cdot x^{(i)} = b_i,
    \end{equation}
    где $i = \overline{1, n}, \ A' = M_{n-1} \dots M_2 M_1 A$. \par
    \qquad После вычисления всех $x^{(i)}$ мы можем составить матрицу (11.6):
    \begin{equation}\tag{11.6}
        A^{-1} =\begin{pmatrix}
                    x_1^{(1)} & x_1^{(2)} & x_1^{(3)} & \cdots & x_1^{(n)}            \\
                    x_2^{(1)} & x_2^{(2)} & x_2^{(3)} & \cdots & x_2^{(n)}            \\
                    x_3^{(1)} & x_3^{(2)} & x_3^{(3)} & \cdots & x_3^{(n)}            \\
                    \cdots    & \cdots    & \cdots    & \cdots & \cdots               \\
                    x_n^{(1)} & x_n^{(2)} & x_n^{(3)} & \cdots &      x_n^{(n)}       \\
                \end{pmatrix}.
    \end{equation}
\item {\large \textbf{Нормы матриц}} \par
    \qquad Нормы матриц [1,4,9,13,20] имеют следующие формулы (11.7)-(11.9):
    
    \begin{equation}\tag{11.7}
    \centering
    \text{1.}
	\||A||_m=\max_{1\leq i \leq n} \sum\limits_{j = 1}^n |a_{ij}|, \ \text{$m$ - норма}
    \end{equation}

    \begin{equation}\tag{11.8}
    \centering
    \text{2.}
	\ ||A||_l=\max_{1\leq j \leq n} \sum\limits_{i=1}^n|a_{ij}|, \ \text{$l$ - норма}
    \end{equation}

    \begin{equation}\tag{11.9}
    \centering
    \text{3.}
	\  ||A||_k= \sqrt{\sum\limits_{\substack{i = 1 \\ j = 1}}^n|a_{ij}|^2}, \ \text{$k$ - норма}
    \end{equation}
\end{enumerate}
\large\item {\large\textbf{Результаты численных расчётов}}\par
\begin{enumerate}
 \item Заданная матрица С имеет вид:
    \begin{center} $
	\begin{pmatrix}
	      -0.990199  &  0.109900   & 0.219800  &  0.329700  &  0.439600     \\
 	     -0.090100  & -3.980199   & 1.069500  &  1.289400  &  1.509300     \\
  	    -0.180200  & -0.930500   & 0.299801  &  0.439200  &  0.519100     \\
 	     -0.270300  & -1.110600   & 0.439200  &-14.900199  & -2.641000     \\
  	    -0.360400  & -1.290700   & 0.519100  &  5.359000  & -6.530199     \\
        \end{pmatrix} $
    \end{center}
    Определитель матрицы C, вычисленный методом Гаусса, равен  $= 5.566859$.

\item Заданная матрица A имеет вид:
    \begin{center} $
        \begin{pmatrix}
   	 5.000000    &  -0.808081    &  0.204061    & -0.068707   & 0.034701       \\
    	  -19.800000  &     4.000000  &   -1.010101  &    0.340101 &  -0.171768     \\
   	   44.104500   &   -8.910000   &   3.000000   &  -1.010101  &  0.510152      \\
   	   -58.217940  &    11.761200  &   -3.960000  &    2.000000 &  -1.010101     \\
   	   28.817880   &   -5.821794   &   1.960200   &  -0.990000  &  1.000000      \\
        \end{pmatrix} $
    \end{center}

\item Обратная матрица $A^{-1}$, вычисленная методом Гаусса, имеет вид:
    \begin{center} $
        \begin{pmatrix}
		5.000000    &  -0.808081    &  0.204061    & -0.068707   & 0.034701       \\
		-19.800000  &     4.000000  &   -1.010101  &    0.340101 &  -0.171768     \\
		44.104500   &   -8.910000   &   3.000000   &  -1.010101  &  0.510152      \\
		-58.217940  &    11.761200  &   -3.960000  &    2.000000 &  -1.010101     \\
		28.817880   &   -5.821794   &   1.960200   &  -0.990000  &  1.000000      \\
        \end{pmatrix} $
    \end{center}

\item Нормы матриц \par
    Таблица 1 — Нормы m, l, k для матриц $A, \ A^{-1}$.
    \begin{table}[h!]
        \centering
        \begin{tabular}{|c|c|c|c|} \hline
            & m-норма & l-норма & k-норма \\ \hline
   		$A$      &  76.949241 & 155.940320  & 82.995927\\ \hline
   		$A^{-1}$ & 7.286700 &  5.950000  &  7.526849\\ \hline
        \end{tabular}
    \end{table} \par
\end{enumerate}

\item {\large \textbf{Вывод}}\par
\qquad В ходе выполнения лабораторной работы были вычислены определитель матрицы методом Гаусса, обратная матрица методом Гаусса, m-норма, l-норма и k-норма для заданной и посчитанной обратной матриц.
\end{enumerate}

\newpage
\begin{center}
\refstepcounter{section} %гиперссылка
\addcontentsline{toc}{section}{Лабораторная работа №12}
\section*{\large Лабораторная работа №12\\
Решение систем линейных уравнений.\\Методы Гаусса, простой итерации и Зейделя.}
\end{center}
\renewcommand{\labelenumi}{\textbf{\arabic{enumi}.}}
\renewcommand{\labelenumii}{\textbf{\arabic{enumi}.\arabic{enumii}}}
\renewcommand{\labelenumiii}{\textbf{\arabic{enumi}.\arabic{enumii}.\arabic{enumiii}}}
\renewcommand{\labelenumiv}{\textbf{\arabic{enumi}.\arabic{enumii}.\arabic{enumiii}.\arabic{enumiv}}}

\begin{enumerate}
\large\item {\large \textbf{Постановка задачи}}
\begin{enumerate}
\item Решить методом Гаусса систему линейных уравнений  $A\vec{x}=\vec{b}$
\begin{center} $
    A= \begin{pmatrix}
	\alpha^2  &  1.1      & 1.2        & 1.3        & 1.4         \\
	1.4\alpha & 2\alpha^2 & 1.1+\alpha & 1.2+\alpha & 1.3+\alpha  \\
	1.3\alpha & 1.4\alpha & 3\alpha^2  & 1.1+\alpha & 1.2+2\alpha \\
	1.2\alpha & 1.3\alpha & 1.4\alpha  & 4\alpha^2  & 1.1+3\alpha \\
	1.1\alpha & 1.2\alpha & 1.3\alpha  & 1.4\alpha  & 5\alpha^2   \\
    \end{pmatrix} $
\end{center}

\begin{center} $
\vec{b}= \begin{pmatrix}
	   3.21\alpha^2+1.934               \\
	   19.58\alpha^2-1.326\alpha-8.253  \\
	   16.74\alpha^2-0.631\alpha-13.251 \\
	   -17.16\alpha^2+3.061\alpha-7.821 \\
	   -35.55\alpha^2+16.527\alpha      \\
    \end{pmatrix} $
\end{center}
Положим параметр $\alpha=0.01+0.002N$.

\item Решить систему уравнений $A\vec{x}=\vec{b}$ методом простой итерации и методом Зейделя с точностью $\varepsilon=10^{-3}$. Сравнить полученные числа итераций.
\begin{center}
    $ A= 
    \begin{small}
        \begin{pmatrix}
		  10          & \alpha+0.3               &\alpha+0.2                &\alpha+0.1               & \alpha+1                 \\
		  10\alpha-3  & \alpha^2+9.91            &\alpha^2+0.9\alpha+0.24   &\alpha^2+0.8\alpha+0.17  & \alpha^2+1.7\alpha+1.7   \\
		10\alpha-2  & \alpha^2+10.1\alpha-3.06 &2\alpha^2+9.87            &2\alpha^2+0.8\alpha+0.22 & 2\alpha^2+3.5\alpha+0.22 \\
		10\alpha-1  & \alpha^2+10.2\alpha-2.03 &2\alpha^2+10.2\alpha-3.08 &3\alpha^2+9.86           & 3\alpha^2+6.4\alpha+2.6  \\
		10\alpha-10 &\alpha^2+9.3\alpha-20.3  &2\alpha^2+7.5\alpha-30.8  &3\alpha^2+4.6\alpha-41.4 & 4\alpha^2-20             \\
        \end{pmatrix} 
    \end{small}$
\end{center}

\begin{center} $
\vec{b}= \begin{pmatrix}
	   35.07\alpha+25.07                    \\
	   35.07\alpha^2+49.379\alpha+40.122    \\
	   69.9\alpha^2+88.55\alpha+63.1691     \\
	   103.2\alpha^2+155.906\alpha+169.1416 \\
	   124.38\alpha^2+71.759\alpha-979.664  \\
    \end{pmatrix} $
\end{center}

Положим параметр $\alpha=0.02+0.001N$.
\end{enumerate}

\large\item {\large \textbf{Теоретический материал}}
\begin{enumerate}
\item {\large\textbf{Решение СЛАУ методом Гаусса}} \par
\qquad СЛАУ вида (12.1):
\begin{equation}\tag{12.1}
    A \cdot \vec{x} = \vec{b},
\end{equation}
где $A \ -$ заданная матрица, $\vec{x} \ -$ вектор-столбец неизвестных, $\vec{b} \ -$ вектор-столбец свободных членов, можно решить методом Гаусса [1,4,9,13,20], для чего сначала необходимо привести матрицу $A$ к треугольному (ступенчатому) виду. \par
\qquad Для приведения матрицы $A$ к ступенчатому виду необходимо последовательно умножать её слева на матрицы $M_i, \ i=\overline{1, n-1}$ (формула матрицы $M_i$ была приведена в лабораторной работе №11), пока $i-$ое умножение не даст верхнетреугольную матрицу. \par
\qquad После необходимо применить обратный ход метода Гаусса: последовательно находится столбец неизвестных $\vec{x}$, начиная с последнего уравнения полученной системы.

\item {\large\textbf{Решение СЛАУ методом простых итераций}} \par
\qquad Чтобы решить СЛАУ вида (12.1) методом простых итераций [1,4,9,13,20], необходимо убедиться, что выполняется диагональное преобладание элементов матрицы $A$, т.е. для элементов матрицы $A$ должно выполняться условие (12.2):
\begin{equation}\tag{12.2}
    \begin{pmatrix}
        |a_{11}|>|a_{12}|+ \dots +|a_{1n}|         \\
        |a_{22}|>|a_{21}|+ \dots +|a_{2n}|         \\
        \dots\dots\dots\dots\dots\dots\dots\dots   \\
        |a_{nn}|>|a_{n1}|+ \dots +|a_{nn-1}|       \\
    \end{pmatrix}.
\end{equation}
\qquad В случае, если (12.2) не выполняется, необходимо добиться выполнения данного условия путём элементарных преобразований. \par
\qquad Далее необходимо преобразовать систему (12.1) к виду (12.3):
\begin{equation}\tag{12.3}
    \vec{x} = \vec{\beta} + \alpha \cdot \vec{x}, 
\end{equation}
где $\vec{\beta}$ - вектор-столбец, чьи элементы вычисляются по формуле $\beta_i = \frac{b_i}{a_{ii}}$, a $\alpha$ - матрица $n \times n$, состоящая из элементов $\alpha_{ij} = \begin{cases}
        -\frac{a_{ij}}{a_{ii}}, & i\neq j;   \\
        0,                      & i=j.       \\
    \end{cases}$ \par
\qquad Решение данного СЛАУ вычисляется с помощью последовательных приближений вида (12.4):
\begin{equation}\tag{12.4}
    \vec{x^{k+1}} = \vec{\beta} + \alpha \vec{x^k}.
\end{equation}
\qquad Начальное приближение $\vec{x^(0)}$ принимается равным $\vec{\beta}$.
\item {\large\textbf{Решение СЛАУ методом Зейделя}} \par
\qquad Метод Зейделя [1,4,9,13,20] представляет собой модификацию метода простой итераций. Основная его идея заключается в том, что при вычислении $(k+1)$-го приближения неизвестной $x_i$ учитываются уже вычисленные ранее $(k+1)$ приближения неизвестных $x_1, x_2, \dots x_{i-1}$. В этом методе, как и в методе простой итерации, для элементов матрицы $A$ должно выполнятся условие (12.2). \par
\qquad Итерационная схема Зейделя имеет вид (12.5):
\begin{equation}\tag{12.5}
    \begin{matrix*}[l]
        x_1^{k+1} = \beta_1 + \sum\limits_{j=2}^n \alpha_{ij} x_j^k                                                   \\
        x_2^{k+1} = \beta_2 + \sum\limits_{j=3}^n \alpha_{ij} x_j^k + \alpha_{21} x_1^{k+1}                           \\
        \dots\dots\dots\dots\dots\dots\dots\dots\dots\dots\dots\dots                                                  \\
        x_i^{k+1} = \beta_i + \sum\limits_{j=i+1}^n \alpha_{ij} x_j^k + \sum\limits_{j=1}^{i-1} \alpha_{ij} x_j^{k+1} \\
        \dots\dots\dots\dots\dots\dots\dots\dots\dots\dots\dots\dots                                                  \\                       
        x_n^{k+1} = \beta_n + \alpha_{nn-1} x_{n-1}^k + \sum\limits_{j=1}^{n-2} \alpha_{ij} x_j^{k+1}                 \\
        k = 0, 1, 2, \dots                             
    \end{matrix*}
\end{equation}
\end{enumerate}
\large\item {\large \textbf{Результаты численных расчётов}}\par
\qquad 1. Заданная матрица $A_1$ и вектор-столбец свободных членов $\vec{b_1}$ \par имеют вид: \par
    $A_1 = \begin{pmatrix}
        0.001444 & 1.100000 & 1.200000 & 1.300000 & 1.400000 \\
        0.053200 & 0.002888 & 1.138000 & 1.238000 & 1.338000 \\
        0.049400 & 0.053200 & 0.004332 & 1.138000 & 1.276000 \\
        0.045600 & 0.049400 & 0.053200 & 0.005776 & 1.214000 \\
        0.041800 & 0.045600 & 0.049400 & 0.053200 & 0.007220 \\
    \end{pmatrix}$,
    \ $b_1 = \begin{pmatrix}
          1.934462   \\
	-8.266093  \\
	-13.256162 \\
	-7.786739  \\
	0.193205   \\
    \end{pmatrix}$.

Таблица 1 — Посчитанные методом Гаусса значения вектора-столбца неизвестных.
\begin{table}[H]
    \centering
    \begin{tabular}{|c|c|c|c|c|c|} \hline
	& $x_1$ & $x_2$ & $x_3$ & $x_4$ & $x_5$ \\ \hline
        Значение & 3.210000  &  9.790001  &  5.579999  & -4.290000 &  -7.110000 \\ \hline
    \end{tabular}
\end{table} \par

\qquad 2. Заданная матрица $A_2$ и вектор-столбец свободных членов $\vec{b_2}$ \par имеют вид: \par
    $A_2 = \begin{pmatrix}
      10.000000  &  0.321000  &  0.221000  &   0.121000  &  1.021000    \\
      -2.790000  &  9.910441  &  0.259341  &   0.187241  &  1.736141    \\
      -1.790000  & -2.847459  &  9.870882  &   0.237682  &  0.294382    \\
      -0.790000  & -1.815359  & -2.864918  &   9.861323  &  2.735723    \\
      -9.790000  &-20.104258  & -30.641617 & -41.302079 & -19.998236   \\
    \end{pmatrix}$, \par
    \ $b_2 = \begin{pmatrix}
  	 25.806470  \\
 	  41.174427  \\
 	  65.059478  \\
 	  172.461139 \\
 	  -978.102211 \\
    \end{pmatrix}$. \par
\qquad В пятой строке заданной матрицы не выполняется диагональное преобладание - значит нам нужно привести матрицу к соответствующему виду с помощью элементраных преобразований. \par
\qquad Заданная матрица $A_2$ приведенная к виду с диагональным преобладанием и вектор-столбец свободных членов $\vec{b_2}$ имеют вид: \par
    $A_2^{'} = \begin{pmatrix}
      10.000000  &  0.321000  &  0.221000  &  0.121000   &  1.021000       \\ 
      -2.790000  &  9.910441  &  0.259341  &  0.187241   &  1.736141       \\ 
      -1.790000  & -2.847459  &  9.870882  &  0.237682   &  0.294382       \\ 
      -0.790000  & -1.815359  & -2.864918  &  9.861323   &  2.735723       \\ 
      -1.711999  & -0.164955  & -0.991366  &  0.156356   &  1.680420      \\  
    \end{pmatrix}$, \par
    \ $b_2^{'} = \begin{pmatrix}
	25.806470 \\
	41.174427  \\
	65.059478  \\
	172.461139  \\
	203.281838   \\ 
    \end{pmatrix}$. \par

Таблица 2 — Посчитанные значения вектора-столбца неизвестных $\vec{x}$ методом простой итерации c точностью $\varepsilon=10^{-3}$ и количество итераций, необходимых для достижения заданной точности.
\begin{table}[h!]
    \centering
    \resizebox{\textwidth}{!}{%
    \begin{tabular}{|c|c|c|c|c|c|c|} \hline
        & $x_1$  & $x_2$  & $x_3$ & $x_4$ & $x_5$ & Количество итераций \\ \hline
Метод простой итерации & -7.952071  & -17.166660  &  -2.689019  & -17.957840   &  111.268728  & 14                  \\ \hline
Точное решение         & -7.952111  & -17.166765  &  -2.689088  & -17.957949   &  111.268667  &                     \\ \hline
Абсолютная погрешность &   $3.963312\cdot 10^{-5}$  &   $1.047611\cdot 10^{-4}$  &    $6.928919\cdot 10^{-5}$  &    $1.092394\cdot 10^{-4}$ &   $6.038011\cdot 10^{-5}$  &                     \\ \hline

    \end{tabular}}
\end{table} \par

Таблица 3 — Посчитанные значения вектора-столбца неизвестных $\vec{x}$ методом Зейделя c точностью $\varepsilon=10^{-3}$ и количество итераций, необходимых для достижения заданной точности.
\begin{table}[h!]
    \centering
    \resizebox{\textwidth}{!}{%
    \begin{tabular}{|c|c|c|c|c|c|c|} \hline
        & $x_1$ & $x_2$ & $x_3$ & $x_4$ & $x_5$ & Количество итераций \\ \hline
Метод простой итерации & -7.952162  &  -17.166847   &  -2.689141  &  -17.958068  &  111.268587\  & 9                   \\ \hline
Точное решение         & -7.952111  &  -17.166765   &  -2.689088  &  -17.957949  &  111.268667\  &                     \\ \hline
Абсолютная погрешность &  $5.119429\cdot 10^{-5}$  &    $8.186405\cdot 10^{-5}$   &  $5.296958\cdot 10^{-5}$  &    $1.194146\cdot 10^{-4}$ &    $8.033083\cdot 10^{-5}$ &                     \\ \hline
    \end{tabular}}
\end{table} \par
\newpage
\large\item {\large \textbf{Вывод}}\par
\qquad В результате выполнения данной лабораторной работы были вычислены значения вектора-столбца неизвестных для исходной СЛАУ методом Гаусса; вычислены значения вектора-столбца неизвестных для заданной СЛАУ методами простых итераций и Зейделя с заданной точностью.
\end{enumerate}

\newpage
\begin{center}
\refstepcounter{section} %гиперссылка
\addcontentsline{toc}{section}{Лабораторная работа №13}
\section*{\large Лабораторная работа №13\\
Интегрирование обыкновенных дифференциальных уравнений.\\Методы неопределенных коэффициентов, степенных рядов и Пикара.}
\end{center}

\renewcommand{\labelenumi}{\textbf{\arabic{enumi}.}}
\renewcommand{\labelenumii}{\textbf{\arabic{enumi}.\arabic{enumii}}}
\renewcommand{\labelenumiii}{\textbf{\arabic{enumi}.\arabic{enumii}.\arabic{enumiii}}}
\renewcommand{\labelenumiv}{\textbf{\arabic{enumi}.\arabic{enumii}.\arabic{enumiii}.\arabic{enumiv}}}

\begin{enumerate}
\large\item {\large \textbf{Постановка задачи}}
\begin{enumerate}
	\item Найти решение дифференциального уравнения Эйлера методом неопределенных коэффициентов:

\begin{equation*}
	x^2y''+0.6y'x+(0.08+0.02N)y=0.
\end{equation*}

Решение уравнения должно удовлетворять начальным условиям:
\begin{equation*}
	y(2)=2, y'(2)=4 .
\end{equation*}
Пользуясь полученным решением, найти значения приближенного решения  дифференциального  уравнения в точках $x_i=2+0.01i$ для $i=1,2,\dots,10$ и сравнить полученные значения с точными значениями решения в этих же точках.

\item Написать 6 членов разложения в степенной ряд решения $y=y(x)$ дифференциального уравнения :

\begin{dmath*}
 \left[ (3a+2)x^3-(a+3)x^2+(a+3)x+(2a-1)\right]y''+
\\+\left[6(3a+2)x^2-4(a+3)x+2(a+3)\right]y'+
\\+\left[6(3a+2)x-2(a+3)\right]y=0
\end{dmath*}

$a=-3+0.01N$.

Решение уравнения должно удовлетворять начальным условиям:
\begin{equation*}
	y(2)=\frac{1}{3}, y'(2)=-\frac{1}{3} .
\end{equation*}

Пользуясь полученным разложением, найти значения приближенного решения  дифференциального  уравнения в точках $x_i=2+0.01i$ для $i=1,2,\dots,10$ и сравнить полученные значения с точными значениями решения в этих же точках.

\item Проинтегрировать методом Пикара систему дифференциальных уравнений:
    

\begin{equation*} 
 \begin{cases}
   y'_1=\frac{y_2-\frac{3}{2}\sqrt{x}}{x},\\
  y'_2=\frac{y_3+2\sqrt{x}}{x},\\
 y'_3=\frac{y_1-\frac{3}{2}\sqrt{x}}{x}.
 \end{cases}
\end{equation*}

С точностью $\varepsilon=10^{-4}$ на интервале $x\in[0.64,0.81]$ с шагом $h=0.01$.
Решение системы должно удовлетворять следующим начальным условиям:
\begin{equation*}
y_1(0.64)=0.64a+0.8,y_2(0.64)=0.64a+1.6,y_3(0.64)=0.64a-0.8, a=-1.0+0.001N.
\end{equation*}

Определить формулу вычисления первого приближения и обосновать ее. Показать число итераций, которое было получено при достижении заданной точности.

\end{enumerate}

\large\item {\large \textbf{Теоретический материал}}
\begin{enumerate}
\item {\large\textbf{Метод неопределенных коэффициентов}} \par
\qquad Метод неопределенных коэффициентов [3,10] применяется при решении линейных дифференциальных уравнений с переменными коэффициентами. Пусть задано линейное дифференциальное уравнение второго порядка вида (13.1)
\begin{equation}\tag{13.1}
    y'' + p(x) y' + q(x) y = r(x)
\end{equation}
с начальными условиями $y(0) = y_0$, $y'(0) = y'_0$. Пусть каждый из коэффициентов уравнения (13.1) можно разложить в ряд Тейлора по степеням $x$:
\begin{equation*}
    p(x) = \sum\limits_{n=0}^{\infty} p_n x^n, \ \ \ \ 
    q(x) = \sum\limits_{n=0}^{\infty} q_n x^n, \ \ \ \ 
    r(x) = \sum\limits_{n=0}^{\infty} r_n x^n.
\end{equation*}
Решение данного уравнения ищется в виде ряда (13.2)
\begin{equation}\tag{13.2}
    y(x) = \sum\limits_{n=0}^{\infty} c_n x^n,
\end{equation}
где коэффициенты $c_n$ неизвестны. \par
\qquad Дифференцируем обе части равенства (13.2) два раза по $x$:
\begin{equation*}
    y'(x) = \sum\limits_{n=1}^{\infty} n c_n x^{n-1}, \  y''(x) = \sum\limits_{n=2}^{\infty} n (n-1) c_n x^{n-2}.
\end{equation*}
Подставляя полученные ряды для $y$, $y'$, $y''$, $p$, $q$, $r$ в уравнение (13.1), получим выражение (13.3):
\begin{equation}\tag{13.3}
    \sum\limits_{n=2}^{\infty} n (n-1) c_n x^{n-2} + \sum\limits_{n=0}^{\infty} p_n x^n \cdot \sum\limits_{n=1}^{\infty} n c_n x^{n-1} +  \sum\limits_{n=0}^{\infty} q_n x^n \cdot \sum\limits_{n=0}^{\infty} c_n x^n = \sum\limits_{n=0}^{\infty} r_n x^n.
\end{equation}

Если умножить ряды и приравнять коэффициенты при одинаковых степенях $x$ в левой и в правой частях равенства (13.3), то получится система (13.4):
\begin{equation}\tag{13.4}
    \begin{cases}
        x^0| \ 2c_2 + c_1 p_0 + c_0 q_0 = r_0,  \\
        x^1| \ 3\cdot 2 c_3 + 2 c_2 p_0 + c_1 p_1 + c_1 q_0 + c_0 q_1 = r_1,  \\
        x^2| \ 4 \cdot 3  c_4 + 3 c_3 p_0 + 2 c_2 p_1 + c_1 p_2 + c_2 q_0 + c_1 q_1 + c_0 q_2 = r_2,  \\
        \ \cdot \ | \ \cdots \cdots \cdots \cdots \cdots \cdots \cdots \cdots \cdots \cdots \cdots \cdots \cdots \cdots   \\
        x^n| \ (n+2) (n+1) c_{n+2} + L(c_{n+1}, c_n, \dots, c_1, c_0) = r_n,
    \end{cases}
\end{equation}
где $L(c_{n+1}, c_n, \dots, c_1, c_0)$ означает линейную функцию аргументов $c_0$, $c_1$, $\dots$, $c_n$, $c_{n+1}$. \par
\qquad Каждое уранение системы (13.4) содержит на одно неизвестное больше по сравнению с предыдущим уравнением. Коэффициенты $c_0$, $c_1$ определяются из начальных условий, а все остальные последовательно определяются из системы (13.4). Если начальные условия заданы при $x=x_0$, то рекомендуется сделать замену $x-x_0 = t$.\par

\item {\large\textbf{Метод последовательного дифференцирования}} \par

\qquad Рассмотрим дифференциальное уравнение $n-$ого порядка (14.5):
\begin{equation}\tag{13.5}
    y^{(n)} = f(x, y, y', \dots, y^{(n-1)})
\end{equation}
с начальными условиями (13.6):
\begin{equation}\tag{13.6}
    y(x_0) = y_0, \ y'(x_0) = y'_0, \ \dots, \ y^{(n-1)} (x_0) = y^{(n-1)}_0.
\end{equation}

\qquad Предположим, что искомое частное решение $y=y(x)$ может быть разложено в ряд Тейлора по степеням разности $x-x_0$:
\begin{equation*}
    y(x) = y(x_0) + \frac{y'(x_0)}{1!} (x-x_0) + \frac{y''(x_0)}{2!} (x-x_0)^2 + \dots + \frac{y^{(n)}(x_0)}{n!} (x-x_0)^n + \dots
\end{equation*}

\qquad Начальные условия (13.6) непосредственно дают нам значения $y^{(k)}(x_0)$ при $(k=0,1,2,\dots, n-1)$. Значение $y^{(n)}(x_0)$ найдем из уравнения (13.5), подставляя $x=x_0$ и используя начальные условия (13.6):
\begin{equation*}
    y^{(n)}(x_0) = f(x_0, y_0, y'_0, \dots, y^{(n-1)}_0).
\end{equation*}

Значения $y^{(n+1)}(x_0)$, $y^{(n+2)}(x_0)$, $\dots$ последовательно определяются дифференцированием [3,10] уравнения (13.5) и подстановкой $x=x_0$, $y^{(k)}(x_0) = y_{0k}$ $(k=0,1,2,\dots)$.

\item {\large\textbf{Метод Пикара}} \par

\qquad Рассмотрим задачу Коши для дифференциального уравнения первого порядка (14.7):
\begin{equation}\tag{13.7}
    y' = f(x,y)
\end{equation}
с начальным условием (13.8):
\begin{equation}\tag{13.8}
    y(x_0) = y_0.
\end{equation}

\qquad Метод последовательных приближений [3,10] состоит в том, что решение $y(x)$ получают как предел последовательности функций $y_n(x)$, которые находятся по рекуррентной формуле (13.9):
\begin{equation}\tag{13.9}
    y_n(x) = y_0 + \int\limits_{x_0}^x f(x, y_{n-1}(x)) dx.
\end{equation}
Для системы дифференциальных уравнений метод последовательных приближений применяется аналогично, последовательно приближая каждое из уравнений. \par
\qquad В качестве начального приближения $y_0(x)$ можно взять любую функцию, достаточно близкую к точному решению. Иногда, например, выгодно в качестве $y_0(x)$ брать приближенное решение уравнения (13.7), полученное в виде частичной суммы степенного ряда. \par
\qquad На каждой итерации интегрирование выполняется либо точно, либо приближенно с помощью какой-либо квадратурной формулы. Процесс вычисления приближений выполняем до тех пор, пока разность между итерациями не будет меньше заданной точности.
\end{enumerate}

\item {\large\textbf{Результаты численных расчётов}} \par
\begin{enumerate}
    \item Дано уравнение Эйлера с начальными условиями:
    \begin{equation*}
        \begin{cases}
            x^2y''+0.6y'x+(0.08+0.02N)y=0, \\
            y(2) = 2,  \\
            y'(2) = 4.
        \end{cases}.
    \end{equation*}
    Для получения его точного решения сделаем замену (13.10):
    \begin{equation*}\tag{13.10}
        \begin{gathered}
            x = e^t \ \to \ dx=e^t dt \\
            y' = \frac{dy}{dx} = \frac{dy}{e^t dt} = e^{-t} \frac{dy}{dt} = e^{-t} y'_t  \\
            y'' = e^{-2t} (y''_{tt} - y'_t)
        \end{gathered}
    \end{equation*}

Подставив эту замену в заданное уравнение Эйлера, составим характеристическое уравнение:
\begin{equation*}
    \lambda^2 - 0.4 \lambda + (0.08+0.02N) = 0.
\end{equation*}
    Корнями этого уравнения являются корни:
    \begin{equation*}
        \lambda_{1,2} = 0.2 \pm i \sqrt{0.04+0.02N}.
    \end{equation*}
\qquad Тогда общее решение этого уравнения будет представлено в виде:
\begin{equation*}
    y(x) = e^{0.2t} (C_1 \cos{(t\sqrt{0.04+0.02N})} + C_2 \sin{(t\sqrt{0.04+0.02N})}).
\end{equation*}
Далее необходимо сделать обратную замену $t = \ln{(x)}$, после чего получим:
\begin{equation*}
    y(x) = x^{0.2} (C_1 \cos{(\sqrt{0.04+0.02N}\ln{(x)})} + C_2 \sin{(\sqrt{0.04+0.02N}\ln{(x)})}).
\end{equation*}
\qquad Подставив это уравнение в начальные условия, найдем коэффициенты $c_1$ и $c_2$:
\begin{equation*}
\begin{gathered}   
    C_{1}=\frac{2\sqrt{2N+4}\cos\left(\sqrt{0.04+0.02N}\ln\left(2\right)\right)-76\sin\left(\sqrt{0.04+0.02N}\ln\left(2\right)\right)}{\sqrt[5]{2}\sqrt{2N+4}},  \\
    C_{2}=\frac{2\sqrt{2N+4}\sin\left(\sqrt{0.04+0.02N}\ln\left(2\right)\right)+76\cos\left(\sqrt{0.04+0.02N}\ln\left(2\right)\right)}{\sqrt[5]{2}\sqrt{2N+4}}. 
\end{gathered}
\end{equation*}
Методом неопределенных коэффициентов вычисляются первые 6 членов разложения в ряд Тейлора
приближенного решения с учётом $w=0.08+0.02N$:
\begin{equation*}
\begin{gathered}
	c_{0}=2 \\
	c_{1}=4 \\
	c_{2}=-\frac{12+5w}{20} \\
	c_{3}=\frac{96-35w}{600} \\
	c_{4}=\frac{-1248+1505w+125w^{2}}{24000} \\
	c_{5}=\frac{11232-22620w-1625w^{2}}{600000}
\end{gathered}
\end{equation*}

На отрезке $[2, 2.1]$ функция $y(x)$ приблизительно равна:
\begin{equation*}
y\left(x\right) \approx c_{0}+c_{1}x+c_{2}x^{2}+c_{3}x^{3}+c_{4}x^{4}+c_{5}x^{5}
\end{equation*}

Результаты представлены в таблице 1.

Таблица 1 — Точное и приближенное решения, абсолютная и относительная погрешности.
\begin{table}[H]
    \centering
    \resizebox{\textwidth}{!}{%
    \begin{tabular}{|c|c|c|c|c|}
     \hline
     $x$ & Точное решение & Приближенное решение & Абсолютная погрешность & Относительная погрешность   \\ \hline
    2.00  &  2.00 & 2.00 & 0.0 & 0.0 \\ \hline
    2.01  &    2.039938  &    2.039969  &$3.112196\cdot 10^{-5}$       &$1.525633\cdot 10^{-5}$\\ \hline
    2.02  &    2.079751  &    2.079875  &$1.239791\cdot 10^{-4}$       &$5.961250\cdot 10^{-5}$\\ \hline
    2.03  &    2.119442  &    2.119719  &$2.778163\cdot 10^{-4}$       &$1.310799\cdot 10^{-4}$\\ \hline
    2.04  &    2.159010  &    2.159502  &$4.918883\cdot 10^{-4}$       &$2.278305\cdot 10^{-4}$\\ \hline
    2.05  &    2.198456  &    2.199222  &$7.654600\cdot 10^{-4}$       &$3.481806\cdot 10^{-4}$\\ \hline
    2.06  &    2.237783  &    2.238881  &$1.097806\cdot 10^{-3}$       &$4.905775\cdot 10^{-4}$\\ \hline
    2.07  &    2.276989  &    2.278478  &$1.488210\cdot 10^{-3}$       &$6.535870\cdot 10^{-4}$\\ \hline
    2.08  &    2.316077  &    2.318013  &$1.935968\cdot 10^{-3}$       &$8.358823\cdot 10^{-4}$\\ \hline
    2.09  &    2.355047  &    2.357487  &$2.440381\cdot 10^{-3}$       &$1.036234\cdot 10^{-3}$\\ \hline
    2.10  &    2.393900  &    2.396901  &$3.000762\cdot 10^{-3}$       &$1.253503\cdot 10^{-3}$\\ \hline
    \end{tabular}}
\end{table} \par

\begin{figure}[H]
    \centering
    \includegraphics[width=0.9\linewidth]{../pics/task1.png}\par
    Рисунок 1 — График сравнения точного решения и приближенного решения, полученного методом неопределенных коэффициентов.\\
    $1$ — точное решение, $2$ — приближенное решение.
\end{figure}\par


\item Дано дифференциальное уравнение:
\begin{dmath*}
 \left[ (3a+2)x^3-(a+3)x^2+(a+3)x+(2a-1)\right]y''+
\\+\left[6(3a+2)x^2-4(a+3)x+2(a+3)\right]y'+
\\+\left[6(3a+2)x-2(a+3)\right]y=0
\end{dmath*}
с начальными условиями $y(2)=\frac{1}{3}$, \ $y'(2)=-\frac{1}{3}$.

Для упрощения решения уравнения вводится следующая замена:
\begin{equation*}
    t(x) = (3a+2)x^3-(a+3)x^2+(a+3)x+(2a-1),
\end{equation*}
и тогда данное дифференциальное уравнение примет вид:
\begin{equation*}
    t y'' + 2 t' y' + t'' y = 0
\end{equation*}
или
\begin{equation*}
    (ty)'' = 0,
\end{equation*}
Тогда общее решение этого уравнения имеет вид:
\begin{equation*}
    y(x) = \frac{c_1 x + c_2}{t(x)}.
\end{equation*}

Подставляя это уравнение в начальные условия, найдем коэффициенты $c_1$, $c_2$:
\begin{equation*}
\begin{gathered}
    c_1 = 3a + 2,  \\
    c_2 =2a - 1.
\end{gathered}
\end{equation*}
\qquad Вычислим значения функции на $[2,2.1]$ с шагом $0.01$ приближенно и точно и сравним их в одних и тех же точках. \par

\qquad Введённые обозначения:
\begin{equation*}
\begin{gathered}
	w=18a+12;\ \ \ \ t_{0}=24a+9;\ \ \ \ t_{1}=33a+15;\ \ \ \ t_{2}=34a+18
\end{gathered}
\end{equation*}
Первые 6 членов приближенного решения найдем методом последовательного дифференцирования:\par
\begin{equation*}
\begin{gathered}
	y_{0}=\frac{1}{3};\ \ \ \ y_{1}=-\frac{1}{3};\ \ \ \ y_{2}=-\frac{2t_{1}^{2}+t_{2}y_{0}}{t_{0}} \\
	y_{3}=\frac{y_{1}\left(3t_{1}t_{2}-wt_{0}\right)+3y_{0}t_{1}\left(2t_{1}-t_{0}\right)}{t_{0}^{2}} \\
	y_{4}=\frac{y_{0}\left(4wt_{0}t_{1}+6t_{0}t_{2}^{2}-12t_{1}^{2}t_{2}\right)+y_{1}\left(24t_{0}t_{1}t_{2}-4wt_{0}^{2}-24t_{1}^{3}\right)}{t_{0}^{3}} \\
	y_{5}=\frac{20\left(wt_{0}-3t_{1}t_{2}\right)\left(t_{0}t_{2}-t_{1}^{2}\right)y_{0}+10\left(t_{0}^{2}\left(4wt_{1}+3t_{2}^{2}\right)+6t_{1}^{2}\left(2t_{1}^{2}-3t_{0}t_{2}\right)\right)y_{1}}{t_{0}^{4}} \\
\end{gathered}
\end{equation*}
На отрезке $[2, 2.1]$ функция $y(x)$ приблизительно равна:
\begin{equation*}
y\left(x\right) \approx y_{0}+y_{1}\left(x-2\right)+\frac{y_{2}}{2}\left(x-2\right)^{2}+\frac{y_{3}}{6}\left(x-2\right)^{3}+\frac{y_{4}}{24}\left(x-2\right)^{4}+\frac{y_{5}}{120}\left(x-2\right)^{5}
\end{equation*}

Результаты представлены в таблице 2.

Таблица 2 — Точное и приближенное решения, абсолютная и относительная погрешности.
\begin{table}[H]
    \centering
    \resizebox{\textwidth}{!}{%
    \begin{tabular}{|c|c|c|c|c|}
     \hline
     $x$ & Точное решение & Приближенное решение & Абсолютная погрешность & Относительная погрешность   \\ \hline
    2.00  &    0.333333  &    0.333333  & 0.0        & 0.0   \\ \hline
    2.01  &    0.330022  &    0.330011  &$1.110579\cdot 10^{-5}$       &$3.365168\cdot 10^{-5}$  \\ \hline
    2.02  &    0.326755  &    0.326710  &$4.442319\cdot 10^{-5}$       &$1.359527\cdot 10^{-4}$  \\ \hline
    2.03  &    0.323530  &    0.323430  &$9.995219\cdot 10^{-5}$       &$3.089422\cdot 10^{-4}$  \\ \hline
    2.04  &    0.320349  &    0.320171  &$1.776927\cdot 10^{-4}$       &$5.546856\cdot 10^{-4}$  \\ \hline
    2.05  &    0.317209  &    0.316931  &$2.776448\cdot 10^{-4}$       &$8.752752\cdot 10^{-4}$  \\ \hline
    2.06  &    0.314110  &    0.313710  &$3.998082\cdot 10^{-4}$       &$1.272829\cdot 10^{-3}$  \\ \hline
    2.07  &    0.311052  &    0.310507  &$5.441828\cdot 10^{-4}$       &$1.749493\cdot 10^{-3}$  \\ \hline
    2.08  &    0.308034  &    0.307323  &$7.107682\cdot 10^{-4}$       &$2.307437\cdot 10^{-3}$  \\ \hline
    2.09  &    0.305055  &    0.304155  &$8.995637\cdot 10^{-4}$       &$2.948860\cdot 10^{-3}$  \\ \hline
    2.10  &    0.302115  &    0.301004  &$1.110568\cdot 10^{-3}$       &$3.675982\cdot 10^{-3}$  \\ \hline
    \end{tabular}}
\end{table} \par


\begin{figure}[H]
    \centering
    \includegraphics[width=0.9\linewidth]{../pics/task2.png}\par
    Рисунок 2 — График сравнения точного решения и приближенного решения, полученного методом степенных рядов.\\
    $1$ — точное решение, $2$ — приближенное решение.
\end{figure}\par


\item Дана система дифференциальных уравнений:
\begin{equation*} 
 \begin{cases}
   y'_1=\frac{y_2-\frac{3}{2}\sqrt{x}}{x},\\
  y'_2=\frac{y_3+2\sqrt{x}}{x},\\
 y'_3=\frac{y_1-\frac{3}{2}\sqrt{x}}{x}.
 \end{cases}
\end{equation*}

с начальными условиями:
\begin{equation*}
y_1(0.64)=0.64a+0.8,y_2(0.64)=0.64a+1.6,y_3(0.64)=0.64a-0.8, a=-1.0+0.001N.
\end{equation*}
\qquad Для того, чтоб найти точное решение этой системы, выразим из первого уравнения $y_2(x)$, посчитаем $y'_2(x)$, подставим его во второе уравнение, выразим $y_3(x)$, посчитаем $y'_3(x)$, подставим его в третье уравнение и получим выражение:
\begin{equation*}
    y_1(x) = x^3 y'''_1 + 3 x^2 y''_1 +\frac{7}{8} \sqrt{x},
\end{equation*}
решением которого является:
\begin{equation*}
    y_1(x) = \frac{c_2 \sin{(\frac{\sqrt{3} \ln{(x)}}{2})} - c_1 \cos{(\frac{\sqrt{3} \ln{(x)}}{2})}}{\sqrt{cx}} + c_3 x + \sqrt{x}.
\end{equation*}
Подставляя это решение в начальные условия, посчитаем общее решение:
\begin{equation*}
    y_1(x) = \sqrt{x} - a x.
\end{equation*}
Последовательно подставляя это решение и начальные условия в остальные уравнения, получим итоговую систему решений:
\begin{equation*}
    \begin{cases}
        y_1(x) =  \sqrt{x} + a x \\
        y_2(x) = 2\sqrt{x} + a x \\
        y_3(x) = -\sqrt{x} + a x 
    \end{cases}.
\end{equation*}

\qquad Введённые обозначения:
\begin{equation*}
\begin{gathered}
w_{1}=0.64a+0.8;\ \ \ \ w_{2}=0.64a+1.6;\ \ \ \ w_{3}=0.64a-0.8 \\
s=0.64;\ \ \ \ l=\ln\left(\frac{x}{s}\right);\ \ \ \ x_{s}=\sqrt{x}-\sqrt{s};\ \ \ \ l_{x}=\ln\left(x\right);\ \ \ \ l_{s}=\ln\left(s\right) \\
\end{gathered}
\end{equation*}

\qquad Первые приближения решения методом Пикара:
\begin{equation*}
\begin{gathered}
	y_{1}^{0}=w_{1}+w_{2}l-3x_{s}  \\
	y_{2}^{0}=w_{2}+w_{3}l+4x_{s}  \\
	y_{3}^{0}=w_{3}+w_{1}l-3x_{s}
\end{gathered}
\end{equation*}

\qquad Вторые приближения решения методом Пикара:
\begin{equation*}
\begin{gathered}
	y_{1}^{1}=w_{1}+lw_{2}+\frac{1}{2}l^{2}w_{3}-4l\sqrt{s}+5x_{s} \\
	y_{2}^{1}=\frac{1}{2}l^{2}w_{1}+w_{2}+lw_{3}+3l\sqrt{s}-2x_{s} \\
	y_{3}^{1}=lw_{1}+\frac{1}{2}l^{2}w_{2}+w_{3}+3l\sqrt{s}-9x_{s}
\end{gathered}
\end{equation*}

\qquad Третьи приближения решения методом Пикара:
\begin{equation*}
\begin{gathered}
	y_{1}^{2}=\left(\frac{1}{6}l_{x}^{3}-\frac{1}{2}l_{s}l_{x}^{2}+\frac{1}{2}l_{s}^{2}l_{x}-\frac{1}{6}l_{s}^{3}+1\right)w_{1}+\left(l_{x}-l_{s}\right)w_{2}+\left(\frac{1}{2}l_{x}^{2}-l_{s}l_{x}+\frac{1}{2}l_{s}^{2}\right)w_{3}+ \\
	+ \left(\frac{3}{2}l_{x}^{2}+\left(2-3l_{s}\right)l_{x}+\frac{3}{2}l_{s}^{2}-2l_{s}+7\right)\sqrt{s}-7\sqrt{x} \\
	y_{2}^{2}=\frac{1}{2}l^{2}w_{1}+\left(\frac{1}{6}l_{x}^{3}-\frac{1}{2}l_{s}l_{x}^{2}+\frac{1}{3}l_{s}^{3}+\frac{1}{2}ll_{s}^{2}+1\right)w_{2}+lw_{3}+ \\
	+\left(\frac{3}{2}l^{2}+9l+14\right)\sqrt{s}-14\sqrt{x} \\
	y_{3}^{2}=lw_{1}+\frac{1}{2}l^{2}w_{2}\left(\frac{1}{6}l_{x}^{3}-\frac{1}{2}l_{s}l_{x}^{2}+\frac{1}{3}l_{s}^{3}+\frac{1}{2}ll_{s}^{2}+1\right)w_{3}-\\
	-\left(2l^{2}+5l+7\right)\sqrt{s}+7\sqrt{x}
\end{gathered}
\end{equation*}

\end{enumerate}

Таблица 3 — Точное и приближенное решения для первого приближения $y_1$, абсолютная и относительная погрешности при шаге 0.01.
\begin{table}[H]
    \centering
    \resizebox{\textwidth}{!}{%
    \begin{tabular}{|c|c|c|c|c|}
     \hline
     $x$ & Точное решение & Приближенное решение & Абсолютная погрешность & Относительная погрешность   \\ \hline
    0.64  &    0.160640  &    0.160640  & 0.0      & 0.0 \\ \hline
    0.65  &    0.156876  &    0.156857  &$1.915756\cdot 10^{-5}$       &$1.221193\cdot 10^{-4}$\\ \hline
    0.66  &    0.153064  &    0.152989  &$7.487567\cdot 10^{-5}$       &$4.891793\cdot 10^{-4}$\\ \hline
    0.67  &    0.149205  &    0.149041  &$1.646360\cdot 10^{-4}$       &$1.103419\cdot 10^{-3}$\\ \hline
    0.68  &    0.145301  &    0.145015  &$2.860637\cdot 10^{-4}$       &$1.968765\cdot 10^{-3}$\\ \hline
    0.69  &    0.141352  &    0.140915  &$4.369177\cdot 10^{-4}$       &$3.090982\cdot 10^{-3}$\\ \hline
    0.70  &    0.137360  &    0.136745  &$6.150820\cdot 10^{-4}$       &$4.477882\cdot 10^{-3}$\\ \hline
    0.71  &    0.133325  &    0.132506  &$8.185573\cdot 10^{-4}$       &$6.139565\cdot 10^{-3}$\\ \hline
    0.72  &    0.129248  &    0.128203  &$1.045454\cdot 10^{-3}$       &$8.088738\cdot 10^{-3}$\\ \hline
    0.73  &    0.125130  &    0.123836  &$1.293985\cdot 10^{-3}$       &$1.034110\cdot 10^{-2}$\\ \hline
    0.74  &    0.120973  &    0.119410  &$1.562460\cdot 10^{-3}$       &$1.291583\cdot 10^{-2}$\\ \hline
    0.75  &    0.116775  &    0.114926  &$1.849278\cdot 10^{-3}$       &$1.583620\cdot 10^{-2}$\\ \hline
    0.76  &    0.112540  &    0.110387  &$2.152924\cdot 10^{-3}$       &$1.913033\cdot 10^{-2}$\\ \hline
    0.77  &    0.108266  &    0.105794  &$2.471959\cdot 10^{-3}$       &$2.283218\cdot 10^{-2}$\\ \hline
    0.78  &    0.103956  &    0.101151  &$2.805024\cdot 10^{-3}$       &$2.698278\cdot 10^{-2}$\\ \hline
    0.79  &    0.099609  &    0.096459  &$3.150827\cdot 10^{-3}$       &$3.163181\cdot 10^{-2}$\\ \hline
    0.80  &    0.095227  &    0.091719  &$3.508142\cdot 10^{-3}$       &$3.683971\cdot 10^{-2}$\\ \hline
    0.81  &    0.090810  &    0.086934  &$3.875809\cdot 10^{-3}$       &$4.268042\cdot 10^{-2}$\\ \hline
    \end{tabular}}
\end{table} \par


\begin{figure}[H]
    \centering
    \includegraphics[width=0.9\linewidth]{../pics/task3_y1_0.png}\par
    Рисунок 3 — График сравнения точного решения и первого приближения $y_1$.\\
    $1$ — точное решение, $2$ — приближенное решение.
\end{figure}\par

Таблица 4 — Точное и приближенное решения для первого приближения $y_2$, абсолютная и относительная погрешности при шаге 0.01.
\begin{table}[H]
    \centering
    \resizebox{\textwidth}{!}{%
    \begin{tabular}{|c|c|c|c|c|}
     \hline
     $x$ & Точное решение & Приближенное решение & Абсолютная погрешность & Относительная погрешность   \\ \hline
        0.64  &    0.960640  &    0.960640  & 0.0        & 0.0   \\ \hline
        0.65  &    0.963102  &    0.963227  &$1.254437\cdot 10^{-4}$       &$1.302497\cdot 10^{-4}$  \\ \hline
        0.66  &    0.965468  &    0.965964  &$4.961863\cdot 10^{-4}$       &$5.139336\cdot 10^{-4}$  \\ \hline
        0.67  &    0.967741  &    0.968845  &$1.104140\cdot 10^{-3}$       &$1.140946\cdot 10^{-3}$  \\ \hline
        0.68  &    0.969922  &    0.971864  &$1.941594\cdot 10^{-3}$       &$2.001804\cdot 10^{-3}$  \\ \hline
        0.69  &    0.972015  &    0.975016  &$3.001188\cdot 10^{-3}$       &$3.087596\cdot 10^{-3}$  \\ \hline
        0.70  &    0.974020  &    0.978296  &$4.275896\cdot 10^{-3}$       &$4.389946\cdot 10^{-3}$  \\ \hline
        0.71  &    0.975940  &    0.981699  &$5.759001\cdot 10^{-3}$       &$5.900979\cdot 10^{-3}$  \\ \hline
        0.72  &    0.977776  &    0.985220  &$7.444084\cdot 10^{-3}$       &$7.613280\cdot 10^{-3}$  \\ \hline
        0.73  &    0.979531  &    0.988856  &$9.325002\cdot 10^{-3}$       &$9.519867\cdot 10^{-3}$  \\ \hline
        0.74  &    0.981205  &    0.992601  &$1.139587\cdot 10^{-2}$       &$1.161416\cdot 10^{-2}$  \\ \hline
        0.75  &    0.982801  &    0.996452  &$1.365107\cdot 10^{-2}$       &$1.388996\cdot 10^{-2}$  \\ \hline
        0.76  &    0.984320  &    1.000405  &$1.608519\cdot 10^{-2}$       &$1.634143\cdot 10^{-2}$  \\ \hline
        0.77  &    0.985763  &    1.004456  &$1.869306\cdot 10^{-2}$       &$1.896303\cdot 10^{-2}$  \\ \hline
        0.78  &    0.987132  &    1.008602  &$2.146971\cdot 10^{-2}$       &$2.174958\cdot 10^{-2}$  \\ \hline
        0.79  &    0.988429  &    1.012839  &$2.441037\cdot 10^{-2}$       &$2.469613\cdot 10^{-2}$  \\ \hline
        0.80  &    0.989654  &    1.017165  &$2.751047\cdot 10^{-2}$       &$2.779806\cdot 10^{-2}$  \\ \hline
        0.81  &    0.990810  &    1.021576  &$3.076561\cdot 10^{-2}$       &$3.105097\cdot 10^{-2}$  \\ \hline
    \end{tabular}}
\end{table} \par

\begin{figure}[H]
    \centering
    \includegraphics[width=0.9\linewidth]{../pics/task3_y2_0.png}\par
    Рисунок 4 — График сравнения точного решения и первого приближения $y_2$.\\
    $1$ — точное решение, $2$ — приближенное решение.
\end{figure}\par


Таблица 5 — Точное и приближенное решения для первого приближения $y_3$, абсолютная и относительная погрешности при шаге 0.01.
\begin{table}[H]
    \centering
    \resizebox{\textwidth}{!}{%
    \begin{tabular}{|c|c|c|c|c|}
     \hline
     $x$ & Точное решение & Приближенное решение & Абсолютная погрешность & Относительная погрешность   \\ \hline
        0.64 &    -1.439360 &    -1.439360 & 0.0 & 0.0  \\ \hline
        0.65 &    -1.455576 &    -1.455547 & $2.904286\cdot 10^{-5}$      & $1.995283\cdot 10^{-5}$  \\ \hline
        0.66 &    -1.471744 &    -1.471628 & $1.154783\cdot 10^{-4}$      & $7.846360\cdot 10^{-5}$  \\ \hline
        0.67 &    -1.487865 &    -1.487607 & $2.582894\cdot 10^{-4}$      & $1.735973\cdot 10^{-4}$  \\ \hline
        0.68 &    -1.503941 &    -1.503485 & $4.564890\cdot 10^{-4}$      & $3.035285\cdot 10^{-4}$  \\ \hline
        0.69 &    -1.519972 &    -1.519263 & $7.091178\cdot 10^{-4}$      & $4.665333\cdot 10^{-4}$  \\ \hline
        0.70 &    -1.535960 &    -1.534945 & $1.015244\cdot 10^{-3}$      & $6.609834\cdot 10^{-4}$  \\ \hline
        0.71 &    -1.551905 &    -1.550531 & $1.373962\cdot 10^{-3}$      & $8.853391\cdot 10^{-4}$  \\ \hline
        0.72 &    -1.567808 &    -1.566024 & $1.784392\cdot 10^{-3}$      & $1.138144\cdot 10^{-3}$  \\ \hline
        0.73 &    -1.583670 &    -1.581425 & $2.245677\cdot 10^{-3}$      & $1.418020\cdot 10^{-3}$  \\ \hline
        0.74 &    -1.599493 &    -1.596736 & $2.756984\cdot 10^{-3}$      & $1.723662\cdot 10^{-3}$  \\ \hline
        0.75 &    -1.615275 &    -1.611958 & $3.317504\cdot 10^{-3}$      & $2.053832\cdot 10^{-3}$  \\ \hline
        0.76 &    -1.631020 &    -1.627093 & $3.926447\cdot 10^{-3}$      & $2.407357\cdot 10^{-3}$  \\ \hline
        0.77 &    -1.646726 &    -1.642143 & $4.583046\cdot 10^{-3}$      & $2.783125\cdot 10^{-3}$  \\ \hline
        0.78 &    -1.662396 &    -1.657110 & $5.286554\cdot 10^{-3}$      & $3.180080\cdot 10^{-3}$  \\ \hline
        0.79 &    -1.678029 &    -1.671993 & $6.036241\cdot 10^{-3}$      & $3.597219\cdot 10^{-3}$  \\ \hline
        0.80 &    -1.693627 &    -1.686796 & $6.831398\cdot 10^{-3}$      & $4.033590\cdot 10^{-3}$  \\ \hline
        0.81 &    -1.709190 &    -1.701519 & $7.671333\cdot 10^{-3}$      & $4.488286\cdot 10^{-3}$  \\ \hline
    \end{tabular}}
\end{table} \par


\begin{figure}[H]
    \centering
    \includegraphics[width=0.9\linewidth]{../pics/task3_y3_0.png}\par
    Рисунок 5 — График сравнения точного решения и первого приближения $y_3$.\\
    $1$ — точное решение, $2$ — приближенное решение.
\end{figure}\par


Таблица 6 — Точное и приближенное решения для второго приближения $y_1$, абсолютная и относительная погрешности при шаге 0.01.
\begin{table}[H]
    \centering
    \resizebox{\textwidth}{!}{%
    \begin{tabular}{|c|c|c|c|c|}
     \hline
     $x$ & Точное решение & Приближенное решение & Абсолютная погрешность & Относительная погрешность   \\ \hline
        0.64  &    0.160640  &    0.160640  & 0.0      & 0.0 \\ \hline
        0.65  &    0.156876  &    0.156876  &$6.476236\cdot 10^{-7}$       &$4.128258\cdot 10^{-6}$ \\ \hline
        0.66  &    0.153064  &    0.153069  &$5.078925\cdot 10^{-6}$       &$3.318174\cdot 10^{-5}$ \\ \hline
        0.67  &    0.149205  &    0.149222  &$1.680788\cdot 10^{-5}$       &$1.126493\cdot 10^{-4}$ \\ \hline
        0.68  &    0.145301  &    0.145340  &$3.907525\cdot 10^{-5}$       &$2.689260\cdot 10^{-4}$ \\ \hline
        0.69  &    0.141352  &    0.141427  &$7.486989\cdot 10^{-5}$       &$5.296683\cdot 10^{-4}$ \\ \hline
        0.70  &    0.137360  &    0.137487  &$1.269480\cdot 10^{-4}$       &$9.241996\cdot 10^{-4}$ \\ \hline
        0.71  &    0.133325  &    0.133523  &$1.978514\cdot 10^{-4}$       &$1.483978\cdot 10^{-3}$ \\ \hline
        0.72  &    0.129248  &    0.129538  &$2.899229\cdot 10^{-4}$       &$2.243150\cdot 10^{-3}$ \\ \hline
        0.73  &    0.125130  &    0.125536  &$4.053221\cdot 10^{-4}$       &$3.239199\cdot 10^{-3}$ \\ \hline
        0.74  &    0.120973  &    0.121519  &$5.460386\cdot 10^{-4}$       &$4.513740\cdot 10^{-3}$ \\ \hline
        0.75  &    0.116775  &    0.117489  &$7.139045\cdot 10^{-4}$       &$6.113483\cdot 10^{-3}$ \\ \hline
        0.76  &    0.112540  &    0.113450  &$9.106061\cdot 10^{-4}$       &$8.091414\cdot 10^{-3}$ \\ \hline
        0.77  &    0.108266  &    0.109404  &$1.137694\cdot 10^{-3}$       &$1.050828\cdot 10^{-2}$ \\ \hline
        0.78  &    0.103956  &    0.105353  &$1.396595\cdot 10^{-3}$       &$1.343447\cdot 10^{-2}$ \\ \hline
        0.79  &    0.099609  &    0.101298  &$1.688617\cdot 10^{-3}$       &$1.695238\cdot 10^{-2}$ \\ \hline
        0.80  &    0.095227  &    0.097242  &$2.014962\cdot 10^{-3}$       &$2.115953\cdot 10^{-2}$ \\ \hline
        0.81  &    0.090810  &    0.093187  &$2.376730\cdot 10^{-3}$       &$2.617256\cdot 10^{-2}$ \\ \hline

    \end{tabular}}
\end{table} \par


\begin{figure}[H]
    \centering
    \includegraphics[width=0.9\linewidth]{../pics/task3_y1_1.png}\par
    Рисунок 6 — График сравнения точного решения и второго приближения $y_1$.\\
    $1$ — точное решение, $2$ — приближенное решение.
\end{figure}\par

\begin{figure}[H]
    \centering
    \includegraphics[width=0.9\linewidth]{../pics/task3_y1_1_reltol.png}\par
    Рисунок 7 — График относительной погрешности для второго приближения $y_1$.\\
    $1$ — относительная погрешность.
\end{figure}\par

\newpage
Таблица 7 — Точное и приближенное решения для второго приближения $y_2$, абсолютная и относительная погрешности при шаге 0.01.
\begin{table}[H]
    \centering
    \resizebox{\textwidth}{!}{%
    \begin{tabular}{|c|c|c|c|c|}
     \hline
     $x$ & Точное решение & Приближенное решение & Абсолютная погрешность & Относительная погрешность   \\ \hline
        0.64  &    0.960640  &    0.960640  & 0.0        & 0.0   \\ \hline
        0.65  &    0.963102  &    0.963102  &$1.497400\cdot 10^{-7}$       &$1.554768\cdot 10^{-7}$  \\ \hline
        0.66  &    0.965468  &    0.965469  &$1.178931\cdot 10^{-6}$       &$1.221099\cdot 10^{-6}$  \\ \hline
        0.67  &    0.967741  &    0.967744  &$3.916555\cdot 10^{-6}$       &$4.047113\cdot 10^{-6}$  \\ \hline
        0.68  &    0.969922  &    0.969931  &$9.139899\cdot 10^{-6}$       &$9.423332\cdot 10^{-6}$  \\ \hline
        0.69  &    0.972015  &    0.972032  &$1.757799\cdot 10^{-5}$       &$1.808408\cdot 10^{-5}$  \\ \hline
        0.70  &    0.974020  &    0.974050  &$2.991481\cdot 10^{-5}$       &$3.071272\cdot 10^{-5}$  \\ \hline
        0.71  &    0.975940  &    0.975987  &$4.679216\cdot 10^{-5}$       &$4.794574\cdot 10^{-5}$  \\ \hline
        0.72  &    0.977776  &    0.977845  &$6.881246\cdot 10^{-5}$       &$7.037649\cdot 10^{-5}$  \\ \hline
        0.73  &    0.979531  &    0.979627  &$9.654120\cdot 10^{-5}$       &$9.855862\cdot 10^{-5}$  \\ \hline
        0.74  &    0.981205  &    0.981336  &$1.305092\cdot 10^{-4}$       &$1.330092\cdot 10^{-4}$  \\ \hline
        0.75  &    0.982801  &    0.982972  &$1.712152\cdot 10^{-4}$       &$1.742115\cdot 10^{-4}$  \\ \hline
        0.76  &    0.984320  &    0.984539  &$2.191272\cdot 10^{-4}$       &$2.226179\cdot 10^{-4}$  \\ \hline
        0.77  &    0.985763  &    0.986038  &$2.746848\cdot 10^{-4}$       &$2.786520\cdot 10^{-4}$  \\ \hline
        0.78  &    0.987132  &    0.987470  &$3.383007\cdot 10^{-4}$       &$3.427106\cdot 10^{-4}$  \\ \hline
        0.79  &    0.988429  &    0.988839  &$4.103626\cdot 10^{-4}$       &$4.151665\cdot 10^{-4}$  \\ \hline
        0.80  &    0.989654  &    0.990146  &$4.912344\cdot 10^{-4}$       &$4.963697\cdot 10^{-4}$  \\ \hline
        0.81  &    0.990810  &    0.991391  &$5.812579\cdot 10^{-4}$       &$5.866492\cdot 10^{-4}$  \\ \hline
    \end{tabular}}
\end{table} \par


\begin{figure}[H]
    \centering
    \includegraphics[width=0.9\linewidth]{../pics/task3_y2_1.png}\par
    Рисунок 8 — График сравнения точного решения и второго приближения $y_2$.\\
    $1$ — точное решение, $2$ — приближенное решение.
\end{figure}\par

\begin{figure}[H]
    \centering
    \includegraphics[width=0.9\linewidth]{../pics/task3_y2_1_reltol.png}\par
    Рисунок 9 — График относительной погрешности для второго приближения $y_2$.\\
    $1$ — относительная погрешность.
\end{figure}\par


Таблица 8 — Точное и приближенное решения для второго приближения $y_3$, абсолютная и относительная погрешности при шаге 0.01.
\begin{table}[H]
    \centering
    \resizebox{\textwidth}{!}{%
    \begin{tabular}{|c|c|c|c|c|}
     \hline
     $x$ & Точное решение & Приближенное решение & Абсолютная погрешность & Относительная погрешность   \\ \hline
        0.64 &    -1.439360 &    -1.439360 & 0.0 & 0.0   \\ \hline
        0.65 &    -1.455576 &    -1.455576 & $9.920178\cdot 10^{-8}$      & $6.815294\cdot 10^{-8}$  \\ \hline
        0.66 &    -1.471744 &    -1.471745 & $7.710649\cdot 10^{-7}$      & $5.239124\cdot 10^{-7}$  \\ \hline
        0.67 &    -1.487865 &    -1.487868 & $2.529106\cdot 10^{-6}$      & $1.699822\cdot 10^{-6}$  \\ \hline
        0.68 &    -1.503941 &    -1.503947 & $5.827780\cdot 10^{-6}$      & $3.875005\cdot 10^{-6}$  \\ \hline
        0.69 &    -1.519972 &    -1.519983 & $1.106794\cdot 10^{-5}$      & $7.281676\cdot 10^{-6}$  \\ \hline
        0.70 &    -1.535960 &    -1.535979 & $1.860182\cdot 10^{-5}$      & $1.211087\cdot 10^{-5}$  \\ \hline
        0.71 &    -1.551905 &    -1.551934 & $2.873745\cdot 10^{-5}$      & $1.851753\cdot 10^{-5}$  \\ \hline
        0.72 &    -1.567808 &    -1.567850 & $4.174278\cdot 10^{-5}$      & $2.662493\cdot 10^{-5}$  \\ \hline
        0.73 &    -1.583670 &    -1.583728 & $5.784929\cdot 10^{-5}$      & $3.652861\cdot 10^{-5}$  \\ \hline
        0.74 &    -1.599493 &    -1.599570 & $7.725537\cdot 10^{-5}$      & $4.829992\cdot 10^{-5}$  \\ \hline
        0.75 &    -1.615275 &    -1.615376 & $1.001293\cdot 10^{-4}$      & $6.198902\cdot 10^{-5}$  \\ \hline
        0.76 &    -1.631020 &    -1.631146 & $1.266121\cdot 10^{-4}$      & $7.762761\cdot 10^{-5}$  \\ \hline
        0.77 &    -1.646726 &    -1.646883 & $1.568200\cdot 10^{-4}$      & $9.523139\cdot 10^{-5}$  \\ \hline
        0.78 &    -1.662396 &    -1.662587 & $1.908465\cdot 10^{-4}$      & $1.148021\cdot 10^{-4}$  \\ \hline
        0.79 &    -1.678029 &    -1.678258 & $2.287649\cdot 10^{-4}$      & $1.363295\cdot 10^{-4}$  \\ \hline
        0.80 &    -1.693627 &    -1.693898 & $2.706296\cdot 10^{-4}$      & $1.597929\cdot 10^{-4}$  \\ \hline
        0.81 &    -1.709190 &    -1.709506 & $3.164784\cdot 10^{-4}$      & $1.851628\cdot 10^{-4}$  \\ \hline
    \end{tabular}}
\end{table} \par


\begin{figure}[H]
    \centering
    \includegraphics[width=0.9\linewidth]{../pics/task3_y3_1.png}\par
    Рисунок 10 — График сравнения точного решения и второго приближения $y_3$.\\
    $1$ — точное решение, $2$ — приближенное решение.
\end{figure}\par

\begin{figure}[H]
    \centering
    \includegraphics[width=0.9\linewidth]{../pics/task3_y3_1_reltol.png}\par
    Рисунок 11 — График относительной погрешности для второго приближения $y_3$.\\
    $1$ — относительная погрешность.
\end{figure}\par

\newpage
Таблица 9 — Точное и приближенное решения для третьего приближения $y_1$, абсолютная и относительная погрешности при шаге 0.01.
\begin{table}[H]
    \centering
    \resizebox{\textwidth}{!}{%
    \begin{tabular}{|c|c|c|c|c|}
     \hline
     $x$ & Точное решение & Приближенное решение & Абсолютная погрешность & Относительная погрешность   \\ \hline
        0.64  &    0.160640  &    0.160640  & 0.0 & 0.0 \\ \hline
        0.65  &    0.156876  &    0.156876  &$5.795735\cdot 10^{-10}$       &$3.694474\cdot 10^{-9}$ \\ \hline
        0.66  &    0.153064  &    0.153064  &$9.043847\cdot 10^{-9}$       &$5.908545\cdot 10^{-8}$ \\ \hline
        0.67  &    0.149205  &    0.149205  &$4.466574\cdot 10^{-8}$       &$2.993576\cdot 10^{-7}$ \\ \hline
        0.68  &    0.145301  &    0.145301  &$1.377572\cdot 10^{-7}$       &$9.480813\cdot 10^{-7}$ \\ \hline
        0.69  &    0.141352  &    0.141353  &$3.282955\cdot 10^{-7}$       &$2.322532\cdot 10^{-6}$ \\ \hline
        0.70  &    0.137360  &    0.137361  &$6.646951\cdot 10^{-7}$       &$4.839072\cdot 10^{-6}$ \\ \hline
        0.71  &    0.133325  &    0.133326  &$1.202711\cdot 10^{-6}$       &$9.020899\cdot 10^{-6}$ \\ \hline
        0.72  &    0.129248  &    0.129250  &$2.004458\cdot 10^{-6}$       &$1.550860\cdot 10^{-5}$ \\ \hline
        0.73  &    0.125130  &    0.125134  &$3.137533\cdot 10^{-6}$       &$2.507411\cdot 10^{-5}$ \\ \hline
        0.74  &    0.120973  &    0.120977  &$4.674231\cdot 10^{-6}$       &$3.863878\cdot 10^{-5}$ \\ \hline
        0.75  &    0.116775  &    0.116782  &$6.690843\cdot 10^{-6}$       &$5.729668\cdot 10^{-5}$ \\ \hline
        0.76  &    0.112540  &    0.112549  &$9.267028\cdot 10^{-6}$       &$8.234446\cdot 10^{-5}$ \\ \hline
        0.77  &    0.108266  &    0.108279  &$1.248525\cdot 10^{-5}$       &$1.153197\cdot 10^{-4}$ \\ \hline
        0.78  &    0.103956  &    0.103973  &$1.643029\cdot 10^{-5}$       &$1.580503\cdot 10^{-4}$ \\ \hline
        0.79  &    0.099609  &    0.099631  &$2.118877\cdot 10^{-5}$       &$2.127185\cdot 10^{-4}$ \\ \hline
        0.80  &    0.095227  &    0.095254  &$2.684878\cdot 10^{-5}$       &$2.819445\cdot 10^{-4}$ \\ \hline
        0.81  &    0.090810  &    0.090843  &$3.349951\cdot 10^{-5}$       &$3.688967\cdot 10^{-4}$ \\ \hline
    \end{tabular}}
\end{table} \par

\begin{figure}[H]
    \centering
    \includegraphics[width=0.9\linewidth]{../pics/task3_y1_2.png}\par
    Рисунок 12 — График сравнения точного решения и третьего приближения $y_1$.\\
    $1$ — точное решение, $2$ — приближенное решение.
\end{figure}\par

\begin{figure}[H]
    \centering
    \includegraphics[width=0.9\linewidth]{../pics/task3_y1_2_reltol.png}\par
    Рисунок 13 — График относительной погрешности для третьего приближения $y_1$.\\
    $1$ — относительная погрешность
\end{figure}\par


Таблица 10 — Точное и приближенное решения для третьего приближения $y_2$, абсолютная и относительная погрешности при шаге 0.01.
\begin{table}[H]
    \centering
    \resizebox{\textwidth}{!}{%
    \begin{tabular}{|c|c|c|c|c|}
     \hline
     $x$ & Точное решение & Приближенное решение & Абсолютная погрешность & Относительная погрешность   \\ \hline
        0.64  &    0.960640  &    0.960640  & 0.0       & 0.0 \\ \hline
        0.65  &    0.963102  &    0.963102  &$3.849615\cdot 10^{-10}$       &$3.997101\cdot 10^{-10}$ \\ \hline
        0.66  &    0.965468  &    0.965468  &$5.945755\cdot 10^{-9}$       &$6.158420\cdot 10^{-9}$ \\ \hline
        0.67  &    0.967741  &    0.967741  &$2.906775\cdot 10^{-8}$       &$3.003671\cdot 10^{-8}$ \\ \hline
        0.68  &    0.969922  &    0.969922  &$8.875050\cdot 10^{-8}$       &$9.150270\cdot 10^{-8}$ \\ \hline
        0.69  &    0.972015  &    0.972015  &$2.093996\cdot 10^{-7}$       &$2.154285\cdot 10^{-7}$ \\ \hline
        0.70  &    0.974020  &    0.974020  &$4.197808\cdot 10^{-7}$       &$4.309776\cdot 10^{-7}$ \\ \hline
        0.71  &    0.975940  &    0.975939  &$7.521141\cdot 10^{-7}$       &$7.706562\cdot 10^{-7}$ \\ \hline
        0.72  &    0.977776  &    0.977775  &$1.241292\cdot 10^{-6}$       &$1.269505\cdot 10^{-6}$ \\ \hline
        0.73  &    0.979531  &    0.979529  &$1.924204\cdot 10^{-6}$       &$1.964414\cdot 10^{-6}$ \\ \hline
        0.74  &    0.981205  &    0.981202  &$2.839156\cdot 10^{-6}$       &$2.893540\cdot 10^{-6}$ \\ \hline
        0.75  &    0.982801  &    0.982797  &$4.025368\cdot 10^{-6}$       &$4.095813\cdot 10^{-6}$ \\ \hline
        0.76  &    0.984320  &    0.984314  &$5.522550\cdot 10^{-6}$       &$5.610526\cdot 10^{-6}$ \\ \hline
        0.77  &    0.985763  &    0.985756  &$7.370532\cdot 10^{-6}$       &$7.476982\cdot 10^{-6}$ \\ \hline
        0.78  &    0.987132  &    0.987123  &$9.608952\cdot 10^{-6}$       &$9.734210\cdot 10^{-6}$ \\ \hline
        0.79  &    0.988429  &    0.988417  &$1.227699\cdot 10^{-5}$       &$1.242071\cdot 10^{-5}$ \\ \hline
        0.80  &    0.989654  &    0.989639  &$1.541317\cdot 10^{-5}$       &$1.557429\cdot 10^{-5}$ \\ \hline
        0.81  &    0.990810  &    0.990791  &$1.905512\cdot 10^{-5}$       &$1.923186\cdot 10^{-5}$ \\ \hline
    \end{tabular}}
\end{table} \par


\begin{figure}[H]
    \centering
    \includegraphics[width=0.9\linewidth]{../pics/task3_y2_2.png}\par
    Рисунок 14 — График сравнения точного решения и третьего приближения $y_2$.\\
    $1$ — точное решение, $2$ — приближенное решение.
\end{figure}\par

\begin{figure}[H]
    \centering
    \includegraphics[width=0.9\linewidth]{../pics/task3_y2_2_reltol.png}\par
    Рисунок 15 — График относительной погрешности для третьего приближения $y_2$.\\
    $1$ — относительная погрешность.
\end{figure}\par

\newpage
Таблица 11 — Точное и приближенное решения для третьего приближения $y_3$, абсолютная и относительная погрешности при шаге 0.01.
\begin{table}[H]
    \centering
    \resizebox{\textwidth}{!}{%
    \begin{tabular}{|c|c|c|c|c|}
     \hline
     $x$ & Точное решение & Приближенное решение & Абсолютная погрешность & Относительная погрешность   \\ \hline
        0.64 &    -1.439360 &    -1.439360 & 0.0 & 0.0 \\ \hline
        0.65 &    -1.455576 &    -1.455576 & $4.351169\cdot 10^{-9}$      & $2.989311\cdot 10^{-9}$  \\ \hline
        0.66 &    -1.471744 &    -1.471744 & $4.080693\cdot 10^{-8}$      & $2.772693\cdot 10^{-8}$  \\ \hline
        0.67 &    -1.487865 &    -1.487865 & $1.938629\cdot 10^{-7}$      & $1.302960\cdot 10^{-7}$  \\ \hline
        0.68 &    -1.503941 &    -1.503941 & $5.924539\cdot 10^{-7}$      & $3.939342\cdot 10^{-7}$  \\ \hline
        0.69 &    -1.519972 &    -1.519971 & $1.405322\cdot 10^{-6}$      & $9.245709\cdot 10^{-7}$  \\ \hline
        0.70 &    -1.535960 &    -1.535957 & $2.835242\cdot 10^{-6}$      & $1.845909\cdot 10^{-6}$  \\ \hline
        0.71 &    -1.551905 &    -1.551900 & $5.113920\cdot 10^{-6}$      & $3.295253\cdot 10^{-6}$  \\ \hline
        0.72 &    -1.567808 &    -1.567800 & $8.497483\cdot 10^{-6}$      & $5.419976\cdot 10^{-6}$  \\ \hline
        0.73 &    -1.583670 &    -1.583657 & $1.326250\cdot 10^{-5}$      & $8.374534\cdot 10^{-6}$  \\ \hline
        0.74 &    -1.599493 &    -1.599473 & $1.970247\cdot 10^{-5}$      & $1.231795\cdot 10^{-5}$  \\ \hline
        0.75 &    -1.615275 &    -1.615247 & $2.812470\cdot 10^{-5}$      & $1.741171\cdot 10^{-5}$  \\ \hline
        0.76 &    -1.631020 &    -1.630981 & $3.884760\cdot 10^{-5}$      & $2.381798\cdot 10^{-5}$  \\ \hline
        0.77 &    -1.646726 &    -1.646674 & $5.219822\cdot 10^{-5}$      & $3.169817\cdot 10^{-5}$  \\ \hline
        0.78 &    -1.662396 &    -1.662328 & $6.851016\cdot 10^{-5}$      & $4.121169\cdot 10^{-5}$  \\ \hline
        0.79 &    -1.678029 &    -1.677941 & $8.812168\cdot 10^{-5}$      & $5.251498\cdot 10^{-5}$  \\ \hline
        0.80 &    -1.693627 &    -1.693516 & $1.113740\cdot 10^{-4}$      & $6.576067\cdot 10^{-5}$  \\ \hline
        0.81 &    -1.709190 &    -1.709051 & $1.386101\cdot 10^{-4}$      & $8.109697\cdot 10^{-5}$  \\ \hline
    \end{tabular}}
\end{table} \par

\begin{figure}[H]
    \centering
    \includegraphics[width=0.9\linewidth]{../pics/task3_y3_2.png}\par
    Рисунок 16 — График сравнения точного решения и третьего приближения $y_3$.\\
    $1$ — точное решение, $2$ — приближенное решение.
\end{figure}\par

\begin{figure}[H]
    \centering
    \includegraphics[width=0.9\linewidth]{../pics/task3_y3_2_reltol.png}\par
    Рисунок 17 — График относительная погрешность для третьего приближения $y_3$.\\
    $1$ — относительная погрешность.
\end{figure}\par


\item {\large\textbf{Вывод}} \par


\qquad В ходе выполнения лабораторной работы методом неопределенных коэффициентов было посчитано приближенное решение задачи 1.1, методом последовательного дифференцирования приближенное решение задачи 1.2, методом Пикара приближенное решение задачи 1.3. Для всех пунктов аналитически были посчитаны точные решения и проведено сравнение точных решений с приближенными.

\end{enumerate}

\newpage
\begin{center}
\refstepcounter{section} %гиперссылка
\addcontentsline{toc}{section}{Лабораторная работа №14}
\section*{\large Лабораторная работа №14 \\
Интегрирование обыкновенных дифференциальных уравнений.\\Методы Рунге-Кутты, Адамса и Милна. Построение разностных схем интегро-интерполяционным методом.}
\end{center}




\renewcommand{\labelenumi}{\textbf{\arabic{enumi}.}}
\renewcommand{\labelenumii}{\textbf{\arabic{enumi}.\arabic{enumii}}}
\renewcommand{\labelenumiii}{\textbf{\arabic{enumi}.\arabic{enumii}.\arabic{enumiii}}}
\renewcommand{\labelenumiv}{\textbf{\arabic{enumi}.\arabic{enumii}.\arabic{enumiii}.\arabic{enumiv}}}

\begin{enumerate}
\large\item {\large \textbf{Постановка задачи}}
\begin{enumerate}

Для задачи:
\begin{equation*}
y'-\frac{y^2}{4c}=-\frac{x}{4c}-\frac{5c}{4x^2},
\end{equation*}

\begin{equation*}
	y(1)=1+c , c=1+0.001N.
\end{equation*}

\item 	На трехточечном шаблоне с постоянным шагом интегроинтерполяционным методом построить устойчивые разностные схемы (формулы Милна и Адамса). 

\item 	Исследовать полученные схемы на аппроксимацию и сходимость к исходной дифференциальной задаче. 

\item Используя построенные схемы, проинтегрировать заданное дифференциальное уравнение с точностью $\varepsilon=10^{-4}$ на интервале $[1;2]$. Для нахождения "начального отрезка" \ $y_0,y_1,y_2$ использовать метод Рунге-Кутта.

Далее использовать:
\begin{itemize}
\item метод Милна;
\item метод Адамса;
\item метод Рунге-Кутта.
\end{itemize}
\end{enumerate}


\large\item {\large \textbf{Теоретический материал}}
\begin{enumerate}
\item {\large\textbf{Метод Рунге-Кутта}} \par
\qquad Рассмотрим задачу Коши для дифференциального уравнения 
\begin{equation}\tag{14.1}
    y'= f(x,y)
\end{equation}
с начальным условием:
\begin{equation}\tag{14.2}
    y(x_0)= y_0.
\end{equation}
\qquad Обозначим через $y_i$ приближённое значение искомого решения в точке $x_i$. Рунге [3,10,20] предложил приближённое значение $y_{i+1}$ в узле $x_i+h \ (i = 0,1,2,...)$ искать в виде:
\begin{equation}\tag{14.3}
    y_{i+1}=y_i+P_{q1}k_1(h)+P_{q2}k_2(h)+...+P_{qq}k_q(h),
\end{equation}
где 
   \[ k_1(h)=hf(x_i,y_i), \]
   \[ k_2(h)=hf(x_i+\alpha_2h,y_i+\beta_{21}k_1(h)), \]
   \[ \cdots \cdots \cdots \cdots \cdots \cdots \cdots \cdots \cdots \cdots \cdots \cdots, \]
   \[ k_q(h)=hf(x_i+\alpha_qh, y_i+\beta_{q1}k_1+...+\beta_{qq-1}k_{q-1}(h)), \]
а $P_{q1},...,P_{qq}$, $\alpha_2,...,\alpha_q$, $\beta_{21},...,\beta_{qq-1}$ $-$ некоторые постоянные числа. Эти числа выбираются так, чтобы разложение выражения (14.3) в ряд по степеням $h$ совпало при произвольной функции $f(x,y)$ и произвольном $h$ до максимально высокой степени с разложением $y_{i+1}$ в ряд Тейлора вокруг точки $x=x_i$:
\begin{equation}\tag{14.4}
    y_{i+1}=y_i+hy_i^{'}+\frac{h^2}{2}y_i^{''}+...+\frac{h^s}{s!}y_0^{(s)}+\frac{h^{s+1}}{(s+1)!}y^{(s+1)}(\xi),
\end{equation}
где
\[
    y_i^{(k)}=y^{(k)}(x_i), x_i<\xi<x_{i+1}.
\]
Введем вспомогательную функцию 
\begin{equation}\tag{14.5}
    \varphi_q(h)=y(x_i+h)-y(x_i)-\sum \limits_{j=1} ^q P_{qj}k_j(h).
\end{equation}
Тогда сформулированное выше условие для выбора чисел $\{P\}$, {$\{\alpha\}$, $\{\beta\}$} эквивалентно тому, что разложение $\varphi_q(h)$ по степеням $h$ должно начинаться с максимально высокой степени
 
\begin{equation}\tag{14.6}
    \varphi_q(h)=\frac{h^{s+1}}{(s+1)!}\varphi_q^{(s+1)}(0)+O(h^{s+2}).
    \end{equation}
    \qquad Пусть постоянные $\alpha_i$, $\beta_{ij}$, $P_{qi}$ в формуле (14.3) выбраны таким образом, что имеет место разложение (14.6). Тогда говорят, что формула (14.4) имеет пордок точности $s$. Величина 
\[\rho_1=\frac{h^{s+1}}{(s+1)!} \varphi_q^{(s+1)}(0)\]
называется главным членом погрешности метода на шаге $h$.\par
\qquad Положим в (14.3) $q=4$. Тогда решение в точке $x_i+h$ будет отыскиваться в виде:
\[
y_{i+1}=y_i+P_{41}k_1(h)+P_{42}k_2(h)+P_{43}k_3(h)+P_{44}k_4(h),
\]
где
   \[ k_1(h)=hf(x_i,y_i), \]
   \[ k_2(h)=hf(x_i+\alpha_2h,y_i+\beta_{21}k_1(h)), \]
   \[ k_3(h)=hf(x_i+\alpha_3h,y_i+\beta_{31}k_1(h)+\beta_{32}k_2(h)),\]
   \[ k_4(h)=hf(x_i+\alpha_4h, y_i+\beta_{41}k_1(h)+\beta_{42}k_2(h)+\beta_{43}k_3(h)). \]
У нас в распоряжении тринадцать параметров, которые мможно выбрать так, чтобы 
\[
\varphi_4(h)=\frac{h^{5}}{120}\varphi_4^{(5)}(0)+O(h^6).
\]
Уравнения этих коэффицинтов имеют вид:
\[
\begin{cases}
\begin{aligned}

&P_1 + P_2 + P_3 + P_4 = 1, \\

&P_2\alpha_2 + P_3\alpha_3 + P_4\alpha_4 = \tfrac{1}{2}, \\
&P_2\beta_{21} + P_3(\beta_{31}+\beta_{32}) + P_4(\beta_{41}+\beta_{42}+\beta_{43}) = \tfrac{1}{2}, \\

&P_2\alpha_2^2 + P_3\alpha_3^2 + P_4\alpha_4^2 = \tfrac{1}{3}, \\
&P_2\alpha_2\beta_{21} + P_3(\alpha_3\beta_{31}+\alpha_3\beta_{32}) + P_4(\alpha_4\beta_{41}+\alpha_4\beta_{42}+\alpha_4\beta_{43}) = \tfrac{1}{3}, \\
&P_3\beta_{32}\beta_{21} + P_4(\beta_{42}\beta_{21}+\beta_{43}\beta_{31}+\beta_{43}\beta_{32}) = \tfrac{1}{6}, \\

&P_2\alpha_2^3 + P_3\alpha_3^3 + P_4\alpha_4^3 = \tfrac{1}{4}, \\
&P_2\alpha_2^2\beta_{21} + P_3(\alpha_3^2\beta_{31}+\alpha_3^2\beta_{32}) + P_4(\alpha_4^2\beta_{41}+\alpha_4^2\beta_{42}+\alpha_4^2\beta_{43}) = \tfrac{1}{4}, \\
&P_2\alpha_2\beta_{21}^2 + P_3(\alpha_3\beta_{31}^2+\alpha_3\beta_{32}^2) + P_4(\alpha_4\beta_{41}^2+\alpha_4\beta_{42}^2+\alpha_4\beta_{43}^2) = \tfrac{1}{12}, \\
&P_3\alpha_3\beta_{32}\beta_{21} + P_4(\alpha_4\beta_{42}\beta_{21}+\alpha_4\beta_{43}\beta_{31}+\alpha_4\beta_{43}\beta_{32}) = \tfrac{1}{8}.
\end{aligned}
\end{cases}
\]
Эта система имеет следующее решение:
\[
\begin{cases}
\begin{aligned}
P_1 &= \frac{1}{6}, \quad 
P_2 = \dfrac{1}{3}, \quad 
P_3 = \dfrac{1}{3}, \quad 
P_4 = \dfrac{1}{6}, \\[0.5em]

\alpha_2 &= \dfrac{1}{2}, \quad 
\alpha_3 = \dfrac{1}{2}, \quad 
\alpha_4 = 1, \\[0.5em]

\beta_{21} &= \dfrac{1}{2}, \\[0.5em]

\beta_{31} &= 0, \quad 
\beta_{32} = \dfrac{1}{2}, \\[0.5em]

\beta_{41} &= 0, \quad 
\beta_{42} = 0, \quad 
\beta_{43} = 1.
\end{aligned}
\end{cases}
\]
\qquad Таким образом, получаем семейство формул Рунге-Кутты четвёртого порядка точности:
\begin{equation}\tag{14.7}
         \begin{cases}
         y_{i+1} = y_i + \Delta y_i, \\
         \Delta y_i = \frac{1}{6}   (K_1^{(i)}+2K_2^{(i)}+2K_3^{(i)}+K_4^{(i)}),
         \end{cases}
\end{equation}

где
\begin{equation*}
         \begin{cases}
            K_1^{(i)} = hf(x_i,y_i) , \\
            K_2^{(i)} = hf(x_i + \frac{h}{2},y_i + \frac{K_1^{(i)}}{2}) , \\
            K_3^{(i)} = hf(x_i + \frac{h}{2},y_i + \frac{K_2^{(i)}}{2}) , \\
            K_4^{(i)} = hf(x_i + h,y_i + K_3^{(i)}) . \\
         \end{cases}
\end{equation*}

\item {\large\textbf{Явные конечно-разностные формулы}} \par
\qquad Формулы для многошаговых методов c $\alpha_k = 1$ [3,10,20] можно получить, используя квадратурные формулы для интегралов.\par
\qquad Рассмотрим задачу (14.1) с начальным условием (14.2) и проинтегрируем уравнение на отрезке $[x_{n-j},x_{n+1}]$, тогда имеем следующее:
\begin{equation}\tag{14.8}
    y(x_{n+1}) - y(x_{n-j}) = \int\limits_{x_{n-j}}^{x_{n+1}} f(x,y(x)) dx = \int\limits_{x_{n-j}}^{x_{n+1}} y'(x)dx.
\end{equation}\par
\qquad Пусть функция $y(x)$ уже вычислена при $x = x_n, x_{n-1},...$ . Обозначим $f_{n-i} = f(x_{n-i}, y_{n-i})$ и построим интерполяционный многочлен Лагранжа $L_{kn}$, который в узлах $x_m$, $m = n-k, \ n-k+1,...,n$ принимает значения $f_m$. Он будет иметь вид (14.9):
    \begin{equation}\tag{14.9}
         L_{kn}(x) = \sum\limits_{i=0}^{k} f_{n-i}\Phi_i(x),
    \end{equation} 
где
    \begin{equation*}
        \Phi_i(x) = \prod\limits_{\substack{m=n-k \\ m\neq n-i}}^{n} \frac{x-x_m}{x_{n-i}-x_m} .
    \end{equation*} 
Заменяя в (14.8) $f(x,y(x))$ на $L_{kn}(x)$, получим (14.10).

\begin{equation}\tag{14.10}
    y(x_{n+1}) - y(x_{n-j}) = h\sum\limits_{i=0}^{k}B_if_{n-i}+\rho_{n+1}
\end{equation}
где 
\begin{equation*}
    \begin{gathered}
    B_i = \frac{1}{h} \int\limits_{x_{n-j}}^{x_{n+1}}\Phi_i(x) dx;\\
    \rho_{n+1} = O(h^{k+2}),
    \end{gathered}
\end{equation*}
так как погрешность интерполяции $O(h^{k+1})$.\par
\qquad Выражениям для $B_i$ можно придать другой вид. 

\begin{equation}\tag{14.11}
    B_i = \frac{1}{h} \int\limits_{x_{n-j}}^{x_{n+1}}\frac{(x-x_{n-k})\dots(x-x_{n-i+1})(x-x_{n-i-1})\dots(x-x_n)}{(x_{n-i}-x_{n-k})\dots(x_{n-i}-x_{n-i+1})(x_{n-i}-x_{n-i-1})\dots(x_{n-i}-x_n)} dx.
\end{equation}
Сделаем замену 
\begin{equation*}
    t = \frac{x-x_{n}}{h}\implies \frac{x-x_{n-}}{h} = t+l.
\end{equation*}
Тогда выражение (14.11) примет вид (14.12).
\begin{equation}\tag{14.12}
  \begin{gathered}
   B_i = \int\limits_{-j}^1 \frac{(t+k)(t+k-1)\dots t}{(t+i)i(i-1) \dots 1(-1)(-2)\dots(i-k)} dt =\\
   = \frac{(-1)^i}{i!(k-i)!}\int\limits_{-j}^{1}\frac{t(t+1)\dots(t+k)}{(t+i)}dt.
 \end{gathered}
\end{equation}
Из выражения (14.12) видно, что $B_i$ - универсальные числа, не зависящие от $x,h$.\par
\qquad Таким образом, получаем разностное уравнение (14.13).
\begin{equation}\tag{14.13}
    y(x_{n+1}) - y(x_{n-j}) = h\sum\limits_{i=0}^{k}B_if(x_{n-i},y_{n-i}).
\end{equation}\par

\item {\large\textbf{Неявные конечно-разностные формулы}} \par
\qquad Рассмотрим задачу (14.1) с начальным условием (14.2). Пусть функция $y(x)$ уже вычислена в узлах  $x_m$, $m = n-k+1, \ n-k+2,...,n$.
Проинтегрируем OДУ на отрезке $[x_{n-j},x_{n+1}]$, тогда имеем следующее:
\begin{equation}\tag{14.14}
    y(x_{n+1}) - y(x_{n-j}) = \int\limits_{x_{n-j}}^{x_{n+1}} f(x,y(x)) dx = \int\limits_{x_{n-j}}^{x_{n+1}} y'(x)dx.
\end{equation}
Построим интерполяционный многочлен Лагранжа $L_{kn}$, который в точках $x_m$, $m = n-k+1,...,n+1$ принимает значения $f_m$. Он будет иметь вид (14.15):
    \begin{equation}\tag{14.15}
         L_{kn}(x) = \sum\limits_{i=0}^{k} f_{n+1-i}\Phi_i(x),
    \end{equation} 
где
    \begin{equation*}
        \Phi_i(x) = \prod\limits_{\substack{m=n+1-k \\ m\neq n+1-i}}^{n} \frac{x-x_m}{x_{n+1-i}-x_m} .
    \end{equation*} 
Из формулы (14.14), подставив в неё вместо $f \ L_{kn}$, получим (14.16):
\begin{equation}\tag{14.16}
    y(x_{n+1}) - y(x_{n-j}) = h\sum\limits_{i=0}^{k}b_if_{n+1-i}+\rho_{n+1},
\end{equation}
где $\rho_{n+1} = O(h^{k+2})$ - локальная погрешность формулы (14.16).\par
\qquad Представим формулу для $b_i$ в другом виде. Сделаем замену $t = \frac{x-x_n}{h}$, тогда получим (14.17).
\begin{equation}\tag{14.17}
    \begin{gathered}
   b_i = \int \limits_{-j}^1 \frac{[ht+x_n-x_{n+1-k}]\dots[ht+x_n-x_{n+1-(i+1)}][ht+x_n - x_{n+1-(i-1)}]\dots}{[x_{n+1-i}-x_{n+1-k}]\dots[x_{n+1-i}-x_{n+1-(i+1)}][x_{n+1-i} - x_{n+1-(i-1)}]\dots} \\ \frac{\dots [ht+x_n -x_{n+1}]}{\dots [x_{n+1-i} -x_{n+1}]} dt =   \int \limits_{-j}^1 \frac{(t+k-1)(t+k-2)\dots t(t-1)dt}{(t+i-1)(k-i)!i!}(-1)^i =   \\
   = \frac{(-1)^i}{(k-i)!i!} \int \limits_{-j}^1 \frac{(t-1)t\dots (t+k-1)}{t+i-1}dt = \frac{(-1)^i}{(k-i)!i!} \int \limits_{-(j+1)}^0 \frac{t(t+1)\dots (t+k)}{t+i} dt.
    \end{gathered}
\end{equation}
Формула для коэффициентов $b_i$ отличается от формул для коэффициентов $B_i$ в явных схемах только пределами интегрирования.\par
\qquad Таким образом, получаем разностное уравнение (14.18):
\begin{equation}\tag{14.18}
    y(x_{n+1}) - y(x_{n-j}) = h\sum\limits_{i=0}^{k}b_if(x_{n+1-i},y_{n+1-i}).
\end{equation}\par
\item {\large\textbf{Метод Адамса}} \par
\qquad Экстраполяционные и интерполяционные формулы Адамса [3,10,20] получаются из уравнений (14.13) и (14.18) при $j=0$.\par
\qquad Таким образом экстраполяционные формулы Адамса будут вычислятся по формуле (14.19).
    \begin{equation}\tag{14.19}
         y(x_{n+1}) = y(x_{n}) + h\sum\limits_{i=0}^{k}B_if(x_{n-i},y_{n-i}).
    \end{equation}
Формула (14.19) называется предиктором. \par
\qquad Интерполяционные формулы Адамса будут вычислятся по формуле (14.20):
\begin{equation}\tag{14.20}
    y(x_{n+1}) = y(x_{n}) + h\sum\limits_{i=0}^{k}b_if(x_{n+1-i},y_{n+1-i}).
\end{equation}
Формула (14.20) называется корректором.\par
\qquadТрёхчленная экстраполяционная формула Адамса примет вид (14.21):
    \begin{equation}\tag{14.21}
         y_{n+1}^{\text{пред}} = y_n + h(\frac{23}{12}f(x_n,y_n) - \frac{4}{3}f(x_{n-1},y_{n-1}) + \frac{5}{12}f(x_{n-2},y_{n-2})) .
    \end{equation}\par
\qquadТрёхчленная интерполяционная формула Адамса примет вид (14.22):
    \begin{equation}\tag{14.22}
         y_{n+1}^{\text{кор}} = y_n + h(\frac{5}{12}f(x_{n+1},y_{n+1}) + \frac{2}{3}f(x_{n},y_{n}) - \frac{1}{12}f(x_{n-1},y_{n-1})) .
    \end{equation}\par
\qquad Исследуем на аппроксимацию разностную схему (14.22), построенную с помощью интерполяционной формулы Адамса.
Для этого найдём невязку:
    \begin{equation*}
    \begin{gathered}
                \delta f^{(h)} = y_{n+1} - y_n - h(\frac{5}{12} y'_{n+1} + \frac{2}{3}y'_{n} - \frac{1}{12}y'_{n-1}) =\\
                = y_n + y'_nh + \frac{h^2}{2!}y''_n + \frac{h^3}{3!}y'''_n + \frac{h^4}{4!}y^{(IV)}_n + \frac{h^5}{5!}y^{(V)}_n+\dots-\\
                -y_n-h(\frac{5}{12}(y'_n + y''_nh + \frac{h^2}{2!}y'''_n + \frac{h^3}{3!}y^{(IV)}_n + \frac{h^4}{4!}y^{(V)}_n + \frac{h^5}{5!}y^{(VI)}_n+\dots)+\\
                + \frac{2}{3}y'_n  - \frac{1}{12}(y'_n - y''_nh + \frac{h^2}{2!}y'''_n - \frac{h^3}{3!}y^{(IV)}_n + \frac{h^4}{4!}y^{(V)}_n - \frac{h^5}{5!}y^{(VI)}_n+\dots)) =\\
                = - \frac{h^4}{24}y^{(IV)}_n + O(h^5).
    \end{gathered}
    \end{equation*}\par
Таким образом, разностная схема имеет порядок аппроксимации равный $O(h^4)$.\par
\qquad Исследуем разностную схему (14.22) на устойчивость. Для этого разложим функцию $f(y)=y^2$ в ряд Тейлора с центром разложения $y_0 = 1+c$:
    \begin{equation*}
    \begin{gathered}
    y^2 = y_0^2 + (y-y_0)f'(y_0) +\frac{(y-y_0)^2}{2!}f''(y_0)+\dots = y_0^2 + (y-y_0)2y_0+\dots = \\
    =(1+c)^2 +2(1+c)y - 2(1+c)^2+\dots = 2y(1+c) - (1+c)^2+\dots
    \end{gathered}
    \end{equation*}\par
Подставляя в разностную схему получим:
    \begin{equation*}
    \begin{gathered}
    y_{n+1} = y_n + h(\frac{5}{12}\frac{2y_{n+1}(1+c)-(1+c)^2}{4c}+\\
    +\frac{2}{3}\frac{2y_{n}(1+c)-(1+c)^2}{4c} - \frac{1}{12}\frac{2y_{n-1}(1+c)-(1+c)^2}{4c})
    \end{gathered}
    \end{equation*}\par
Выразим $y_{n+1}$:
 \begin{equation*}
    \begin{gathered}
    y_{n+1} = \frac{1+\frac{h}{3c}(1+c)}{1-\frac{5h}{24c}(1+c)}y_n -\frac{\frac{h}{24c}(1+c)}{1-\frac{5h}{24c}(1+c)} y_{n-1}+\rho_h
    \end{gathered}
    \end{equation*}\par
Приведём схему к виду $u_{n+1} = R_hu_n+\rho_h$, где 
\begin{center}
    $u_n =
    \begin{pmatrix}
        y_n\\
        y_{n-1}
    \end{pmatrix}$.
\end{center}
Получим 
\begin{center}
    $\begin{pmatrix}
        y_{n+1}\\
        y_{n}
    \end{pmatrix}$ 
    = $R_h \cdot 
        \begin{pmatrix}
        y_{n}\\
        y_{n-1}
    \end{pmatrix}$, где \\
    \qquad\par
    $R_h =  \begin{pmatrix}
            \frac{1+\frac{h}{3c}(1+c)}{1-\frac{5h}{24c}(1+c)} &\ -\frac{\frac{h}{24c}(1+c)}{1-\frac{5h}{24c}(1+c)} \\
            1 &\ 0
            \end{pmatrix}$
\end{center}\par
Для того, чтобы исследовать схему на устойчивость введём норму \begin{small}$ \|R_h\| = \max \limits_i \sum\limits_j |R_h|$\end{small}. Если $\|R_h\| \leq 1+ch, h \to 0 $, то схема устойчива.
Посчитаем нормы первой и второй строк матрицы $R_h$:
\begin{equation*}
    \begin{gathered}
        R_{h_{1}} = \frac{1+\frac{h}{3c}(1+c)}{1-\frac{5h}{24c}(1+c)} - \frac{\frac{h}{24c}(1+c)}{1-\frac{5h}{24c}(1+c)} = \frac{1+(1+c)(\frac{h}{3c}-\frac{h}{24c})}{1-\frac{5h}{24c}(1+c)} =\\
        = \frac{1+\frac{7h}{24c}(1+c)}{1-\frac{5h}{24c}(1+c)} = \frac{1-\frac{5h}{24c}(1+c) + \frac{12h}{24c}(1+c)}{1-\frac{5h}{24c}(1+c)} =\\
        = 1 + \frac{ \frac{12h}{24c}(1+c)}{1-\frac{5h}{24c}(1+c)} \leq 1 +ch
    \end{gathered}
\end{equation*}
\begin{equation*}
    \begin{gathered}
        R_{h_{2}} = 1 + 0 = 1 \leq 1 +ch
    \end{gathered}
\end{equation*}
Следовательно разностная схема устойчива.
\item {\large\textbf{Метод Милна}} \par
\qquad Экстраполяционные и интерполяционные формулы Милна [3,10,20] получаются из уравнений (14.13) и (14.18) при $j\neq0$.\par
\qquad Такми образом, полагая, что j=1, k=2 получим трёхчленную экстраполяционную и трёхчленную интерполяционную формулы Милна.

\qquadТрёхчленная экстраполяционная формула Милна примет вид (14.23):
    \begin{equation}\tag{14.23}
         y_{n+1}^{\text{пред}} = y_{n-1} + \frac{h}{3}(7f(x_n,y_n) - 2f(x_{n-1},y_{n-1}) + f(x_{n-2},y_{n-2})) .
    \end{equation}\par
\qquadТрёхчленная интерполяционная формула Милна примет вид (14.24):
    \begin{equation}\tag{14.24}
         y_{n+1}^{\text{кор}} = y_{n-1} + \frac{h}{3}(f(x_{n-1},y_{n-1}) + 4f(x_{n},y_{n}) + f(x_{n+1},y_{n+1})).
    \end{equation}\par
    
\qquad Исследуем на аппроксимацию разностную схему (14.24), построенную с помощью интерполяционной формулы Милна.
Для этого найдём невязку:
   \begin{equation*}
    \begin{gathered}
                \delta f^{(h)} = y_{n+1}-y_{n-1} - \frac{h}{3}(y'_{n-1} + 4y'_{n} + y'_{n+1}) =\\
                =y_n + y'_nh + \frac{h^2}{2!}y''_n + \frac{h^3}{3!}y'''_n + \frac{h^4}{4!}y^{(IV)}_n + \frac{h^5}{5!}y^{(V)}_n+\dots-\\
                -y_n + y'_nh - \frac{h^2}{2!}y''_n + \frac{h^3}{3!}y'''_n - \frac{h^4}{4!}y^{(IV)}_n + \frac{h^5}{5!}y^{(V)}_n+\dots-\\
                -\frac{h}{3}(y'_n - y''_nh + \frac{h^2}{2!}y'''_n - \frac{h^3}{3!}y^{(IV)}_n + \frac{h^4}{4!}y^{(V)}_n - \frac{h^5}{5!}y^{(VI)}_n+\dots + 4y'_{n} + \\
                +y'_n + y''_nh + \frac{h^2}{2!}y'''_n + \frac{h^3}{3!}y^{(IV)}_n + \frac{h^4}{4!}y^{(V)}_n + \frac{h^5}{5!}y^{(VI)}_n+\dots) = \\
                = -\frac{h^5}{90}y^{(V)}_n + O(h^6) .
    \end{gathered}
    \end{equation*}\par
Таким образом, разностная схема имеет порядок аппроксимации равный $O(h^5)$.\par

\qquad Исследуем разностную схему (14.24) на устойчивость. Для этого разложим функцию $f(y)=y^2$ в ряд Тейлора с центром разложения $y_0 = 1+c$:
    \begin{equation*}
    \begin{gathered}
    y^2 = y_0^2 + (y-y_0)f'(y_0) +\frac{(y-y_0)^2}{2!}f''(y_0)+\dots = y_0^2 + (y-y_0)2y_0+\dots = \\
    =(1+c)^2 +2(1+c)y - 2(1+c)^2+\dots = 2y(1+c) - (1+c)^2+\dots
    \end{gathered}
    \end{equation*}\par
Подставляя в разностную схему получим:
\begin{equation*}
    \begin{gathered}
        y_{n+1} = y_{n-1} + \frac{h}{3} \Big(\frac{2y_{n+1} (1+c) - (1+c)^2}{4c} + \\ 
        + 4 \frac{2y_{n+1} (1+c) - (1+c)^2}{4c} +  \frac{2y_{n+1} (1+c) - (1+c)^2}{4c} \Big ). \\
    \end{gathered}
\end{equation*}
Выразим $y_i+1$:
 \begin{equation*}
    \begin{gathered}
        y_{n+1} = y_n \frac{4h(1+c)}{6c-h(1+c)} + y_{n-1} \frac{6c+h(1+c)}{6c-h(1+c)} + \rho_h .
    \end{gathered}
    \end{equation*}\par
Приведём схему к виду $u_{n+1} = R_hu_n+\rho_h$, где 
\begin{center}
    $u_h =
    \begin{pmatrix}
        y_n\\
        y_{n-1}
    \end{pmatrix}$.
\end{center}
Получим 
\begin{center}
    $\begin{pmatrix}
        y_{n+1}\\
        y_{n}
    \end{pmatrix}$ 
    = $R_h \cdot 
        \begin{pmatrix}
        y_{n}\\
        y_{n-1}
    \end{pmatrix}$, где \\
    \qquad\par
    $R_h =  \begin{pmatrix}
        \frac{4h(1+c)}{6c-h(1+c)} & \frac{6c+h(1+c)}{6c-h(1+c)}  \\
        1 & 0
            \end{pmatrix}$
\end{center}\par
Для того, чтобы исследовать схему на устойчивость введём норму $ \|R_h\| = \\ = \max \limits_i \sum\limits_j |R_h|$. Если $\|R_h\| \leq 1+ch, h \to 0 $, то схема устойчива. Посчитаем нормы первой и второй строк матрицы $R_h$:
\begin{equation*}
    \begin{gathered}
    R_{h1} = \frac{4h(1+c)}{6c-h(1+c)} + \frac{6c+h(1+c)}{6c-h(1+c)} = \frac{6c+5h(1+c)}{6c-h(1+c)} = 1 + \frac{6h(1+c)}{6c - h(1+c)} \leq 1+ch,
    \end{gathered}
\end{equation*}
\begin{equation*}
    \begin{gathered}
        R_{h_{2}} = 1 + 0 = 1 \leq 1 +ch.
    \end{gathered}
\end{equation*}
Следовательно разностная схема устойчива.

\end{enumerate}
\qquad

\large\item {\large \textbf{Результаты численных расчётов}}\par
\begin{enumerate}
С помощью программы WolframAlpha было получено точное решение уравнения $y^{\prime}-\frac{y^{2}}{4 c}=-\frac{x}{4 c}-\frac{5 c}{4 x^{2}}$:
\begin{equation*}
    \begin{gathered}
    y(x) =\frac{c + x^{\frac{3}{2}}}{x}.
    \end{gathered}
\end{equation*}\par
\qquad Выбранный шаг h при расчетах равен 0.1.\par
Таблица 1 — Заданные точки, точные значения, приближенные значения, полученные методом Рунге-Кутта, абсолютная и относительная погрешности.
\begin{table}[H]
    \centering
    \resizebox{\textwidth}{!}{%
    \begin{tabular}{|c|c|c|c|c|}
        \hline
         x & Точное значение & Приближенное значение & Абсолютная погрешность& Относительная погрешность\\         \hline
  1.00  &  2.001000  &  2.001000  &  0.0  & 0.0 \\ \hline
  1.10  &  1.958808  &  1.958807  &  $0.960396\cdot 10^{-6}$  &  $0.490296\cdot 10^{-6}$   \\ \hline
  1.20  &  1.929611  &  1.929610  &  $1.656053\cdot 10^{-6}$  &  $0.858231\cdot 10^{-6}$   \\ \hline
  1.30  &  1.910175  &  1.910173  &  $2.212857\cdot 10^{-6}$  &  $1.158457\cdot 10^{-6}$   \\ \hline
  1.40  &  1.898215  &  1.898213  &  $2.698735\cdot 10^{-6}$  &  $1.421722\cdot 10^{-6}$   \\ \hline
  1.50  &  1.892078  &  1.892075  &  $3.153053\cdot 10^{-6}$  &  $1.666450\cdot 10^{-6}$   \\ \hline
  1.60  &  1.890536  &  1.890532  &  $3.600385\cdot 10^{-6}$  &  $1.904425\cdot 10^{-6}$   \\ \hline
  1.70  &  1.892664  &  1.892659  &  $4.057364\cdot 10^{-6}$  &  $2.143731\cdot 10^{-6}$   \\ \hline
  1.80  &  1.897751  &  1.897747  &  $4.536291\cdot 10^{-6}$  &  $2.390350\cdot 10^{-6}$   \\ \hline
  1.90  &  1.905246  &  1.905241  &  $5.047127\cdot 10^{-6}$  &  $2.649067\cdot 10^{-6}$   \\ \hline
  2.00  &  1.914713  &  1.914707  &  $5.598656\cdot 10^{-6}$  &  $2.924017\cdot 10^{-6}$   \\ \hline
    \end{tabular}}
\end{table} \par


\begin{figure}[H]
    \centering
    \includegraphics[width=0.9\linewidth]{../pics/rk4.png}\par
    Рисунок 1 — График сравнения точного и приближенного значения, посчитанного методом Рунге-Кутта.\\
    $T(x)$ — точное решение, $P(x)$ — приближенное решение.
\end{figure}\par

\begin{figure}[H]
    \centering
    \includegraphics[width=0.9\linewidth]{../pics/rk4_reltol.png}\par
    Рисунок 2 — График относительной погрешности для приближенного значения, посчитанного методом Рунге-Кутта.\\
    $1$ — относительная погрешность. 
\end{figure}\par


Таблица 2 — Заданные точки, точные значения, приближенные значения, полученные методом Адамса, абсолютная и относительная погрешности.
\begin{table}[H]
    \centering
    \resizebox{\textwidth}{!}{%
    \begin{tabular}{|c|c|c|c|c|}
        \hline
        x & Точное значение & Приближенное значение & Абсолютная погрешность& Относительная погрешность\\
        \hline
  1.00   &    2.001000 &  2.001000  & 0.0 & 0.0   \\ \hline
  1.10   &    1.958808 &  1.958807  & $0.960396\cdot 10^{-6}$  &  $0.490296\cdot 10^{-6}$   \\ \hline
  1.20   &    1.929611 &  1.929610  & $1.656053\cdot 10^{-6}$  &  $0.858231\cdot 10^{-6}$   \\ \hline
  1.30   &    1.910175 &  1.910192  & $1.676222\cdot 10^{-5}$  &  $8.775229\cdot 10^{-6}$   \\ \hline
  1.40   &    1.898215 &  1.898247  & $3.140549\cdot 10^{-5}$  &  $1.654474\cdot 10^{-5}$   \\ \hline
  1.50   &    1.892078 &  1.892121  & $4.359967\cdot 10^{-5}$  &  $2.304327\cdot 10^{-5}$   \\ \hline
  1.60   &    1.890536 &  1.890590  & $5.441435\cdot 10^{-5}$  &  $2.878250\cdot 10^{-5}$   \\ \hline
  1.70   &    1.892664 &  1.892728  & $6.453789\cdot 10^{-5}$  &  $3.409897\cdot 10^{-5}$   \\ \hline
  1.80   &    1.897751 &  1.897826  & $7.444827\cdot 10^{-5}$  &  $3.922972\cdot 10^{-5}$   \\ \hline
  1.90   &    1.905246 &  1.905331  & $8.449636\cdot 10^{-5}$  &  $4.434929\cdot 10^{-5}$   \\ \hline
  2.00   &    1.914713 &  1.914808  & $9.495615\cdot 10^{-5}$  &  $4.959287\cdot 10^{-5}$   \\ \hline
    \end{tabular}}
\end{table} \par


\begin{figure}[H]
    \centering
    \includegraphics[width=0.9\linewidth]{../pics/adams3.png}\par
    Рисунок 3 — График сравнения точного и приближенного значения, посчитанного методом Адамса.\\
    $1$ — точное решение, $2$ — приближенное решение.
\end{figure}\par

\begin{figure}[H]
    \centering
    \includegraphics[width=0.9\linewidth]{../pics/adams3_reltol.png}\par
    Рисунок 4 — График относительной погрешности для приближенного значения, посчитанного методом Адамса.\\
    $1$ — относительная погрешность.
\end{figure}\par


Таблица 3 — заданные точки, точные значения, приближенные значения, полученные методом Милна, абсолютная и относительная погрешности.
\begin{table}[H]
    \centering
    \resizebox{\textwidth}{!}{%
    \begin{tabular}{|c|c|c|c|c|}
        \hline
x & Точное значение & Приближенное значение & Абсолютная погрешность& Относительная погрешность\\ \hline
  1.00  &    2.001000  &   2.001000   &   0.0 & 0.0  \\ \hline
  1.10  &    1.958808  &   1.958807   &   $0.960396\cdot 10^{-6}$  &  $0.490296\cdot 10^{-6}$ \\ \hline
  1.20  &    1.929611  &   1.929610   &   $1.656053\cdot 10^{-6}$  &  $0.858231\cdot 10^{-6}$ \\ \hline
  1.30  &    1.910175  &   1.910157   &   $1.816478\cdot 10^{-5}$  &  $9.509487\cdot 10^{-6}$ \\ \hline
  1.40  &    1.898215  &   1.898200   &   $1.500593\cdot 10^{-5}$  &  $7.905283\cdot 10^{-6}$ \\ \hline
  1.50  &    1.892078  &   1.892049   &   $2.844105\cdot 10^{-5}$  &  $1.503164\cdot 10^{-5}$ \\ \hline
  1.60  &    1.890536  &   1.890511   &   $2.461380\cdot 10^{-5}$  &  $1.301948\cdot 10^{-5}$ \\ \hline
  1.70  &    1.892664  &   1.892627   &   $3.689422\cdot 10^{-5}$  &  $1.949327\cdot 10^{-5}$ \\ \hline
  1.80  &    1.897751  &   1.897718   &   $3.351841\cdot 10^{-5}$  &  $1.766217\cdot 10^{-5}$ \\ \hline
  1.90  &    1.905246  &   1.905201   &   $4.545624\cdot 10^{-5}$  &  $2.385845\cdot 10^{-5}$ \\ \hline
  2.00  &    1.914713  &   1.914670   &   $4.305334\cdot 10^{-5}$  &  $2.248552\cdot 10^{-5}$ \\ \hline
    \end{tabular}}
\end{table} \par


\begin{figure}[H]
    \centering
    \includegraphics[width=0.9\linewidth]{../pics/miln3.png}\par
    Рисунок 5 — График сравнения точного и приближенного значения, посчитанного методом Милна.\\
    $1$ — точное решение, $2$ — приближенное решение.
\end{figure}\par

\begin{figure}[H]
    \centering
    \includegraphics[width=0.9\linewidth]{../pics/miln3_reltol.png}\par
    Рисунок 6 — График относительной погрешности для приближенного значения, посчитанного методом Милна.\\
    $1$ — относительная погрешность.
\end{figure}\par

\end{enumerate}
\large\item {\large \textbf{Вывод}}\par \qquad
В ходе выполнения лабораторной работы был рассмотрен метод Рунге-Кутты, а также были построены устойчивые разностные схемы Адамса и Милна для заданной дифференциальной задачи.
Полученные схемы были исследованы на аппроксимацию и устойчивость. С помощью метода Рунге-Кутты и полученных схем Адамса и Милна заданное дифференциальное уравнение было проинтегрировано на исходном интервале с заданной точностью.
\end{enumerate}

\newpage
\begin{center}
\refstepcounter{section} %гиперссылка
\addcontentsline{toc}{section}{Лабораторная работа №15}
\section*{\large Лабораторная работа №15\\
Многошаговые методы решения обыкновенных дифференциальных уравнений. Построение разностных схем методом неопределенных коэффициентов.}
\end{center}

\renewcommand{\labelenumi}{\textbf{\arabic{enumi}.}}
\renewcommand{\labelenumii}{\textbf{\arabic{enumi}.\arabic{enumii}}}
\renewcommand{\labelenumiii}{\textbf{\arabic{enumi}.\arabic{enumii}.\arabic{enumiii}}}
\renewcommand{\labelenumiv}{\textbf{\arabic{enumi}.\arabic{enumii}.\arabic{enumiii}.\arabic{enumiv}}}

\begin{enumerate}
\large\item {\large \textbf{Постановка задачи}}
\begin{enumerate}
	\item Вывести интерполяционную формулу интегрирования с постоянным шагом в ординатной форме для уравнения и определить локальную погрешность выведенной формулы. Исходными данными для вывода формул являются $M=k+1$, $M$ — количество точек шаблона и $s$ — порядок аппроксимации.

$N=1$ формула Хемминга (схема "1/3"): \;\;\;\;\;\;\;\;\;\;\; $k=3, s=4$;

\item Вывести экстраполяционную формулу интегрирования с постоянным шагом в ординатной форме для уравнения и определить локальную погрешность выведенной формулы. Исходными данными для вывода формул являются $M=k+1$, $M$ — количество точек шаблона и $s$ — порядок аппроксимации.

$N=1$ формула Адамса: \;\;\;\;\;\;\;\;\;\;\; $k=3, s=3$;

\item Исследовать полученные разностные схемы на устойчивость.

\item По полученным в 1. и 2. формулам численно проинтегрировать дифференциальное уравнение из задания №14  с теми же начальными данными.
\end{enumerate}

\large\item {\large \textbf{Теоретический материал}}
\begin{enumerate}
\item {\large\textbf{Многошаговые методы решения ОДУ}} \par
\qquad Общая формула (схема) для многошаговых методов решения ОДУ [3,10,20] имеет вид (15.1):
\begin{equation}\tag{15.1}
    y_{n+k} = F (f; x_{n+k}, x_{n+k-1},\dots , x_{n+}; y_{n+k},\dots, y_{n}).
\end{equation}
Из класса (15.1) выделим многошаговые методы вида (15.2):
\begin{equation}\tag{15.2}
    \sum \limits_{i=0}^k \alpha_i y_{n+i} = h \sum \limits_{i=0}^k \beta_i f(x_{n+i},y_{n+i}), \; \; \; \alpha_k \neq 0.
\end{equation}
Методы вида (15.2) применяются на сетке с постоянным шагом $h$ : $x_n =x_0 +nh$,  $n=0,1,2,\dots$.\par
\qquad Уравнение (15.2) является разностным уравнением $k-$порядка и его решение зависит от $k$ параметров. Чтобы выделить единственное решение, надо задать $k-$условий на функции $ \{ y_n \}$. Этими условиями являются значения $y_n$, при $n=0,1,\dots, k-1$ :
\begin{equation}\tag{15.3}
    y_0 =q_0, y_1 =q_1, \dots, y_{k-1}=q{k-1},
\end{equation}
где $ \{ q_i \}$ - известные числа. Зная данные (15.3), из (15.2), полагая последовательно $n=0,1,\dots$, мы определим $y_k, y_{k+1},\dots$. \par
\qquad Если искомое решение $y_{n+k}$ входит в правую часть (15.2), что бывает при $\beta_k \neq 0$, то формула (15.2) определяет неявный метод. Если $\beta_k = 0$, то мы можем разрешить (15.2) относительно $y_{n+k}$, в этом случае (15.2) определяет явный метод. 
\item {\large\textbf{Построение схем методом неопределённых коэффициентов}} \par
\qquad Коэффициенты $\alpha_i, \beta_i$ в (15.2) выбираются так, чтобы схема (15.2)
аппроксимировала ОДУ с достаточно гладкой правой частью с некоторым
порядком $s$. Это означает, что для точного решения дифференциальной задачи $y(x)$
величина (15.4) 
\begin{equation}\tag{15.4}
    \rho_{n+k} = \sum \limits_{i=0}^k\alpha_i y(x_{n+i}) - h \sum \limits_{i=0}^k \beta_i y'(x_{n+i})
\end{equation}
является величиной $O(h^{s+1})$. Число $s$ называется степенью разностного
уравнения (15.2). Разложим $y(x_{n+i}), y'(x_{n+i})$ в ряд Тейлора в точке $x=x_n$:
\begin{equation}\tag{15.5}
\begin{small}
\begin{gathered}
     y(x_{n+i}) = \sum \limits_{j=0}^{s+1} \frac{(ih)^j}{j!}y^{(j)}(x_n) + O(h^{s+2});\\
    y'(x_{n+i}) = \sum \limits_{j=0}^{s} \frac{(ih)^j}{j!}y^{(j+1)}(x_n) + O(h^{s+1}).
\end{gathered}
\end{small}
\end{equation}
Подставим разложения (15.5) в (15.4). Тогда получим разложение $\rho_{n+k}$ в ряд по степеням $h$:
\begin{equation}\tag{15.6}
\begin{small}
    \begin{gathered}
        \rho_{n+k} = \left( \sum \limits_{i=0}^{k} \alpha_i \right) y(x_n) + h \left(\sum \limits_{i=1}^{k} i\alpha_i - \sum \limits_{i=0}^{k} \beta_i\right)y'(x_n) +\\
        + \frac{h^2}{2}\left( \sum \limits_{i=1}^{k} i^2\alpha_i - 2\sum \limits_{i=1}^{k} i\beta_i\right)y''(x_n) +\dots + \frac{h^S}{s!}\left( \sum \limits_{i=1}^{k} i^s\alpha_i - s\sum \limits_{i=1}^{k} i^{s-1}\beta_i\right)y^{(s)}(x_n) +\\
        +\frac{h^{(s+1)}}{(s+1)!}\left( \sum \limits_{i=1}^{k} i^{s+1}\alpha_i - (s-1)\sum \limits_{i=1}^{k} i^{s}\beta_i\right)y^{(s+1)}(x_n) + O\left(h^{(s+2)}\right).
    \end{gathered}
\end{small}
\end{equation}
Чтобы $\rho_{n+k}$ была величиной порядка $ O\left(h^{(s+1)}\right)$, необходимо, чтобы выполнялось (15.7):
\begin{equation}\tag{15.7}
    \begin{cases}
        \sum \limits_{i=0}^{k} \alpha_i =0;\\
        \sum \limits_{i=1}^{k} i\alpha_i - \sum \limits_{i=0}^{k} \beta_i =0;\\
        \sum \limits_{i=1}^{k} i^{\nu}\alpha_i -\nu \sum \limits_{i=1}^{k} i^{\nu -1} \beta_i =0, \; \; \nu = 2, 3,\dots,s.
    \end{cases}
\end{equation}
Если $\{ \alpha_i, \beta_i\}$ определены согласно (15.7), то погрешность аппроксимации имеет вид (15.8):
\begin{equation}\tag{15.8}
\begin{gathered}
        r_{n+k} = \frac{h^{s}}{(s+1)!}\left( \sum \limits_{i=1}^{k} i^{s+1}\alpha_i -(s+1) \sum \limits_{i=1}^{k} i^{s} \beta_i \right)y^{(s+1)}(x_n) + O\left( h ^{(s+1)}\right) =\\
        = c_{s+1}h^sy^{(s+1)}+O\left( h ^{(s+1)}\right).
\end{gathered}
\end{equation}



\item {\large\textbf{Формула Хемминга "1/3" (k=3, s=4)}} \par
формула (15.2) примет вид:
\begin{equation}\tag{15.9}
\begin{gathered}
    \alpha_0y_n + \alpha_1y_{n+1}+ \alpha_2y_{n+2} +\alpha_3y_{n+3} = \\
    = h(\beta_0y'_n+ \beta_1y'_{n+1}+\beta_2y'_{n+2}+\beta_3y'_{n+3}).
\end{gathered}
\end{equation}

Для формулы Хемминга "1/3" (k=3, s=4): 
\begin{equation}\tag{15.10}
    \alpha_3 = 1, \;\; \alpha_2 = \alpha_1 = \alpha_0 =  -\frac{1}{3}.
\end{equation}
Учитывая (15.10) система (15.7) примет следующий вид:
\begin{equation}\tag{15.11}
    \begin{small}
\begin{gathered}
        \begin{cases}
            \beta_0 + \beta_1 + \beta_2 + \beta_3 = 2    \\
            2\beta_1 + 4\beta_2 + 6\beta_3 = \frac{22}{3} \\
            3\beta_1 + 12\beta_2 + 27\beta_3 = 24 \\
            4\beta_1 + 32\beta_2 + 108\beta_3 = \frac{226}{3}
        \end{cases}
\end{gathered}
    \end{small}
\end{equation}
Решая систему находим:
$\left[
    \begin{gathered}
        \beta_0 = \frac{5}{36}  \\
        \beta_1 = \frac{15}{36} \\
        \beta_2 = \frac{39}{36} \\
        \beta_3 = \frac{13}{36}
    \end{gathered}
\right.$ \par
\qquad\par
Подставив найденные коэффициенты в (15.9), получим формулу Хемминга "1/3" (k=3, s=4):
\begin{equation}\tag{15.12}
    y_{n+3} =\frac{y_{n+2} + y_{n+1} + y_{n}}{3}+ \frac{1}{36}h \left( 5y'_n + 15y'_{n+1} + 39y'_{n+2} + 13y'_{n+3} \right).
\end{equation}

Невязка этой схемы будет иметь вид:
\begin{equation*}
    \rho_{n+3} = \frac{h^5}{120} \left ( \sum\limits_{i=1}^3 i^5  \alpha_i - 5 \sum\limits_{i=1}^3 i^4 \beta_i \right ) y_n^{(5)} + o(h^6) = -\frac{1}{40} h^5 y_n^{(5)}.
\end{equation*}

Для исследования схемы на устойчивость применим спектральный признак устойчивости и найдём характеристический многочлен:

\begin{equation*}
    \rho(\lambda) = \lambda^3 - \frac{1}{3}\lambda^2 - \frac{1}{3}\lambda - \frac{1}{3}
\end{equation*}
и найдём корни полученного характеристического многочлена:
\begin{equation*}
\begin{gathered}
   \lambda_1 = 1 \\
   \lambda_{2,3} = \frac{-1 \pm \sqrt{2}i}{3}
\end{gathered}
\end{equation*}
Поскольку помимо главного корня на границе комплексного единичного корня не лежит посторонних корней,
то формула Хемминга "1/3" (k=3, s=4) является сильно устойчивой.




\item {\large\textbf{Формула Адамса (k=3, s=3)}} \par
формула (15.2) примет вид:
\begin{equation}\tag{15.13}
\begin{gathered}
    \alpha_0y_n + \alpha_1y_{n+1}+ \alpha_2y_{n+2} +\alpha_3y_{n+3} =\\
    = h(\beta_0y'_n+ \beta_1y'_{n+1}+\beta_2y'_{n+2}+\beta_3y'_{n+3}).
\end{gathered}
\end{equation}

Для формулы Адамса (k=3, s=3): 
\begin{equation}\tag{15.14}
    \alpha_n =1, \; \; \alpha_{n-1} = -1, \; \; \alpha_{n-2} = \alpha_{n-3} = \dots = \alpha_0 = 0, \beta_n = 0. 
\end{equation}
Учитывая (15.14) система (15.7) примет следующий вид:
\begin{equation}\tag{15.15}
    \begin{small}
\begin{gathered}
        \begin{cases}
            \beta_0 + \beta_1 + \beta_2 + \beta_3 = 1    \\
            2\beta_1 + 4\beta_2 = 5 \\
            3\beta_1 + 12\beta_2 = 19
        \end{cases}
\end{gathered}
    \end{small}
\end{equation}
Решая систему находим:
$\left[
    \begin{gathered}
        \beta_0 = \frac{5}{12}  \\
        \beta_1 = -\frac{16}{12} \\
        \beta_2 = \frac{23}{12}
    \end{gathered}
\right.$ \par
\qquad\par
Подставив найденные коэффициенты в (15.13), получим формулу Адамса (k=3, s=3):
\begin{equation}\tag{15.16}
    y_{n+3} = y_{n+2} + \frac{h}{12} \left( 5y'_n - 16y'_{n+1} + 23y'_{n+2} \right).
\end{equation}

Невязка этой схемы будет иметь вид:
\begin{equation*}
    \rho_{n+3} = \frac{h^4}{24} \left ( \sum\limits_{i=1}^3 i^4  \alpha_i - 4 \sum\limits_{i=1}^3 i^3 \beta_i \right ) y_n^{(4)} + o(h^5) = \frac{9}{24} h^4 y_n^{(4)}.
\end{equation*}

Для исследования схемы на устойчивость применим спектральный признак устойчивости и найдём характеристический многочлен:

\begin{equation*}
    \rho(\lambda) = \lambda^3 - \lambda^2
\end{equation*}
и найдём корни полученного характеристического многочлена:
\begin{equation*}
\begin{gathered}
   \lambda_{1,2} = 0 \\
   \lambda_3 = 1
\end{gathered}
\end{equation*}
Поскольку все корни характеристического многочлена, за исключением главного корня, лежат внутри комплексного единичного круга, то формула Адамса является сильно устойчивой.
\end{enumerate}




\item {\large\textbf{Результаты численных расчётов}} \par
\qquad С помощью выыеденных формул численно проинтегрируем дифференциальное уравнение
\begin{equation*}
y'-\frac{y^2}{4c}=-\frac{x}{4c}-\frac{5c}{4x^2},
\end{equation*}
\begin{equation*}
	y(1)=1+c , c=1+0.001N.
\end{equation*}
на интервале [1; 2] с шагом $h=0.1$.\par 
\qquad В качестве "корректора" используется формула Хемминга "1/3" (k=3, s=4), а в качестве
"предиктора" используется формула Адамса (k=3, s=3).\par
 Значения y(1.1),y(1.2),y(1.3) были вычислены с помощью метода Рунге-Кутты.\par

\qquad Таблица 1 — Заданные точки, точные значения, приближенные значения, абсолютная и относительная погрешности.
\begin{table}[H]
    \centering
    \resizebox{\textwidth}{!}{%
    \begin{tabular}{|c|c|c|c|c|}
        \hline
        x & Точное значение & Приближенное значение & Абсолютная погрешность& Относительная погрешность\\ \hline
       1.0  &    2.001000  &    2.001000  & 0.0 & 0.0  \\ \hline
       1.1  &    1.958809  &    1.958808  &$9.603961\cdot 10^{-7}$       &$4.902960\cdot 10^{-7}$ \\ \hline
       1.2  &    1.929612  &    1.929610  &$1.656053\cdot 10^{-6}$       &$8.582313\cdot 10^{-7}$ \\ \hline
       1.3  &    1.910175  &    1.910288  &$1.123252\cdot 10^{-4}$       &$5.880360\cdot 10^{-5}$ \\ \hline
       1.4  &    1.898216  &    1.898597  &$3.814173\cdot 10^{-4}$       &$2.009346\cdot 10^{-4}$ \\ \hline
       1.5  &    1.892078  &    1.892838  &$7.601697\cdot 10^{-4}$       &$4.017644\cdot 10^{-4}$ \\ \hline
       1.6  &    1.890536  &    1.891813  &$1.276624\cdot 10^{-3}$       &$6.752712\cdot 10^{-4}$ \\ \hline
       1.7  &    1.892664  &    1.894607  &$1.942615\cdot 10^{-3}$       &$1.026392\cdot 10^{-3}$ \\ \hline
       1.8  &    1.897752  &    1.900511  &$2.758705\cdot 10^{-3}$       &$1.453670\cdot 10^{-3}$ \\ \hline
       1.9  &    1.905247  &    1.908987  &$3.740462\cdot 10^{-3}$       &$1.963242\cdot 10^{-3}$ \\ \hline
       2.0  &    1.914714  &    1.919616  &$4.902169\cdot 10^{-3}$       &$2.560262\cdot 10^{-3}$ \\ \hline
    \end{tabular}}
\end{table} \par


\begin{figure}[H]
    \centering
    \includegraphics[width=0.9\linewidth]{../pics/method_15.png}\par
    Рисунок 1 — График сравнения точного решения и приближенного, посчитанного по схеме.\\
    $1$ — точное решение, $2$ — приближенное решение.
\end{figure}\par

\begin{figure}[H]
    \centering
    \includegraphics[width=0.9\linewidth]{../pics/reltol_15.png}\par
    Рисунок 2 — График относительной погрешности для приближенного решения, посчитанного по схеме.\\
    $1$ — относительная погрешность.
\end{figure}\par


\large\item {\large \textbf{Вывод}}\par
\qquad В ходе выполнения лабораторной работы были выведены интерполяционная формула интегрирования Хемминга "1/3" (k=3, s=4) и экстраполяционная формула интегрирования Адамса (k = 3, s = 3). Для каждой выведенной формулы была найдена локальная погрешность, и каждая формула была исследована на устойчивость. По полученным формулам было численно проинтегрированно дифференциальное уравнение из лабораторной работы №14. Также были найдены абсолютная и относительная погрешности. 
\end{enumerate}

\newpage
\begin{center}
\refstepcounter{section} %гиперссылка
\addcontentsline{toc}{section}{Лабораторная работа №16}
\section*{\large Лабораторная работа №16}
\bf{Исследование аппроксимации и устойчивости разностной схемы системы обыкновенных дифференциальных уравнений.}
\end{center}
\renewcommand{\labelenumi}{\textbf{\arabic{enumi}.}}
\renewcommand{\labelenumii}{\textbf{\arabic{enumi}.\arabic{enumii}}}
\renewcommand{\labelenumiii}{\textbf{\arabic{enumi}.\arabic{enumii}.\arabic{enumiii}}}
\renewcommand{\labelenumiv}{\textbf{\arabic{enumi}.\arabic{enumii}.\arabic{enumiii}.\arabic{enumiv}}}

\begin{enumerate}
\large\item {\large \textbf{Постановка задачи}}
\begin{enumerate}

\item Исследовать во внутренних узлах устойчивость разностной схемы
\begin{equation*}
\begin{cases}
	\frac{u_{i+1}-u_{i-1}}{2h}-\left(a_1v_{i+1}+\left(1-\frac{2}{N}\right)v_{i}+b_1v_{i-1}\right)=0,\\
	a_1+b_1=\frac{2}{N},\\
		\frac{v_{i+1}-v_{i-1}}{2h}-4\left(a_2u_{i+1}+\left(1-\frac{1}{N}\right)u_{i}+b_2u_{i-1}\right)=0,\\
	a_2+b_2=\frac{1}{N}
\end{cases}
\end{equation*}
\normalsize
для дифференциальной задачи
\begin{equation*}
\begin{cases}
	\frac{du}{dx}-v=0, u(1)=1;\\
	\frac{dv}{dx}-4u=0, v(1)=6;
\end{cases}
\end{equation*}
Для доказательства устойчивости необходимо построить оператор перехода(матрицу $4 	
\times4$) $R_h=(r_{ij})$.

\item Подобрать параметры $a_1, b_1, a_2, b_2$ таким образом, чтобы аппроксимация схемы во внутренних узлах была 2 порядка.

\item Решить систему любым методом на отрезке $[1.0, 1.1]$ с шагом, равным $0.01$.
\end{enumerate}


\large\item {\large \textbf{Теоретический материал}}
\begin{enumerate}
\item {\large\textbf{Аппроксимация}} \par
\qquad Найдём порядок аппроксимации [3,21] разностной схемы (16.1).
\begin{equation}\tag{16.1}
\begin{cases}
	\frac{u_{i+1}-u_{i-1}}{2h}-\left(a_1v_{i+1}+\left(1-\frac{2}{N}\right)v_{i}+b_1v_{i-1}\right)=0,\\
		\frac{v_{i+1}-v_{i-1}}{2h}-4\left(a_2u_{i+1}+\left(1-\frac{1}{N}\right)u_{i}+b_2u_{i-1}\right)=0,\\
  	a_1=\frac{2}{N} - b_1,\\
	a_2=\frac{1}{N} - b_2.
\end{cases}
\end{equation}

Для этого разложим в ряд Тейлора функции $u_{i\pm 1}$, $v_{i\pm 1}$:
\begin{equation}\tag{16.2}
    \begin{gathered}
        u_{i\pm 1} = u_i \pm hu'_i + \frac{h^2}{2}u''_i \pm  \frac{h^3}{6}u_i''' +\dots\\
        v_{i\pm 1} = v_i \pm hv'_i + \frac{h^2}{2}v''_i \pm  \frac{h^3}{6}v'''_i +\dots
    \end{gathered}
\end{equation}
Полученые разложения (16.2) подставим в первое уравнение системы (16.1), а затем во второе.
\begin{small}
\begin{equation*}
    \begin{gathered}
 \frac{u_i + hu'_i + \frac{h^2}{2}u''_i+  \frac{h^3}{6}u'''_i+ \dots \ - u_i + hu'_i - \frac{h^2}{2}u''_i+  \frac{h^3}{6}u'''_i+ \dots}{2h}-\\
 - \left(a_1(v_i + hv'_i+ \frac{h^2}{2}v''_i+  \frac{h^3}{6}v'''_i+ \dots \ ) +\left(1-\frac{2}{N}\right)v_{i} + b_1(v_i - hv'_i+ \frac{h^2}{2}v''_i-  \frac{h^3}{6}v'''_i+ \dots \ ) \right)=0,
    \end{gathered}
\end{equation*}
\end{small}

\begin{small}
\begin{equation*}
    \begin{gathered}
	\frac{v_i + hv'_i+ \frac{h^2}{2}v''_i+  \frac{h^3}{6}v'''_i+ \dots \ - v_i + hv'_i- \frac{h^2}{2}v''_i+  \frac{h^3}{6}v'''_i+ \dots \ }{2h}- \\ - 4\left(a_2(u_i + hu'_i + \frac{h^2}{2}u''_i+  \frac{h^3}{6}u'''_i+ \dots )+\left(1-\frac{1}{N}\right)u_{i}+b_2(u_i - hu'_i + \frac{h^2}{2}u''_i-  \frac{h^3}{6}u'''_i+ \dots )\right)=0.
    \end{gathered}
\end{equation*}
\end{small}
Выполним преобразования: 
\begin{small}
\begin{equation*}
    \begin{gathered}
  u'_i = v_i(a_1 + b_1 + 1 - \frac{2}{N}) + hv'_i(a_1 - b_1) + \frac{h^2}{2}v''_i(a_1 + b_1) + \frac{h^3}{6}v'''_i(a_1-b_1)+\dots \ - \frac{h^2}{6}u'''+ \dots \ 
    \end{gathered}
\end{equation*}
\end{small}
\begin{small}
\begin{equation*}
    \begin{gathered}
  v'_i  = 4u_i(a_2 + b_2 + 1 - \frac{1}{N}) + 4hu'_i(a_2 - b_2) + \frac{4h^2}{2}v''_i(a_2 + b_2) + \frac{4h^3}{6}v'''_i(a_2-b_2)+\dots \ - \frac{h^2}{6}v'''+ \dots \
    \end{gathered}
\end{equation*}
\end{small}
Для получения второго порядка аппроксимации нужно выполним следующее:
\begin{equation*}
    \begin{cases}
        a_1 - b_1 = 0\\
        a_2 - b_2 = 0 
    \end{cases}.
\end{equation*}

Затем найдём коэффициенты $a_1,a_2,b_1,b_2$:

\begin{equation*}
\begin{cases}
        a_1 - b_1 = 0,\\
  	a_1=\frac{2}{N} - b_1
\end{cases}
\implies \left[ 
      \begin{gathered} 
        a_1 = \frac{1}{N}; \\ 
        b_1 = \frac{1}{N} \\ 
      \end{gathered} 
\right.
\end{equation*}

\begin{equation*}
\begin{cases}
        a_2 - b_2 = 0,\\
  	a_2=\frac{1}{N} - b_2
\end{cases}
\implies \left[ 
      \begin{gathered} 
        a_2 = \frac{1}{2N}; \\ 
        b_2 = \frac{1}{2N} \\ 
      \end{gathered} 
\right.
\end{equation*}
В конечном итоге, схема (16.1), которая имеет второй порядок аппроксимации, примет вид (16.3).
\begin{equation}\tag{16.3}
\begin{cases}
	\frac{u_{i+1}-u_{i-1}}{2h}-\left(\frac{1}{N}v_{i+1}+\left(1-\frac{2}{N}\right)v_{i}+\frac{1}{N}v_{i-1}\right)=0,\\
		\frac{v_{i+1}-v_{i-1}}{2h}-4\left(\frac{1}{2N}u_{i+1}+\left(1-\frac{1}{N}\right)u_{i}+\frac{1}{2N}u_{i-1}\right)=0.
\end{cases}
\end{equation}
\item {\large\textbf{Устойчивость}} \par
\qquad Исследуем разностную схему (16.3) на устойчивость [3,7,21]. \par
Для этого выразим $u_{i+1}$,  $v_{i+1}$:
\begin{equation*}
\begin{cases}
	u_{i+1} = u_{i-1} + \frac{2h}{N}(v_{i+1} + (N-2)v_i+v_{i-1})\\
	v_{i+1} = v_{i-1} + \frac{8h}{N}(\frac{1}{2}u_{i+1} + (N-1)u_i+u_{i-1})\\
\end{cases}
\end{equation*}
Теперь подставим перовое выражение во второе, затем второе выражение в первое. В итоге, получим (16.4):
\begin{equation}\tag{16.4}
\begin{cases}
	u_{i+1} = \frac{\frac{16h^2}{N^2}(N-1)}{1-\frac{8h^2}{N^2}} u_{i} + 
                \frac{1+\frac{8h^2}{N^2}}{1-\frac{8h^2}{N^2}}u_{i-1}+     \frac{\frac{2h}{N}(N-2)}{1-\frac{8h^2}{N^2}}v_{i}+ \frac{\frac{4h}{N}}{1-\frac{8h^2}{N^2}}v_{i-1}  \\
                \\
	v_{i+1} = \frac{\frac{8h^2}{N^2}(N-2)}{1-\frac{8h^2}{N^2}} v_{i} + 
                \frac{1+\frac{8h^2}{N^2}}{1-\frac{8h^2}{N^2}}v_{i-1}+     \frac{\frac{8h}{N}(N-1)}{1-\frac{8h^2}{N^2}}u_{i}+ \frac{\frac{8h}{N}}{1-\frac{8h^2}{N^2}}u_{i-1}  \\
\end{cases}.
\end{equation}


Приведём схему к виду $u_{n+1} = R_hu_n+\rho_h$, где 


\begin{center}
\begin{small}
    $u_n =
    \begin{pmatrix}
        u_n\\
        v_n\\
        u_{n-1}\\
        v_{n-1}
    \end{pmatrix}$
\end{small}
\end{center}
Получим 
    \begin{small}
    $\begin{pmatrix}
        u_{n+1}\\
        v_{n+1}\\
        u_{n}\\
        v_{n}
    \end{pmatrix}$ 
    = $R_h \cdot 
        \begin{pmatrix}
        u_{n}\\
        v_{n}\\
        u_{n-1}\\
        v_{n-1}
    \end{pmatrix}$ 
    \end{small}, где матрица перехода $R_h$ имеет вид:\\
    \qquad\par
    \begin{center}
    $R_h =  \begin{pmatrix}
            \frac{16h^2(N-1)}{N^2-8h^2} &\ \frac{2hN(N-2)}{N^2-8h^2} &\ \frac{N^2+8h^2}{N^2-8h^2} &\ \frac{4hN}{N^2-8h^2} \\
            \frac{8hN(N-1)}{N^2-8h^2} &\ \frac{8h^2(N-2)}{N^2-8h^2} &\ \frac{8hN}{N^2-8h^2} &\ \frac{N^2+8h^2}{N^2-8h^2} \\
            1 &\ 0 &\ 0 &\ 0 \\
            0 &\ 1 &\ 0 &\ 0 \\
            \end{pmatrix}.$
\end{center}\par
Для того, чтобы исследовать схему на устойчивость введём норму \begin{small}$ \|R_h\| = \max \limits_i \sum\limits_j |R_h|$\end{small}. Если $\|R_h\| \leq 1+ch, h \to 0 $, то схема устойчива.\par
Посчитаем норму каждой строки матрицы $R_h$:
\begin{small}
\begin{equation*}
    \begin{gathered}
        R_{h_{1}} =  \frac{16h^2(N-1)}{N^2-8h^2} + \frac{2hN(N-2)}{N^2-8h^2} + \frac{N^2+8h^2}{N^2-8h^2} + \frac{4hN}{N^2-8h^2} \leq 1 +ch \text{ при } h \to 0
    \end{gathered}
\end{equation*}
\end{small}
\begin{small}
\begin{equation*}
    \begin{gathered}
        R_{h_{2}} =  \frac{8hN(N-1)}{N^2-8h^2} + \frac{8h^2(N-2)}{N^2-8h^2} + \frac{8hN}{N^2-8h^2} + \frac{N^2+8h^2}{N^2-8h^2} \leq 1 +ch \text{ при } h \to 0
    \end{gathered}
\end{equation*}
\end{small}

\begin{small}
\begin{equation*}
    \begin{gathered}
        R_{h_{3}} =  1+0+0+0 \leq 1 +ch 
    \end{gathered}
\end{equation*}
\end{small}
\begin{small}
\begin{equation*}
    \begin{gathered}
        R_{h_{4}} =  0+1+0+0 \leq 1 +ch 
    \end{gathered}
\end{equation*}
\end{small}
Следовательно разностная схема устойчива.

\item {\large\textbf{Точное решение}} \par
\qquad Найдём точное решение системы дифференциальных уравнений [23]:
\begin{equation}\tag{16.5}
\begin{cases}
	\frac{du}{dx}-v=0, u(1)=1;\\
	\frac{dv}{dx}-4u=0, v(1)=6;
\end{cases}
\end{equation}
Приведём систему (16.5) к виду (16.6):
\begin{equation}\tag{16.6}
\begin{cases}
	\frac{du}{dx}=v\\
	\frac{dv}{dx}=4u
\end{cases}
\end{equation}
Продифференцируем первое выражение системы (16.6) и подставим во второе выражение:
\begin{equation*}
 \frac{du}{dx}=v  \implies \frac{d^2u}{dx^2}=\frac{dv}{dx}
\end{equation*}
\begin{equation*}
 \frac{dv}{dx}=4u \implies \frac{d^2u}{dx^2} = 4u
\end{equation*}
Общее решение для функции $u$: $u = C_1e^{2x} +  C_2e^{-2x}$.\par
Продифференцировав общее решение для функции  $u$, получим общее решение для функции $v$: $v = 2C_1e^{2x} -  2C_2e^{-2x}$.\par
\qquad Общее решение фистемы примет вид (16.7):
\begin{equation}\tag{16.7}
\begin{cases}
	u = C_1e^{2x} +  C_2e^{-2x}\\
	v = 2C_1e^{2x} -  2C_2e^{-2x}
\end{cases}
\end{equation}\par
\qquad Из начальных условий найдем $C_1$ и $C_2$:
\begin{equation*}
\begin{cases}
	u(1) = C_1e^{2} +  C_2e^{-2} = 1\\
	v(1) = 2C_1e^{2} -  2C_2e^{-2} = 6
\end{cases}
\implies \left[ 
      \begin{gathered} 
        C_1 = 2e^{-2} \\ 
        C_2 = -e^2 \\ 
      \end{gathered} 
\right.
\end{equation*}\par
 Подставив $C_1$ и $C_2$ найдём частное решение системы (16.8):
\begin{equation}\tag{16.8}
\begin{cases}
	u = 2e^{2x-2} - e^{2-2x}\\
	v = 4e^{2x-2} +  2e^{2-2x}
\end{cases}
\end{equation}\par

\end{enumerate}

\newpage
\large\item {\large \textbf{Результаты численных расчётов}}\par
\begin{enumerate}
\qquadПосчитаем приближенные значения функции u(x) на отрезке $[1.0, 1.1]$ с шагом $h = 0.01$ методом Рунге-Кутта и сравним их с точным решением. Результаты представлены в таблице 1.\par
Таблица 1 — Заданные точки, точные значения, приближенные значения, полученные методом Рунге-Кутта, абсолютная и относительная погрешности.
\begin{table}[H]
    \centering
    \resizebox{\textwidth}{!}{%
    \begin{tabular}{|c|c|c|c|c|}
        \hline
        x & Точное значение & Приближенное значение & Абсолютная погрешность& Относительная погрешность\\
        \hline
    1.00  &  1.000000  &  1.000000  &  0.0 &  0.0   \\ \hline
    1.01  &  1.060204  &  1.060204  &  $8.008949\cdot 10^{-11}$  &  $7.554158\cdot 10^{-11}$    \\ \hline
    1.02  &  1.120832  &  1.120832  &  $1.612887\cdot 10^{-10}$  &  $1.439009\cdot 10^{-10}$    \\ \hline
    1.03  &  1.181909  &  1.181909  &  $2.436941\cdot 10^{-10}$  &  $2.061869\cdot 10^{-10}$    \\ \hline
    1.04  &  1.243458  &  1.243458  &  $3.274036\cdot 10^{-10}$  &  $2.633009\cdot 10^{-10}$    \\ \hline
    1.05  &  1.305504  &  1.305504  &  $4.125155\cdot 10^{-10}$  &  $3.159817\cdot 10^{-10}$    \\ \hline
    1.06  &  1.368073  &  1.368073  &  $4.991298\cdot 10^{-10}$  &  $3.648414\cdot 10^{-10}$    \\ \hline
    1.07  &  1.431189  &  1.431189  &  $5.873477\cdot 10^{-10}$  &  $4.103913\cdot 10^{-10}$    \\ \hline
    1.08  &  1.494878  &  1.494878  &  $6.772713\cdot 10^{-10}$  &  $4.530613\cdot 10^{-10}$    \\ \hline
    1.09  &  1.559165  &  1.559165  &  $7.690035\cdot 10^{-10}$  &  $4.932151\cdot 10^{-10}$    \\ \hline
    1.10  &  1.624075  &  1.624075  &  $8.626492\cdot 10^{-10}$  &  $5.311635\cdot 10^{-10}$    \\ \hline
    \end{tabular}}
\end{table} \par

\begin{figure}[H]
    \centering
    \includegraphics[width=0.9\linewidth]{../pics/laba16_u.png}\par
    Рисунок 1 — График сравнения точного решения и приближенного решения $u$, посчитанного методом Рунге-Кутта.\\
    $1$ — точное решение, $2$ — приближенное решение.
\end{figure}\par

\begin{figure}[H]
    \centering
    \includegraphics[width=0.9\linewidth]{../pics/laba16_u_reltol.png}\par
    Рисунок 2 — График относительной погрешности для приближенного решения $u$, посчитанного методом Рунге-Кутта.\\
    $1$ — относительная погрешность.
\end{figure}\par



\qquad Посчитаем приближённые значения функции v(x) на отрезке $[1.0, 1.1]$ с шагом $h = 0.01$ методом Рунге-Кутта и сравним их с точным решением. Результаты представлены в таблице 2.\par

Таблица 2 — Заданные точки, точные значения, приближенные значения, полученные методом Рунге-Кутта, абсолютная и относительная погрешности.
\begin{table}[H]
    \centering
    \resizebox{\textwidth}{!}{%
    \begin{tabular}{|c|c|c|c|c|}
        \hline
        x & Точное значение & Приближенное значение & Абсолютная погрешность& Относительная погрешность\\
        \hline
    1.00  &  6.000000  &  6.000000  &  0.0    &  0.0     \\ \hline
    1.01  &  6.041203  &  6.041203  &  $5.386713\cdot 10^{-11}$  &  $8.916623\cdot 10^{-12}$   \\ \hline
    1.02  &  6.084822  &  6.084822  &  $1.141629\cdot 10^{-10}$  &  $1.876191\cdot 10^{-11}$   \\ \hline
    1.03  &  6.130875  &  6.130875  &  $1.809574\cdot 10^{-10}$  &  $2.951576\cdot 10^{-11}$   \\ \hline
    1.04  &  6.179381  &  6.179381  &  $2.543227\cdot 10^{-10}$  &  $4.115667\cdot 10^{-11}$   \\ \hline
    1.05  &  6.230359  &  6.230359  &  $3.343378\cdot 10^{-10}$  &  $5.366270\cdot 10^{-11}$   \\ \hline
    1.06  &  6.283828  &  6.283828  &  $4.210880\cdot 10^{-10}$  &  $6.701138\cdot 10^{-11}$   \\ \hline
    1.07  &  6.339812  &  6.339812  &  $5.146620\cdot 10^{-10}$  &  $8.117939\cdot 10^{-11}$   \\ \hline
    1.08  &  6.398331  &  6.398331  &  $6.151559\cdot 10^{-10}$  &  $9.614318\cdot 10^{-11}$   \\ \hline
    1.09  &  6.459410  &  6.459410  &  $7.226672\cdot 10^{-10}$  &  $1.118782\cdot 10^{-10}$   \\ \hline
    1.10  &  6.523073  &  6.523073  &  $8.373044\cdot 10^{-10}$  &  $1.283604\cdot 10^{-10}$   \\ \hline
    \end{tabular}}
\end{table} \par

\begin{figure}[H]
    \centering
    \includegraphics[width=0.9\linewidth]{../pics/laba16_v.png}\par
    Рисунок 3 — График сравнения точного решения и приближенного решения $v$, посчитанного методом Рунге-Кутта.\\
    $1$ — точное решение, $2$ — приближенное решение.
\end{figure}\par

\begin{figure}[H]
    \centering
    \includegraphics[width=0.9\linewidth]{../pics/laba16_v_reltol.png}\par
    Рисунок 4 — График относительной погрешности для приближенного решения $v$, посчитанного методом Рунге-Кутта.\\
    $1$ — относительная погрешность.
\end{figure}\par

\qquad Используя первое точное решение из условия задачи и второе значение функции u, посчитанное методом Рунге-Кутта, посчитаем с помощью разностной схемы приближенные значения функции u на отрезке [1.0, 1.1] с шагом h = 0.01 и сравним их с точным решением.Результаты представлены в таблице 3.\par
\newpage
Таблица 3 $-$ Заданные точки, точные значения, приближенные значения, полученные с помощью разностной схемы, абсолютная и относительная погрешности.

\begin{table}[H]
    \centering
    \resizebox{\textwidth}{!}{%
    \begin{tabular}{|c|c|c|c|c|}
        \hline
        x & Точное значение & Приближенное значение & Абсолютная погрешность& Относительная погрешность\\
        \hline
  1.00  &  1.000000  &  1.000000   &  0.0    &  0.0     \\ \hline
  1.01  &  1.060204  &  1.060204   &  $8.008949\cdot 10^{-11}$  &  $7.554158\cdot 10^{-11}$   \\ \hline
  1.02  &  1.120832  &  1.120723   &  $1.084663\cdot 10^{-4}$  &  $9.677308\cdot 10^{-5}$   \\ \hline
  1.03  &  1.181908  &  1.181794   &  $1.145164\cdot 10^{-4}$  &  $9.689110\cdot 10^{-5}$   \\ \hline
  1.04  &  1.243457  &  1.243228   &  $2.291573\cdot 10^{-4}$  &  $1.842904\cdot 10^{-4}$   \\ \hline
  1.05  &  1.305504  &  1.305263   &  $2.413979\cdot 10^{-4}$  &  $1.849077\cdot 10^{-4}$   \\ \hline
  1.06  &  1.368073  &  1.367710   &  $3.624671\cdot 10^{-4}$  &  $2.649472\cdot 10^{-4}$   \\ \hline
  1.07  &  1.431189  &  1.430808   &  $3.810585\cdot 10^{-4}$  &  $2.662530\cdot 10^{-4}$   \\ \hline
  1.08  &  1.494877  &  1.494369   &  $5.088299\cdot 10^{-4}$  &  $3.403822\cdot 10^{-4}$   \\ \hline
  1.09  &  1.559164  &  1.558630   &  $5.339523\cdot 10^{-4}$  &  $3.424605\cdot 10^{-4}$   \\ \hline
  1.10  &  1.624074  &  1.623406   &  $6.687203\cdot 10^{-4}$  &  $4.117546\cdot 10^{-4}$   \\ \hline
    \end{tabular}}
\end{table} \par


\begin{figure}[H]
    \centering
    \includegraphics[width=0.9\linewidth]{../pics/laba16_u_scheme.png}\par
    Рисунок 5 — График сравнения точного решения и приближенного решения $u$, посчитанного по схеме.\\
    $1$ — точное решение, $2$ — приближенное решение.
\end{figure}\par
\begin{figure}[H]
    \centering
    \includegraphics[width=0.9\linewidth]{../pics/laba16_u_reltol_scheme.png}\par
    Рисунок 6 — График относительной погрешности для приближенного решения $u$, посчитанного по схеме.\\
    $1$ — относительная погрешность.
\end{figure}\par


\qquad Используя первое точное решение из условия задачи и второе значение функции v, посчитанное методом Рунге-Кутта, посчитаем с помощью разностной схемы приближенные значения функции v на отрезке [1.0, 1.1] с шагом h = 0.01 и сравним их с точным решением. Результаты представлены в таблице 4.\par

Таблица 4 $-$ Заданные точки, точные значения, приближенные значения, полученные с помощью разностной схемы, абсолютная и относительная погрешности.

\begin{table}[H]
    \centering
    \resizebox{\textwidth}{!}{%
    \begin{tabular}{|c|c|c|c|c|}
        \hline
        x & Точное значение & Приближенное значение & Абсолютная погрешность& Относительная погрешность\\
        \hline
  1.00  &  6.000000  &  6.000000   &  0.0   &  0.0     \\ \hline
  1.01  &  6.041202  &  6.041202   &  $5.386713\cdot 10^{-11}$  &  $8.916623\cdot 10^{-12}$   \\ \hline
  1.02  &  6.084821  &  6.084817   &  $4.456007\cdot 10^{-6}$  &  $7.323151\cdot 10^{-7}$   \\ \hline
  1.03  &  6.130875  &  6.130862   &  $1.276797\cdot 10^{-5}$  &  $2.082568\cdot 10^{-6}$   \\ \hline
  1.04  &  6.179380  &  6.179362   &  $1.854965\cdot 10^{-5}$  &  $3.001863\cdot 10^{-6}$   \\ \hline
  1.05  &  6.230358  &  6.230322   &  $3.567079\cdot 10^{-5}$  &  $5.725319\cdot 10^{-6}$   \\ \hline
  1.06  &  6.283828  &  6.283785   &  $4.327788\cdot 10^{-5}$  &  $6.887184\cdot 10^{-6}$   \\ \hline
  1.07  &  6.339811  &  6.339741   &  $6.972554\cdot 10^{-5}$  &  $1.099804\cdot 10^{-5}$   \\ \hline
  1.08  &  6.398331  &  6.398251   &  $7.967150\cdot 10^{-5}$  &  $1.245192\cdot 10^{-5}$   \\ \hline
  1.09  &  6.459409  &  6.459293   &  $1.159847\cdot 10^{-4}$  &  $1.795593\cdot 10^{-5}$   \\ \hline
  1.10  &  6.523072  &  6.522943   &  $1.287984\cdot 10^{-4}$  &  $1.974506\cdot 10^{-5}$   \\ \hline
    \end{tabular}}
\end{table} \par


\begin{figure}[H]
    \centering
    \includegraphics[width=0.9\linewidth]{../pics/laba16_v_scheme.png}\par
    Рисунок 7 — График сравнения точного решения и приближенного решения $v$, посчитанного по схеме.\\
    $1$ — точное решение, $2$ — приближенное решение.
\end{figure}\par


\begin{figure}[H]
    \centering
    \includegraphics[width=0.9\linewidth]{../pics/laba16_v_reltol_scheme.png}\par
    Рисунок 8 — График относительной погрешности для приближенного решения $v$, посчитанного по схеме.\\
    $1$ — относительная погрешность.
\end{figure}\par


\end{enumerate}
\large\item {\large \textbf{Вывод}}\par
\qquad В ходе выполнения лабораторной работы было проведено исследование аппроксимации и устойчивости разностной схемы системы обыкновенных дифференциальных уравнений. Методом Рунге-Кутта и с помощью разностной схемы на отрезке [1.0,1.1] с шагом, равным 0.01 были посчитаны приближенные значения функций u(x), v(x) данной системы. Также были найдены точное решение системы ОДУ, абсолютная и относительная погрешности. 
\end{enumerate}

\newpage
\begin{center}
\refstepcounter{section} %гиперссылка
\addcontentsline{toc}{section}{Лабораторная работа №17}
\section*{\large Лабораторная работа №17\\
Интегрирование обыкновенных дифференциальных уравнений.\\ Метод прогонки.}
\end{center}

\renewcommand{\labelenumi}{\textbf{\arabic{enumi}.}}
\renewcommand{\labelenumii}{\textbf{\arabic{enumi}.\arabic{enumii}}}
\renewcommand{\labelenumiii}{\textbf{\arabic{enumi}.\arabic{enumii}.\arabic{enumiii}}}
\renewcommand{\labelenumiv}{\textbf{\arabic{enumi}.\arabic{enumii}.\arabic{enumiii}.\arabic{enumiv}}}

\begin{enumerate}
\large\item {\large \textbf{Постановка задачи}}
\begin{enumerate}
 \item Численно проинтегрировать дифференциальное уравнение:
\begin{equation*}
x^2y''-xy'-8y=-\alpha\left(x^2\cos{x}-x\sin{x}+8cos{x}\right)-\frac{x^2\sin{x}+3x\cos{x}+5\sin{x}}{2x}
\end{equation*}
с точностью $\varepsilon=10^{-4}$ на интервале $x\in[1,2]$ с шагом, равным 0.01. $\alpha=1+0.001N$.

Граничные условия:
\begin{equation*}
\begin{gathered}
y(1)+y'(1)=\alpha(\sin{1}+\cos{1})+6+\frac{\cos{1}}{2},\\
2y(2)-y'(2)=\alpha(\sin{2}+2\cos{2})+\frac{5}{8}\sin{2}-\frac{\cos{2}}{4}-\frac{3}{4}.
\end{gathered}
\end{equation*}
\item Для решения подобрать устойчивую разностную схему 2 порядка точности (выбор схемы обосновать). 
\item Численно продемонстрировать порядок сходимости численных результатов к точному решению данного дифференциального уравнения (т.е. получить числинные значения для шага h и для $\frac{h}{10}$ и сравнить в одинаковых точках между собой и с точным решением).
\end{enumerate}

\large\item {\large \textbf{Теоретический материал}}
\begin{enumerate}
\item {\large\textbf{Метод прогонки}} \par
\qquad Рассмотрим систему уравнений вида (17.1):
\begin{equation}\tag{17.1}
    \begin{cases}
        y^{''} + p(x) y^{'} + q(x) y = f(x) \\
        \alpha_0 \cdot y(a) + \alpha_1 \cdot y^{'} (a) = A  \\
        \beta_0 \cdot y(b) + \beta_1 \cdot y^{'} (b) = B
    \end{cases}.
\end{equation}
\qquad Введем обозначения (17.2):
\begin{equation}\tag{17.2}
\begin{cases}
    p(x_i) = p_i \\ 
    q(x_i) = q_i \\ 
    f(x_i) = f_i \\
    y(x_i) = y_i
\end{cases}.
\end{equation}

\qquad Составим для системы (17.1) разностную схему [10] (17.3):
\begin{equation}\tag{17.3}
    \frac{y_{i+1} - 2y_i + y_{i-1}}{h^2} + \frac{y_{i+1} - y_{i-1}}{2h} p_i + y_i q_i = f_i.
\end{equation}

\qquad Выразив $y_{i+1}$, получим выражение (17.4):
\begin{equation}\tag{17.4}
    y_{i+1} + y_i \left( \frac{-4 + 2h^2 q_i}{2 + h p_i} \right) + y_{i-1} \left( \frac{2 - h p_i}{2 + h p_i} \right ) = \frac{2 f_i h^2}{2 + h p_i}.
\end{equation}

\qquad Введем обозначения (17.5):
\begin{equation}\tag{17.5}
    m_i = \frac{-4 + 2h^2 q_i}{2 + h p_i}, \ n_i = \frac{2 - h p_i}{2 + h p_i}, \ F_i = \frac{2 f_i}{2 + h p_i},
\end{equation}

с учетом которого выражение (17.4) примет вид (17.6):
\begin{equation}\tag{17.6}
    y_{i+1} + m_i y_i + n_i y_{i-1} = h^2 F_i.
\end{equation}

\qquad Введем (17.7):
\begin{equation}\tag{17.7}
\begin{gathered}    
    y_i = c_i (d_i - y_{i+1}), \\
    y_{i-1} = c_{i-1} (d_{i-1} - y_{i}).
\end{gathered}
\end{equation}


\qquad Подставим это выражение в (17.6) и получим (17.8):
\begin{equation}\tag{17.8}
    y_{i+1} + m_i y_i + n_i c_{i-1} (d_{i-1} - y_i) = h^2 F_i.
\end{equation}
\qquad Выразим из него $y_i$:
\begin{center}
    $y_i = \frac{h^2 F_i - n_i c_{i-1} d_{i-1} - y_{i+1}}{m_i - n_i c_{i-1}}$.
\end{center}
\qquad С учетом выраженного $y_i$ и введенной замены получим, что:
\begin{center}
    $c_i = \frac{1}{m_i - n_i c_{i-1}}, \ d_i = h^2 F_i - n_i c_{i-1} d_{i-1} \ (i=1, 2, \dots, n-1)$.
\end{center}

\qquad С учетом первого граничного условия и выражения (17.6) получим систему (17.9):
\begin{equation}\tag{17.9}
    \begin{cases}
        \alpha_0 y_0 + \alpha_1 \frac{-y_2 + 4 y_1 - 3 y_0}{2h} = A \\
        y_2 + m_1 y_1 + n_1 y_0 = h^2 F_1
    \end{cases}
\end{equation}
Выразим из второго уравнения $y_2$, подставим в первое и, выразив $y_0$, получим:
\begin{center}
    $y_0 = \frac{h(2A + \alpha_1 F_1 h) - (m_1 + 4) \alpha_1 y_1}{2h \alpha_0 + \alpha_1 (n_1 - 3)}$.
\end{center}
С учетом того, что $y_0 = c_0 (d_0 - y_1)$, получим:
\begin{center}
    $c_0 = \frac{(m_1 + 4) \alpha_1}{2h \alpha_0 + \alpha_1 (n_1 - 3)}, \ d_0 = \frac{h(2A + \alpha_1 F_1 h)}{(m_1 + 4) \alpha_1}$.
\end{center}
Затем прямым ходом определяются коэффициенты $c_i$, $d_i$ $(i = 1, 2, \dots , n-1)$.
\qquad Для определения $y_n$ воспользуемся вторым граничным условием и выражением (17.7) и получим систему (17.10):
\begin{equation}\tag{17.10}
    \begin{cases}
        \beta_0 y_n - \beta_1 \frac{3 y_n - 4 y_{n-1} + y_{n-2}}{2h} = B  \\
        y_{n-1} = c_{n-1} (d_{n-1} - y_n)  \\
        y_{n-2} = c_{n-2} (d_{n-2} - y_{n-1})
    \end{cases}.
\end{equation}

Подставив второе и третье уравнение в первое и выразив $y_n$, получим:
\begin{center}
    $y_n = \frac{2Bh - (4+c_{n-2}) c_{n-1} d_{n-1} + c_{n-2} d_{n-2} }{4h - 3 - 4 c_{n-1} - c_{n-2} c_{n-1} }$.
\end{center}

Используя полученное выражение $y_n$ и формулу (17.7), с помощью обратного хода прогонки посчитаем все $y_i$ $(i = n-1, \dots, 1, 0)$.

\item {\large\textbf{Аппроксимация}} \par
\qquad Для исследования схемы на аппроксимацию, подставим выражения для $y^{'}$ и $y^{''}$ в заданное дифференциальное уравнения, в результате чего получим систему вида (17.11):
\begin{equation}\tag{17.11}
    \begin{cases}
        y_{i+1} - 2 y_i + y_{i-1} - \frac{h}{2x_i} (y_{i+1} - y_{i-1}) - \frac{8 h^2}{x^2_i} y_i = \Phi_i \\
        y_0 + \frac{-y_2 + 4 y_1 - 3 y_0}{2h} = A  \\
        2 y_n - \frac{3 y_n - 4 y_{n-1} + y_{n-2}}{2h} = B
    \end{cases}
\end{equation}
После разложения функций $y_{i+1}$, $y_{i-1}$ в ряд Тейлора относительно точки $y_i$, а также после разложения функций $y_{0}$, $y_{2}$ в ряд Тейлора относительно точки $y_1$, мы можем посчитать невязку $\delta f^{(h)}$:
\begin{equation*}
    \delta f^{(h)} = 
    \begin{cases}
        \frac{h^3}{6} \left ( y^{'''}(\xi_1) - y^{'''}(\eta_1) \right ) + \frac{h^4}{12 x_i} \left ( y^{'''}(\xi_1) + y^{'''}(\eta_1) \right ) \\
        \frac{h^2}{3} \left ( y^{'''}(\xi_2) - 2y^{'''}(\xi_3) \right ) \\
        -\frac{2 h^2}{3} y^{'''}(\eta_2) + \frac{4 h^2}{6} y^{'''}(\eta_3) 
    \end{cases},
\end{equation*}
из которой видно, что схема имеет второй порядок аппроксимации.
\item {\large\textbf{Устойчивость}} \par
\qquad Для исследования на устойчивость используем выражение (17.4) и составим матрицу перехода $R_h$:
\begin{center}
    $R_h = \left (
    \begin{matrix}
        \frac{4 - 2 q_i h^2}{2 + p_i h} & -\frac{2 - p_i h}{2 + p_i h}  \\
        1                               &                            0
    \end{matrix} \right). $
\end{center}
Посчитаем нормы первой и второй строк матрицы $R_h$:
\begin{equation*}   
    R_{h_1} = \frac{4 - 2 q_i h^2}{2 + p_i h} - \frac{2 - p_i h}{2 + p_i h} = \frac{2 + p_i h - 2 q_i h^2}{2 + p_i h} = 1 - \frac{2 q_i h^2}{2 + p_i h} \leq 1 + ch,
\end{equation*}

\begin{equation*}
    R_{h_2} = 1 + 0 = 1 \leq 1 + ch,
\end{equation*}
следовательно построенная схема устойчива.

\item {\large\textbf{Точное решение}} \par
\qquad Заданное дифференциальное уравнение будем решать аналитически. \par
\qquad Общее решение уравнения [23] будет представлено в виде (17.12):
\begin{equation}\tag{17.12}
    y(x) = \frac{sin(x)}{2x} + \alpha cos(x) + c_1 x^4 + \frac{c_2}{x^2}.
\end{equation}

Подставляя общее решение в граничные условия, найдем коэффициенты $c_1$, $c_2$:
\begin{center}
    $c_1 = \frac{2 \alpha sin(1) + 5}{5}$ \\
    $c_2 = -1$.
\end{center}

Таким образом общее решение заданного дифференциального уравнения будет иметь вид (17.13):
\begin{equation}\tag{17.13}
    y(x) = \frac{sin(x)}{2x} + \alpha cos(x) + \frac{2 \alpha sin(1) + 5}{5} x^4 - \frac{1}{x^2}.
\end{equation}

\end{enumerate}
\item {\large\textbf{Результаты численных расчётов}} \par
\qquad Вычислим значения функции на $[1,2]$ с шагом 0.01 и 0.001 приближенно и точно и сравним их в одних и тех же точках.

Таблица 1 — Точное и приближенное решения, абсолютная и относительная погрешности при шаге 0.01.
\begin{table}[H]
    \centering
    \resizebox{\textwidth}{!}{%
    \begin{tabular}{|c|c|c|c|c|}
     \hline
$x$ & Точное решение & Приближенное решение & Абсолютная погрешность & Относительная погрешность \\ \hline
    1.0 &    1.298503  &    1.295724  & $2.779152\cdot 10^{-3}$       & $2.140274\cdot 10^{-3}$ \\ \hline
    1.1 &    1.990090  &    1.987724  & $2.779152\cdot 10^{-3}$       & $1.396496\cdot 10^{-3}$ \\ \hline
    1.2 &    2.828873  &    2.826805  & $2.779152\cdot 10^{-3}$       & $9.824239\cdot 10^{-4}$ \\ \hline
    1.3 &    3.865041  &    3.863184  & $2.779152\cdot 10^{-3}$       & $7.190486\cdot 10^{-4}$ \\ \hline
    1.4 &    5.147810  &    5.146095  & $2.779152\cdot 10^{-3}$       & $5.398707\cdot 10^{-4}$ \\ \hline
    1.5 &    6.727045  &    6.725412  & $2.779152\cdot 10^{-3}$       & $4.131312\cdot 10^{-4}$ \\ \hline
    1.6 &    8.654185  &    8.652586  & $2.779152\cdot 10^{-3}$       & $3.211339\cdot 10^{-4}$ \\ \hline
    1.7 &   10.982803  &   10.981191  & $2.779152\cdot 10^{-3}$       & $2.530458\cdot 10^{-4}$ \\ \hline
    1.8 &   13.768946  &   13.767278  & $2.779152\cdot 10^{-3}$       & $2.018420\cdot 10^{-4}$ \\ \hline
    1.9 &   17.071345  &   17.069579  & $2.779152\cdot 10^{-3}$       & $1.627963\cdot 10^{-4}$ \\ \hline
    2.0 &   20.951561  &   20.949654  & $2.779152\cdot 10^{-3}$       & $1.326465\cdot 10^{-4}$ \\ \hline
    \end{tabular}}
\end{table} \par


\begin{figure}[H]
    \centering
    \includegraphics[width=0.9\linewidth]{../pics/laba17_1.png}\par
    Рисунок 1 — График сравнения точного решения и приближенного решения, посчитанного с шагом $h=0.01$.\\
    $1$ — точное решение, $2$ — приближенное решение.
\end{figure}\par

\begin{figure}[H]
    \centering
    \includegraphics[width=0.9\linewidth]{../pics/laba17_1_reltol.png}\par
    Рисунок 2 — График относительной погрешности для приближенного решения, посчитанного с шагом $h=0.01$.\\
    $1$ — относительная погрешность.
\end{figure}\par

\newpage
Таблица 2 — Точное и приближенное решения, абсолютная и относительная погрешности при шаге 0.001.
\begin{table}[H]
    \centering
    \resizebox{\textwidth}{!}{%
    \begin{tabular}{|c|c|c|c|c|c|}
     \hline
     $x$ & Точное решение & Приближенное решение & Абсолютная погрешность & Относительная погрешность & Погрешность между приближенными решениями     \\ \hline
    1.0  &    1.298503  &    1.298475  & $2.779152\cdot 10^{-3}$       & $2.140274\cdot 10^{-3}$  & $2.751297\cdot 10^{-3}$ \\ \hline
    1.1  &    1.990090  &    1.990066  & $2.779152\cdot 10^{-3}$       & $1.396496\cdot 10^{-3}$  & $2.342205\cdot 10^{-3}$ \\ \hline
    1.2  &    2.828873  &    2.828852  & $2.779152\cdot 10^{-3}$       & $9.824239\cdot 10^{-4}$  & $2.047172\cdot 10^{-3}$ \\ \hline
    1.3  &    3.865041  &    3.865023  & $2.779152\cdot 10^{-3}$       & $7.190486\cdot 10^{-4}$  & $1.838576\cdot 10^{-3}$ \\ \hline
    1.4  &    5.147810  &    5.147793  & $2.779152\cdot 10^{-3}$       & $5.398707\cdot 10^{-4}$  & $1.698663\cdot 10^{-3}$ \\ \hline
    1.5  &    6.727045  &    6.727028  & $2.779152\cdot 10^{-3}$       & $4.131312\cdot 10^{-4}$  & $1.615974\cdot 10^{-3}$ \\ \hline
    1.6  &    8.654185  &    8.654169  & $2.779152\cdot 10^{-3}$       & $3.211339\cdot 10^{-4}$  & $1.583207\cdot 10^{-3}$ \\ \hline
    1.7  &   10.982803  &   10.982787  & $2.779152\cdot 10^{-3}$       & $2.530458\cdot 10^{-4}$  & $1.595895\cdot 10^{-3}$ \\ \hline
    1.8  &   13.768946  &   13.768929  & $2.779152\cdot 10^{-3}$       & $2.018420\cdot 10^{-4}$  & $1.651565\cdot 10^{-3}$ \\ \hline
    1.9  &   17.071345  &   17.071328  & $2.779152\cdot 10^{-3}$       & $1.627963\cdot 10^{-4}$  & $1.749191\cdot 10^{-3}$ \\ \hline
    2.0  &   20.951561  &   20.951542  & $2.779152\cdot 10^{-3}$       & $1.326465\cdot 10^{-4}$  & $1.888827\cdot 10^{-3}$ \\ \hline
    \end{tabular}}
\end{table} \par

\begin{figure}[H]
    \centering
    \includegraphics[width=0.9\linewidth]{../pics/laba17_2.png}\par
    Рисунок 3 — График сравнения точного решения и приближенного решения, посчитанного с шагом $h=0.001$.\\
    $1$ — точное решение, $2$ — приближенное решение.
\end{figure}\par

\begin{figure}[H]
    \centering
    \includegraphics[width=0.9\linewidth]{../pics/laba17_2_reltol.png}\par
    Рисунок 4 — График относительной погрешности для приближенного решения, посчитанного с шагом $h=0.001$.\\
    $1$ — относительная погрешность.
\end{figure}\par


\item {\large\textbf{Вывод}} \par

\qquad В ходе выполнения лабораторной работы была построена разностная схема второго порядка точности, которая была исследована на аппроксимацию и устойчивость, и по которой методом прогонки были посчитаны приближенные значения заданного дифференциального уравнения. Затем аналитически было посчитано точное решение заданного дифференциального уравнения и проведено сравнение точного решения с приближенным.

\end{enumerate}

\newpage
\begin{center}
\refstepcounter{section} %гиперссылка
\addcontentsline{toc}{section}{Лабораторная работа №18}
\section*{\large Лабораторная работа №18\\
Интегрирование обыкновенных дифференциальных уравнений.\\ Разностная схема 4 порядка.}
\end{center}

\renewcommand{\labelenumi}{\textbf{\arabic{enumi}.}}
\renewcommand{\labelenumii}{\textbf{\arabic{enumi}.\arabic{enumii}}}
\renewcommand{\labelenumiii}{\textbf{\arabic{enumi}.\arabic{enumii}.\arabic{enumiii}}}
\renewcommand{\labelenumiv}{\textbf{\arabic{enumi}.\arabic{enumii}.\arabic{enumiii}.\arabic{enumiv}}}

\begin{enumerate}
\large\item {\large \textbf{Постановка задачи}}
\begin{enumerate}
\item Для задачи:
$$
\begin{gathered}
-\frac{d^2y}{dx^2}+(2b-3)y=2(b-2)xe^x;\\
y(0)=\frac{1}{b-2}; y'(0)=\frac{b-1}{b-2};
\end{gathered}
$$
\begin{enumerate}
\item На трехточечном шаблоне с постоянным шагом построить методом неопределенных коэффициентов устойчивую разностную схему 4 порядка точности
\item По полученной разностной схеме численно проинтегрировать уравнение с точностью $\varepsilon=10^{-4}$ на интервале $x\in[0,1]$ с шагом, равным 0.1. Положим параметр $b=1+0.001N$ для конкретизации задачи. Сравнить полученные результаты с точным решением уравнения в точках $x=0.1;0.5;1.$
\end{enumerate}
\end{enumerate}

\large\item {\large \textbf{Теоретический материал}}
\begin{enumerate}
\item {\large\textbf{Построение разностной схемы}} \par
\qquad Построим разностную схему четвертого порядка точности по трехточечному. Для этого введем обозначение (18.1):
\begin{equation}\tag{18.1}
    \frac{d^2y}{dx^2} = f(x,y) = (2b-3)y - 2(b-2)x e^x    
\end{equation}
и, используя метод неопределенных коэффициентов, составим схему (18.2):
\begin{equation}\tag{18.2}
    b_1y_{i-1}+b_2y_i+b_3y_{i+1}=a_1f_{i-1}+a_2f_i+a_3f_{i+1}.
\end{equation}
\qquad Разложим в ряд Тейлора функции $y_{i+1}$, $y_{i-1}$, $f_{i+1}$, $f_{i-1}$ относительно точки $y_i$ и $f_i$ соотвественно, а также будем учитывать, что $y^{'''} = f^{'}$ и $y^{''''} = f^{''}$. \par
\qquad Тогда, подставив все функции в выражение (18.2), получим (18.3):
\begin{equation}\tag{18.3}
\begin{gathered}
    b_{1}\left(y_{i}-hy_{i}'+\frac{h^{2}}{2}y_{i}''-\frac{h^{3}}{6}y_{i}'''+\frac{h^{4}}{24}y_{i}''''-\frac{h^{5}}{120}y_{i}'''''\right) + \\
    + b_{2}y_{i} + \\
    + b_{3}\left(y_{i}+hy_{i}'+\frac{h^{2}}{2}y_{i}''+\frac{h^{3}}{6}y_{i}'''+\frac{h^{4}}{24}y_{i}''''+\frac{h^{5}}{120}y_{i}'''''\right) = \\
    = a_{1}\left(y_{i}''-hy_{i}'''+\frac{h^{2}}{2}y_{i}''''-\frac{h^{3}}{6}y_{i}'''''\right) + \\
    + a_{2}y_{i}'' + \\
    + a_{3}\left(y_{i}''+hy_{i}'''+\frac{h^{2}}{2}y_{i}''''+\frac{h^{3}}{6}y_{i}'''''\right) + O(h^5).
\end{gathered}
\end{equation}
\qquad Для того, чтобы схема была четвертого порядка точности, необходимо выполнение системы (18.4):
\begin{equation}\tag{18.4}
    \begin{cases}
        b_{1}+b_{2}+b_{3}=0 \\
        -b_{1}+b_{3}=0 \\
        \frac{h^{2}}{2}\left(b_{1}+b_{3}\right)=a_{1}+a_{2}+a_{3} \\
        \frac{h^{2}}{6}\left(-b_{1}+b_{3}\right)=-a_{1}+a_{3} \\
        \frac{h^{2}}{12}\left(b_{1}+b_{3}\right)=a_{1}+a_{3} \\
        \frac{h^{2}}{20}\left(-b_{1}+b_{3}\right)=-a_{1}+a_{3}
    \end{cases},
\end{equation}
которая в итоге сводится к системе (18.5):
\begin{equation}\tag{18.5}
    \begin{cases}
        a_{1}=\frac{h^{2}}{12}b_{1}  \\
        a_{2}=\frac{5h^{2}}{6}b_{1}  \\
        a_{3}=\frac{h^{2}}{12}b_{1} \\
        b_{2}=-2b_{1} \\
        b_{3}=b_{1} \\
    \end{cases}.
\end{equation}

\qquad Итоговая схема имеет вид (18.6):
\begin{equation}\tag{18.6}
        y_{i+1} -2y_i + y_{i-1} = \frac{h^2}{12} (f_{i+1} + 10f_{i} + f_{i-1}).
\end{equation}

\item {\large\textbf{Устойчивость}} \par
\qquad Для исследования разностной схемы на устойчивость сначала необходимо привести её к каноническому виду. Для этого подставим в выражение (18.6) значения функций $f_{i-1}$, $f_i$ и $f_{i+1}$ и получим (18.7):
\begin{equation}\tag{18.7}
\begin{gathered}
    y_{i+1} -2y_i + y_{i-1} = \frac{h^2}{12} \cdot \Big( (2b-3)y_{i+1} -2(b-2)x_{i+1} e^{x_{i+1}} + (2b-3)y_{i} -\\ 
    -2(b-2)x_{i} e^{x_{i}} + (2b-3)y_{i-1} -2(b-2)x_{i-1} e^{x_{i-1}} \Big ).
\end{gathered}
\end{equation}

\qquad Выражая из (18.7) $y_{i+1}$ получим канонический вид (18.8):
    
\begin{equation}\tag{18.8}
\begin{gathered}
    y_{i+1} = y_i \frac{24 + 10h^2 (2b-3)}{12 - h^2 (2b-3)} - y_{i-1} -\\
    - \frac{2h^2 (b-2) (x_{i+1} e^{x_{i+1}} + 10 x_i e^{x_i} + x_{i-1} e^{x_{i-1}}}{12 - h^2 (2b-3)}.
\end{gathered}
\end{equation}
\\

\qquad Составим матрицу перехода $R_h$:

\begin{center}
    $R_h = \left(
    \begin{matrix}
        \frac{24 + 10h^2 (2b-3)}{12 - h^2 (2b-3)} & -1  \\
        1                                         & 0
    \end{matrix}\right ) $.
\end{center}
\qquad Вычислим собственные значения матрицы $R_h$, для чего разрешим характеристическое уравнение (18.9):
\begin{equation}\tag{18.9}
    \lambda ^2 - \frac{24 + 10h^2 (2b-3)}{12 - h^2 (2b-3)} \lambda + 1 = 0.
\end{equation}

\qquad Получим корни уравнения (18.10):
\begin{equation}\tag{18.10}
    \lambda_{1,2} = \frac{\frac{24 + 10h^2 (2b-3)}{12 - h^2 (2b-3)} \pm \sqrt{\left(\frac{24 + 10h^2 (2b-3)}{12 - h^2 (2b-3)} \right) ^2 - 4}}{2}.
\end{equation}
\qquad Поскольку $\lim\limits_{h \to 0} \lambda_{1,2} = 1$, то выполняется неравенство (18.11):
\begin{equation}\tag{18.11}
    |\lambda_{1,2}| < 1 + ch,    
\end{equation}
из которого следует, что схема устойчива.

\item {\large\textbf{Первое приближение}} \par
\qquad Для того, чтоб начать считать значения $y_i$ по схеме (18.8), необходимо знать значения $y_0$ и $y_1$. $y_0$ дано при постановке задачи. Чтобы посчитать $y_1$, разложим его в ряд Тейлора по формуле (18.12):
\begin{equation}\tag{18.12}
    y_1 = y_0 + y^{'}_0 h + \frac{y^{''}_0}{2} h^2 + \frac{y^{'''}_0}{6} h^3 + \frac{y^{''''}_0}{24} h^4.
\end{equation}

\qquad Используя начальные дифференциальное уравнение и условия, посчитаем $y^{''}_0$, $y^{'''}_0$, $y^{''''}_0$:
\begin{center}
    $y^{''}_0 = \frac{2b-3}{b-2}$, $ y^{'''}_0 = \frac{3b-5}{b-2}$, $y^{''''}_0 = \frac{4b-7}{b-2}$.
\end{center}
\qquad Таким образом получим формулу (18.13) для первого приближения $y_1$:
\begin{equation}\tag{18.13}
    y_1 = \frac{1}{b-2} + \frac{b-1}{b-2} \cdot h + \frac{2b-3}{b-2} \cdot \frac{h^2}{2} + \frac{3b-5}{b-2} \cdot \frac{h^3}{6} + \frac{4b-7}{b-2}\cdot  \frac{h^4}{24}.
\end{equation}


\item {\large\textbf{Точное решение}} \par
\qquad Посчитаем точное решение заданного дифференциального уравнения [23], составив для него характеристическое уравнение
\begin{center}
    $y^{''} - (2b-3)y = 0$  \\
    $\lambda^{2} - (2b-3) = 0$  \\
    $\lambda^2 = 2b-3$  \\
    $\lambda_{1,2} = \pm \sqrt{2b-3}$  \\
    $y = c_1 e^{x \lambda_1 } + c_2 e^{x \lambda_2}$
\end{center}
\qquad Частное решение $y_0$ будем искать в виде $y_0 = (cx+d) e^x$. Выразив $y^{''}_0$ и подставив всё в заданное дифференциальное уравнение, посчитаем коэффициенты $c$ и $d$:
\begin{center}
    $c = 1$, $d = \frac{1}{b-2}$.
\end{center}
\qquad Таким образом общее решение заданного дифференциального уравнения будет иметь вид (18.14):
\begin{equation}\tag{18.14}
    y = c_1 e^{x \lambda_1 } + c_2 e^{x \lambda_2} + x e^x + \frac{1}{b-2} e^x.
\end{equation}
\qquad Подставив в уравнение (18.14) заданные начальные условия, найдем итоговое решение заданного дифференциального уравнения:
\begin{center}
    $y = \left (x + \frac{1}{b-2} \right ) e^x$.
\end{center}
\end{enumerate}

\item {\large\textbf{Результаты численных расчётов}} \par
\qquad Вычислим точное и приближенное решения заданного дифференциального уравнения в точках $x = 0.1$, $x = 0.5$, $x = 1.0$.\par
Результаты представлены в таблице 1.

Таблица 1 — Точное и приближенное решения, абсолютная и относительная погрешности.
\begin{table}[H]
    \centering
    \resizebox{\textwidth}{!}{%
    \begin{tabular}{|c|c|c|c|c|}
     \hline
     $x$ & Точное решение & Приближенное решение & Абсолютная погрешность & Относительная погрешность   \\ \hline
    0.1  &   -0.995760  &   -0.995760  &    $0.340314\cdot 10^{-6}$  &   $-0.341763\cdot 10^{-6}$ \\ \hline
    0.5  &   -0.826011  &   -0.826012  &    $0.137168\cdot 10^{-5}$  &   $-0.166061\cdot 10^{-5}$ \\ \hline
    1.0  &   -0.002721  &   -0.002722  &    $0.145687\cdot 10^{-5}$  &   $-5.354158\cdot 10^{-4}$ \\ \hline
    \end{tabular}}
\end{table} \par

\item {\large\textbf{Вывод}} \par

\qquad В ходе выполнения лабораторной работы методом неопределенных коэффициентов была выведена разностная схема четвертого порядка, которая была исследована на устойчивость. Затем было посчитано точное решение заданного дифференциального уравнения и проведено сравнение точного решения с приближенным.

\end{enumerate}

\newpage
\begin{center}
\refstepcounter{section} %гиперссылка
\addcontentsline{toc}{section}{Лабораторная работа №19}
\section*{\large Лабораторная работа №19\\
Решение уравнения переноса.\\ Неявная разностная схема.}
\end{center}

\renewcommand{\labelenumi}{\textbf{\arabic{enumi}.}}
\renewcommand{\labelenumii}{\textbf{\arabic{enumi}.\arabic{enumii}}}
\renewcommand{\labelenumiii}{\textbf{\arabic{enumi}.\arabic{enumii}.\arabic{enumiii}}}
\renewcommand{\labelenumiv}{\textbf{\arabic{enumi}.\arabic{enumii}.\arabic{enumiii}.\arabic{enumiv}}}

\begin{enumerate}
\large\item {\large \textbf{Постановка задачи}}
\begin{enumerate}
 \item Для дифференциальной задачи:
$$
\begin{gathered}
\frac{\partial u}{\partial t}+\frac{\alpha}{2\alpha+1}\frac{\partial u}{\partial x}=\alpha t;\\
u(x,0)=3+(2\alpha+1)x; 0\le x\le1;\\
u(0,t)=3-\alpha t+\frac{\alpha t^2}{2}; 0\le t\le1;\\
\alpha=-\frac{1}{2}+0.001N,
\end{gathered}
$$
постороить разностную схему (теоретически обосновать аппроксимацию и устойчивость) на шаблоне:
%\begin{center}
%\begin{figure}[H]
%	\centering
%	\includegraphics[scale=0.8]{diff_scheme}
%	\caption{Шаблон для разностной схемы}
%	\label{fig:mpr}
%\end{figure}
%\end{center}

Для полученной схемы написать расчетные формулы алгоритма и численно решить задачу в квадрате $[0\le x\le1]\times[0\le t \le1]$. Полученное численное решение сравнить с точным в тех же точках, в которых получено численное решение.
\end{enumerate}

\large\item {\large \textbf{Теоретический материал}}
\begin{enumerate}
\qquad Воспользуемся шаблоном разностной схемой для численого решения дифференциальной задачи [3,10,18]:
\begin{figure}[H]
    \centering
    \includegraphics[width=0.3\linewidth]{../pics/corner_scheme.png}\par
    Рисунок 1 - Шаблон разностной схемы.\\
\end{figure}
\item {\large\textbf{Построение разностной схемы}} \par

\qquad Используя шаблон, заменим частные производные конечными разностями:
\begin{equation*}
    \frac{\partial u}{\partial t} = \frac{u_{m-1}^n - u_{m-1}^{n-1}}{\tau}, \ \frac{\partial u}{\partial x} = \frac{u_{m}^n - u_{m-1}^{n}}{h}.
\end{equation*}

Тогда разностная схема примет вид:
\begin{equation*}
    \frac{u_{m-1}^n - u_{m-1}^{n-1}}{\tau} + \frac{\alpha}{2\alpha+1} \cdot \frac{u_{m}^n - u_{m-1}^{n}}{h}=\alpha n \tau.
\end{equation*}

\item {\large\textbf{Аппроксимация}} \par
\qquad Для исследования на аппроксимацию разложим функции $u_{m}^{n}$ и $u_{m-1}^{n-1}$ в ряд Тейлора в окрестности точки $u_{m-1}^{n}=u$:
\begin{equation*}
    \begin{gathered}
        u_m^n = u + h \frac{\partial u}{\partial x} + \frac{h^2}{2} \cdot \frac{\partial^2 u}{\partial x^2} + \dots  \\
        u_{m-1}^{n-1} = u - \tau \frac{\partial u}{\partial t} + \frac{\tau^2}{2} \cdot \frac{\partial^2 u}{\partial t^2} + \dots  
    \end{gathered}
\end{equation*}
и подставим их в исходную схему:
\begin{equation*}
    \frac{u - (u - \tau \frac{\partial u}{\partial t} + \frac{\tau^2}{2} \cdot \frac{\partial^2 u}{\partial t^2} + \dots)}{\tau} + \frac{\alpha}{2\alpha+1} \frac{u + h \frac{\partial u}{\partial x} + \frac{h^2}{2} \cdot \frac{\partial^2 u}{\partial x^2} + \dots - u}{h} = \alpha n \tau.
\end{equation*}

Выполнив преобразования, останется:
\begin{equation*}
    - \frac{\tau}{2} \frac{\partial^2 u}{\partial t^2} + o(\tau^2) + \frac{\alpha}{2\alpha+1} \frac{h}{2} \frac{\partial^2 u}{\partial x^2} + o(h^2) = 0,
\end{equation*}
откуда следует, что заданная схема имеет порядок аппроксимации $o(\tau + h)$.

\item {\large\textbf{Устойчивость. Принцип минимума/максимума}} \par
\qquad Пусть имеется двухслойная разностная схема (19.1):
\begin{equation}\tag{19.1}
    \sum\limits_{k} \alpha_{k+m} u_{k+m}^n = \sum\limits_{l} \beta_{l+m} u_{l+m}^{n-1} + \varphi_m^{n-1}.
\end{equation}
\qquad Для того, чтоб разностная схема (19.1) была устойчива, необходимо выполнение неравенства [3,21] (19.2):
\begin{equation}\tag{19.2}
    |\alpha_m| - \sum\limits_{k\neq 0} |\alpha_{k+m}| \geq \sum\limits_{l} |\beta_{l+m}|.
\end{equation}

\qquad Если для схемы (19.1) выполняется неравенство (19.3):
\begin{equation}\tag{19.3}
    |\alpha_m| - \sum\limits_{k\neq 0} |\alpha_{k+m}| \geq \frac{\chi}{\tau},
\end{equation}
то схема считается устойчивой по начальным данным.

\qquad Теперь рассмотрим нашу разностную схему и домножим ее слева и справа на $\tau$:
\begin{equation*}
    u_{m-1}^n + \frac{\alpha \tau }{h (2\alpha + 1)} (u_m^n - u_{m-1}^n) + \alpha n \tau^2 + u_{m-1}^{n-1}.
\end{equation*}
Запишем число Куранта $K$:
\begin{equation*}
    K = \frac{\alpha \tau }{h (2\alpha + 1)}.
\end{equation*}
Тогда неравенство (19.2) примет вид (19.4):
\begin{equation*}\tag{19.4}
    |K| - |1-K| \geq 1.
\end{equation*}
Решая данное неравенство получим, что наша разностная схема устойчива при:
\begin{equation*}
    \frac{\alpha \tau }{h (2\alpha + 1)} \geq 1.
\end{equation*}
Как видим, т.к. устойчивость нашей схемы зависит от выбора шагов $\tau$ и $h$, то наша схема считается условно устойчивой.

\item {\large\textbf{Точное решение}} \par

\qquad Перепишем начальное дифференциальное уравнение в виде [23]:
\begin{equation*}
    \frac{dt}{1} = \frac{dx}{\frac{\alpha }{2 \alpha + 1}} = \frac{du}{\alpha t},
\end{equation*}
который можно записать в виде системы:
\begin{equation*}
    \begin{cases}
        dt = \frac{du}{\alpha t}  \\
        dt = \frac{dx}{\frac{\alpha}{2 \alpha + 1}}
    \end{cases}
    \to
    \begin{cases}
        u = \frac{\alpha t^2}{2} + c_1  \\
        x = \frac{\alpha t}{2 \alpha + 1} + c_2
    \end{cases}
    \to
    \begin{cases}
        c_1 = u - \frac{\alpha t^2}{2}  \\
        c_2 = x - \frac{\alpha t}{2 \alpha + 1}
    \end{cases}
\end{equation*}
Общее решение будет иметь вид:
\begin{equation*}
    u(x,t) = \frac{\alpha t^2}{2} + f(x - \frac{\alpha t}{2 \alpha + 1}).
\end{equation*}

Подставляя это выражение в начальное условие получим:
\begin{equation*}
    f(x) = 3 + (2 \alpha + 1) x.
\end{equation*}
Делая замену $x = x - \frac{\alpha t}{2 \alpha + 1}$, получим:
\begin{equation*}
    f(x - \frac{\alpha t}{2 \alpha + 1}) = 3 + (2 \alpha + 1)x - \alpha t.
\end{equation*}
Таким образом, общее решение заданного дифференциального уравнения будет иметь вид:
\begin{equation*}
    u(x,t) = 3 + (2 \alpha + 1)x + \frac{\alpha t^2}{2} - \alpha t.
\end{equation*}
\end{enumerate}

\item {\large\textbf{Результаты численных расчётов}} \par
\qquad Вычислим значения функции с шагами $h=0.1$ и $\tau=0.2$ приближенно и точно и сравним их в одних и тех же точках.
Результаты представлены в таблицах 1, 2, 3, 4. \par
Таблица 1 — Приближенное решение. \par
\begin{table}[H]
    \centering
\begin{tabular}{|c|c|c|c|c|c|c|}   \hline
  \diagbox{x}{t} &       0.0 &       0.2 &       0.4 &       0.6 &       0.8 &       1.0   \\ \hline
 0.0 &  3.000000  &  3.089820  &  3.159680  &  3.209580  &  3.239520  &  3.249500   \\ \hline
 0.1 &  3.000200  &  3.090040  &  3.159900  &  3.209800  &  3.239740  &  3.249720   \\ \hline
 0.2 &  3.000400  &  3.090260  &  3.160120  &  3.210020  &  3.239960  &  3.249940   \\ \hline
 0.3 &  3.000600  &  3.090480  &  3.160340  &  3.210240  &  3.240180  &  3.250160   \\ \hline
 0.4 &  3.000800  &  3.090700  &  3.160560  &  3.210460  &  3.240400  &  3.250380   \\ \hline
 0.5 &  3.001000  &  3.090920  &  3.160780  &  3.210680  &  3.240620  &  3.250600   \\ \hline
 0.6 &  3.001200  &  3.091141  &  3.161000  &  3.210900  &  3.240840  &  3.250820   \\ \hline
 0.7 &  3.001400  &  3.091361  &  3.161220  &  3.211120  &  3.241060  &  3.251040   \\ \hline
 0.8 &  3.001600  &  3.091581  &  3.161440  &  3.211340  &  3.241280  &  3.251260   \\ \hline
 0.9 &  3.001800  &  3.091801  &  3.161660  &  3.211560  &  3.241500  &  3.251480   \\ \hline
 1.0 &  3.002000  &  3.092022  &  3.161880  &  3.211780  &  3.241720  &  3.251700   \\ \hline
\end{tabular}
\end{table} \par

Таблица 2 — Точное решение. \par
\begin{table}[H]
    \centering
\begin{tabular}{|c|c|c|c|c|c|c|}   \hline
  \diagbox{x}{t} &       0.0 &       0.2 &       0.4 &       0.6 &       0.8 &       1.0   \\ \hline
 0.0 & 3.000000  &  3.089820  &  3.159680  &  3.209580  &  3.239520  &  3.249500   \\ \hline
 0.1 & 3.000200  &  3.090020  &  3.159880  &  3.209780  &  3.239720  &  3.249700   \\ \hline
 0.2 & 3.000400  &  3.090220  &  3.160080  &  3.209980  &  3.239920  &  3.249900   \\ \hline
 0.3 & 3.000600  &  3.090420  &  3.160280  &  3.210180  &  3.240120  &  3.250100   \\ \hline
 0.4 & 3.000800  &  3.090620  &  3.160480  &  3.210380  &  3.240320  &  3.250300   \\ \hline
 0.5 & 3.001000  &  3.090820  &  3.160680  &  3.210580  &  3.240520  &  3.250500   \\ \hline
 0.6 & 3.001200  &  3.091020  &  3.160880  &  3.210780  &  3.240720  &  3.250700   \\ \hline
 0.7 & 3.001400  &  3.091220  &  3.161080  &  3.210980  &  3.240920  &  3.250900   \\ \hline
 0.8 & 3.001600  &  3.091420  &  3.161280  &  3.211180  &  3.241120  &  3.251100   \\ \hline
 0.9 & 3.001800  &  3.091620  &  3.161480  &  3.211380  &  3.241320  &  3.251300   \\ \hline
 1.0 & 3.002000  &  3.091820  &  3.161680  &  3.211580  &  3.241520  &  3.251500   \\ \hline
\end{tabular}
\end{table} \par
\newpage
Таблица 3 — Абсолютная погрешность. \par
\begin{table}[H]
    \centering
    \resizebox{\textwidth}{!}{%
\begin{tabular}{|c|c|c|c|c|c|c|}   \hline
  \diagbox{x}{t} & 0.0 & 0.2 & 0.4 & 0.6 & 0.8 & 1.0   \\ \hline
 0.0 & 0.0  &  0.0  &  0.0 &  0.0 &  0.0 &  0.0 \\ \hline
 0.1 & 0.0 &  $0.020000\cdot 10^{-3}$  &  $0.020000\cdot 10^{-3}$  &  $0.020000\cdot 10^{-3}$  &  $0.020000\cdot 10^{-3}$  &  $0.020000\cdot 10^{-3}$   \\ \hline
 0.2 & 0.0 &  $0.040040\cdot 10^{-3}$  &  $0.040000\cdot 10^{-3}$  &  $0.040000\cdot 10^{-3}$  &  $0.040000\cdot 10^{-3}$  &  $0.040000\cdot 10^{-3}$   \\ \hline
 0.3 & 0.0 &  $0.060120\cdot 10^{-3}$  &  $0.060000\cdot 10^{-3}$  &  $0.060000\cdot 10^{-3}$  &  $0.060000\cdot 10^{-3}$  &  $0.060000\cdot 10^{-3}$   \\ \hline
 0.4 & 0.0 &  $0.080241\cdot 10^{-3}$  &  $0.080000\cdot 10^{-3}$  &  $0.080000\cdot 10^{-3}$  &  $0.080000\cdot 10^{-3}$  &  $0.080000\cdot 10^{-3}$   \\ \hline
 0.5 & 0.0 &  $0.100402\cdot 10^{-3}$  &  $0.099999\cdot 10^{-3}$  &  $0.100000\cdot 10^{-3}$  &  $0.100000\cdot 10^{-3}$  &  $0.100000\cdot 10^{-3}$   \\ \hline
 0.6 & 0.0 &  $0.120603\cdot 10^{-3}$  &  $0.119998\cdot 10^{-3}$  &  $0.120000\cdot 10^{-3}$  &  $0.120000\cdot 10^{-3}$  &  $0.120000\cdot 10^{-3}$   \\ \hline
 0.7 & 0.0 &  $0.140845\cdot 10^{-3}$  &  $0.139997\cdot 10^{-3}$  &  $0.140000\cdot 10^{-3}$  &  $0.140000\cdot 10^{-3}$  &  $0.140000\cdot 10^{-3}$   \\ \hline
 0.8 & 0.0 &  $0.161127\cdot 10^{-3}$  &  $0.159995\cdot 10^{-3}$  &  $0.160000\cdot 10^{-3}$  &  $0.160000\cdot 10^{-3}$  &  $0.160000\cdot 10^{-3}$   \\ \hline
 0.9 & 0.0 &  $0.181450\cdot 10^{-3}$  &  $0.179993\cdot 10^{-3}$  &  $0.180000\cdot 10^{-3}$  &  $0.180000\cdot 10^{-3}$  &  $0.180000\cdot 10^{-3}$   \\ \hline
 1.0 & 0.0 &  $0.201813\cdot 10^{-3}$  &  $0.199990\cdot 10^{-3}$  &  $0.200000\cdot 10^{-3}$  &  $0.200000\cdot 10^{-3}$  &  $0.200000\cdot 10^{-3}$   \\ \hline
\end{tabular}}
\end{table} \par
Таблица 4 — Относительная погрешность. \par
\begin{table}[H]
    \centering
    \resizebox{\textwidth}{!}{%
\begin{tabular}{|c|c|c|c|c|c|c|}   \hline
  \diagbox{x}{t} & 0.0 & 0.2 & 0.4 & 0.6 & 0.8 & 1.0   \\ \hline
 0.0 & 0.0 &  0.0 &  0.0 &  0.0 &  0.0 &  0.0 \\ \hline
 0.1 & 0.0 &  $0.064725\cdot 10^{-4}$  &  $0.063294\cdot 10^{-4}$  &  $0.062310\cdot 10^{-4}$  &  $0.061734\cdot 10^{-4}$  &  $0.061544\cdot 10^{-4}$   \\ \hline
 0.2 & 0.0 &  $0.129570\cdot 10^{-4}$  &  $0.126579\cdot 10^{-4}$  &  $0.124611\cdot 10^{-4}$  &  $0.123460\cdot 10^{-4}$  &  $0.123081\cdot 10^{-4}$   \\ \hline
 0.3 & 0.0 &  $0.194538\cdot 10^{-4}$  &  $0.189856\cdot 10^{-4}$  &  $0.186905\cdot 10^{-4}$  &  $0.185178\cdot 10^{-4}$  &  $0.184610\cdot 10^{-4}$   \\ \hline
 0.4 & 0.0 &  $0.259627\cdot 10^{-4}$  &  $0.253125\cdot 10^{-4}$  &  $0.249192\cdot 10^{-4}$  &  $0.246889\cdot 10^{-4}$  &  $0.246131\cdot 10^{-4}$   \\ \hline
 0.5 & 0.0 &  $0.324838\cdot 10^{-4}$  &  $0.316385\cdot 10^{-4}$  &  $0.311470\cdot 10^{-4}$  &  $0.308592\cdot 10^{-4}$  &  $0.307645\cdot 10^{-4}$   \\ \hline
 0.6 & 0.0 &  $0.390172\cdot 10^{-4}$  &  $0.379636\cdot 10^{-4}$  &  $0.373741\cdot 10^{-4}$  &  $0.370288\cdot 10^{-4}$  &  $0.369151\cdot 10^{-4}$   \\ \hline
 0.7 & 0.0 &  $0.455628\cdot 10^{-4}$  &  $0.442878\cdot 10^{-4}$  &  $0.436004\cdot 10^{-4}$  &  $0.431976\cdot 10^{-4}$  &  $0.430650\cdot 10^{-4}$   \\ \hline
 0.8 & 0.0 &  $0.521206\cdot 10^{-4}$  &  $0.506110\cdot 10^{-4}$  &  $0.498259\cdot 10^{-4}$  &  $0.493657\cdot 10^{-4}$  &  $0.492141\cdot 10^{-4}$   \\ \hline
 0.9 & 0.0 &  $0.586908\cdot 10^{-4}$  &  $0.569332\cdot 10^{-4}$  &  $0.560507\cdot 10^{-4}$  &  $0.555329\cdot 10^{-4}$  &  $0.553625\cdot 10^{-4}$   \\ \hline
 1.0 & 0.0 &  $0.652733\cdot 10^{-4}$  &  $0.632544\cdot 10^{-4}$  &  $0.622747\cdot 10^{-4}$  &  $0.616994\cdot 10^{-4}$  &  $0.615101\cdot 10^{-4}$   \\ \hline
\end{tabular}}
\end{table} \par

\item {\large\textbf{Вывод}} \par

\qquad В ходе выполнения лабораторной работы была построена разностная схема, которая была исследована на аппроксимацию и устойчивость, и по которой в квадрате $[0\le x\le1]\times[0\le t \le1]$ были посчитаны приближенные значения заданного дифференциального уравнения. Затем аналитически было посчитано точное решение заданного дифференциального уравнения и проведено сравнение точного решения с приближенным.

\end{enumerate}

\newpage
\begin{center}
\refstepcounter{section} %гиперссылка
\addcontentsline{toc}{section}{Лабораторная работа №20}
\section*{\large Лабораторная работа №20\\
Решение уравнения теплопроводности.\\ Разностная схема повышенной точности.}
\end{center}

\renewcommand{\labelenumi}{\textbf{\arabic{enumi}.}}
\renewcommand{\labelenumii}{\textbf{\arabic{enumi}.\arabic{enumii}}}
\renewcommand{\labelenumiii}{\textbf{\arabic{enumi}.\arabic{enumii}.\arabic{enumiii}}}
\renewcommand{\labelenumiv}{\textbf{\arabic{enumi}.\arabic{enumii}.\arabic{enumiii}.\arabic{enumiv}}}

\begin{enumerate}
\large\item {\large \textbf{Постановка задачи}}
\begin{enumerate}
 \item Для дифференциальной задачи:
$$
\begin{gathered}
\frac{\partial u}{\partial t}=\frac{\partial^2 u}{\partial x^2},\;\;\;\;\;\;0<x<1,\;\;0<t<\infty\\
u(x,0)=2\alpha, \;\;\;\;\;\;u(0,t)=3\alpha,\;\;u(1,t)=\alpha;
\end{gathered}
$$
методом конечных разностей численно найти решение при $t=20$. Для решения использовать разностную схему:
$$
\begin{gathered}
\frac{1}{\tau}(u_m^{n+1}-u_m^n)=
\frac{\sigma}{h^2}(u_{m+1}^{n+1}-2u_m^{n+1}+u_{m-1}^{n+1})+\frac{1-\sigma}{h^2}(u_{m+1}^n-2u_m^n+u_{m-1}^n) \\
1 \le m \le M - 1; n = 0, 1, 2, \dots; \sigma = \frac{3}{2 \alpha(\alpha+2)}, \\
u_m^0 = 2\alpha,
u_0^n = 3\alpha,
u_M^n = \alpha, \;\;\;
(1 \le m \le M - 1);
\end{gathered}
$$
Положим $\alpha=-2.8+0.5N$.

\qquad За счет подбора  соотношения между $\tau$  (шаг по времени) и  $h$ (шаг по пространству) использовать при численном решении схему повышенной точности.

\qquad Полученное численное решение сравнить с точным решением в тех же точках по переменной  $x$ при $t=20$ , в которых получено численное решение.
\end{enumerate}

\large\item {\large \textbf{Теоретический материал}}
\begin{enumerate}

\qquad Воспользуемся шаблоном разностной схемой для численого решения дифференциальной задачи:
\begin{figure}[H]
    \centering
    \includegraphics[width=0.3\linewidth]{../pics/H_scheme.png}\par
    Рисунок 1 — Шаблон разностной схемы.\\
\end{figure}
\item {\large\textbf{Аппроксимация}} \par
\qquad Для исследования схемы на аппроксимацию разложим функции $u_{m-1}^{n}$, $u_{m}^{n}$, $u_{m+1}^{n}$, $u_{m-1}^{n+1}$, $u_{m}^{n+1}$, $u_{m+1}^{n+1}$,  в ряд Тейлора в окрестности точки $u_{m}^{n+\frac{1}{2}}=u$:

\begin{flushleft}
$
u_{m-1}^{n}   = u - \frac{\tau}{2} u_t - h u_x + \frac{\tau^2}{8} u_{tt} + \frac{\tau}{2} h u_{tx} + \frac{h^2}{2} u_{xx} - \frac{\tau^3}{48} u_{ttt} - \frac{\tau^2}{8} h u_{ttx} - \frac{\tau}{4} h^2 u_{txx} - \ - \frac{h^3}{6} u_{xxx} + \frac{\tau^4}{384} u_{tttt} + \frac{\tau^3}{48} h u_{tttx} + \frac{\tau^2}{16} h^2 u_{ttxx} + \frac{\tau}{12} h^3 u_{txxx} + \frac{h^4}{24} u_{xxxx} + \dots,$  \par
\medskip
$u_{m}^{n}     = u - \frac{\tau}{2} u_t + \frac{\tau^2}{8} u_{tt} - \frac{\tau^3}{48} u_{ttt} + \frac{\tau^4}{384} u_{tttt} + \dots,$ \par
\medskip
$u_{m+1}^{n}   = u - \frac{\tau}{2} u_t + h u_x + \frac{\tau^2}{8} u_{tt} - \frac{\tau}{2} h u_{tx} + \frac{h^2}{2} u_{xx} - \frac{\tau^3}{48} u_{ttt} + \frac{\tau^2}{8} h u_{ttx} - \frac{\tau}{4} h^2 u_{txx} + \ + \frac{h^3}{6} u_{xxx} + \frac{\tau^4}{384} u_{tttt} - \frac{\tau^3}{48} h u_{tttx} + \frac{\tau^2}{16} h^2 u_{ttxx} - \frac{\tau}{12} h^3 u_{txxx} + \frac{h^4}{24} u_{xxxx} + \dots,$ \par
\medskip
$u_{m-1}^{n+1} = u + \frac{\tau}{2} u_t - h u_x + \frac{\tau^2}{8} u_{tt} - \frac{\tau}{2} h u_{tx} + \frac{h^2}{2} u_{xx} + \frac{\tau^3}{48} u_{ttt} - \frac{\tau^2}{8} h u_{ttx} + \frac{\tau}{4} h^2 u_{txx} - \ - \frac{h^3}{6} u_{xxx} + \frac{\tau^4}{384} u_{tttt} - \frac{\tau^3}{48} h u_{tttx} + \frac{\tau^2}{16} h^2 u_{ttxx} - \frac{\tau}{12} h^3 u_{txxx} + \frac{h^4}{24} u_{xxxx} + \dots,$ \par
\medskip
$u_{m}^{n+1}   = u + \frac{\tau}{2} u_t + \frac{\tau^2}{8} u_{tt} + \frac{\tau^3}{48} u_{ttt} + \frac{\tau^4}{384} u_{tttt} + \dots,$ \par
\medskip
$u_{m+1}^{n+1} = u + \frac{\tau}{2} u_t + h u_x + \frac{\tau^2}{8} u_{tt} + \frac{\tau}{2} h u_{tx} + \frac{h^2}{2} u_{xx} + \frac{\tau^3}{48} u_{ttt} + \frac{\tau^2}{8} h u_{ttx} + \frac{\tau}{4} h^2 u_{txx} + \ + \frac{h^3}{6} u_{xxx} + \frac{\tau^4}{384} u_{tttt} + \frac{\tau^3}{48} h u_{tttx} + \frac{\tau^2}{16} h^2 u_{ttxx} + \frac{\tau}{12} h^3 u_{txxx} + \frac{h^4}{24} u_{xxxx} + \dots.$ \par

\end{flushleft}

\qquad Подставим разложенные функции в исходное дифференциальное уравнение и посчитаем невязку:
\begin{flushleft}
$\delta f^{(h)} = \frac{1}{\tau} \Big( u + \frac{\tau}{2} u_t + \frac{\tau^2}{8} u_{tt} + \frac{\tau^3}{48} u_{ttt} + \frac{\tau^4}{384} u_{tttt} - \big(u - \frac{\tau}{2} u_t + \frac{\tau^2}{8} u_{tt} - \frac{\tau^3}{48} u_{ttt} + \frac{\tau^4}{384} u_{tttt}\big) \Big) -$ \par
$ -\frac{\sigma}{h^2} \Big(u + \frac{\tau}{2} u_t + h u_x + \frac{\tau^2}{8} u_{tt} + \frac{\tau}{2} h u_{tx} + \frac{h^2}{2} u_{xx} + \frac{\tau^3}{48} u_{ttt} + \frac{\tau^2}{8} h u_{ttx} + \frac{\tau}{4} h^2 u_{txx} + \frac{h^3}{6} u_{xxx} + \frac{\tau^4}{384} u_{tttt} + \frac{\tau^3}{48} h u_{tttx} + \frac{\tau^2}{16} h^2 u_{ttxx} + \frac{\tau}{12} h^3 u_{txxx} + \frac{h^4}{24} u_{xxxx}
-2 \cdot \big(u + \frac{\tau}{2} u_t + \frac{\tau^2}{8} u_{tt} + \frac{\tau^3}{48} u_{ttt} + \frac{\tau^4}{384} u_{tttt} \big) + u + \frac{\tau}{2} u_t - h u_x + \frac{\tau^2}{8} u_{tt} - \frac{\tau}{2} h u_{tx} + \frac{h^2}{2} u_{xx} + \frac{\tau^3}{48} u_{ttt} -\frac{\tau^2}{8} h u_{ttx} + \frac{\tau}{4} h^2 u_{txx} - \frac{h^3}{6} u_{xxx} + \frac{\tau^4}{384} u_{tttt} - \frac{\tau^3}{48} h u_{tttx} + \frac{\tau^2}{16} h^2 u_{ttxx} - \frac{\tau}{12} h^3 u_{txxx}  + \frac{h^4}{24} u_{xxxx} \Big) + \frac{1-\sigma}{h^2} \Big(u - \frac{\tau}{2} u_t + h u_x + \frac{\tau^2}{8} u_{tt} - \frac{\tau}{2} h u_{tx} + \frac{h^2}{2} u_{xx} - \frac{\tau^3}{48} u_{ttt} + \frac{\tau^2}{8} h u_{ttx} - \frac{\tau}{4} h^2 u_{txx} + \frac{h^3}{6} u_{xxx} + \frac{\tau^4}{384} u_{tttt} - \frac{\tau^3}{48} h u_{tttx} + \frac{\tau^2}{16} h^2 u_{ttxx} - \frac{\tau}{12} h^3 u_{txxx} + \frac{h^4}{24} u_{xxxx} -$ \par
$- 2\cdot \big(u - \frac{\tau}{2} u_t + \frac{\tau^2}{8} u_{tt} - \frac{\tau^3}{48} u_{ttt} + \frac{\tau^4}{384} u_{tttt}\big) + u - \frac{\tau}{2} u_t - h u_x + \frac{\tau^2}{8} u_{tt} + \frac{\tau}{2} h u_{tx} + \frac{h^2}{2} u_{xx} - \frac{\tau^3}{48} u_{ttt} - \frac{\tau^2}{8} h u_{ttx} - \frac{\tau}{4} h^2 u_{txx} - \frac{h^3}{6} u_{xxx} + \frac{\tau^4}{384} u_{tttt} + \frac{\tau^3}{48} h u_{tttx} + \frac{\tau^2}{16} h^2 u_{ttxx} + \frac{\tau}{12} h^3 u_{txxx} + \frac{h^4}{24} u_{xxxx} \Big)$
\end{flushleft}
\qquad Выполнив преобразования и сгруппировав коэффициенты перед производными, посчитаем итоговую невязку:
\begin{flushleft}
    $\delta f^{(h)} = \frac{\tau^2}{24} u_{ttt} - (\frac{h^2}{12} - \frac{\tau}{2} + \sigma \tau) u_{txx} + \frac{\tau^2}{8} u_{ttxx}$,
\end{flushleft}
из которой видно, что при $\sigma \neq \frac{1}{2}$ схема имеет порядок аппрокисмации $o(\tau + h^2)$, а при $\sigma = \frac{1}{2}$ схема имеет порядок аппроксимации $o(\tau^2 + h^2)$. Если же взять $\sigma = \frac{1}{2} - \frac{h^2}{12 \tau}$, занулив тем самым скобку $(\frac{h^2}{12} - \frac{\tau}{2} + \sigma \tau)$, то порядок аппроксимации схемы будет $o(\tau^2 + h^4)$\par

\item {\large\textbf{Устойчивость}} \par
\qquad Для исследования разностной схемы на устойчивость используем спектральный признак [3,21]. Сделаем замену $u_m^n = \rho_q^n e^{i q x_m}$ и подставим в заданную разностную схему:
\begin{tiny}
\begin{equation*}
\begin{gathered}
    \frac{1}{\tau}\left(\rho_q^{n+1} e^{i q x_m}-\rho_q^n e^{i q x_m}\right)=\frac{\sigma}{h^2}\left(\rho_q^{n+1} e^{i q x_{m+1}}-2 \rho_q^{n+1} e^{i q x_m}+\rho_q^{n+1} e^{i q x_{m-1}}\right)+\frac{1-\sigma}{h^2}\left(\rho_q^n e^{i q x_{m+1}}-2 \rho_q^n e^{i q x_m}+\rho_q^n e^{i q x_{m-1}}\right) \\
    \rho_q - 1 = \frac{\tau \sigma }{h^2} \left(\rho_q e^{i q h} -2 \rho_q + \rho_q e^{-i q h}\right ) + \frac{\tau (1-\sigma)}{h^2} \left(e^{i q h} - 2 + e^{-i q h}\right ) \\
    e^{i q h} - 2 + e^{-i q h} = \cos{qh} + isin{qh} + cos{qh} - isin{qh} - 2 = 2(cos{qh} - 1) = -4 sin^2{\frac{qh}{2}} \\
    \rho_q - 1 = \frac{\tau \sigma }{h^2} \rho_q \left(-4 sin^2{\frac{qh}{2}} \right ) + \frac{\tau (1-\sigma)}{h^2} \left(-4 sin^2{\frac{qh}{2}} \right) \\
    \rho_q \left( 1 + \frac{4 \tau \sigma}{h^2} \sin^2{\frac{qh}{2}}\right) = \frac{4 \tau (\sigma - 1)}{h^2} \sin^2{\frac{qh}{2}} + 1 \\
    \rho_q = \frac{1 - \frac{4 \tau (1 - \sigma)}{h^2} \sin^2{\frac{qh}{2}} }{ 1 + \frac{4 \tau \sigma}{h^2} \sin^2{\frac{qh}{2}}}
\end{gathered}
\end{equation*}
\end{tiny}
Для устойчивости схемы необходимо, чтобы $|\rho_q|\leq 1$. Условие $\rho_q \leq 1$ выполняется при любых значениях $\sigma$. Рассмотрим условие $\rho_q \geq -1$:
\begin{equation*}
    \begin{gathered}
        1 - \frac{4 \tau (1 - \sigma)}{h^2} \sin^2{\frac{qh}{2}} \geq  -1 - \frac{4 \tau \sigma}{h^2} \sin^2{\frac{qh}{2}} \\
         1 - \frac{4 \tau}{h^2} \sin^2{\frac{qh}{2}} + \frac{4 \tau \sigma)}{h^2} \sin^2{\frac{qh}{2}} \geq  -1 - \frac{4 \tau \sigma}{h^2} \sin^2{\frac{qh}{2}}  \\
         \frac{8 \tau \sigma}{h^2} \sin^2{\frac{qh}{2}} \geq \frac{4 \tau}{h^2} \sin^2{\frac{qh}{2}} - 2  \\
         4 \tau \sigma \sin^2{\frac{qh}{2}} \geq 2 \tau \sin^2{\frac{qh}{2}} - h^2  \\
         \sigma \geq \frac{2 \tau \sin^2{\frac{qh}{2}} - h^2}{4 \tau \sin^2{\frac{qh}{2}}} \geq \frac{1}{2} - \frac{h^2}{4 \tau}.
    \end{gathered}
\end{equation*}
Отсюда мы можем сделать вывод, что схема является условно устойчивой.

\item {\large\textbf{Метод прогонки}} \par

\qquad Выполним преобразования и перепишем схему в следующем виде:
\begin{small}
\begin{equation*}
    u_{m+1}^{n+1} - u_{m}^{n+1} \left( \frac{h^2 + 2 \tau \sigma}{\tau \sigma } \right) + u_{m-1}^{n+1} = u_{m+1}^{n} \left( \frac{\tau (\sigma - 1)}{\tau \sigma} \right) + u_{m}^{n} \left( \frac{2\tau (1 - \sigma) - h^2}{\tau \sigma} \right) + u_{m-1}^{n} \left( \frac{\tau (\sigma - 1)}{\tau \sigma} \right).
\end{equation*}
\end{small}
Введем замену:
\begin{equation*}
    k=\frac{\tau}{h^{2}},\ \ \ d_{m}=u_{m}^{n}+k\left(1-\sigma \right)\left(u_{m+1}^{n}-2u_{m}^{n}+u_{m-1}^{n}\right).
\end{equation*}

Подставив эту замену в разностную схему получим:
\begin{equation*}
    -\sigma ku_{m-1}^{n+1}+\left(1+2\sigma k\right)u_{m}^{n+1}-\sigma ku_{m+1}^{n+1}=d_{m}.
\end{equation*}

Введем обозначение:
\begin{equation*}
\begin{gathered}
    u_{m}^{n+1}=a_{m}^{n+1}u_{m+1}^{n+1}+b_{m}^{n+1}  \\
    u_{m-1}^{n+1}=a_{m-1}^{n+1}u_{m}^{n+1}+b_{m-1}^{n+1}   
\end{gathered}
\end{equation*}
и подставим его в разностную схему:
\begin{equation*}
    -\sigma k\left(a_{m-1}^{n+1}u_{m}^{n+1}+b_{m-1}^{n+1}\right)+\left(1+2\sigma k\right)u_{m}^{n+1}-\sigma ku_{m+1}^{n+1}=d_{m}.
\end{equation*}

Выражая из нее $u_m^{n+1}$, получим:
\begin{equation*}
    u_{m}^{n+1}=\frac{sk}{1+2\sigma k-a_{m-1}^{n+1}\sigma k}u_{m+1}^{n+1}+\frac{d_{m}+b_{m-1}^{n+1}\sigma k}{1+2\sigma k-a_{m-1}^{n+1}\sigma k}.
\end{equation*}

Сравнивая это выражение с введенной заменой, получим, что:
\begin{equation*}
    \begin{gathered}
        a_{m}^{n+1}=\frac{sk}{1+2\sigma k-a_{m-1}^{n+1}\sigma k},  \\
        b_{m}^{n+1}=\frac{d_{m}+b_{m-1}^{n+1}\sigma k}{1+2\sigma k-a_{m-1}^{n+1}\sigma k}.
    \end{gathered}
\end{equation*}

Для $m=1$ будем иметь:
$ a_{1}^{n+1}=\frac{\sigma k}{1+2\sigma k}, \ 
b_{1}^{n+1}=\frac{d_{1}+3\sigma k\alpha}{1+2\sigma k}$.

\item {\large\textbf{Точное решение}} \par

Решение заданного дифференциального уравнения будем искать в виде:
\begin{equation*}
    u(x, t)=U(x, t)+V(x, t).
\end{equation*}
$U(x,t)$ будем искать методом разделения переменных в виде:
\begin{equation*}
    U(x,t) = X(x)T(t).
\end{equation*}
Подставляя это выражение в заданное уравнение, получим задачу Штурма-Луивилля:
\begin{equation*}
        \frac{X^{''}(x)}{X(x)}=\frac{T^{'}(t)}{T(t)}=-\lambda^2 
\end{equation*}
или 
\begin{equation*}
    \begin{cases}
        X^{''}(x)+\lambda^2 X(x)=0 \\
        T^{'}(t)+\lambda^2 T(t)=0 
    \end{cases}.
\end{equation*}
Решая эту систему, получим общее решение:
\begin{equation*}
    \begin{cases}
        X(x)=C_1 \cos \lambda x+C_2 \sin \lambda x  \\
        T(t)=C_3 e^{-\lambda^2 t} 
    \end{cases} .
\end{equation*}
Подставляя это решение в граничные условия, получим:
\begin{equation*}
    \begin{cases}
        c_1 = 0  \\
        c_2 \sin{\lambda}
    \end{cases}
\end{equation*}
$c_2$ берем не равным нулю, следовательно $\sin{\lambda} = 0$, т.е. $\lambda = \pi m$, где $m \in Z$. \par
Тогда общее решение уравнения примет вид:
\begin{equation*}
    U(x,t) = \sum\limits_{m=1}^{\infty} U_m (x,t) = \sum\limits_{m=1}^{\infty} c_m e^{-(\pi m)^2 t} \sin{\pi m x}. 
\end{equation*}

$V(x,t)$ будем искать в виде $V(x,t) = c_1 x + c_2$. Подставим его в граничные условия и получим систему:
\begin{equation*}
    \begin{cases}
        c_1 = -2 \alpha  \\
        c_2 =  3 \alpha
    \end{cases}.
\end{equation*}
Решение $V(x,t) = \alpha (3 - 2x)$. Тогда общее решение можно записать в виде:
\begin{equation*}
    u(x,t) = \alpha (3 - 2x) + \sum\limits_{m=1}^{\infty} c_m e^{-(\pi m)^2 t} \sin{\pi m x}.
\end{equation*}
Так как сравнивать решения мы будем при $t=20$, то суммой можно пренебречь и общее решение будет иметь вид:
\begin{equation*}
    u(x,t) =  \alpha (3 - 2x).
\end{equation*}


\end{enumerate}
\item {\large\textbf{Результаты численных расчётов}} \par
\qquad Вычислим значения функции на $x\in [0,1]$ с шагом $h=0.1$ по $x$ при $t=20$.
Результаты представлены в таблице 1. \par
Таблица 1 — Точное и приближенное решения, абсолютная и относительная погрешности.
\begin{table}[H]
    \centering
    \resizebox{\textwidth}{!}{%
    \begin{tabular}{|c|c|c|c|c|}
     \hline
	$x$ & Точное решение & Приближенное решение & Абсолютная погрешность & Относительная погрешность   \\ \hline
0.0 & -6.900000 & -6.900000 & 0.0 & 0.0 \\ \hline
0.1 & -6.440000 & -6.440000 & $0.888178\cdot 10^{-15}$ & $0.137916\cdot 10^{-15}$\\ \hline
0.2 & -5.980000 & -5.980000 & $0.888178\cdot 10^{-15}$ & $0.148525\cdot 10^{-15}$\\ \hline
0.3 & -5.520000 & -5.520000 & $0.888178\cdot 10^{-15}$ & $0.160902\cdot 10^{-15}$\\ \hline
0.4 & -5.060000 & -5.060000 & $0.888178\cdot 10^{-15}$ & $0.175529\cdot 10^{-15}$\\ \hline
0.5 & -4.600000 & -4.600000 & 0.0 & 0.0\\ \hline
0.6 & -4.140000 & -4.140000 & $0.888178\cdot 10^{-15}$ & $0.214536\cdot 10^{-15}$\\ \hline
0.7 & -3.680000 & -3.680000 & $0.133227\cdot 10^{-14}$ & $0.362029\cdot 10^{-15}$\\ \hline
0.8 & -3.220000 & -3.220000 & $0.888178\cdot 10^{-15}$ & $0.275832\cdot 10^{-15}$\\ \hline
0.9 & -2.760000 & -2.760000 & $0.444089\cdot 10^{-15}$ & $0.160902\cdot 10^{-15}$\\ \hline
1.0 & -2.300000 & -2.300000 & 0.0  & 0.0 \\ \hline
    \end{tabular}}
\end{table} \par
\newpage
\item {\large\textbf{Вывод}} \par

\qquad В ходе выполнения лабораторной работы была построена разностная схема, которая была исследована на аппроксимацию и устойчивость, и по которой методом прогонки были посчитаны приближенные значения заданного дифференциального уравнения. Затем аналитически было посчитано точное решение заданного дифференциального уравнения и проведено сравнение точного решения с приближенным.

\end{enumerate}

\newpage
\begin{center}
\refstepcounter{section} %гиперссылка
\addcontentsline{toc}{section}{Лабораторная работа №21}
\section*{\large Лабораторная работа №21 \\
Решение уравнения гиперболического типа.\\ Разностная схема «крест».}
\end{center}
\renewcommand{\labelenumi}{\textbf{\arabic{enumi}.}}
\renewcommand{\labelenumii}{\textbf{\arabic{enumi}.\arabic{enumii}}}
\renewcommand{\labelenumiii}{\textbf{\arabic{enumi}.\arabic{enumii}.\arabic{enumiii}}}
\renewcommand{\labelenumiv}{\textbf{\arabic{enumi}.\arabic{enumii}.\arabic{enumiii}.\arabic{enumiv}}}

\begin{enumerate}
\large\item {\large \textbf{Постановка задачи}}
\par
Для дифференциальной задачи:
$$
\begin{gathered}
\frac{\partial^2u}{\partial t^2}=\frac{1}{a^2}\frac{\partial^2u}{\partial x^2};\\
u(x,0)=2\sin{\pi x},\;\;\;\frac{\partial u}{\partial t}(x,0)=0,\;\;\;0\le x\le 1;\\
u(0,t)=0,\;\;\;u(1,t)=0,\;\;\;0\le t<\infty
\end{gathered}
$$
методом конечных разностей численно найти решение при   $t=\frac{a}{2}+1$. Положим $a=2+0.01N$.

Для численного решения использовать шаблон разностной схемы – «крест» с аппроксимацией начальных данных $O(\tau^2)$ , где $\tau$  - шаг по времени.

\large\item {\large \textbf{Теоретический материал}}
\begin{enumerate}
\item {\large\textbf{Разностная схема}} \par
\qquad Воспользуемся шаблоном разностной схемой "крест" [3] для численого решения дифференциальной задачи (21.1): 
\begin{figure}[H]
    \centering
    \includegraphics[width=0.3\linewidth]{../pics/cross_scheme.png}\par
    Рисунок 1 — Шаблон разностной схемы "крест".\\
\end{figure}

\begin{equation}\tag{21.1}
\begin{cases}
\frac{\partial^2u}{\partial t^2}=\frac{1}{a^2}\frac{\partial^2u}{\partial x^2};\\
u(x,0)=2\sin{\pi x},\;\;\;\frac{\partial u}{\partial t}(x,0)=0,\;\;\;0\le x\le 1;\\
u(0,t)=0,\;\;\;u(1,t)=0,\;\;\;0\le t<\infty
\end{cases}
\end{equation}
Заменим производные в задаче следующим образом:
\begin{equation*}
\begin{gathered}
        \frac{\partial^2 u}{\partial t^2}=\frac{u_m^{n+1}-2 u_m^n+u_m^{n-1}}{\tau^2},\\
        \\
        \frac{\partial^2 u}{\partial x^2}=\frac{u_{m+1}^n-2 u_m^n+u_{m-1}^n}{h^2} .
\end{gathered}
\end{equation*}
Дифференциальная задача примет вид (21.2):
\begin{equation}\tag{21.2}
    \begin{cases}
\frac{1}{\tau^2}\left(u_m^{n+1}-2 u_m^n+u_m^{n-1}\right)=\frac{1}{a^2 h^2}\left(u_{m+1}^n-2 u_m^n+u_{m-1}^n\right), \quad 1 \leq m \leq M-1 ; \\
u_m^0=2 \sin (\pi mh) ;  \quad \  \ 1 \leq m \leq M;\\
u_0^{n}=0 ; \quad u_M^{n}=0 ;  \quad 1 \leq n \leq N .
    \end{cases}
\end{equation}
Найдём $u_m^1$ для того, чтобы начать расчёт по схеме. Разложим в ряд Тейлора $u_m^1$ с центром разложения в точке $(x_m, 0)$:
\begin{equation*}
\begin{gathered}
        u_m^1 \cong u_m^0+\tau \frac{\partial u_m^0}{\partial t}+\frac{\tau^2 \partial^2 u_m^0}{2 \partial t^2}+\dots \ = u_m^0+\tau \cdot 0+\frac{\tau^2}{2}\left(\frac{1}{a^2} \frac{d^2\left(2 \sin \pi x_m\right)}{d x^2}\right)=\\
        =u_m^0-\frac{\tau^2 \pi^2 \sin (\pi mh)}{a^2} .
\end{gathered}
\end{equation*}
\item {\large\textbf{Аппроксимация}} \par
\qquad Исследуем на аппроксимацию разностную схему (21.3).
\begin{equation}\tag{21.3}
\frac{1}{\tau^2}\left(u_m^{n+1}-2 u_m^n+u_m^{n-1}\right)=\frac{1}{a^2 h^2}\left(u_{m+1}^n-2 u_m^n+u_{m-1}^n\right).
\end{equation}

Для этого разложим в ряд Тейлора функции $u_{m\pm 1}^n,u_m^{n \pm 1}$  в точке $(x_m, t_n)$:
\begin{equation}\tag{21.4}
    \begin{aligned}
    & u_{m \pm 1}^n=u \pm h u_x+\frac{h^2}{2} u_{x x} \pm \frac{h^3}{6} u_{x x x}+\frac{h^4}{24} u_{x x x x}+\cdots \\
    & u_m^{n\pm 1}=u \pm \tau u_t+\frac{\tau^2}{2} u_{t t} \pm \frac{\tau^3}{6} u_{t t t}+\frac{\tau^4}{24} u_{t t t t}+\cdots
\end{aligned}
\end{equation}
Подставим полученные разложения (21.4) в (21.3):

\begin{equation}\tag{21.5}
\begin{gathered}
\frac{1}{\tau^2}(u + \tau u_t+\frac{\tau^2}{2} u_{t t} + \frac{\tau^3}{6} u_{t t t}+\frac{\tau^4}{24} u_{t t t t}+\cdots \ -2u_m^n+ \\
+u - \tau u_t+\frac{\tau^2}{2} u_{t t} - \frac{\tau^3}{6} u_{t t t}+\frac{\tau^4}{24} u_{t t t t}+\cdots \ )=\\
= \frac{1}{a^2 h^2}(u + h u_x+\frac{h^2}{2} u_{x x} + \frac{h^3}{6} u_{x x x}+\frac{h^4}{24} u_{x x x x}+\cdots\ -2 u_m^n+\\
+u - h u_x+\frac{h^2}{2} u_{x x} - \frac{h^3}{6} u_{x x x}+\frac{h^4}{24} u_{x x x x}+\cdots).
\end{gathered}
\end{equation} 
Преобразуем (21.5) и найдём невязку:
\begin{equation*}
\begin{gathered}
\delta f^{(h)}=\frac{1}{\tau^2}\left(\tau^2 u_{t t}+\frac{\tau^4}{12} u_{t t t}\right)-\frac{1}{a^2 h^2}\left(h^2 u_{x x}+\frac{h^4}{12} u_{x x x}\right)=\frac{\tau^2}{12} \frac{\partial^4 u}{\partial t^4}-\frac{h^2}{12} \frac{\partial^4 u}{\partial x^4} .
\end{gathered}
\end{equation*}
Следовательно, разностная схема имеет порядок аппроксимации $O(\tau^2+h^2)$.
\item {\large\textbf{Устойчивость}} \par
\qquad Исследуем устойчивость разностной схемы (21.3). Для этого полагаем, что $u_m^n=p_q^n e^{i q x_m}$ и подставляем данное выражение в разностную схему:
\begin{equation}\tag{21.6}
\begin{small}
    \begin{gathered}
        \frac{1}{\tau^2}\left(\rho_q^{n+1} e^{i q x_m}-2 \rho_q^n e^{i q x_m}+\rho_q^{n-1} e^{i q x_m}\right)=\frac{1}{a^2 h^2}\left(\rho_q^n e^{i q x_{m+1}}-2 \rho_q^n e^{i q x_m}+\rho_q^n e^{i q x_{m-1}}\right).
    \end{gathered}
\end{small}
\end{equation}
Поделим выражение на $\rho_q^{n-1} e^{i q x_m}$ и умножим на $ \tau^2$:
\begin{equation}\tag{21.7}
\rho_q^2-2 \rho_q+1=\frac{\rho_q \tau^2}{a^2 h^2}\left(e^{i q h}-2+e^{-i q h}\right) .
\end{equation}
Приведём выражение (21.7) к следующему виду:
\begin{equation}\tag{21.8}
    \rho_q^2-2 \rho_q\left(1-2 \frac{\tau^2}{a^2 h^2} \sin ^2 \frac{q h}{2}\right)+1=0 .
\end{equation}

Выражение (21.8) имеет вид $\varepsilon\rho^2 -2\mu\rho + \nu = 0$, где 
\begin{equation*}
    \begin{aligned}
    & \varepsilon = 1 \\
    & \mu = 1-2 \frac{\tau^2}{a^2 h^2} \sin ^2 \frac{q h}{2}\\
    & \nu = 1
\end{aligned}
\end{equation*}
Для того, чтобы разностная схема была устойчива необходимо, чтобы выполнялось (21.9):
\begin{equation}\tag{21.9}
    \begin{aligned}
    & 1) \  \nu \leq \varepsilon\\
    & 2) \  2|\mu| \leq \varepsilon+\nu
\end{aligned}
\end{equation}
Проверим выполнение пункта 1) из (21.9):\par
 $\nu \leq \varepsilon  \implies 1 \leq 1$ — выполняется. \par
 Проверим выполнение пункта 2) из (21.9):
\begin{equation*}
 2|\mu| \leq \varepsilon+\nu \implies 2| 1-2 \frac{\tau^2}{a^2 h^2} \sin ^2 \frac{q h}{2}| \leq 2 \implies | 1-2 \frac{\tau^2}{a^2 h^2} \sin ^2 \frac{q h}{2}| \leq 1
\end{equation*}
 $ \begin{aligned}
    & 1. \ \ \;  1-2 \frac{\tau^2}{a^2 h^2} \sin ^2 \frac{q h}{2} \leq 1 \\
    & \quad -2 \frac{\tau^2}{a^2 h^2} \sin ^2 \frac{q h}{2} \leq 0\\
    & \quad \quad 2 \frac{\tau^2}{a^2 h^2} \sin ^2 \frac{q h}{2} \geq 0 \text{ — выполняется}.
\end{aligned}$ 
 $ \begin{aligned}
    & \\
    & 2. \ \   1-2 \frac{\tau^2}{a^2 h^2} \sin ^2 \frac{q h}{2} \geq -1 \\
    & \quad -2 \frac{\tau^2}{a^2 h^2} \sin ^2 \frac{q h}{2} \geq -2\\
    &  \quad \frac{\tau^2}{a^2 h^2} \sin ^2 \frac{q h}{2} \leq 1\\
    & \quad \frac{\tau^2}{a^2 h^2} \leq 1 \text{ — условная устойчивость}.
\end{aligned}$\par
\qquad\par
Таким образом, выяснили, что схема является условно устойчивой.
\item {\large\textbf{Точное решение}} \par
\qquad Будем искать решение в виде: $u(x,t) = X(x)T(t).$\par
Подставив в $\frac{\partial^2u}{\partial t^2}=\frac{1}{\alpha^2}\frac{\partial^2u}{\partial x^2}$, получим следующее:
\begin{equation}\tag{21.10}
    \begin{gathered}
        T''X = \frac{1}{a^2}TX'' \implies \\
       \implies  a^2\frac{T''}{T} = \frac{X''}{X}= -\lambda^2\\
       \begin{cases}
           X(x)= Acos(\lambda x)+Bsin(\lambda x)\\
           T(t)= Ccos(\frac{\lambda}{a} t)+Dsin(\frac{\lambda}{a} t)\\
       \end{cases}
    \end{gathered}
\end{equation}
Используя граничные условия, найдём $ A, B $:
\begin{equation*}
\begin{gathered}
       \begin{cases}
           A = 0\\
           Acos(\lambda x)+Bsin(\lambda x) =0
       \end{cases}
       \implies 
       \begin{cases}
           A = 0\\
           Bsin(\lambda x) =0
       \end{cases}
       \implies\\ \implies sin(\lambda x) =0 
       \implies \lambda =\pi m
\end{gathered}
\end{equation*}
Решение примет вид (21.11):
\begin{equation}\tag{21.11}
\begin{gathered}
        u(x,t) = \sum\limits_{m=1}^\infty(C_mcos(\frac{\pi m}{a} t)+D_msin(\frac{\pi m}{a} t))sin(\pi mx)
\end{gathered}
\end{equation}
Используя начальные условия, найдём $ C, D $:
\begin{equation*}
\begin{gathered}
       \begin{cases}
           2sin(\pi x) = \sum\limits_{m=1}^\infty C_m sin(\pi m x)\\
           0=\sum\limits_{m=1}^\infty D_m \frac{\pi m}{a} sin(\pi m x)
       \end{cases}
       \implies m =1 \implies\\
       \implies
       \begin{cases}
           C_m = 2\\
           D_m = 0 
       \end{cases}
\end{gathered}
\end{equation*}
В итоге, получим решение вида (21.12):
\begin{equation}\tag{21.12}
\begin{gathered}
        u\left(x,t\right)=2\cos\left(\frac{\pi}{a}t\right)\sin\left(\pi x\right).
\end{gathered}
\end{equation}
\end{enumerate}
\large\item {\large \textbf{Результаты численных расчётов}}\par
\qquad Для поиска точного и приблиближенного решения дифференциальной задачи возьмём шаги $\tau =0.01$ и $ h = 0.1$.\par
Таблица 1 — Заданные точки, точные значения, приближенные значения, абсолютная и относительная погрешности.
\begin{table}[H]
    \centering
    \resizebox{\textwidth}{!}{%
    \begin{tabular}{|c|c|c|c|c|}
        \hline
        x    & Точное значение & Приближенное значение & Абсолютная погрешность & Относительная погрешность\\    \hline
    0.0 &    -0.0 &     0.0 & 0.0       & 0.0 \\ \hline
    0.1 &    -0.618015 &    -0.617785 & $2.298802\cdot 10^{-4}$      & $3.719654\cdot 10^{-4}$  \\ \hline
    0.2 &    -1.175535 &    -1.175097 & $4.372583\cdot 10^{-4}$      & $3.719654\cdot 10^{-4}$  \\ \hline
    0.3 &    -1.617985 &    -1.617383 & $6.018345\cdot 10^{-4}$      & $3.719655\cdot 10^{-4}$  \\ \hline
    0.4 &    -1.902055 &    -1.901347 & $7.074990\cdot 10^{-4}$      & $3.719656\cdot 10^{-4}$  \\ \hline
    0.5 &    -1.999939 &    -1.999195 & $7.439085\cdot 10^{-4}$      & $3.719656\cdot 10^{-4}$  \\ \hline
    0.6 &    -1.902055 &    -1.901347 & $7.074991\cdot 10^{-4}$      & $3.719656\cdot 10^{-4}$  \\ \hline
    0.7 &    -1.617985 &    -1.617383 & $6.018347\cdot 10^{-4}$      & $3.719657\cdot 10^{-4}$  \\ \hline
    0.8 &    -1.175535 &    -1.175097 & $4.372586\cdot 10^{-4}$      & $3.719657\cdot 10^{-4}$  \\ \hline
    0.9 &    -0.618015 &    -0.617785 & $2.298805\cdot 10^{-4}$      & $3.719659\cdot 10^{-4}$  \\ \hline
    1.0 &    -0.0 &     0.0 & 0.0  & 0.0   \\ \hline
    \end{tabular}}
\end{table} \par

\large\item {\large \textbf{Вывод}}\par
\qquad В ходе выполнения лабораторной работы была построена и проверена на
устойчивость разностная схема «крест». При вычислении решения по этой
схеме были получены приближенные значения с точностью до $10^{-4}$.
\end{enumerate}

\newpage

\refstepcounter{section} %гиперссылка
\addcontentsline{toc}{section}{Заключение}
\begin{center}
\section*{\large Заключение}
\end{center}
\qquad В ходе работы над курсовой были изучены и проанализированы основные численные методы, широко применяемые для решения различных задач в современной науке и технике. Особое внимание уделялось не только теоретическим основам, но и практической реализации этих методов, что позволило глубже понять их возможности и ограничения. \par
Для каждой рассмотренной задачи были получены численные решения, которые с высокой точностью совпали с точными аналитическими решениями или эталонными данными из справочных материалов. Это подтверждает корректность и надёжность выбранных численных методов, а также правильность их реализации. \par
Кроме того, для каждого метода были подробно представлены расчётные схемы, что способствует лучшему пониманию алгоритмической структуры и последовательности вычислений. Приведённые численные результаты демонстрируют эффективность и применимость методов к реальным задачам, а также позволяют оценить погрешности и стабильность вычислений. \par
Таким образом, проделанная работа не только закрепила теоретические знания в области численных методов, но и расширила практические навыки их применения, что является важным этапом подготовки специалистов, способных решать сложные инженерные и научные задачи с использованием современных вычислительных технологий.
\newpage

\refstepcounter{section} %гиперссылка
\addcontentsline{toc}{section}{Литература}
\begin{center}
\section*{\large Литература}
\end{center}
\begin{enumerate}
  \item Арушанян О.Б., Залёткин С.Ф. \glqq Численное решение обыкновенных дифференциальных уравнений на ФОРТРАНе\grqq - М.: Изд-во МГУ, 1990. 336c.
  \item Бейтмен Г., Эрдейи А. \glqq Таблицы интегральных преобразований. Том 1. перевод с английского Н.Л. Виленкина\grqq - М.: Наука, 1969. 344с.
  \item Березин И.С., Жидков Н.П. \glqq Методы вычислений: учебное пособие\grqq - М.: Физматгиз, т.1, 1959. 464c.
  \item Бородич Л.И., Герасимович А.И., Кеда Н.П., Мелешко И.Н. \glqq Справочное пособие по приближенным методам решения задач высшей математики\grqq - Минск: Высшая школа 1986. 189c.
  \item Воеводин В.В. \glqq Вычислительные основы линейной алгебры\grqq - М.: Наука, 1977. 300c.
  \item Голубев А.И. \glqq Численные методы. Курс лекций. Часть 3\grqq - Рукопись, ВНИИЭФ, №8/12133, 1995. 131c.
  \item Годунов С.К., Рябенький В.С. \glqq Разностные схемы\grqq - М.: Наука, 1973. 400c.
  \item Двайт Г.Б. \glqq Таблицы интегралов и другие математические формулы: перевод с английского Н.В. Леви под редакцией К.А. Семендяева\grqq
  \item Демидович Б.П., Марон И.А. \glqq Основы вычислительной математики\grqq - М.: Наука, 1966. 664c.
  \item Демидович Б.П., Марон И.А., Шувалова Э.З. \glqq Численные методы анализа\grqq - М.: Наука, 1967.367 .
  \item Дробышевич В.И., Дымников В.П., Ривин Г.С. \glqq Задачи по вычислительной математике: учебное пособие для вузов под редакцией Г.И. Марчука\grqq - М.: Наука, 1980. 144c.
  \item Загускин В.Л. \glqq Справочник по численным методам решений трансцендентных уравнений под редакцией А.М. Лопшина\grqq - М.: Физматгиз, 1960. 216c.
  \item Калиткин Н.Н. \glqq Численные методы\grqq - М.: Наука, 1978. 512c.
  \item Камке Э. \glqq Справочник по обыкновенным дифференциальным уравнениям: перевод с немецкого С.В. Фомина\grqq - М.: Наука, 1976. 576c.
  \item Коллатц Л. \glqq Численные методы решения дифференциальных уравнений: перевод с немецкого\grqq - М.: Издательство иностранной литературы, 1953. 459c.
  \item Крылов А.Н. \glqq Лекции о приближенных вычислениях\grqq - М.: Издательство технической литературы, 1954. 398c.
  \item Ланцош К. \grqq Практические методы прикладного анализа: перевод с английского М.3. Кайнера под редакцией А.М Лопшина\grqq - М.: Физматгиз, 1961. 524c.
  \item Марчук Г.И. \glqq Методы вычислительной математики\grqq - М.: Наука, 1980. 536c.
  \item \glqq Математический практикум под редакцией Г.Н. Положего\grqq - М.: Физматгиз, 1960. 512c.
  \item Самарский А.А., Вабищевич П.Н. \glqq Задачи и упражднения по численным методам: Учебное пособие.\grqq - М.: Эдиториал УРСС, 2000. 208c.
  \item Самарский А.А. \glqq Теория разностных схем\grqq - М.: Наука, 1977. 656c.
  \item Фаддеев Д.К., Фаддеева В.Н. \glqq Вычислительные методы линейной алгебры\grqq - М.: Физматгиз, 1963. 734c.
  \item Филиппов А.Ф. \glqq Сборник задач по дифференциальным уравнениям\grqq - М.: Наука, 1965. 89c.
  \item Форсайт Дж., Молер К. \glqq Численное решение систем алгебраических уравнений: перевод с английского под редакцией Г.И. Марчука\grqq - М.: Мир, 1969. 167c.
\end{enumerate}
\end{document}
